\eng{Computer lingusitics}


\selectlanguage{russian}
\section{Lecture 1} % (fold)
\label{sec:lecture_1}
\eng{2015-01-12: Introduction}
\begin{enumerate}
	\item Практические мелкие задачи -- поверхностный синтаксический анализ
	\item Работоспособное приложение
	\item Теория
\end{enumerate}
\eng{Natural Language Processing}
\eng{Computaional linguistics}
Общая лингвистика \begin{itemize}
	\item Синтаксис, синтактика
	\item Семантика
	\item Прагматика -- естественный язык со своими особенностями развивался из соображений удобства решения пракитических задач гоминидов.
\end{itemize}	
Теория формальных языков, Иерархия грамматик Хомского (\eng{Noam Chomsky})

\eng{Quatitative (statistical) linguistics}

Проблема существования языковой универсали (инвариант):
к сожалению среди всех языков универсалью может лищь считаться существование гласных и согласных.
Среди европейских языков -- существование частей речи.

Семиотика -- теория знаковых систем.
Треугольник Фреге
\begin{description}
	\item[Signifier] Смысловой символ
	\item[Signified] Представление в сознании
	\item[Referent] Целевой предмет или явление
\end{description}

Базовые единицы:
\begin{itemize}
	\item фонемы/графемы
	\item лексемы
\end{itemize}

Уровни:
\begin{itemize}
	\item Синтаксичечкий -- предложение
		словосочетания $\to$ сверхфразовые единства
	\item Графематический -- графемы
	\item Морфологический -- слова
	\item Семантический -- элементарная единица ``сема''
	\item Дискурсивный -- \eng{Coherent connected set of sentences}
\end{itemize}

Невозможность взаимо однозначного отображения лексемы в смысл
\begin{description}
	\item[Полисем\'ия] многозначность языковой единицы
	\item[Синонимия] совпадение единиц по смыслу
	\item[Омонимия] совпадение единиц по форме, существенной различие по смыслу
	лексическая
	морфологическая
\end{description}

\begin{itemize}
	\item Семантические сети
	\item Синтез нового текста -- языковая компетенция
	\item векторная модлеь текста (\eng{bag of words})
\end{itemize}

% section lecture_1 (end)

