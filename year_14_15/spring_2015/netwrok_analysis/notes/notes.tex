\documentclass[a4paper]{article}
\usepackage[utf8]{inputenc}

\usepackage{graphicx, url}

\usepackage{amsmath, amsfonts, amssymb, amsthm}
\usepackage{xfrac, mathptmx}

\newcommand{\obj}[1]{{\left\{ #1 \right \}}}
\newcommand{\clo}[1]{{\left [ #1 \right ]}}
\newcommand{\clop}[1]{{\left [ #1 \right )}}
\newcommand{\ploc}[1]{{\left ( #1 \right ]}}

\newcommand{\brac}[1]{{\left ( #1 \right )}}
\newcommand{\induc}[1]{{\left . #1 \right \vert}}
\newcommand{\abs}[1]{{\left | #1 \right |}}
\newcommand{\nrm}[1]{{\left\| #1 \right \|}}
\newcommand{\brkt}[1]{{\left\langle #1 \right\rangle}}
\newcommand{\floor}[1]{{\left\lfloor #1 \right\rfloor}}

\newcommand{\Real}{\mathbb{R}}
\newcommand{\Cplx}{\mathbb{C}}
\newcommand{\Pwr}{\mathcal{P}}

\newcommand{\defn}{\mathop{\overset{\Delta}{=}}\nolimits}

% \usepackage[english, russian]{babel}
% \newcommand{\eng}[1]{\foreignlanguage{english}{#1}}
% \newcommand{\rus}[1]{\foreignlanguage{russian}{#1}}

\title{Structural analysis and visualization of networks}
\author{Nazarov Ivan, \rus{101мНОД(ИССА)}\\the DataScience Collective}
\begin{document}
\selectlanguage{english}
\maketitle

email: lzhukov@hse.ru

Programming: iPython notebooks
Visualization: yEd, Gephi

Linear algebra prerequisites:
	Spares matrices
	Eigenanalysis


Graph $G=(V,E)$. The set $V$ is the set of vertices and $E$ is a subset of $V\times V$.
An element $(u,v)\in E$ is an edge starting at $u$ and ending in $v$.
The incidence matrix is defined as $a_{ij}=1_E\brac{(i,j)}$, so it denotes and edge $i\to j$.

Random graphs are pure mathematics theory created by Erd\"os and Renyui.
Statistical physics for analysis of complex networks.

Network, social network, complex network just another name for a graph.

Power law (scale free) few vertices with high degree, many nodes with few neighbours.

Complex means that reduction of a system actually destroys the systems.
Cannot predict the whole by the studying the parts.

Facebook network -- many worlds network -- typical for a power law random graph.

Complex networks usually have the following characteristics: \begin{enumerate}
	\item Power law distribution of the vertex degree;
	\item Small diameter and average path length;
	\item High propensity to cluster: the number of triangles in the network.
\end{enumerate}

Let's introduce the following local features of the graph: the vertex degree.
\begin{align*}
	\delta^+(v) &\defn \#\obj{\induc{u\in V}\, (u,v)\in E }\\
	\delta^-(v) &\defn \#\obj{\induc{u\in V}\, (v,u)\in E }
\end{align*}

``Any two people are on average separated no more than by six intermediate connections.''
\begin{item}
\item ``The small-world problem'', Stanley Milgram, 1967
\item ``An experimental study of the small-world problem'', Travers j., Milgram S., 1969
\end{item}

Bethe lattice $(V,E)$ is an infinite cycle-free graph with ever node having the same number of neighbours $z > 1$.
Given an edge $(v,u)\in E$ the end vertex $u$ is connected to $z-1$ other neighbours, which means that $N_k = N_{k-1}\cdot (z-1)$ with $N_1 = z$, since the centre vertex is connected to $z$ other vertices.
Thus $N_k = z \brac{z-1}^{k-1}$.
And the total number of nodes is \[S_n \defn 1+\sum_{k = 1}^n N_k = 1 + z \frac{\brac{z-1}^n-1}{(z-1)-1}\] (check this!)







\end{document}