\documentclass[a4paper]{article}
\usepackage[utf8]{inputenc}
%\usepackage{fullpage}

\usepackage{graphicx, url}

\usepackage{amsmath, amsfonts, xfrac}
\usepackage{mathtools}

% \usepackage{tikz}
% \usetikzlibrary{shapes}
% \colorlet{light blue}{blue!50}

\let\originalleft\left
\let\originalright\right
\renewcommand{\left}{\mathopen{}\mathclose\bgroup\originalleft}
\renewcommand{\right}{\aftergroup\egroup\originalright}

\newcommand{\abs}[1]{\lvert #1\rvert}

\newcommand{\Real}{\mathbb{R}}
\newcommand{\Cplx}{\mathbb{C}}

\newcommand{\Ccal}{\mathcal{C}}

\newcommand{\ex}[0]{{\mathbb{E}}}
\newcommand{\pr}[0]{{\mathbb{P}}}
\newcommand{\Var}[0]{\text{var}}

\newcommand{\defn}{\overset{\Delta}{=}}

\newcommand{\re}{\operatorname{Re}\nolimits}
\newcommand{\im}{\operatorname{Im}\nolimits}

\usepackage[english, russian]{babel}
\newcommand{\eng}[1]{\foreignlanguage{english}{#1}}
\newcommand{\rus}[1]{\foreignlanguage{russian}{#1}}

\title{\rus{Зачётная работа по курсу \\ ``Теоретико-вероятностные методы в статистике''}}
\author{\rus{Назаров Иван,} \rus{101мНОД(ИССА)}}

\begin{document}
\selectlanguage{russian}
\maketitle

\section{У2.18} % (fold)
\label{sec:problem_2_18}

Итак необходимо изучить моменты любого порядка случайной величины с плотностью
\[
p(x\rvert a) = C \frac{e^{-\frac{1}{2}\ln^2 x}}{x}\bigl(1 + a\sin(2\pi \ln x)\bigr)
\]
при $\lvert a\rvert \leq 1$. Именно при этих $a$ функция $p(x\rvert a)$ всюду неотрицательна.

Итак необходимо вычислить при целых $k$ выражение
\[
\ex_{p(x\rvert a)} X^k
= C \int_0^\infty x^k \frac{e^{-\frac{1}{2}\ln^2 x}}{x}\bigl(1 + a\sin(2\pi \ln x)\bigr) dx
\]

Проведём замену пременных в интеграле $y = \ln x$ ($x = e^y$, $dy = \frac{dx}{x}$) и получим:
\[
\ex_{p(x\rvert a)} X^k
= C \int_{-\infty}^\infty e^{ky} e^{-\frac{1}{2}y^2}\bigl(1 + a\sin(2\pi y)\bigr) dy
\]

Подынтегральное выражение есть сумма
\[e^{ky} e^{-\frac{1}{2}y^2} + a e^{ky} e^{-\frac{1}{2}y^2}\sin(2\pi y)\]
в силу линейности, искомый интеграл есть сумма интегралов каждого из слагаемых.

Первое слагаемое (константа $C$ опущена):
\begin{align*}
	\int_{-\infty}^\infty e^{ky} e^{-\frac{1}{2}y^2}
	&= \int_{-\infty}^\infty e^{-\frac{1}{2}( y^2 - 2ky + k^2 - k^2)} dy\\
	&= e^{\frac{1}{2}k^2} \sqrt{2\pi} \int_{-\infty}^\infty \frac{1}{\sqrt{2\pi}} e^{-\frac{1}{2}( y-k)^2 } dy\\
	&= e^{\frac{1}{2}k^2} \sqrt{2\pi}
\end{align*}

Для работы со вторым слагаемым заменим $2\pi$ под синусом на параметр $t$.
Также заметим, что для любго $\theta\in \Real$
\[\sin \theta = \frac{1}{2i} \bigl( e^{i\theta} - e^{-i\theta} \bigr)\]
Тогда второе слагаемое (опять-таки константа $C$ опущена) принимает вид:
\begin{align*}
	a \int_{-\infty}^\infty e^{ky} e^{-\frac{1}{2}y^2} \sin(t y) dy
	&= a \int_{-\infty}^\infty e^{ky} e^{-\frac{1}{2}y^2} \sin(t y) dy\\
	&= \frac{a}{2i} \int_{-\infty}^\infty e^{ky} e^{-\frac{1}{2}y^2} \bigl(e^{t i y} - e^{-t i y}\bigr) dy
\end{align*}

Здесь опять интеграл разбивается на два слагаемых, причём одно отличается
от другого знаком при $t$ и перед интегралом -- рассмотрев подробнее одно,
автоматически поличим из него другое. Итак
\begin{align*}
	\int_{-\infty}^\infty e^{ky} e^{-\frac{1}{2}y^2} e^{it y} dy
	&= \int_{-\infty}^\infty e^{-\frac{1}{2}y^2 + ky - \frac{1}{2}k^2 + \frac{1}{2}k^2 } e^{it (y-k) + it k} dy\\
	&= e^{\frac{1}{2}k^2} e^{it k} \int_{-\infty}^\infty e^{-\frac{1}{2}(y-k)^2} e^{it (y-k)} dy\\
	&= \bigl[ x = y-k \bigr] = e^{\frac{1}{2}k^2} e^{it k} \int_{-\infty}^\infty e^{-\frac{1}{2}x^2} e^{it x} dx
\end{align*}

Исследуем интеграл
\[\int_{-\infty}^\infty e^{-b x^2} e^{it x} dx\]
при произвольном $b>0$ и преобразуем его в следующее выражение:
\begin{align*}
	\int_{-\infty}^\infty e^{-b x^2} e^{it x} dx
	&= \int_{-\infty}^\infty e^{-b (x^2 - 2\frac{it}{2b} x - \frac{t^2}{4b^2} + \frac{1}{4b^2}t^2 )} dx\\
	&= \int_{-\infty}^\infty e^{-b (x - \frac{it}{2b})^2} e^{-\frac{1}{4b}t^2} dx\\
	&= e^{-\frac{1}{4b}t^2} \int_{-\infty}^\infty e^{-b (x - \frac{it}{2b})^2} dx
\end{align*}

Для нахождения
\[\int_{-\infty}^\infty e^{-b (x - \frac{it}{2b})^2} dx\]
заменим $\frac{t}{2b}$ на параметр $\eta$ и рассмотрим всюду аналитическую функцию
комплексного аргумета $f(z) = e^{-b (z - i\eta)^2}$, $z\in \Cplx$. Нужно вычислить
значение 
\[\int_\Real e^{-b (z - i\eta)^2} dz
%% сведение к пределу теоремой о доминируемой сходимости
= \lim_{R\to\infty} \int_{-R}^R e^{-b (z - i\eta)^2} dz \]

По теореме Коши, для любой функции $f(z)$ аналитической в области $R$ и содержащей
замкнутый контур $\Ccal$, такой что $\Ccal\sim 0$ гомеоморфным преобразованием
сводящийся к одной точке, справедливо
\[\oint_\Ccal f(z) dz = 0\]

Возьмём произвольное $R>0$ и рассмотрим прямоугольный контур $\Box$ в $\Cplx$ с
узлами $\pm R$ и $\pm R+i\eta$. Поскольку $f(z)$ аналитична всюду, то для любого
$R>0$ внутри контура не содержатся особенности $f(z)$. Поскольку $\Box\sim 0$ в
$\Cplx$ то
\[\oint_\Box f(z) dz = 0\]
Но был специально выбрам так, чтобы состоять из прямых путей в $\Cplx$. Поэтому
в вольной записи
\[
\oint_\Box f(z) dz
= \Bigl( \int_{-R}^R + \int_R^{R+i\eta} + \int_{R+i\eta}^{-R+i\eta} + \int_{-R+i\eta}^{-R} \Bigr) f(z) dz
\]
Искомый интеграл равен
\[\int_{-R}^R e^{-b (z - i\eta)^2} dz
= - \Bigl(\int_R^{R+i\eta} + \int_{R+i\eta}^{-R+i\eta} + \int_{-R+i\eta}^{-R} \Bigr) e^{-b (z - i\eta)^2} dz\]
Итак на ``отрезке'' $[-R+i\eta,R+i\eta]$ функция $f(z)$ легко интегрируется
\begin{align*}
	\int_{R+i\eta}^{-R+i\eta} e^{-b (z - i\eta)^2} dz
	&= \int e^{-b (z - i\eta)^2} 1_{[-R+i\eta,R+i\eta]}(z) dz\\
	&= \bigl[x = z-i\eta\bigr] = \int_R^{-R} e^{-b x^2} dx = -\int_{-R}^R e^{-b x^2} dx
\end{align*}
На $[R,R+i\eta]$ нужен анализ: 
\begin{align*}
	\int_R^{R+i\eta} e^{-b (z - i\eta)^2} dz
	&= \int e^{-b (z - i\eta)^2} 1_{[R,R+i\eta]}(z) dz\\
	&= \bigl[z = R+is \bigr] = \int e^{-b (R + is - i\eta)^2} 1_{[0,\eta]}(s) ids\\
	&= \int_0^\eta i e^{-b \bigr(R + i(s - \eta)\bigl)^2} ds\\
	&= \bigl[\xi = s-\eta \bigr] = \int_{-\eta}^0 i e^{-b(R + i\xi)^2} d\xi\\
	&= \int_{-\eta}^0 i e^{-b (R^2 - \xi^2 + 2 i R\xi)} d\xi
	= e^{-bR^2} \int_{-\eta}^0 i e^{b\xi^2} e^{- i 2bR\xi} d\xi
\end{align*}
Заметим, что этот интеграл имеет очень хорошее свойство:
\begin{align*}
	\biggl\lvert \int_R^{R+i\eta} e^{-b (z - i\eta)^2} dz \biggr\rvert
	&= e^{-bR^2} \biggl\lvert \int_{-\eta}^0 i e^{b\xi^2} e^{- i 2bR\xi} d\xi \biggr \rvert\\
	&\leq e^{-bR^2} \int_{-\eta}^0 \Bigl\lvert i e^{b\xi^2} e^{- i 2bR\xi} \Bigr\rvert d\xi\\
	&= e^{-bR^2} \int_{-\eta}^0 e^{b\xi^2} \Bigl\lvert e^{- i 2bR\xi} \Bigr\rvert d\xi\\
	&= e^{-bR^2} \int_{-\eta}^0 e^{b\xi^2} d\xi \leq e^{-bR^2} \eta e^{b\eta^2}
\end{align*}
поскольку как ни ``крути'' $\theta$ всегда выполняется $\abs{e^{i\theta}} = 1$.
Поэтому этот интеграл бесонечно мал при $R\to\infty$.

\noindent На $[-R+i\eta, R]$ аналогичные выкладки приводят к
\[
\int_{-R+i\eta}^R e^{-b (z - i\eta)^2} dz
= e^{-bR^2} \int_0^{-\eta} i e^{b\xi^2} e^{i 2bR\xi} d\xi
\]
откуда опять-таки 
\[
\biggl\lvert \int_{-R+i\eta}^R e^{-b (z - i\eta)^2} dz \biggr\rvert
\leq e^{-bR^2} \eta e^{b\eta^2}
\]

\noindent Возвращаясь к интегралу $\int_{-R}^R e^{-b (z - i\eta)^2} dz$ имеем
\begin{multline*}
	\biggl\lvert \int_{-R}^R e^{-b (z - i\eta)^2} dz + \int_{R+i\eta}^{-R+i\eta} e^{-b (z - i\eta)^2} dz \biggr\rvert = \\
	= \biggl\lvert \int_{-R}^R e^{-b (z - i\eta)^2} dz - \int_{-R}^R e^{-b x^2} dx \biggr\rvert \leq 2 e^{-bR^2} \eta e^{b\eta^2}
\end{multline*}
Поскольку $\eta$ -- это конечное число, отсюда вытекает 
\[
\lim_{R\to\infty}\int_{-R}^R e^{-b (z - i\eta)^2} dz
= \lim_{R\to\infty} \int_{-R}^R e^{-b x^2} dx
= \int_{-\infty}^\infty e^{-b x^2} dx
\]
Однако
\[
\int_{-\infty}^\infty e^{-b x^2} dx
= \bigl[ x = \frac{1}{\sqrt{2b}}s\bigr]
= \frac{1}{\sqrt{2b}} \int_{-\infty}^\infty e^{-\frac{s^2}{2}} ds
= \sqrt{\frac{2\pi}{2b}}
\]
Таким образом справедливо, что 
\[
\int_{-\infty}^\infty e^{-b (z - i\eta)^2} dz
= \lim_{R\to\infty} \int_{-R}^R e^{-b (z - i\eta)^2} dz
= \sqrt{\frac{2\pi}{2b}}
\]

\noindent Эти рассуждения позволяют получить следующее равенство:
\[
\int_{-\infty}^\infty e^{-b x^2} e^{it x} dx
= e^{-\frac{1}{4b}t^2} \sqrt{\frac{\pi}{b}}
\]
В свою очередь, при $b=\frac{1}{2}$ отсюда следует, что
\[
\int_{-\infty}^\infty e^{ky} e^{-\frac{1}{2}y^2} e^{it y} dy
= e^{\frac{1}{2}k^2} e^{it k} \int_{-\infty}^\infty e^{-\frac{1}{2}x^2} e^{it x} dx
= e^{\frac{1}{2}k^2} \sqrt{2\pi} e^{-\frac{1}{2}t^2} e^{it k}
\]
Заменив $t$ на $-t$, получаем
\[
\int_{-\infty}^\infty e^{ky} e^{-\frac{1}{2}y^2} e^{-it y}
= e^{\frac{1}{2} k^2} \sqrt{2\pi} e^{-\frac{1}{2}t^2} e^{-it k}
\]

\noindent Далее
\begin{multline*}
	a \int_{-\infty}^\infty e^{ky} e^{-\frac{1}{2}y^2} \sin(t y) dy \\
	= \frac{a}{2i} \Bigl( \int_{-\infty}^\infty e^{ky} e^{-\frac{1}{2}y^2} e^{it y} dy
		- \int_{-\infty}^\infty e^{ky} e^{-\frac{1}{2}y^2} e^{-it y} dy \Bigr)\\
	= a e^{\frac{1}{2}k^2} \sqrt{2\pi} e^{-\frac{1}{2}t^2} \frac{1}{2i} \bigl( e^{it k} - e^{-it k}\bigr)\\
	= a e^{\frac{1}{2}k^2} \sqrt{2\pi} e^{-\frac{1}{2}t^2} \sin(t k)
\end{multline*}
Подставляя $t = 2\pi$ получаем
\[
a \int_{-\infty}^\infty e^{ky} e^{-\frac{1}{2}y^2} \sin(2\pi y) dy
= a e^{\frac{1}{2}k^2} \sqrt{2\pi} e^{-\frac{1}{2}t^2} \sin( 2\pi k)
= 0
\]
поскольку все $k$ -- целые.

Итак $k$-й момент распределения равен
\begin{align*}
	\ex_{p(x\rvert a)} X^k
	&= C \int_{-\infty}^\infty e^{ky} e^{-\frac{1}{2}y^2}\bigl(1 + a\sin(2\pi y)\bigr) dy\\
	&= C \int_{-\infty}^\infty e^{ky} e^{-\frac{1}{2}y^2} + C a \int_{-\infty}^\infty e^{ky} e^{-\frac{1}{2}y^2} \sin(2\pi y) dy\\
	&= C e^{\frac{1}{2}k^2} \sqrt{2\pi} + 0
\end{align*}
В частности для $k=0$ из условия $\int p(x\rvert a) dx = 1$ следует, что $C = \frac{1}{\sqrt{2\pi}}$.
Получается, что моменты распределения не зависят от параметра $a$.

Из проведённого анализа можно сделать вывод, что знания даже счётного числа моментов
распределения не всегда достаточно для восстановления его параметров.

% section problem_2_18 (end)

% [ВШЭ ТВМММ] Зачётная работа Назаров Иван.pdf
\end{document}
