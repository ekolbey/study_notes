\documentclass[a4paper]{article}
\usepackage[utf8]{inputenc}
%\usepackage{fullpage}

\usepackage{graphicx, url}

\usepackage{amsmath, amsfonts, xfrac}
\usepackage{mathtools}

\let\originalleft\left
\let\originalright\right
\renewcommand{\left}{\mathopen{}\mathclose\bgroup\originalleft}
\renewcommand{\right}{\aftergroup\egroup\originalright}

\newcommand{\obj}[1]{{\left\{ #1 \right \}}}
\newcommand{\clo}[1]{{\left [ #1 \right ]}}
\newcommand{\clop}[1]{{\left [ #1 \right )}}
\newcommand{\ploc}[1]{{\left ( #1 \right ]}}

\newcommand{\brac}[1]{{\left ( #1 \right )}}
\newcommand{\crab}[1]{{\left ] #1 \right [}}
\newcommand{\induc}[1]{{\left . #1 \right \vert}}
\newcommand{\abs}[1]{{\left | #1 \right |}}
\newcommand{\nrm}[1]{{\left\| #1 \right \|}}
\newcommand{\brkt}[1]{{\left\langle #1 \right\rangle}}

\newcommand{\floor}[1]{{\left\lfloor #1 \right\rfloor}}

\newcommand{\Rbar}{{\bar{\mathbb{R}}}}
\newcommand{\Real}{{\mathbb{R}}}
\newcommand{\Zinf}{{\clo{ 0, +\infty }}}
\newcommand{\Cplx}{{\mathbb{C}}}
\newcommand{\Tcal}{{\mathcal{T}}}
\newcommand{\Dcal}{{\mathcal{D}}}
\newcommand{\Hcal}{{\mathcal{H}}}
\newcommand{\Ccal}{{\mathcal{C}}}
\newcommand{\Scal}{{\mathcal{S}}}
\newcommand{\Ncal}{{\mathcal{N}}}
\newcommand{\Ecal}{{\mathcal{E}}}
\newcommand{\Fcal}{{\mathcal{F}}}
\newcommand{\borel}[1]{{\mathcal{B}\brac{#1}}}
\newcommand{\Ex}[0]{{\mathbb{E}}}
\newcommand{\pr}[0]{{\mathbb{P}}}
\newcommand{\Var}[1]{{\text{Var}\brac{#1}}}

\newcommand{\pwr}[1]{{\mathcal{P}\brac{#1}}}
\newcommand{\Dyns}[1]{{\mathfrak{D}\brac{#1}}}
\newcommand{\Ring}[1]{{\mathcal{R}\brac{#1}}}
\newcommand{\Supp}[1]{{\operatorname{supp}\nolimits\brac{#1}}}

\newcommand{\defn}{{\mathop{\overset{\Delta}{=}}\nolimits}}
\newcommand{\lpto}{{\mathop{\overset{L^p}{\to}}\nolimits}}

\newcommand{\re}{\operatorname{Re}\nolimits}
\newcommand{\im}{\operatorname{Im}\nolimits}

\usepackage[english, russian]{babel}
\newcommand{\eng}[1]{\foreignlanguage{english}{#1}}
\newcommand{\rus}[1]{\foreignlanguage{russian}{#1}}

\title{\rus{Домашняя работа по курсу \\ ``Теоретико-вероятностные методы в статистике''}}
\author{\rus{Назаров Иван,} \rus{101мНОД(ИССА)}}

\begin{document}
\selectlanguage{russian}
\maketitle

\section{У1.8 -- Задача о сборщике купонов} % (fold)
\label{sec:task_1_8}

Предположим, пачки овсянки пронумерованы в порядке их покупки или согласно
серийному номеру самой упаковки (в дальнейшем от мы откажемся от этого упрощения).

Пусть $\Omega$ -- множество всозможных цветов купонов, $\abs{\Omega} = k \geq 2$.
Каждый набор купонов на $n$ купленных пачек овсянки можно сопоставить набору
$\brac{a_i}_{i=1}^n \in \Omega^n$, где $a_i\in \Omega$ -- цвет купона на пачке $i$.

Подсчитаем количество ``плохих'' конфигураций в $\Omega^n$ -- тех в которых
отсутсвуют купоны хотя бы один цвет купона $c\in\Omega$.
Введём обозначение множества конфигуаций, в которых нет цвета $c\in \Omega$ :
\[A_c \defn \obj{\induc{w\in \Omega^n}\, w_i \neq c \forall i = 1\ldots n }\]
Тогда искомые конфигурации есть ни что иное как объединение $A \defn \bigcup_{c\in \Omega} A_с$.

Воспользуемся формулой включений-исключений, которая в случае данной постановки
задачи имеет вид:
\[\abs{ \bigcup_{c\in \Omega} A_c } = \sum_{\emptyset \neq C\subseteq \Omega}
\brac{-1}^{\abs{C} - 1} \abs{\cup_{c\in C} A_c} \]
Для любого непустого $C\subseteq \Omega$ мощность множества $\cup_{c\in C} A_c$
равна $\brac{\abs{\Omega} - \abs{C}}^n$.

Таким образом \begin{align*}
	\abs{ \bigcup_{c\in \Omega} A_c } &= \sum_{\emptyset \neq C\subseteq \Omega}
		\brac{-1}^{\abs{C} - 1} \brac{\abs{\Omega} - \abs{C}}^n \\
		&= \sum_{m=1}^k C^m_k {(-1)}^{m-1} {(k - m)}^n
\end{align*}
где $C^m_k = \frac{k!}{m!(k-m)!}$ -- число неупорядоченных перестановок из $n$ по $k$ элементов.
При этом количество всевозможных конфигураций цветов купонов на $n$ различимых
пачках составляет $\abs{\Omega}^n$ штук.

Количество конфигураций $n$ различимых упаковок, в которых присутствуют все $k$ цветов,
равно разности числа всех конфигураций и числа конфигураций, хотя бы без одного цвета.
Итак
\begin{align*}
	Q_n & = \abs{\Omega}^n - \abs{ \bigcup_{c\in \Omega} A_c } \\
	& = k^n - \sum_{m=1}^k C^m_k {(-1)}^{m-1} {(k - m)}^n \\
	& = \sum_{m=0}^k C^m_k {(-1)}^m {(k - m)}^n \\
	& = k! \left\{\begin{matrix} n \\ k \end{matrix}\right\}
\end{align*}
где $\left\{\begin{matrix} n \\ k \end{matrix}\right\}$ -- число Стирлинга второго рода,
отражающее количество разбиений $n$ элементного множетсва на $k$ непустых групп.



% k = 4 ; n = 4
% res <- replicate( sample( k, n, replace = TRUE ), n = 10000, simplify = FALSE )
% b <- sapply( sapply( res, unique ), length )
% sum( b == k ) / length( b )

% section task_1_8 (end)

\begin{document}
