\documentclass[a4paper]{article}
\usepackage[utf8]{inputenc}

\usepackage{graphicx, url}

\usepackage{amsmath, amsfonts, amssymb, amsthm}
\usepackage{xfrac, mathptmx}

\newcommand{\obj}[1]{{\left\{ #1 \right \}}}
\newcommand{\clo}[1]{{\left [ #1 \right ]}}
\newcommand{\clop}[1]{{\left [ #1 \right )}}
\newcommand{\ploc}[1]{{\left ( #1 \right ]}}

\newcommand{\brac}[1]{{\left ( #1 \right )}}
\newcommand{\induc}[1]{{\left . #1 \right \vert}}
\newcommand{\abs}[1]{{\left | #1 \right |}}
\newcommand{\nrm}[1]{{\left\| #1 \right \|}}
\newcommand{\brkt}[1]{{\left\langle #1 \right\rangle}}
\newcommand{\floor}[1]{{\left\lfloor #1 \right\rfloor}}

\newcommand{\Real}{\mathbb{R}}
\newcommand{\Cplx}{\mathbb{C}}
\newcommand{\Pwr}{\mathcal{P}}
\newcommand{\Fcal}{\mathcal{F}}
\newcommand{\Ex}{\mathbb{E}}

\newcommand{\defn}{\mathop{\overset{\Delta}{=}}\nolimits}

\usepackage[english, russian]{babel}
\newcommand{\eng}[1]{\foreignlanguage{english}{#1}}
\newcommand{\rus}[1]{\foreignlanguage{russian}{#1}}

\title{Integral operators in finance}
\author{Nazarov Ivan, \rus{101мНОД(ИССА)}\\the DataScience Collective}
\begin{document}
\selectlanguage{english}
\maketitle

\section{Lecture \# 1} % (fold)
\label{sec:lecture_1}

Fundamentals of variational calculus

time evolution of the the density functions, derived from the optimization problems.

Applications of theory to managing financial portfolio.

\rus{Владимир Рубенович, +7 (903) 155-27-97 к515 }

emails:
econ2014magistr@gmail.com
hsefondy@gmail.com
frin1315@gmail.com
hse.datasci.2014@gmail.com
apeconomics2014@gmail.com


The main goal is Forecasting the time evolution probability densities.

An example:
Binary options: touch- and leave-options strike when stock price hits or leaves a specified set.
	Terms one minute base time scale
	$Q$ -- amount of money
	If the rate crosses a given level within one minute, then we get a premium, otherwise loose everything

the table of intervals-premia
	 1 -- 60\% -- break even probability 62.5\% (why?)
	 5 -- 68\%
	15 -- 70\%
	30 -- 75\%
	60 -- 80\%

The first quiestions are \begin{itemize}
	\item Is it a deterministic process (use runs-test
	\item what is the probability of up/down movements
\end{itemize}

$P_t\sim D_t$ and the current level is $Q_0$.
The distribution $D_t$ could be quite asymmetric and multi-modal.
The price process is never stable and is highly volatile.
Loses are asymmetric and harmful.

\begin{enumerate}
	\item An empirical distribution constructed on some sliding window can be used for immediate forecasting.
	\item Forecast the financial variable $Q_{t+dt}$ given $Q_0 = Q(t_0)$. This is complex.
	What is being forecast? Prices of returns?
\end{enumerate}

Consider the following partial differential equation (Fokker-Planck equations)

\[\frac{\partial}{\partial t} p(x,t) = -\frac{\partial}{\partial x} \brac{ \mu(x) p(x,t) } + \frac{\partial^2}{\partial x^2} \brac{ \sigma(x) p(x,t) } \]

The analyst may never know what is more efficient: a stable in the short term distribution function or an evolving one?

What is $x$ in the equation? Kolmogorov's equation is a decomposition of the change in $x$ and $t$ and the
\[x+dx, t+dt \models \int w(x,dx) p(x, dx, dt )\ldots \]
Like the truncated Taylor's expansion in the Ito's formula.

This form of the Fokker-Planck equation is just a convention.
Or use a random variable with all central moments equal to zero.
Suppose $dx \sim \mathcal{N}$ then the odd moments are zero.


Subjective perception of the density with shifting local centre.

% We do finance here not mathematics!
% Either forecast or lost.

Independent identically normally distributed increments yes they give a second order equation, but not the normal variate itself.

Euler-Lagrange equation. \[\frac{d }{dt}F_x - F_{\dot{x}} = 0\]

First derivative in the variational problem.

Derivation of the Fokker-Planck equation.

How to make $p(x,t)$ evolve in time? \begin{description}
	\item[Differential] Use differential operators. This approach requires the knowledge of parameterts governing the instantaneous diffusion.\hfill\\
	the solution of the Fokker-Planck equation is determined by two functions $\mu(x)$ and $\sigma(x)$

	The instantaneous drift term ($\frac{d}{dt} x = \mu(x)$).
	If $\mu(x) = m (k-x)$ -- mean reverting process, stabilizes over the long run (Ornstein-Uhlenbeck process)
	The instantaneous volatility $\sigma(x) = \sigma_0^2$ -- constant. Given the Gaussian distribution.

	But the default $\mu(x)$ and $\sigma(x)$ are not good enough. And in addition we don't know them.

	How to obtain the best approximations of $\mu$ and $\sigma$? How to chose the one 
	\begin{enumerate}
		\item Use the basics of stochastic filed theory
		\item Construct the function of instantaneous diffusion as eigenvalues of some Fredholm integral operator.
	\end{enumerate}

	Non parametric estimation of $\mu$ and $\sigma$	(c.f. Florensen (?) )
	% Diffusion DIFFUSION!!!!!

	\item[Integral] Act as if the $\mu$ and $\sigma$ are unknown (the drift and the spread). \hfill \\
	Do not equation explicit expressions of the drift and spread.

	Will attempt to construct the source-wise representation (\rus{исотокобразное представление}).
	Fredholm's equation.
	
\end{description}

How are the densities obtained?

The operator equation in some function space.
\[L_x^\brkt{2}\clo{p} = T_t^\brkt{1}\clo{P}\]
$L_x^\brkt{n}$ -- differential operator of order $n$ with respect to the variable $x$.

\[p(x,t) = \sum_{n\geq 1} c_n u_n(x) e^{-\lambda_n t}\]


Basis of functions $\brac{u_i(\cdot)}_{i\geq 1}$ is orthonormal is it is complete and
\[\int\limits_a^b u_j(x) u_j(x) w(x)dx = \delta_{ij}\]
Like in linear algebra but with functions!!!

For some $f(x)$  it is true that $f(x) = \sum_i c_i u_i(x)$ (if $u_j$ is complete) \[
c_i = \int_a^b f(x)u_i(x)w(x)dx
\]
Orthonormality allows some bound on approximation error for truncated series.

Use the $L^2$ norm in our functional space.

\[\nrm{f-g}_2 \defn \brac{\int \abs{f-g}^2 w(x) dx}^\frac{1}{2}\]

% If you don't kick the hedgehog, it won't fly.
% Construct the evolution of the exchange rate consensus forecasts available in Bloomberg.
% Central bank - interbank liquidity relations determines the exchange rate: REPO deals closure

% Use MathCAD!!!

Literature:
\begin{itemize}
	\item Quantitative Finance
	\begin{enumerate}
		\item 
	\end{enumerate}
\end{itemize}

Choose the process and simulate
Choose a financial task
Complete two laboratory tasks.

$\sigma(x)$ -- variance not standard deviation!!!

\begin{align*}
	\ln \omega(x) &= \ln N + \int \frac{ - \mu(x) + \frac{d}{dx} \sigma(x)}{\sigma(x)} dx
\end{align*}

$\omega(x)$ -- the source density function. Construct an operator, which would make $\omega$ evolve in time.
Basis functions $u_j(x)$ are also the solutions of the second order equation: eigenfunctions. 

\[
  \sigma(x) \frac{d^2}{dx^2} \omega \\
+ \brac{ 2\frac{d}{dx}\sigma(x) - \mu(x) } \frac{d}{dx} \omega \\
+ \brac{ \frac{d^2}{dx^2}\sigma(x) - \frac{d}{dx}\mu(x) } \omega \\
= 0
\]

The basis functions are the solutions of this equation -- eigenfunctions.
And this basis is complete (for functions on the same domain).

A simplified Fokker-Planck equation with $ \frac{\partial}{\partial t} p(x,t) \equiv 0$


Strum-Liuville equation:
\[L_x^\brkt{2}\clo{f} = 0\]

Cauchy problem: eigenfunctions of the operator. The spectral decomposition determine the effects of the operator.
\[L_x^\brkt{2}\clo{u_i(x)} = \lambda u_i(x)\]

In simple cases the spectral values and eigenfunctions can be obtained by separation of variables or Rodrigues.

Polynomial basis.
The functions $\mu(x)$ and $\sigma(x)$ are not arbitrary when Rodrigues' method is applied.

Bochner's theorem generalizes the Rodrigues's method


%% Home work
The Fokker-Planck equation is
\[\frac{\partial}{\partial t} p(x,t) = - \frac{\partial}{\partial x} \brac{ \mu(x,t) p(x,t) } + \frac{\partial^2}{\partial x^2} \brac{ \sigma(x,t) p(x,t) }\]

Another form of the same equation is
\begin{align*}
	\frac{\partial}{\partial t} p(x,t) &= \sigma(x,t) \frac{\partial^2}{\partial^2 x} p(x,t) \\
	&+ \brac{ 2 \frac{\partial}{\partial x} \sigma(x,t) - \mu(x,t) } \frac{\partial}{\partial x} p(x,t) \\
	&+ \brac{ \frac{\partial^2}{\partial x^2} \sigma(x,t) - \frac{\partial}{\partial x} \mu(x,t) } p(x,t) 
\end{align*}

Assume $\mu$ and $\sigma$ are stationary (time-independent), i.e. \[\mu(x,t) = \mu(x) \text{ and }\sigma(x,t) = \sigma(x)\]
and consider $p(x,t)$ of the form $X(x) \cdot T(t)$ -- with variables separated -- where the functions $X$ and $T$ are non-zero.
We are working in the Hilbert space of square integrable twice differentiable real functions on some interval of $\Real$ with the $L^2$ norm.
Then \begin{align*}
	X(x) \frac{\partial}{\partial t} T(t) &= T(t) \sigma(x) \frac{\partial^2}{\partial^2 x} X(x)\\
	&+ T(t) \brac{ 2 \frac{\partial}{\partial x} \sigma(x) - \mu(x) } \frac{\partial}{\partial x} X(x)\\
	&+ T(t) \brac{ \frac{\partial^2}{\partial x^2} \sigma(x) - \frac{\partial}{\partial x} \mu(x) } X(x)
\end{align*}
After minor simplification we get \begin{align*}
	\frac{1}{T(t)} \frac{\partial}{\partial t} T(t) &= \frac{1}{X(x)}\left \{ \sigma(x) \frac{\partial^2}{\partial^2 x} X(x) + \brac{ 2 \frac{\partial}{\partial x} \sigma(x) - \mu(x) } \frac{\partial}{\partial x} X(x) \right.\\
	&\left . + \brac{ \frac{\partial^2}{\partial x^2} \sigma(x) - \frac{\partial}{\partial x} \mu(x) } X(x) \right \}
\end{align*}

Since the left hand side depends on $t$ alone whilst the right-hand side only on $x$,
it must therefore be true that both sides are constant, say $\lambda\in \Real$. Thus
\[ T(t) = C_1 e^{\lambda t} \] and $X(\cdot)$ satisfies
\[\lambda X(x) = \sigma(x) \frac{\partial^2}{\partial^2 x} X(x) + \brac{ 2 \frac{\partial}{\partial x} \sigma(x) - \mu(x) } \frac{\partial}{\partial x} X(x) + \brac{ \frac{\partial^2}{\partial x^2} \sigma(x) - \frac{\partial}{\partial x} \mu(x) }\]
which means that $X$ is the eigenfunction of the following second order differential operator
\[L^\brkt{2}_x \defn \sigma(x) \frac{\partial^2}{\partial^2 x} + \brac{ 2 \frac{\partial}{\partial x} \sigma(x) - \mu(x) } \frac{\partial}{\partial x} + \brac{ \frac{\partial^2}{\partial x^2} \sigma(x) - \frac{\partial}{\partial x} \mu(x) } \]
whereas $T$ is the eigenfunction of $L^\brkt{1}_t \defn \frac{\partial}{\partial t}$ with the same eigenvalue.

% section lecture_1 (end)

\section{Lecture \# 2} % (fold)
\label{sec:lecture_2}

By constructing evolution of the density function, we attempt to forecast the movements of some market variable.

We can try to fit the learning sample as best as possible (asides of the overfitting), but the structure is ever changing.
An analyst cannot expect the studied process to stabilize around some deterministic trends, and must not wait until the average behaviour.
The Job is to forecast and re-evaluate almost continuously. we must be vulgar: rethink, refit and reestimate.
The estimated structure is fleeting as the market is.  

If the market is far away from the EMH equilibrium, then returns are in principle forecastable.
Thus is makes sense to describe and estimate the time-evolution of the density.

% Fokker, Planck and Fourier.
We are trying to obtain the complete basis functions $\brac{u_n(x)}_{n\geq 1}$ for the following for of a solution.
\[p(x,t) = \sum_{n\geq 1} c_n u_n(x) e^{-\lambda_n t}\]
Bochner's theorem: not every parametrization yields a complete polynomial basis.
The discrete spectrum is dictated by the form of the sought solution.
It is also possible to seek the solution with a continuous spectrum.

% Study the Rodrigues method (see the respective file)
% Risken, H.: The Fokker-Planck Equation. Methods of Solution and Applications, 2nd ed. (Springer-Verlag, Berlin, 1989) – p 26
% Зайцев В. Ф., Полянин А. Д.: Метод разделения переменных в математической физике (Книжный Дом, 2009) – стр. 13

% Check the lecture on differential equations.
The Wronsky determinant \[w_{ij} = \frac{d^{j-1}}{dx^{j-1}} \phi_i(x)\]

% Если f_1(x), \ldots , f_n(x) — линейно зависимы на I, то W(x)=0,\forall x\in I.
% Если определитель Вронского на интервале не равен нулю хотя бы в одной точке,
%   то функции f_1(x), \ldots , f_n(x) являются линейно независимыми. Обратное
%   вообще говоря неверно, но для случая, когда функции являются решениями
%   дифференциального уравнения будут верны более сильные следствия.

% See the equation 12.39 in Mathematical physic book


The variational problem:
\[\Phi(u(\cdot), v, x) =  p(x) v^2 + \brac{ q(x) - \beta w(x) } u^2\]
$v$ is the first derivative of $u$ with respect to $x$.
\[\frac{d}{dx}\Phi_u - \phi_v = 0\]

The solution yields the same Sturm-Liuville equation as in the separation of variables.

How many basis functions must there be in the decomposition?
Just estimate the error of the finite sum approximation of the initial density $f(x)$: since $p(x,0) \approx f(x)$.

% Intraday data for 50 assets spanned across pre dot-com crisis,
%  intercrises period, the Global financial crisis and its aftermath.

Cannot calibrate the speed of time-evolution, since the Bochner-compatible parametrization -- $\sigma \in P^2\clo{x}$
and $\mu \in P^1\clo{x}$ -- yields constant eigenvalues of the Sturm-Liuville equation.

It would be nice if the spectrum depended on the learning sample.

A brief excursion:
let $S$ be the asset price, and \[dS = \mu(S) dt + \sqrt{\sigma(S)} \sqrt{dt} \epsilon\]
where the randomness is $\epsilon\sim \mathcal{N}(0,1)$.
$\mu(S) = m S$ and $\sigma(S) = s^2 S^2$ -- the immediate drift and volatility terms. Then
\[\frac{dS}{S} = m dt + s \sqrt{dt} \epsilon\]
-- the Langewin equation.

\[\frac{\partial}{\partial t} p(x,t) = \frac{\partial^2}{\partial x^2} \brac{\sigma(x) p(x,t)} - \frac{\partial}{\partial x} \brac{\mu(x) p(x,t)}\]

\[\tilde{p}(x) = p(x,0)\]

Financial derivatives stichastic differential equation:
\[\frac{\partial}{\partial t} \Fcal(x,t) + \sigma(x) \frac{\partial^2}{\partial x^2} \Fcal(x,t) + \mu(x) \frac{\partial}{\partial x} \Fcal(x,t) = r \Fcal(x,t)\]
Solution \[\Fcal(x,t) e^{r(t-t)} = \Ex_{x,t} \Phi(x)\]

Either the terminal boundary condition (like a payoff function for an option) or the initial boundary condition.

How these approaches are connected? Since both use the same parametrization of the asset price dynamics.

Options as probabilities -- the market estimate of the probability of some abstract terminal payoff.

Hermite conjugate differential operators

The second method of derivation of the complete basis.

The weights of the decomposition are determined by the dot product in the $L^2$ space of the initial density.

\[c_n \defn \int p(x) u_n(x) w(x) dx\]

Either use infinitely large basis, or manage to approximate the evolved density with another solution.

% section lecture_2 (end)

\section{Lecture \#3} % (fold)
\label{sec:lecture_3}

Invariant measure

Perron-Fr\"obeinus tranfomrmation

Integral transformations: decompose the kernel into into the finite basis, and obtain a matrix representation of the integral operator.

If one can tune the spectrum, then it is possible to get \textbf{something} to get the invariant density function.

Trigonometric sinusoidal a basis.

Another canonical basis via the Rodrigues method.

The trigonometric basis works great on a compact support: it is orthogonal and normalised.

In the polynomial case it is hard to find a suitable orthogonality region.

It is time to recall the Gramm-Schmidt orthogonalization procedure.


The weighting function is ``spices for the axe'', and it does not influence the final porridge.
% Why do we need the weighting function at all?

% See hand-written notes.

% If I want to get some files from the lecturer, I should notice him before hand by SMS on Monday.


$\detla = y'$ -- the density function.
\[F(y,\delta, x) = \delta \log \delta + \brac{\alpha \brac{x-\mu} + \beta \brac{x-\sigma}^2 + \epsilon \brac{x-\eta}^3 + \zeta \brac{x-\xi}^4} \delta \]

Shannon's entropy is used for emphasizing the localisation of the data.

Solving the Euler-Lagrange equation for the variational problem with the above functional with respect to the density of $y$ ($\delta$).

\[\lambda \defn \max\abs{r} + \sqrt{\sigma^2(r)}\]

No derivatives or gradients written out in stack -- only the numerical methods.
% Conjugate gradients or Newton-Raphson can easily fail to converge.

% Эрудит -- это библиотека с уже существующими ящичками и шкафчиками
% non-linear Fokker-Planck equation.

If one uses the Perron-Fr\"obenius :
\begin{align*}
	p(x) = f'(x) g(y)\\
	y'(x) = p(x) \frac{1}{q(y)} = p(x) \phi(x)
	x = \int \frac{1}{\phi(y)} dy + c
\end{align*}

Do note try to solve the problem analytically -- it usually brings about non integrable functions, -- resort to numerical methods.

In designing a trading rule on could use quantile functions.

% кривая требует стоп лося

% Не в праве просто совать кошку в микроволновку; её нужно сначала причесать

Enter a market at some ask, hold, and then leave when the bid is satisfactory.
One must forecast both the bid and the ask price series.

We are interested in support of of the empirical process.


The general formulation of the model:
Two approaches to forecasting in finance
\begin{itemize}
	\item We know from behavioural finance that there exists a malevolent player : the next state contrasts the current state
	\[H_{KL} = \int \log\frac{p(x)}{q(x)} p(x) dx + \int \log\frac{q(x)}{p(x)} q(x) dx\]
	Informational divergence between $p$ and $q$. Find such that the derivative of $H$ is zero.
	\[p(x) = \int K(x,\epsilon) g(\epsilon) d\epsilon \]
	see the hand-written notes.

	Extremum of the informational gain due to $H$.

	\item If the kernel is decomposed in another basis, then the kernel is a non-diagonal matrix. If $M$ is hermitian then the spectral theorem 
	% Hamilton-Keli theorem: SVD

% В четверг: СМС в среду


\end{itemize}


% Здравствуйте Владимир Рубенович!

% Подскажите, пожалуйста, возможно ли подойти к Вам на Шаболовку завтра (Среда 4 Февраля) после 1900, для того, чтобы посмотреть Вашу электронную библиотеку (по финансам и уравнениям математической физики в финансах), о которой Вы упоминали на Ваших лекциях?

% В пятницу в конце прошлой лекции Вы обещали прислать прислать на почтовые ящики слушателей письмо со следующим содержимым :
% 	* небольшая подборка интересных книг, связанных с курсом;
% 	* MathCAD файлы к лекции и к Вашему курсу в прошлом семестре;
% 	* материалы по задаче применения информационного расхождения Кульбака-Лейблера (informational gain);
% 	* список ключевых слов для самостоятельного поиска статей и ссылок.

% с уважением,
% Назаров Иван
% Группа 101мНОД(ИССА)


% section lecture_3 (end)

\section{Lecture \# 4} % (fold)
\label{sec:lecture_4}

\subsection{Which basis to choose?} % (fold)
\label{sub:the_choice_of_basis}

Fokker-Planck:
\[\frac{\partial}{\partial t} p(x,t) = \frac{\partial^2}{\partial x^2} \brac{\sigma(x) p(x,t) } - \frac{\partial}{\partial x}\brac{\mu(x) p(x,t)}\]

Separation of variables yields the Cauchy problem (with arbitrary $\lambda$):
\[\sigma(x) f''(x) + \brac{2\sigma'(x) - \mu(x)} f'(x) + \brac{\sigma''(x) - \mu'(x)} f(x) = \lambda f(x)\]

The Sturm-Liuville problem is when $\lambda=0$.

Three types of solutions:
\begin{itemize}
	\item plynomial (with non-zero Wronsy determinant);
	\item trigonometric (sines and cosines).
\end{itemize}

The following density function solves the Sturm-Liuville problem:
\[n e^{-\int \frac{\sigma'(s)-\mu(s)}{\sigma(s)} ds}\]

The selection of basis is of utter importance in financial analysis.

The basic idea is to choose the evolution engine by selecting the parametrization function: the drift ($\mu$) and the direction cone around it ($\sigma$).

\noindent\textbf{A good approach}\hfill \\
Estimate (using the ML) a family of densities defined just above on the learning sample to infer the $\mi(x)$ and $\sigma(x)$.

Diebold's method 
Make an integral CDF transformation of the data, then back to the normal variate (using the inversion method by normal quantiles). Then select the sample, which is best describe by the normal distribution.

% subsection the_choice_of_basis (end)

\subsection{How many basis function to use?} % (fold)
\label{sub:the_effective_subspace_dimension}

This significantly affects the evolution properties and the accuracy of fitting.

Recall that $f(x) = \induc{p(x,t)}_{t=0}$ is the boundary condition on the partial differential equation problem of Fokker-Planck. Thus the initial density is decomposed as:
\[f(x) = p(x, 0) \sim \sum_n^{\text{finite}} c_n u_n(x)\]
where $c_n = \int_a^b p(x) u_n(x) w(x) dx$.

Use as many as needed in order to achieve acceptable approximation error of the initial (sample) distribution:
\[\epsilon = \nrm{f(\cdot) - p(\cdot,0)}_2 = \brac{\int \abs{f(x)-p(x,0)}^2 dx}^\frac{1}{2}\]

Having selected a finite basis $\brac{u_n}_{n=1}^k$, the density now evolves according to
\[p(x,t) \approx \sum_{k=1}^n c_n e^{-\lambda_n t} u_n(x)\]

Each basis function has a corresponding $\lambda_n$.

Thus the first problem is also concerned with the selection of the spectrum.

The evolution of the density is nothing but an evolution of an $k$ dimensional vector, since $p(x, t) \approx C_t' U(x)$
where the vector evolves according to the following decay
\[C_t = \brac{ c_1 e^{-\lambda_1 t},\,\ldots,\, c_k e^{-\lambda_k t} }\]

The solution spectrum depends on the parametrization of the problem!

Note that each basis function is a density function of the Sturm-Liuville problem where .

% subsection the_effective_subspace_dimension (end)

\subsection{How to correct for destabilization} % (fold)
\label{sub:how_to_correct_for_destabilization}

Since a finite basis has been chosen, for any $t\neq0$ the evolved function ceases to be a density function at all, since $C_t\to 0$ as $t\uparrow$.

The general form of the solution does not guarantee that there is no loss or gain of probability in $p(x,t)$.
\[f(x) \leftarrow \sum_n \brac{c_n e^{-\lambda_n t}} u_n(x)\]

% (?) (what is this $g(x)$)
For some $t>0$ one has to re-approximate:
\[\int g(x) u_n(x) w(x) dx \approx c_{tn}\]
where 
\[g(x) = N e^{F(x,\alpha,\beta,\eta)}\]

%% Where does the weighting function come from????

Consider a constrained variational minimization problem:
\[\Fcal(F, f, x) \defn - f \log f + \sum_n \beta_n f(x) u_n(x) w(x) \]
where $f = \frac{d}{dx} F$.

the informational gain is condensed in the entropy term:
\[\int p(x) \log\frac{1}{p(x)} dx\]
%% See the InfoMax files sent by email.

The solution if given by the Euler-Lagrange equation:
\[\frac{d}{dx}\Fcal_f - \Fcal_F = 0\]
where $\Fcal_F = \frac{\partial}{\partial F}\Fcal = 0$.
This the solution involves exponents of integrals.

The problem of integrating exponents of exponents one may use polynomial expansion, but it is very dangerous, or inaccurate, and requires the knowledge of the parameters of $q$.
ALso the problem of numerical integration (but much consideration should be devoted to the precision).

% Бунин: Встреча как разлука
% Шукшин: ненависть являющаяся какрсисом всего -- Любовь как ненависть

% subsection how_to_correct_for_destabilization (end)

\subsection{The actual lecture} % (fold)
\label{sub:the_actual_lecture}

What to is come after the break:
\begin{itemize}
	\item Gaussian weighting, constant spectrum, $\text{dim}=3$;
	\item Gaussian weighting, constant spectrum, $\text{dim}=4$;
	\item Quasi-Gaussian weights, variable spectrum, $\text{dim}=3$.
\end{itemize}

% Risk neutral density is for the option price.

Real pirates regard options (non-linear derivatives) not as a delta-hedging, but as the main asset of trading.

Derivatives prices are considerably more volatile that the underlying assets. The market is however is not liquid enough.

% Diversifying, futures, indices, forex pairs
Every invest banker does not solve new stochastic differential equations, but plugs the known parameters into Black-Scholes formula for a fair price and adds a certain non-zero fee.

Fokker-Planck
The density function must be defined on the same domain as the basis function.
%% Cf. FXCM-long_short.xmcd

Weighting function is $\Real$, while the one, being decomposed, is defined over a finite interval. Thus one needs to transform the domain (probabilistically): random variable transform $Y=G(X)$
\[f(y) = \induc{\frac{1}{\abs{G_x'(x)}}}_{x=G^{-1}(y)}} h\brac{G^{-1}(y)}\]

% c.f. transfrom15.xmcd and Malinkevich

The following \textbf{L}east \textbf{S}quares fitting problem
\[L_t = \sum_{n=1}^k \brac{ \int \Fcal(s;\Theta) u_n(s) w(s) ds - c_n e^{-\lambda_n t} }^2 \to \min_\Theta\]

% Кролики жрут траву с постоянной скоростью так, что при поедании на поверхности травы может образовываться воронка (если наполнить коробку с ним рубленой тарвой до краёв). Куда же девается в них съеденная трава?

Fitness criteria \begin{description}
	\item[Binary options trading] price-change directions; leave-option -- median, entry -- ``close value'';
	\item[Traditional investing] Bid-ask spread, it there a price within the 30 minute interval, where the ask price is higher than the bid price on entry.
\end{description}

%При точности 65% не придерётся ни эконометрист нудний, ни Петрович, который дал нам денги.

% Single-symbol parameters are for the function at $t=0$, double-symbol -- at $t=1$

Implicit functions 
Bivariate density function considered as an implicit function, might uncover different dependence structure between $x$ and $y$.
%% jointimplicit.xmcd


% subsection the_actual_lecture (end)

% section lecture_4 (end)

\section{Lectire \# 5} % (fold)
\label{sec:lectire_5}

Considering $p(x,t)$ agianst $p(x,0) = f(x)$ and $q(x) = p(x,1)$.

Consider the Kullback-Leibler divergence:
\[H_{\text{KL}} \defn \int \log \frac{p(x)}{q(x)} p(x) dx + \int \log \frac{q(x)}{p(x)} q(x) dx \]

The idea is to find the maximum of this information gain between $0$ and $1$.

Give a scheme for finding a parametrisation of the differential operator, so that the information gain between $t_0$ and $t_1$ is maximal.

The parametrisation is the coefficients of the spectral decomposition of and the weights density. What this basis should look like?

the main result: for any $\lambda$ it is true that
\[\sigma(x) p''(x) + \brac{2\sigma'(x) - \mu(x) + (\lambda-1) \sigma'(x)} p'(x) + \brac{\sigma''(x) - \mu'(x) + (\lambda-1) \sigma'(x) f(x) } p(x) = 0\]
where $f(x) = \frac{\sigma'(x) - \mu(x)}{\sigma(x)}$ for \[p(x) = N e^{\int f(s) ds}\]
The problem is that the same solution is also a solution to an infinitely many differential equations.

Solving the variational problem to get a ``good'' density function, with many desirable properties. Then identically the discovered solution with a solution to some parametrized Fokker-Planck equation.

% Entropy plus the determinant of Fisher's information matrix.

% Что сложнее: боковичок или болтанка?

The problem can be reduced to finding the appropriate diffusion parameter $\sigma(x)$. However due to the ``main result'' there are infinitely many possibilities for sigma.

We have $\mu= \sigma'(x) - f(x)\sigma(x)$

It is possible to get $\sigma(x)$ as an eigenfunction of some Fredholm integral operator. Thus may yield a countable of even a finite set of candidates.

How to chose the bes specification of the Fokker-Planck equation?


If $f(x,t)$ is some termed derivative contract for $x$ -- the underlying asset price, then 
\[\frac{\partial}{\partial t} f(x,t) + \mu(x) \frac{\partial}{\partial x}f(x,t) + \frac{1}{2}\sigma^2(x) \frac{\partial^2}{\partial x^2} f(x,t) = r f(x,t)\] 
The probability density function comes here from the terminal value $F(x) = f(x,T)$, then \[f(x,t) = e^{-r(T-t)}\Ex_t F(x)\]

The Langewin equation for $x$ with $\xi\sim \mathcal{N}(0,1)$
\[dx  = \mu(x) dt + \sqrt{\sigma^2(x)dt}\xi\]

%% Утром в Среду -- СМС по поводу встречи с Евстигнеевым в тот же вечер.
%% Заранее написать Email.

\subsection{Integral operators} % (fold)
\label{sub:integral_operators}

Formalize the process, where there is no physical time. A subjective forecasting actually does not support the use of continuous time: the planning happens in discrete time steps. Intuitive foreseeing does not involve smooth time transitions: foreseeing works with discrete events.

The transitions between time moments is discontinuous.

Thus the application of Fokker-Planck operator is philosophically incorrect.

Suppose we have some probability density $p(x)$ and wish to obtain the density $q(x)$ at some future moment.

Recall that $y=f(x)$, then $q(y) = \frac{p\brac{f^{-1}(y)}}{\abs{f'\brac{f^{-1}(y)}}}$ if $f$ is invertible.

This about the following possible transition operator:
\[p(x)\to q(x)\quad : \quad q(x) = \int_a^b K(x,\xi) p(\xi) d\xi\]
where $K(x,\xi)$ is the kernel of the integral operator. 

Consider this equation (Sturm-Liuville when $f = 0$):
\[\sigma(x) \phi''(x) + \brac{2\sigma'(x) - \mu(x)} \phi'(x) + \brac{\sigma''(x) - \mu'(x) } \phi(x) = f(x)\]
where $f(x)$ is an \textbf{external loading} and it makes the equation inhomogeneous. The solution in general is given by
\[\phi(x) = \int G(x,\ix) f(\xi) d\xi\]
where $G(x,\xi)$ is the Greene function, derived form two linearly independent solution of the homogeneous equation.

If we instead consider the Cauchy problem $f=\lambda \phi$, then it is possible to construct a symmetric integral kernel and get a basis of eigenfunctions on the integral operator:
\[u_n(x)\sqrt{\omega(x)} = \lambda_n \int_a^b K(x,\xi) u_n(\xi)\sqrt{\omega(\xi)} d\xi\]

\noindent\textbf{Merser's theorem}\hfill\\
The kernel of the integral operator can be represented as
\[K(x,\xi) = \sum_n \frac{u_n(x)\sqrt{\omega(x)} }{\lambda_n} u_n(\xi)\sqrt{\omega(\xi)} \]

Now define 
\[A_{ij} = \iint u_i(x)\sqrt{\omega(x)} K(x,\xi) u_j(\xi)\sqrt{\omega(\xi)} dxd\xi\]

It is possible to have $A$ be a diagonal matrix.
% $p(x)\to C'U(x) \sqrt{\omega(x)}

Hmaburger's method gives an orthonormal polynomial basis.

Suppose we have a density function $z(x) = ne^{-\int_a^x f(s) ds}$ and a set of linearly independent functions $\brac{y_k(\cdot)}_{k=1}^n$ with
\[\int y_i(s) y_j(s) z(s) ds = \delta_{ij}\]
What is the spectrum? And whether there exists a kernel equal to
\[G(x,\xi)=\sum_{k=1}^n \frac{y_k(x)y_k(\xi)}{\lambda_k}\]

Consider a degenerate kernel (separable kernel).

We still must make additional assumption regarding the specifics of the integral transition operator: it has to be a specialized Fredholm operator.

% Tikhonov Vasilyeva §9, Kornov §15.3.2, Felfund Fomin §27

\subsubsection{Estimating the spectrum of a degenerate kernel} % (fold)
\label{ssub:estimating_the_spectrum_of_a_degenerate_kernel}

Suppose we have a planar grid where the products $u_n(x) u_n(\xi)$ are evaluated. Uniform grid won't do, since it results in ``wild'' eigenvalues.

For \[K(x,\xi) = \sum_{n=1}^p \frac{u_n(x) u_n(\xi)}{\lambda_n}\]
we must place the grid knots uniformly on the scale of the cumulative weighting function in order to get better approximation.



% subsubsection estimating_the_spectrum_of_a_degenerate_kernel (end)

% subsection integral_operators (end)

% section lectire_5 (end)

\end{document}

