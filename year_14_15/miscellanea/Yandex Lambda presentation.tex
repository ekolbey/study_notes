Устюжании Андрей Евгениевич
% Лямбда -- Лаборатория методов анализа больших данных
LAMBDA -- Laboratory of Methods for Big Data Analysis

What is this laboratory?
Competence in big data analysis
Group in the following fields:
	CERN -- particle physics and big data
	Computational biology

Other fields of study that require big data algorithms, mathematics and approach

Big data analysis in particle physics? No need for in depth physical background to solve the problems.

The basic process is the same:
... -> hypothesis -> experiment -> refined hypothesis -> Experiment -> ...

Analysis of particle collision data
The need for deeper theory spurred the need for finer experiments, which in turn required more precision and yielded more data.

Four particle collision chambers: LHCb CMS ATALS ALICE.

Huge experiments require processing of enormous volumes of data
% Millennium Falcon

Scientists are interested in what happen in the immediate vicinity of a proton-proton particle collision.

It's like learning how the telephone works by studying the shards and their trajectories from a collision with a rock.

Particle jets, particle tracks

\begin{itemize}
	\item The standard model;
	\item the new physics (supper symmetry, strings et c.).
\end{itemize}

Just compare the probability of a particular particle fission under one hypothesis and another.

An event in particle physics is a photograph of a certain particle collision with tracks, jets, hits and vectors and momenta.

LHCb \underset{10^7 \text{per sec.}}{\to} Farm \underset{<10^4 \text{per sec.}}{\to} GRID

Hypothesis testing: a collection of events with a predefined set of features (further selected for a particular experiment).
The is a need for splitting the data into signal and background events.
The goal is to train a model, which could discriminate \emph{the signal from the noise}

The criteria of a decision rule:
\begin{itemize}
	\item Efficiency
	\item Bias
	\item ... (?)
\end{itemize}

$X$ o=is extracted from each event
a clasifier $g(\cdot)$ is trained to discriminate \emph{the signal from the background}

The background is estimated in the selection region
\[G \defm \obj{\induc{x}\, g(x\geq s)}\]

A \emph{discovery} is made when the number of real events $n$ in $G$ is significantly higher than $b$

Particle collision simulator with particle detector simulators (Monte Carlo methods) are used to generate a training sample for the purposes of classifier training.


Simple cuts classifier, later evolves into some more sophisticated discriminative separating hypersurface.

If we measure a new signal sufficiently inconsistent with the background-only hypothesis we can claim a discovery of a new physics phenomenon.

The magnitude is measured in terms of significance.

Analysis Value chain (process)

Data \to Pre-selection \to Pre-processing \to Event selection (cut based / MVA based) \to Counting/fitting \to Systematics estimation \to Significance estimation.

% An outlier can be a signal!!!
Improvements in big data analysis and algorithms:
\begin{itemize}
	\item Newer discrimination (deep learning, binary-decision tree)
	\item Improved feature
	\item sophisticated training scenarios
\end{itemize}
There is a certain inertia (as usual) in adaptation of newer algorithms.

A too flexible decision boundary too closely conforms to the training sample -- overfitting.
The need for independent cross validation.

Likelihood
Misclassification
False positive Rate
Punzi measure

Reproducibility is important everywhere, and most importantly in data analysis.
% iPython <- git <- 
% Pinball-effect by James Birk

LHCb simulation - Data -> classifier

Its behaviour and compare the distribution of the input data and the manner in which the classifier 
% arxiv.org/abs/1410.4140v1

% Homomorphic encryption 
% Privacy encoding


Phase space: SVM even with very intricate kernels still are beaten by boosting and binary decision trees.
Neural Network topology with the trained weights are still uninterpretable black boxes.

Rectangular cuts (rectal cuts)

On interesting goal is how to approximate a classifier by cuts or binary trees?

% Physics and machine learning synergy: Bishop (yes, that one) is a physics to machine learning convert.


anaderi@yandex-team.ru

Мотивационное письмо плюс краткое резюме о прослушанных курсах



