\documentclass[a4paper]{article}
\usepackage[utf8]{inputenc}

\usepackage{graphicx, url}

\usepackage{amsmath, amsfonts, amssymb, amsthm}
\usepackage{xfrac, mathptmx}

\newcommand{\obj}[1]{{\left\{ #1 \right \}}}
\newcommand{\clo}[1]{{\left [ #1 \right ]}}
\newcommand{\clop}[1]{{\left [ #1 \right )}}
\newcommand{\ploc}[1]{{\left ( #1 \right ]}}

\newcommand{\brac}[1]{{\left ( #1 \right )}}
\newcommand{\induc}[1]{{\left . #1 \right \vert}}
\newcommand{\abs}[1]{{\left | #1 \right |}}
\newcommand{\nrm}[1]{{\left\| #1 \right \|}}
\newcommand{\brkt}[1]{{\left\langle #1 \right\rangle}}
\newcommand{\floor}[1]{{\left\lfloor #1 \right\rfloor}}

\newcommand{\Real}{\mathbb{R}}
\newcommand{\Cplx}{\mathbb{C}}
\newcommand{\Pwr}{\mathcal{P}}

\newcommand{\defn}{\mathop{\overset{\Delta}{=}}\nolimits}

\usepackage[english, russian]{babel}
\newcommand{\eng}[1]{\foreignlanguage{english}{#1}}
\newcommand{\rus}[1]{\foreignlanguage{russian}{#1}}

\title{Combinatorial search methods}
\author{Nazarov Ivan, \rus{101мНОД(ИССА)}\\the DataScience Collective}
\begin{document}
\selectlanguage{english}
\maketitle

\selectlanguage{russian}

Кабатянский ГА
Комбинаторная тория поиска

знаем как искать одну фальшивую монетку из 1кк
аналог: угадать заагданное число из миллиона с помощью бинарных вопросов.
-- бинарный поиск; $O(log N)$
-- постоянное общение вопрос-ответ
-- что если задть все вопросы сразу? те же двадцать вопросов

-- задача Улама (\eng{Renyi-Ulham}): что если ответ на вопрос может и не быть правдивым?
	детерминированное кол-во искажений
	задавать одновременно или последовательно не играет роли

дискретные комбинаторные задчи начали возникать в начале \eng{XX} века.

смесь крови от роты солдат -- есть реакция, то есть заражённый -- искать в семси

В $n$ мерном дискретном пр-ве $Q^n$ расставить ладьи так, чтобы покрывалось в $Q^n$.
Ладья бьёт по простым направлениям.
Количество ладей -- \[M \defn \frac{Q^n}{n (Q-1)+1}\]

Полк $(F,\cdot,+)$, $(F,\cdot)$ и $(F,+)$ -- группы 

Если $q = P^m$ то $\exists$ своершенная расстановка ладей $\Leftrigharrow$ $n=\frac{q^r-1}{q-1}$
Если $q\neq P^m$, то расстановки нет

Задача комбинаторного поиска:
найти интересущий нас обёект среди остальных

$M$ -- finte set, $X\subseteq M$ which must be located.
Allowed questions $A\subseteq M$. Responses то $(A,X)$:\begin{itemize}
	\item binary response whether $A\cap X = \emptyset$ or $\neq \emptyset$
	\item the cardinality of $A\cap X$
	\item $\abs{A\cap X}$ if $\abs{A\cap X}\leq T$, $\prep$ otherwise
\end{itemize}

\selectlanguage{english}

Let $\chi_A = \brac{1_A(m)}_{m\in M}$. Then $\abs{A\cap X} = \brkt{\chi_A,\chi_X}$.

A strange model:
Response $\chi_A+\chi_X\mod{2}$ -- the core of coding theory.

Consider question-response $(A,X)\to \abs{A\cap X}$
-- the task of finding counterfieted coins on exact scales.

Given a large $M$ find $X\subseteq M$ by suggexting (asking) $A\subseteq M$.
How many questions are required?
Is there a difference between sequential and simultaneous asking?

(?) Using the information quantity:
infromativeness of every answer is $\frac{\log n}{n+1}$

the complexity of a boolean function
construction a given function using AND, OR, NOT, NAND and XOR gates.
works well on functions of small number of arguments

boolean function of $n$ varaibles requires $K=\sfrac{2^n}{n}$ number of gates.
suppose less, the the number of schemes is less than the number of boolean functions $2^{2^n}$.

Given $(\chi_A,\chi_X)\to \brkt{\chi_A,\chi_X}$.
know the ardinality if $X$: ask $A=M$.

Hamming distance between vectors $u$ and $v$ from some space $\Sigma^M$.
\[d_H\brac{u,v} = \abs{\induc{m\in M}\, u_m \neq v_m} \]

know the cardinality of the simmetric difference $X\Delta A$:
ask for the Hamming distance.
\[d_H(\chi_A,\chi_X) = \abs{\chi_A}+\abs{\chi_X}-2\brkt{\chi_A,\chi_X}\]

$2\frac{n}{\log_2 n}$

Hamming space
$H_q^n = \prod_{k=1}^n H_q$, where $H_q$ is some finite alphabet with a metric given by 
\[S^n(c,t)\defn \obj{\induc{x\in H_q^n}\, d_H(x,c)=t}\]
$\abs{S^n(c,t)} = C^t_n \brac{q-1}^t$

\[B^n(c,t)\defn \obj{\induc{x\in H_q^n}\, d_H(x,c)\leq t}\]
$\abs{B^n(c,t)} = \sum_{k=0}^t C^k_n \brac{q-1}^k$

\[h_2(x) \defn x\log_2{\frac{1}{x}} + (1-x)\log_2{\frac{1}{1-x}}\]
$\abs{B^n(c,t)}\leq 2^{n h\brac{\frac{t}{n}}}$ -- similar for $\frac{t}{n}\leq \frac{1}{2}$

$h_2(x)$ looks like a parabola with the maximum at $\sfrac{1}{2}$, where it is equal to $1$.

A code $C$ is a subset of $H_q^n$ 
\[d(C) \defn \inf\obj{\induc{d(c,d)}\,c,d\in C\, c\neq d}\]
The measure of the adherence within $C$ \rus{как скеливается $C$}.
Code is a set of maximally ``independent'' points. is related to the problem of sphere packing.

For some $t$ the collection $\brac{B^n(c,t)}_{c\in C}$ is pairwise disjoint. Thus $d(C)\leq 2t+1$.

problem of tower placement is equivalent to the coding problem with $t=1$

$\abs{C}\leq \frac{q^n}{\abs{B^n(c,t)}}$ because of the property of code $C$

what is the lower bound on $\abs{C}$? The Hilbert bound (not that Hilbert)
\[\frac{q^n}{\abs{B^n(c,t)}} \leq \abs{C}\]

% Can all of this be formulated in an abstract metric space $(X,d)$?

Any optimal code $C$ induces a $2t$-covering of $H_q^n$.

If $\abs{C} = \frac{q^n}{\abs{B^n(c,t)}}$ and $C$ is an optimal code, then either $q=2$, $t=3$, $n=2,3$, or $q=3$, $t=2$, $n=10,11$

If $t=O(1)$ or $t=O(n)$ then there are two asymptotic processes.
It $t=O(1)$, then $\abs{B^n(c,t)} \approx 1+n (q-1)+\ldots + \frac{n!}{t!}\brac{q-1}^t\approx c n^t$
In $q=2$ and $t=O(n)$, $C$ cannot be chosen better than at random. and the lower (Hilbert) bound is exact.

An old geometric problem of cannon ball packing. in euclidean space the lower bound is known as Minkowski bound (that Minkowski).


\end{document}





