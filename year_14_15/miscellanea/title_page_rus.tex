% This file is written in Windows-1251 (CP-1251) encoding. 
% If the russian letters show as abrcadabra, either tell your editor 
% to show the file in correct encoding, either convert the file into 
% the encoding you are using.
% Do not forget to insert the corresponging \usepackage[..]{inputenc}
% into the main LaTeX file (ex. \usepackage[cp1251]{inputenc})
%
% ШАБЛОН ТИТУЛЬНОГО ЛИСТА ДЛЯ ДИПЛОМНЫХ РАБОТ СТУДЕНТОВ ВШОПФ
% Использование:
% 1. Замените ниже все параметры (название работы и т.п.) на ваши
% 2. Вставьте в основной LaTeX'овский файл команду \input{title.tex}
% 3. Не забудьте в основной LaTeX-файл вставить \usepackage{array} и \usepackage[russian]{babel}
%
\thispagestyle{empty}
\vbox to \textheight{\renewcommand{\baselinestretch}{1}\selectfont%чтобы не было 1.5 интервала в заголовке
\begin{center}
{\sc
ФЕДЕРАЛЬНОЕ АГЕНТСТВО ПО ОБРАЗОВАНИЮ\\[0.5cm]

Государственное образовательное учреждение\\
высшего профессионального образования\\
"<Нижегородский государственный университет им. Н.""И. Лобачевского">\\
(ННГУ)\\[0.5cm]

\large Высшая школа общей и прикладной физики\\[1.5cm]
}%sc

\vfill

{\Large\bf
Электромагнитно-индуцированная прозрачность\\
в сжатом вакууме

}
\end{center}

\vspace{1.5cm}
\vfill

% для бакалаврской работы заменить текст ниже на
%      Дипломная работа\par
%      студента IV курса\par
%      Андреева Ивана Сергеевича\par
%      на соискание степени бакалавра\par

\begin{flushright}
Дипломная работа\par
студента 2 курса магистратуры\par
Андреева Ивана Сергеевича\par
на соискание степени магистра\par
\end{flushright}
\vspace{0.5cm}

% строки вида "&\hfill доктор физико-математических наук\\" можно добавлять и удалять
\noindent\begin{tabular}{@{}p{0.5\textwidth}@{}>{\hfill}p{0.5\textwidth}@{}}
\underline{Научный руководитель}&\hfill ведущий научный сотрудник ИПФ РАН\\
&\hfill кандидат физ-мат. наук\\[0.3cm]
&\underline{\hspace{3cm}} В.""А. Миронов\\[1.2cm]

\underline{Рецензент}&\hfill старший научный сотрудник ИПФ РАН\\
&\hfill кандидат физико-математических наук\\[0.3cm]
&\underline{\hspace{3cm}} Е.Н.Радионычев\\[1.2cm]

\underline{Декан ВШОПФ}&доктор физико-математических наук\\
&профессор\\[0.3cm]
&\underline{\hspace{3cm}} М.""Д. Токман
\end{tabular}

\vfill
\begin{center}
Нижний Новгород\\
2009
\end{center}
}%конец группы для страницы и \vbox
\eject
