\documentclass[a4paper]{article}
\usepackage[utf8]{inputenc}

\usepackage{graphicx, url}

\usepackage{amsmath, amsfonts, amssymb, amsthm}
\usepackage{xfrac, mathptmx}

\newcommand{\obj}[1]{{\left\{ #1 \right \}}}
\newcommand{\clo}[1]{{\left [ #1 \right ]}}
\newcommand{\clop}[1]{{\left [ #1 \right )}}
\newcommand{\ploc}[1]{{\left ( #1 \right ]}}

\newcommand{\brac}[1]{{\left ( #1 \right )}}
\newcommand{\induc}[1]{{\left . #1 \right \vert}}
\newcommand{\abs}[1]{{\left | #1 \right |}}
\newcommand{\nrm}[1]{{\left\| #1 \right \|}}
\newcommand{\brkt}[1]{{\left\langle #1 \right\rangle}}
\newcommand{\floor}[1]{{\left\lfloor #1 \right\rfloor}}

\newcommand{\Real}{\mathbb{R}}
\newcommand{\Cplx}{\mathbb{C}}
\newcommand{\Pwr}{\mathcal{P}}

\newcommand{\defn}{\mathop{\overset{\Delta}{=}}\nolimits}

\usepackage[russian, english]{babel}
\newcommand{\eng}[1]{\foreignlanguage{english}{#1}}
\newcommand{\rus}[1]{\foreignlanguage{russian}{#1}}

\title{the Generalized marriage problem}
\author{Nazarov Ivan, \rus{101мНОД(ИССА)}\\the DataScience Collective}
\begin{document}
\selectlanguage{english}
\maketitle

\section{generalised matching problem} % (fold)
\label{sec:generalised_matching_problem}

the goal of economic mechanism design to devise a system with rules and roles
to facilitate the interaction of economic agents in achieving a preferred common goal.
Usually this is based on preferences of agents.

In a bipartite graph $\big((U, V), E\big)$ the marriage is a set of non-adjacent edges
$H\subseteq E$ such that $(u,v)\in H$ with $u\in U$ and $v\in V$.

In the mechanism design, the central planner might maximize the combined utility of each node (agent).

Generalized matches take into account the matching preferences among $U$ and $V$:
\begin{itemize}
	\item $U, V\subseteq A$ -- finite no-overlapping subsets of agents;
	\item a set $\mu$, a matching map, such that $\mu(u) = v$ if and only if $\mu(v)=u$;
	\item $\mu(u)\in V\cup \{u\}$ and $\mu(v)\in U\cup \{v\}$ -- it is not prohibited to be forever alone.
\end{itemize}

A stable generalised matching is 
\begin{description}
	\item[Engagement] for all $x\in U\cup V = A$ with $\mu(x) \neq x$ is is true that $\mu(x)\preceq_x x$;
	\item[Pairwise stability] there is no $x\in U$ and $y\in V$ such that $x\preceq_y \mu(y)$
	\textbf{and} $y\preceq_x \mu(x)$, meaning that no pair of agents mutually prefer each other
	to the partners assigned by $\mu$.
\end{description}

The matching (\textbf{deferred acceptance}) algorithm $O\big(|U||V|\big)$:
\begin{itemize}
	\item match each $U$ with the most preferred alternative in $V$;
	\item for each $v\in V$ keep the match with the most preferred among those $u\in U$ with $\mu(u)=v$;
	\item for those $u\in U$, whose choices were rejected choose the next best alternative among the remaining $V$;
	\item the iterations continue until every $u\in U$ has a match, or exhausted the list of admissible preferences.
\end{itemize}

the $\preceq_x$ for $x\in A$ are linear orders, but in general it is important to ensure that
the preferences be acyclic.

The algorithm run on $U$ and $V$ provides different results, and in general matching from $U$ 
produces $\mu$ that is sub-optimal with respect to $V$, but is optimal from the point of view of $U$, and vice versa.

Properties of deferred acceptance mechanism
\begin{itemize}
	\item stable matches;
	\item the agents $u\in U$ have incentive not to distort their preferences $\preceq_u$,
	whereas $v\in V$ may manipulate their preferences $\preceq_v$ in order to get a better match.
	The proposing side always has incentives to be honest.
\end{itemize}
Preference manipulation is better done in the complete information setting.
Preference truncation by making some alternatives unacceptable.

% section generalised_matching_problem (end)

\section{One-to-many} % (fold)
\label{sec:one_to_many}

In practice, a more interesting problem is one to many matching
\begin{itemize}
	\item $U, V\subseteq A$ -- finite no-overlapping subsets of agents;
	\item elements $u$ and $v$ form their preferences and each $v\in V$ sets a quota $q_v$ of matches;
	\item a set $\mu$, a matching map, such that $\mu(u) = \{v\}$ if and only if $u\in \mu(v)$
	% \item $\mu(u)\in V\cup \{u\}$ and $\mu(v)\in U\cup \{v\}$ -- it is not prohibited to be forever alone.
	%% missed the last items..
\end{itemize}

Constraints on admissible $\mu$: \begin{itemize}
	\item individual rationality for $U$ and $V$;
	\item absence of blocking pairs in $U\time V$. A pair $a,b$ is blocking if \begin{description}
		\item[either] $b\preceq_a\mu(a)$, $a\prec_b b$ and $\lvert\mu(b)\rvert<q_b$;
		\item[or] $b\preceq_a\mu(a)$ and $\exist a'\in \mu(b)$, such that $a\prec_b a'$.
	\end{description}
\end{itemize}

One-to-many deferred choice constructs
\begin{itemize}
	\item a stable matching, which is the most preferable Pareto efficient for the \textbf{proposing} side;
	\item the accepting side may manipulate the preferences.
\end{itemize}

% section one_to_many (end)

\section{Applications} % (fold)
\label{sec:applications}

Real life applications:
\begin{itemize}
	\item university acceptance;
	\item patent-donor matching;
	\item allocation of medical graduates to internships: UK's National Resident Match Program.
	One of the main problems is that the preferences may be changing in time, and thus this matching
	is dynamically unstable. Also there might be coupled preferences between pairs of future interns
	(family ties).
\end{itemize}

Coupled preferences may cause the loss of stability. Roth and Peranson (?) devised a mechanism,
which minimizes the number of blocking pairs. In practice it turned out, that if there share
of coupled preferences is low, then the probability of unstable matching is low as well.

How to incorporate the interview process in the matching mechanism. Interviews permit preference
discovery. Which is more important: to reveal the preferences, or to get a stable match?

School matching is a different problem, since schools (in general) are not allowed to have preferences.
Also is there an issue of stability? On the one hand it is meaningless, since schools have no preferences.
However, in practice (Boston), from the point of view of fairness, stability is important so that 
individual rights are not violated.



% section applications (end)

\end{document}