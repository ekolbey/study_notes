\documentclass[a4paper]{article}
\usepackage[utf8]{inputenc}

\usepackage{graphicx, url}

\usepackage{amsmath, amsfonts, xfrac}
\usepackage{mathtools}

\newcommand{\obj}[1]{{\left\{ #1 \right \}}}
\newcommand{\clo}[1]{{\left [ #1 \right ]}}
\newcommand{\clop}[1]{{\left [ #1 \right )}}
\newcommand{\ploc}[1]{{\left ( #1 \right ]}}

\newcommand{\brac}[1]{{\left ( #1 \right )}}
\newcommand{\induc}[1]{{\left . #1 \right \vert}}
\newcommand{\abs}[1]{{\left | #1 \right |}}
\newcommand{\nrm}[1]{{\left\| #1 \right \|}}
\newcommand{\brkt}[1]{{\left\langle #1 \right\rangle}}
\newcommand{\floor}[1]{{\left\lfloor #1 \right\rfloor}}

\newcommand{\Real}{\mathbb{R}}
\newcommand{\Cplx}{\mathbb{C}}
\newcommand{\Pwr}{\mathcal{P}}

\newcommand{\Ctx}{\mathbb{K}}
\newcommand{\Pat}{\mathbb{P}}
\newcommand{\supp}{\text{supp}}
\newcommand{\conf}{\text{conf}}


\newcommand{\defn}{\mathop{\overset{\Delta}{=}}\nolimits}

\usepackage[english, russian]{babel}
\newcommand{\eng}[1]{\foreignlanguage{english}{#1}}
\newcommand{\rus}[1]{\foreignlanguage{russian}{#1}}

\begin{filecontents}{references.bib}
@book{diestel2006,
  title={Graph Theory},
  author={Diestel, R.},
  isbn={9783540261834},
  lccn={99057468},
  series={Electronic library of mathematics},
  url={http://books.google.ru/books?id=aR2TMYQr2CMC},
  year={2006},
  publisher={Springer}
}
\end{filecontents}


\title{Summary of Diestel's graph theory}
\author{Nazarov Ivan, \rus{101мНОД(ИССА)}\\the DataScience Collective}

\begin{document}
\maketitle

\begin{abstract}
	A summary of a great book on graph theory.
\end{abstract}

\tableofcontents
\clearpage

Summary of book~\cite{diestel2006}.

\section{Chapter 1} % (fold)
\label{sec:chapter_1}

\subsection{Introduction to graphs} % (fold)
\label{sub:introduction_to_graphs}

A graph is an ordered pair $G=(V,E)$ with $E\subseteq V\times V$.
The number of vertices in $G$, $\abs{V}$, is known as the \textbf{order} of $G$ and is also denoted by $\abs{G}$.
At the same time $\abs{E}$ the number of edges in $G$ is the \textbf{size} of $G$, denoted by $\nrm{G}$.

A vertex $v\in V$ is \textbf{incident} to an edge $e\in E$, written as $v\in e$, if $e=(u,v)$ or $(v,u)$ for some $u\in V$.
Vertices $x,y\in V$ are \textbf{adjacent}, or neighbours in $G$, if $(x,y)$ or $(y,x)\in E$. Otherwise, vertices $x$ and $y$ are \textbf{independent}.
Edges $e,f\in V$, $e\neq f$, are \textbf{adjacent} if there exists $v\in V$ such that $v\in e,f$.
A graph $(V,E)$ is \textbf{complete} if every $e_1, e_2\in E$ are adjacent.

A collection of nodes $S\subseteq V$ or edges $S\subseteq E$ is \textbf{stable}(\rus{внутренне устойчиво}), if every $x,y\in S$ are non-adjacent in $G$.





% subsection introduction_to_graphs (end)


% section chapter_1 (end)

\bibliographystyle{plain}
\bibliography{references}

\end{document}
