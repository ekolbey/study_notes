\documentclass[a4paper]{article}
\usepackage{fullpage}
%% Date of the last edit : 2014-09-22

\usepackage{graphicx, url}

\title{probability.net Theorems}
\author{Nazarov Ivan}

\usepackage{amsmath, amsfonts, xfrac}
\usepackage{mathtools}

\newcommand{\obj}[1]{\left\{ #1 \right \}}
\newcommand{\clo}[1]{\left [ #1 \right ]}
\newcommand{\clop}[1]{\left [ #1 \right )}
\newcommand{\ploc}[1]{\left ( #1 \right ]}

\newcommand{\brac}[1]{\left ( #1 \right )}
\newcommand{\crab}[1]{\left ] #1 \right [}
\newcommand{\induc}[1]{\left . #1 \right \vert}
\newcommand{\abs}[1]{\left | #1 \right |}
\newcommand{\nrm}[1]{\left\| #1 \right \|}
\newcommand{\brkt}[1]{\left\langle #1 \right\rangle}

\newcommand{\floor}[1]{\left\lfloor #1 \right\rfloor}

\newcommand{\Rbar}{{\bar{\mathbb{R}}}}
\newcommand{\Real}{\mathbb{R}}
\newcommand{\Zinf}{\clo{ 0, +\infty }}
\newcommand{\Cplx}{\mathbb{C}}
\newcommand{\Tcal}{\mathcal{T}}
\newcommand{\Dcal}{\mathcal{D}}
\newcommand{\Hcal}{\mathcal{H}}
\newcommand{\Ccal}{\mathcal{C}}
\newcommand{\Scal}{\mathcal{S}}
\newcommand{\Mcal}{\mathcal{M}}
\newcommand{\Ecal}{\mathcal{E}}
\newcommand{\Fcal}{\mathcal{F}}
\newcommand{\borel}[1]{\mathcal{B}\brac{#1}}
\newcommand{\pwr}[1]{\mathcal{P}\brac{#1}}
\newcommand{\Dyns}[1]{\mathfrak{D}\brac{#1}}
\newcommand{\Ring}[1]{\mathcal{R}\brac{#1}}
\newcommand{\Supp}[1]{\operatorname{supp}\nolimits\brac{#1}}

\newcommand{\defn}{\mathop{\overset{\Delta}{=}}\nolimits}
\newcommand{\lpto}{\mathop{\overset{L^p}{\to}}\nolimits}

\newcommand{\re}{\operatorname{Re}\nolimits}
\newcommand{\im}{\operatorname{Im}\nolimits}

\begin{document}
\maketitle
\abstract{This file contains the list of major theorems and particularly interesting results, obtained while studying the wonderful tutorials from \url{http://probability.net}. The results  are presented along with sketches of their proof and brief discussions.}
\tableofcontents
\clearpage

\section{Dynkin Systems} % (fold)
\label{sec:tut_1}
\url{http://probability.net/PRTdynkin.pdf}

\noindent \textbf{Definition} 1.
A Dynkin system (or $\lambda$-system) on $\Omega$ is a collection $\mathfrak{D}$ of subsets of $\Omega$, with \begin{enumerate}
	\item $\Omega\in \mathfrak{D}$.
	\item If $A,B\in \mathfrak{D}$ with $A\subseteq B$, then $B\setminus A\in \mathfrak{D}$.
	\item If $\brac{A_n}_{n\geq 1}\in \mathfrak{D}$ and $A_n\subseteq A_{n+1}$ for all $n\geq 1$ then $\bigcup_{n\geq 1}A_n \in \mathfrak{D}$.
\end{enumerate}

\noindent \textbf{Definition} 2.
A $\sigma$-algebra on $\Omega$ is a collection $\Fcal$ of subsets of $\Omega$, with \begin{enumerate}
	\item $\Omega\in \Fcal$.
	\item If $A\in \Fcal$, then $A^c \defn \Omega\setminus A\in \Fcal$.
	\item If $\brac{A_n}_{n\geq 1}\in \Fcal$, then $\bigcup_{n\geq 1}A_n \in \Fcal$.
\end{enumerate}

\noindent \textbf{Definition} 5.
A $\pi$-system is a collection $\Ccal$ of sets such that $A\cap B\in \Ccal$ for every $A,B\in \Ccal$.

\label{thm:dynkin_pi} \noindent \textbf{Theorem} 1-1.
Let $\Ccal$ be a $\pi$-system on $\Omega$. Then $\Dyns{\Ccal}$ is a $\pi$-system.

For any $A\in \Dyns{\Ccal}$ define the following auxiliary structure \[\Gamma_A = \obj{\induc{ B\in\Dyns{\Ccal} } A\cap B \in \Dyns{\Ccal}}\] First, $A\cap \Omega = A$ implies $\Omega \in \Gamma_A$. Next, if $E,F\in \Gamma_A$ with $E\subseteq F$, then $E, F\in \Dyns{\Ccal}$ implies $E\setminus F\in \Dyns{\Ccal}$, whereas $E\cap A, F\cap A\in \Dyns{\Ccal}$ and $E\cap A\subseteq F\cap A$ imply \[E\setminus F \cap A = \brac{E\cap A}\setminus \brac{F\cap A}\in \Dyns{\Ccal}\] Therefore $E\setminus F\in \Gamma_A$. Finally, if $\brac{E_n}_{n\geq1}\in \Gamma_A$ with $F_n\subseteq F_{n+1}$, then $\bigcup_{n\geq1}F_n\in \Dyns{\Ccal}$. However $\brac{F_n\cap A}_{n\geq1}\in \Dyns{\Ccal}$ with $F_n\cap \subseteq F_{n+1} \cap A$ implies that \[\bigcup_{n\geq1}\brac{F_n\cap A} = \brac{\bigcup_{n\geq1}F_n}\cap A\in \Dyns{\Ccal}\] whence $\bigcup_{n\geq1}F_n \in \Gamma_A$. Therefore $\Gamma_A$ is a Dynkin system on $\Omega$ for any $A\in \Ccal$.

Now, suppose $A\in \Ccal$. If $E\in \Ccal$, then $E\in \Dyns{\Ccal}$ and $E\cap A\in \Ccal\subseteq \Dyns{\Ccal}$, whence $E\in \Gamma_A$. Therefore $\Dyns{\Ccal}\subseteq \Gamma_A$ for any $A\in \Ccal$.

Suppose $B\in \Dyns{\Ccal}$. If $A\in \Ccal$, then the above implies $B\in \Gamma_A$, whence $A\cap B\in \Dyns{\Ccal}$. Thus $A\in \Gamma_B$, since $\Ccal\subseteq\Dyns{\Ccal}$. Therefore $\Dyns{\Ccal}\subseteq \Gamma_B$ for any $B\in \Dyns{\Ccal}$.

Now, if $A,B\in \Dyns{\Ccal}$ then $A\in \Gamma_B$ and $A\cap B\in \Dyns{\Ccal}$. Therefore, the Dynkin system on $\Omega$ generated by a $\pi$-system is itself a $\pi$-system.\\

\label{thm:dynkin_pi_sigma} \noindent \textbf{Theorem} 1-2.
Let $\Fcal$ be a collection of subsets of $\Omega$, which is a $\pi$-system as well as a Dynkin system on $\Omega$. Then $\Fcal$ is a $\sigma$-algebra on $\Omega$.

Indeed, $\Omega\in \Fcal$, since $\Fcal$is a Dynkin system. Next, if $A\in \Fcal$, then $A\subset \Omega$ implies that $A^c\in \Fcal$.

If $A,B\in \Fcal$, then $A^c, B^c\in \Fcal$, whence $A^c\cap B^c\in \Fcal$. Therefore $A\cup B = \brac{A^c\cap B^c}^c\in \Fcal$ and $\Fcal$ is closed under finite unions.

Finally, let $\brac{A_n}_{n\geq1}\in \Fcal$. By the above, for any $n\geq1$ the set $B_n\defn \bigcup_{k=1}^n A_k$ is in $\Fcal$. In addition, $B_n\subseteq B_{n+1}$ for all $n\geq1$, whence $\bigcup_{n\geq1}A_n \in \Fcal$, since $\Fcal$ is a Dynkin system on $\Omega$. Therefore $\Fcal$ is a $\sigma$-algebra on $\Omega$.\\

\label{thm:dynkin_system} \noindent \textbf{Theorem} 1 (Dynkin system).
Let $\Ccal$ be a collection of subsets of $\Omega$, which is closed under pairwise intersection. If $\mathfrak{D}$ is a Dynkin system, containing $\Ccal$, then $\mathfrak{D}$ also contains the $\sigma$-algebra $\sigma\brac{\Ccal}$ on $\Omega$ generated by $\Ccal$.

Indeed, by theorem 1-1 $\Dyns{\Ccal}$ is a $\pi$-system as well as a Dynkin system on $\Omega$. By theorem 1-2 $\Dyns{\Ccal}$ is a $\sigma$-algebra on $\Omega$ with $\Ccal\subseteq\Dyns{\Ccal}$. Therefore $\sigma\brac{\Ccal}\subseteq\Dyns{\Ccal}$.\\

% section tut_1 (end)

\section{Caratheodory's Extension} % (fold)
\label{sec:tut_2}
\url{http://probability.net/PRTcaratheodory.pdf}

\noindent \textbf{Definition} 6.
A semi-ring on $\Omega$ is a collection $\Scal$ of subsets of $\Omega$, with\begin{enumerate}
	\item $\emptyset\in \Scal$.
	\item If $A,B\in \Scal$, then $A\cap B\in \Scal$.
	\item If $A,B\in \Scal$, then there exist $n\geq 0$ and $\brac{A_k}_{k=1}^n\in \Scal$ -- pairwise disjoint, such that $A\setminus B = \bigcup_{k=1}^n A_k$ (when $n=0$ the union is $\emptyset$). 
\end{enumerate}

\noindent \textbf{Definition} 7.
A ring on $\Omega$ is a collection $\mathcal{R}$ of subsets of $\Omega$, with \begin{enumerate}
	\item $\emptyset\in \mathcal{R}$.
	\item If $A,B\in \mathcal{R}$, then $A\cup B\in \mathcal{R}$.
	\item If $A,B\in \mathcal{R}$, then $A\setminus B\in \mathcal{R}$.
\end{enumerate}

If $\Omega\in \mathcal{R}$ then the ring $\mathcal{R}$ is an algebraic ring with addition $A\oplus B \defn A\Delta B \defn A\setminus B \cup B\setminus A$ and multiplication $A\otimes B \defn A\cap B$. In this case the multiplicatively neutral element is $\Omega$ and additively neutral element is $\emptyset$.

\label{thm:ring_semi_ring_pairwise} \noindent \textbf{Theorem} 2-1.
Let $\Scal$ be a semi-ring on $\Omega$ and put \[R\defn \obj{ \induc{\biguplus_{A\in F} A}\, F\subseteq \Scal\,\text{-- finite} }\] Then $\Ring{\Scal} = R$.

Indeed, let $A,B\in R$. Then there exist $m,n\geq 0$ and $\brac{A_i}_{i=1}^n, \brac{B_j}_{j=1}^m\in S$ such that $A\defn \uplus_{i=1}^n A_i$ and $B\defn \uplus_{j=1}^m B_j$. Since $A\cap B = \uplus_{i,j} A_i\cap B_j$, $A\cap B\in R$.

Furthermore, by basic properties of set difference and relations between set operations for $m\geq1$ \[A\setminus B = A\cap B^c = \brac{\biguplus_{i=1}^n A_i} \cap \bigcap_{j=1}^m B_j^c = \bigcap_{j=1}^m \biguplus_{i=1}^n \brac{A_i \cap B_j^c} = \bigcap_{j=1}^m \biguplus_{i=1}^n \brac{A_i \setminus B_j}\]  since $A_i,B_j\in S$ for all $i,j$, by definition of a semi-ring there exists $p_{ij}\geq 0$ and $\brac{E_{ijk}}_{k=1}^{p_{ij}}\in S$ pairwise disjoint such that $A_i\setminus B_j = \biguplus_{k=1}^{p_{ij}} E_{ijk}$. Therefore \[A\setminus B = \bigcap_{j=1}^m \biguplus_{i=1}^n \biguplus_{k=1}^{p_{ij}} E_{ijk}\] Since finite union of nested finite unions is still finite, and $R$ is closed under finite intersections, it has to be true that $A\setminus B\in R$. If $m=0$ then $A\setminus B = A\in R$. Therefore $R$ is closed under set difference.

Now, for any $A,B\in R$ it is true that $A \cup B = \brac{A\setminus B} \uplus B$. Since $A\setminus B\in R$ there exists $p\geq 0$ and pairwise disjoint $\brac{E_k}_{k=1}^p\in S$ such that $A\setminus B = \biguplus_{k=1}^p E_k$. Thus \[A\cup B = \brac{\biguplus_{k=1}^p E_k }\uplus \biguplus_{j=1}^m B_j\] whence $A\cup B\in R$. Therefore $R$ is a ring on $\Omega$.

Obviously, $S\subseteq R$ which implies that $\Ring{\Scal}\subseteq R$. Conversely, if $A\in R$ then there is $n\geq0$ and $\brac{A_i}_{i=1}^n\in S$ such that $A=\biguplus_{i=1}^n A_i$. Since $S\subseteq \Ring{\Scal}$ and any ring is closed under finite unions, $A\in \Ring{\Scal}$. Thus $\Ring{\Scal} = R$ and the ring generated by a semi-ring is actually a collection of all finite unions of pairwise disjoint elements of $S$.\\

\label{thm:ring_semi_ring} \noindent \textbf{Theorem} 2-2.
Let $S$ be a semi-ring on $\Omega$ and put \[R'\defn \obj{ \induc{\bigcup_{A\in F} A}\, F\subseteq S\,\text{-- finite} }\] Then $\Ring{\Scal} = R$.

First, $\Ring{\Scal}\subseteq R'$ because, obviously, $S\subseteq R'$. Second, for any $A\in R'$ there is $n\geq0$ and $\brac{A_i}_{i=1}^n\in S$ such that $A=\bigcup_{i=1}^n A_i$. Since $R$ is closed under finite union and $S\subseteq R$, it is therefore true that $A\in R$, whence $R'\subseteq \Ring{\Scal}$ by theorem 2-1. Therefore the ring generated by a semi-ring is nothing but a collection of all finite unions of elements of $S$.\\

\noindent \textbf{Definition} 9.
Let $\Ccal$ be a collection of subsets of $\Omega$ with $\emptyset \in \Ccal$. A measure on $\Ccal$ is any map $\mu:\Ccal\to\Zinf$, such that\begin{enumerate}
	\item $\mu\brac{\emptyset} = 0$.
	\item For all $\brac{A_n}_{n\geq1}\in \Ccal$ with $\uplus_{n\geq 1} A_n\in \Ccal$, the map is such that $\mu\brac{\uplus_{n\geq 1} A_n} = \sum_{n\geq 1} \mu\brac{A_n}$.
\end{enumerate}

In other words, for any $A\in \Ccal$ and any partition $\brac{A_n}_{n=1}^\infty\in \Ccal$ of $A$, i.e. $\brac{A_n}_{n=1}^\infty$ are pairwise disjoint and $A = \cup_{n=1}^\infty A_n$, the series $\sum_{n=1}^\infty \mu\brac{A_n}$ converges to $\mu\brac{A}$ in $\Zinf$ and it does so regardless of the order of summation, since $\brac{A_{\sigma\brac{n}}}_{n\geq 1}$ is still a partition of $A$ for any $\phi:\mathbb{N}\to\mathbb{N}$. This seemingly trivial note shows a very important link between measures and complex measures. Note that $\uplus_{n\geq1} A_n$ emphasises the fact that the union is in no particular order.

\label{thm:meas_fintely_additive} \noindent \textbf{Theorem} 2-3.
Any measure is finitely additive.

Indeed, for all $n\geq 1$ and any $\brac{A_k}_{k=1}^n\in \mathcal{A}$ with $A = \biguplus_{k=1}^n A_k \in \mathcal{A}$, it is true that $\mu\brac{A} = \sum_{k=1}^n \mu\brac{A_k}$. Indeed, setting $A_k=\emptyset$ for all $k>n$ gives \[\mu\brac{A} = \mu\brac{\biguplus_{k\geq 1} A_k} = \sum_{k\geq1}\mu\brac{A_k} = \sum_{k=1}^n \mu\brac{A_k}\]

\label{thm:meas_set_inclusison} \noindent \textbf{Theorem} 2-4.
Let $\mu$ be a measure on $\mathcal{A}$ and $A,B\in \mathcal{A}$ with $B\setminus A\in \mathcal{A}$. Then $A\subseteq B$ implies that $\mu\brac{A}\leq \mu\brac{B}$.

Indeed, since $B=B\setminus A \uplus A$ finite additivity of $\mu$ (theorem 2-3) implies \[\mu\brac{B} = \mu\brac{B\setminus A}+\mu\brac{A}\geq \mu\brac{A}\] Therefore $\mu\brac{A}\leq\mu\brac{B}$ for any $A, B\in \mathcal{A}$.\\

\label{thm:extension1} \noindent \textbf{Theorem} 2.
Let $\Scal$ be a semi-ring on $\Omega$ and $\mu:\Scal\to\Zinf$ be a measure. Then there exists a unique measure $\bar{\mu}:\Ring{\Scal}\to\Zinf$ with $\induc{\bar{\mu}}_{\Scal}=\mu$.

First, for any $A\in \Ring{\Scal}$, by theorem 2-1 there are $n\geq0$ and $\brac{A_k}\in \Scal$ such that $A=\biguplus_{k=1}^n A_k$. Thus define the $\bar{\mu}\brac{A}$ as \[\bar{\mu}\brac{A}\defn \sum_{k=1}^n \mu\brac{A_k}\]

Let $A\in \Ring{\Scal}$ and suppose there exist $n,m\geq 0$ and $\brac{A_i}_{i=1}^n, \brac{B_j}_{j=1}^m\in \Scal$ such that $A = \biguplus_{i=1}^n A_i = \biguplus_{j=1}^m B_j$. Then $A_i = \biguplus_{j=1}^m A_i\cap B_j$ for each $i=1\ldots n$. Since $\mu$ is a measure on $\Scal$ and any semi-ring is closed under finite set intersection, it is true that $\mu\brac{A_i}=\sum_{j=1}^m \mu\brac{A_i\cap B_j}$ for each $i=1\ldots n$. Similarly, $\mu\brac{B_j}=\sum_{i=1}^n \mu\brac{A_i\cap B_j}$ for each $j=1\ldots m$. Since finite summation is commutative, it therefore must be true that \[\sum_{i=1}^n \mu\brac{A_i} = \sum_{i=1}^n \sum_{j=1}^m \mu\brac{A_i\cap B_j} = \sum_{j=1}^m \sum_{i=1}^n \mu\brac{A_i\cap B_j} = \sum_{j=1}^m\mu\brac{B_j}\] Thus by studying a finer partition of $A$, it has been shown that $\bar{\mu}\brac{A}$ is independent of the particular representation. Thus $\bar{\mu}:\Ring{\Scal}\to\Zinf$ is a well defined map.

Since for any $A\in \Scal$ the set $A$ is a finite representation of itself via elements of $\Scal$, $\bar{\mu}\brac{A} = \mu\brac{A}$, whence $\induc{\bar{\mu}}_{\Scal} = \mu$. Therefore the map $\bar{\mu}$ indeed extends the original measure $\mu$. Since, $\emptyset\in \Scal$, it must be true that $\bar{\mu}\brac{\emptyset} = \mu\brac{\emptyset}=0$.

Let $\brac{A_k}_{k\geq1}\in \Ring{\Scal}$ be a sequence of pairwise disjoint sets such that $A\defn \biguplus_{k\geq1}A_k \in \Ring{\Scal}$. By theorem 2-1 there exist $\brac{p_k}_{k\geq 1}\geq 0$ and $\brac{A_i^k}_{i=1}^{p_k}\in \Scal$ such that $A_k=\biguplus_{i=1}^{p_k} A_i^k$ for all $k\geq1$. At the same time there exists $p\geq0$ and $\brac{B_j}_{j=1}^p\in\Scal$ such that $A=\biguplus_{j=1}^p B_j$.

For any $j=1\ldots p$ it is true that \[B_j = \biguplus_{k\geq1} B_j\cap A_k = \biguplus_{k\geq1} \biguplus_{i=1}^{p_k} B_j\cap A_i^k\] Since a countable union of finite nested unions is still a countable union and $\Scal$ is a $\pi$-system, this last observation means that for each $j=1\ldots p$ there exists a collection $\brac{C_m}_{m\geq1}\in\Scal$ of pairwise disjoint sets such that $B_j = \biguplus_{m\geq1} C_m$. Thus, because $\mu$ is a measure on $\Scal$ and infinite summation of non-negative numbers is associative and commutative by theorem Sup-B-3, it is true that \[\mu\brac{B_j} = \sum_{m\geq1} \mu\brac{C_m} = \sum_{k\geq1} \sum_{i=1}^{p_k} \mu\brac{B_j\cap A_i^k}\]

Now, for any $k\geq1$ and $i=1\ldots p_k$ it is true that $A_i^k = \biguplus_{j=1}^p \brac{B_j\cap A_i^k}$, whence $\mu\brac{A_i^k} = \sum_{j=1}^p \mu\brac{B_j\cap A_i^k}$. Hence by theorem Sup-B-3 and since $\bar{\mu}\brac{A_k} = \sum_{i=1}^{p_k} \mu\brac{A_i^k}$, it has to be true that \begin{align*}\bar{\mu}\brac{A} &= \sum_{j=1}^p \mu\brac{B_j} = \sum_{j=1}^p \sum_{k\geq1} \sum_{i=1}^{p_k} \mu\brac{B_j\cap A_i^k} \\&= \sum_{k\geq1} \sum_{i=1}^{p_k} \sum_{j=1}^p \mu\brac{B_j\cap A_i^k} = \sum_{k\geq1} \sum_{i=1}^{p_k} \mu\brac{A_i^k} = \sum_{k\geq1} \bar{\mu}\brac{A_k}\end{align*} Therefore $\bar{\mu}$ is a measure on $\Ring{\Scal}$.

Let $\mu'$ be any other measure on $\Ring{\Scal}$ with $\induc{\mu'}_{\Scal}=\mu$. If $A\in \Ring{\Scal}$ then by theorem 2-1 there exists $n\geq0$ and $\brac{A_k}_{k=1}^n\in \Scal$ such that $A=\biguplus_{k=1}^n A_k$. Since $\mu'$ is finitely additive by theorem 2-3 and $\brac{A_k}_{k=1}^n\in \Ring{\Scal}$, it is true that \[\mu'\brac{A} = \sum_{k=1}^n \mu'\brac{A_k} = \sum_{k=1}^n \mu\brac{A_k} = \bar{\mu}\brac{A}\] Therefore any extension of $\mu$ from $\Scal$ to $\Ring{\Scal}$ coincides with $\bar{\mu}$.\\

\noindent \textbf{Definition} 10.
An outer measure on $\Omega$ is any map $\mu^\ast:\pwr{\Omega}\to\Zinf$, such that\begin{enumerate}
	\item $\mu^\ast\brac{\emptyset} = 0$.
	\item If $A\subseteq B$, then $\mu^\ast\brac{A}\leq \mu^\ast\brac{B}$.
	\item $\mu^\ast\brac{\bigcup_{n=1}^{+\infty} A_n} \leq \sum_{n=1}^{+\infty} \mu^\ast\brac{A_n}$.
\end{enumerate}

\noindent \textbf{Definition} 11.
The $\sigma$-algebra associated with an outer measure $\mu^\ast$ on $\Omega$ is defined as \[\Sigma\brac{\mu^\ast} \defn \obj{ \induc{ A\subseteq \Omega } \mu^\ast\brac{T} = \mu^\ast\brac{T\cap A} + \mu^\ast\brac{T\cap A^c}, \forall T\subseteq \Omega }\] This is a collection of those subsets of $\Omega$ that slice every other subset in such a way, that the measures of the slices add up to the measure of the whole. Such ``nice'' sets are called measurable with respect to the outer measure $\mu^\ast$.

\label{thm:outermeasure_sigma_algebra} \noindent \textbf{Theorem} 3.
If $\mu^\ast:\pwr{\Omega}\to\Zinf$ be an outer measure on $\Omega$, then $\Fcal \defn \Sigma\brac{\mu^\ast}$ is a $\sigma$-algebra on $\Omega$ and $\induc{\mu^\ast}_\Fcal$ is a measure on $\Fcal$.

Indeed, since $\Omega^c=\emptyset$ and $\mu^\ast\brac{\emptyset}=0$ it is true that $\mu^\ast\brac{T} = \mu^\ast\brac{T\cap \Omega} + \mu^\ast\brac{T\cap \emptyset}$ for any $T\subseteq \Omega$, whence $\Omega\in \Fcal$.

If $A\in \Fcal$, then for any $T\subseteq\Omega$ \[\mu^\ast\brac{T} = \mu^\ast\brac{T\cap A} + \mu^\ast\brac{T\cap A^c}\] which, because $\brac{A^c}^c=A$, by symmetry implies that $A^c\in \Fcal$.

Suppose $A,B\in \Fcal$ and $T\subseteq\Omega$. Then $T\cap A\subseteq \Omega$ implies that \[\mu^\ast\brac{T\cap A } = \mu^\ast\brac{T\cap A \cap B} + \mu^\ast\brac{T\cap A \cap B^c}\] Now, since $B^c, A^c\subseteq \brac{A\cap B}^c$, $T\cap A^c = T \cap \brac{A\cap B}^c\cap A^c$ and $T\cap A\cap B^c = T \cap \brac{A\cap B}^c\cap A$. Thus\begin{align*}\mu^\ast\brac{T\cap \brac{A\cap B}^c}&=\mu^\ast\brac{T\cap \brac{A\cap B}^c \cap A^c} + \mu^\ast\brac{T\cap \brac{A\cap B}^c\cap A}\\&=\mu^\ast\brac{T\cap A^c} + \mu^\ast\brac{T\cap A\cap B^c}\end{align*} Adding $\mu^\ast\brac{T\cap A\cap B}$ to both sides gives\[\mu^\ast\brac{T\cap A^c} + \mu^\ast\brac{T\cap A} = \mu^\ast\brac{T\cap\brac{A\cap B}^c}+\mu^\ast\brac{T\cap\brac{A\cap B}}\] which, since $A\in \Fcal$ implies that $A\cap B\in \Fcal$.

Basic relations among set operations further imply that $\Fcal$ is closed under finite set union and intersection, as well as under set difference.

If $A,B\in \Fcal$ with $A,\cap B=\emptyset$, then \begin{align*}\mu^\ast\brac{T\cap \brac{A\uplus B}} &= \mu^\ast\brac{T\cap \brac{A\uplus B}\cap A} + \mu^\ast\brac{T\cap \brac{A\uplus B}\cap A^c}\\ &= \mu^\ast\brac{T\cap A} + \mu^\ast\brac{T\cap B^c}\end{align*} since $\brac{A\uplus B}\cap A = A$ and $\brac{A\uplus B}\cap A^c = B$.

Let $\brac{A_n}_{n\geq1}\in \Fcal$ and put $B_{n+1}\defn A_{n+1}\setminus\brac{ \bigcup_{k=1}^n A_k}$ with $B_1\defn A_1$. Then $B\defn \brac{B_n}_{n\geq1}\in \Fcal$ are pairwise disjoint and $\biguplus_{n\geq1} B_n = \bigcup_{n\geq1} A_n$. As $\Fcal$ is closed under finite unions $\uplus_{k=1}^N B_k\in \Fcal$ for any $N\geq0$, whence by induction on $N\geq1$ \begin{align*}\mu^\ast\brac{T\cap \uplus_{k=1}^N B_k} &= \mu^\ast\brac{T\cap \uplus_{k=1}^N B_k\cap B_N} + \mu^\ast\brac{T\cap \uplus_{k=1}^N B_k\cap B_N^c}\\&=\mu^\ast\brac{T\cap B_N} + \mu^\ast\brac{T\cap \uplus_{k=1}^{N-1} B_k}\\&=\sum_{k=1}^N \mu^\ast\brac{T\cap B_k}\end{align*}

Since $\uplus_{k=1}^N B_k\subseteq B$ for all $N\geq1$, the basic property of an outer measure on $\Omega$ implies that $\mu^\ast\brac{T\cap B^c}\leq \mu^\ast\brac{T\cap \brac{\uplus_{k=1}^N B_k}^c}$ for any $N\geq1$. Thus, for all $N\geq1$\[\mu^\ast\brac{T\cap B^c} + \sum_{k=1}^N \mu^\ast\brac{T\cap B_k}\leq \mu^\ast\brac{T\cap \brac{\uplus_{k=1}^N B_k}^c} + \mu^\ast\brac{T\cap \uplus_{k=1}^N B_k} = \mu^\ast\brac{T}\] Thus $\mu^\ast\brac{T\cap B^c} + \sum_{n\geq1} \mu^\ast\brac{T\cap B_n}\leq \mu^\ast\brac{T}$.

Now, as $T=T\cap B\uplus T\cap B^c$ by sub-additivity of $\mu^\ast$ it is true that \[\mu^\ast\brac{T}\leq \mu^\ast\brac{T\cap B^c}+\mu^\ast\brac{T\cap B}\leq\mu^\ast\brac{T\cap B^c}+\sum_{n\geq1}\mu^\ast\brac{T\cap B_n}\] Firstly, this implies that $\mu^\ast{T}=\mu^\ast\brac{T\cap B^c}+\mu^\ast\brac{T\cap B}$ for any $T\subseteq\Omega$, whence $B\in \Fcal$. Secondly, this implies that for any $T\subseteq \Omega$ \[\mu^\ast\brac{T} = \mu^\ast\brac{T\cap B^c} + \sum_{n\geq1} \mu^\ast\brac{T\cap B_n}\] which means that in particular for $T\defn B$ it is true that \[\mu^\ast\brac{B}=0+\sum_{n\geq1} \mu^\ast\brac{B_n}\] since $B_n\cap B=B_n$ for any $n\geq1$.

Therefore $\Fcal$ is a $\sigma$-algebra on $\Omega$ and $\induc{\mu^\ast}_{\Fcal}$ is a measure on $\Fcal$.\\

\label{thm:outermeasure_ring} \noindent \textbf{Theorem} 2-5.
Let $\mathcal{R}$ be a ring on $\Omega$ and $\mu:\mathcal{R}\to \Zinf$ be a measure on $\mathcal{R}$. Then the map defined for all $T\subseteq \Omega$ as \[\mu^\ast\brac{T}\defn \inf\obj{ \induc{ \sum_{n\geq1} \mu\brac{A_n} }\, \brac{A_n}_{n\geq1}\in \mathcal{R},\, T\subseteq\bigcup_{n\geq1} A_n}\] is an outer measure on $\Omega$ with $\induc{\mu^\ast}_\mathcal{R} = \mu$. Furthermore, $\mathcal{R}\subseteq\Sigma\brac{\mu^\ast}$.

First, since $A_n\defn \emptyset$ is an $\mathcal{R}$-cover of $\emptyset$, it is true that $\mu^\ast\brac{\emptyset}\leq \sum_{n\geq1}\mu\brac{A_n} = 0$, whence $\mu^\ast\brac{\emptyset}=0$.

Second, if $A\subseteq B$ then any $\mathcal{R}$-cover of $B$ is also an $\mathcal{R}$-cover of $A$, whence $\mu^\ast\brac{A}\leq \mu^\ast\brac{B}$, because of the properties of the greater lower bound.

Suppose $\brac{A_n}_{n\geq1}$ are such that $\mu\brac{A_n}<+\infty$ for all $n\geq 1$ and $\epsilon>0$. Then for each $n\geq1$ there exists an $\mathcal{R}$-cover $\brac{A_k^n}_{k\geq1}$ of $A_n$ such that \[\sum_{k\geq1}\mu\brac{A_k^n} <\mu^\ast\brac{A_n}+\frac{\epsilon}{2^n}\] Since a countable union of nested countable unions is itself countable, $\brac{A_k^n}_{n,k\geq1}$ is $\mathcal{R}$-cover of $\bigcup_{n\geq1} A_n$. By theorem Sup-B-3 and the definition of $\mu^\ast$ then \[\mu^\ast\brac{\bigcup_{n\geq1} A_n}\leq \sum_{n\geq1} \sum_{k\geq1} \mu\brac{A_k^n} \leq \sum_{n\geq1}\mu^\ast\brac{A_n}+\sum_{n\geq1}\frac{\epsilon}{2^n} = \sum_{n\geq1}\mu^\ast\brac{A_n}+\epsilon\] Hence, $\mu^\ast\brac{\bigcup_{n\geq1} A_n}\leq \sum_{n\geq1}\mu^\ast\brac{A_n}$ in this case.

If there is $n\geq1$ with $\mu^\ast\brac{A_n}=+\infty$ then it is trivially true that $\mu^\ast\brac{\bigcup_{n\geq1} A_n}\leq +\infty = \sum_{n\geq1}\mu^\ast\brac{A_n}$. In conclusion, thus defined map $\mu^\ast$ is an outer measure on $\Omega$.

Let $A\in \mathcal{R}$ and $\brac{A_n}_{n\geq1}\in \mathcal{R}$ be such that $\brac{A_n}_{n\geq1}$ is a cover of $A$. For $B_{n+1}\defn \brac{A_{n+1}\cap A}\setminus \bigcup_{k=1}^n \brac{A_k\cap A}$ with $B_1\defn A_1\cap A$, $\brac{B_n}_{n\geq1}\in \mathcal{R}$ since any ring is closed under set difference and finite intersection. Furthermore, the fact that $\mu$ is a measure on $\mathcal{R}$ and that $A=\biguplus_{n\geq1} B_n$ implies that \[\mu\brac{A}=\mu\brac{\biguplus_{n\geq1} B_n} = \sum_{n\geq1}\mu\brac{B_n}\leq \sum_{n\geq1}\mu\brac{A_n}\] whence $\mu\brac{A}\leq \mu^\ast\brac{A}$.

Since for any $A\in \mathcal{R}$ the set $A$ is its own $\mathcal{R}$-cover of $A$, it must be true that $\mu^\ast\brac{A}\leq \mu\brac{A}$, whence $\induc{\mu^\ast}_{\mathcal{R}} = \mu$.

If $A\in \mathcal{R}$ then for any $T\subseteq\Omega$ sub-additivity of $\mu^\ast$ implies $\mu^\ast\brac{T}\leq \mu^\ast\brac{T\cap A}+\mu^\ast\brac{T\cap A^c}$. If $\brac{T_n}_{n\geq1}\in \mathcal{R}$ is a cover of $T$, then $\brac{T_n\cap A}_{n\geq1}$ and $\brac{T_n\cap A^c}_{n\geq1}$ are $\mathcal{R}$-covers of $T\cap A$ and $T\cap A^c$ respectively, whence \begin{align*}\mu^\ast\brac{T\cap A}+\mu^\ast\brac{T\cap A^c}&\leq \sum_{n\geq1}\mu\brac{T_n\cap A}+\sum_{n\geq1}\mu\brac{T_n\cap A^c}\\&= \sum_{n\geq1}\mu\brac{T_n\cap A}+\mu\brac{T_n\cap A^c}\end{align*} Since the combination of these covers is a countable $\mathcal{R}$-cover of $T$, by definition of $\mu^\ast$ it must be true that \[\mu^\ast\brac{T\cap A}+\mu^\ast\brac{T\cap A^c}\leq \mu^\ast\brac{T}\] Therefore $A\in \Sigma\brac{\mu^\ast}$ because $\mu^\ast\brac{T}=\mu^\ast\brac{T\cap A}+\mu^\ast\brac{T\cap A^c}$, whence $\mathcal{R}\subseteq\Sigma\brac{\mu^\ast}$.\\

\label{thm:caratheodory1} \noindent \textbf{Theorem} 4 (Caratheodory's extension).
Let $\mathcal{R}$ be a ring on $\Omega$ and $\mu:\mathcal{R}\to \Zinf$ be a measure. Then there exists a measure $\mu':\sigma\brac{\mathcal{R}}\to \Zinf$, the restriction of which to $\mathcal{R}$ coincides with $\mu$.\\

Indeed, by theorem 2-5 there is an outer measure $\mu^\ast$ on $\Omega$ with $\induc{\mu^\ast}_\mathcal{R}=\mu$ and $\mathcal{R}\subseteq\Sigma\brac{\mu^\ast}$. Since $\Sigma\brac{\mu^\ast}$ is a $\sigma$-algebra on $\Omega$ by theorem 3, $\sigma\brac{\mathcal{R}}\subseteq \Sigma\brac{\mu^\ast}$. Furthermore by the same theorem $\induc{\mu^\ast}_{\Sigma\brac{\mu^\ast}}$ is a measure on $\Sigma\brac{\mu^\ast}$. Therefore, if $\mu'\defn \induc{\mu^\ast}_{\sigma\brac{\mathcal{R}}}$, then $\mu'$ is a measure on $\sigma\brac{\mathcal{R}}$ with $\induc{\mu'}_\mathcal{R} = \mu$.\\

\label{thm:ring_semi_ring_sigma_algebra} \noindent \textbf{Theorem} 2-6.
If $\Scal$ is a semi-ring on $\Omega$, then $\sigma\brac{\Ring{\Scal}}=\sigma\brac{\Scal}$.

Indeed, since by theorem 2-2 $\Ring{\Scal}$ is a collection of finite unions of sets from $\Scal$, it must be a subset of $\sigma{\Scal}$, whence $\sigma\brac{\Ring{\Scal}}\subseteq\sigma\brac{\Scal}$. The converse is trivial, since $\Scal\subseteq\Ring{\Scal}\subseteq\sigma\brac{\Ring{\Scal}}$.\\

\label{thm:caratheodory2} \noindent \textbf{Theorem} 5.
Let $\Scal$ be a semi-ring on $\Omega$ and $\mu:\Scal\to \Zinf$ be a measure. Then there exists a measure $\mu':\sigma\brac{\Scal}\to \Zinf$, the restriction of which to $\Scal$ coincides with $\mu$.

If $\mu$ is a measure on $\Scal$ then by theorem 2 there exists a unique extension $\bar{\mu}$ of $\mu$ from $\Scal$ to $\Ring{\Scal}$. By theorem 5 (Caratheodory's Extension) for $\bar{\mu}:\Ring{\Scal}\to \Zinf$ there exists a measure $\mu':\sigma\brac{\Ring{\Scal}}\to\Zinf$ with $\induc{\mu'}_{\Ring{\Scal}}=\bar{\mu}$. However, by theorem 2-6 $\sigma\brac{\Ring{\Scal}}=\sigma\brac{\Scal}$, which implies that $\mu'$ is actually a measure on $\sigma\brac{\Scal}$. Furthermore $\induc{\mu'}_\Scal = \induc{\bar{\mu}}_\Scal = \mu$, since $\Scal\subseteq\Ring{\Scal}$.\\

% section tut_2 (end)

\section{Stieltjes-Lebesgue Measure} % (fold)
\label{sec:tut_3}
\url{http://probability.net/PRTstieltjesmeas.pdf}

\noindent \textbf{Definition} 12.
Let $\Ccal\subseteq \pwr{\Omega}$ and $\mu:\Ccal\to \Zinf$ be a map. Then $\mu$ is finitely additive when for any $n\geq1$ and $\brac{A_k}_{k=1}^n\in \Ccal$ with $\uplus_{k=1}^n A_k\in \Ccal$ it is true that $\mu\brac{\uplus_{k=1}^n A_k}=\sum_{k=1}^n \mu\brac{A_k}$. The map $\mu$ is finitely sub-additive if for any $n\geq1$ and $A,\brac{A_k}_{k=1}^n\in \Ccal$ with $A\subseteq \cup_{k=1}^n A_k\in \Ccal$ it is true that $\mu\brac{A}\leq \sum_{k=1}^n \mu\brac{A_k}$.

\label{thm:half_open_inetrvals} \noindent \textbf{Theorem} 3-1.
The collection $\Scal$ defined as $\Scal \defn \obj{\induc{\ploc{a,b}}\,a,b\in \Real}$ is a semi-ring on $\Real$.

Indeed, for any $a,b,c,d\in \Real$ it is true that $\ploc{a,b}\cap\ploc{c,d} = \ploc{a\vee c, b\wedge d}$. Furthermore $c\leq d$ implies that $b\wedge c\leq a\vee d$ and \[\ploc{a,b}\setminus\ploc{c,d} = \ploc{a,b\wedge c}\cup \ploc{a\vee d,b}\] Therefore $\Scal$ is a semi-ring on $\Real$.\\

\label{thm:fin_add_is_fin_sub_add} \noindent \textbf{Theorem} 3-2.
Let $\Scal$ be a semi-ring on $\Omega$ and $\mu:\Scal\to\Zinf$ be a finitely additive map. Then $\mu$ is finitely sub-additive.

First, for any $A\in \Ring{\Scal}$, by theorem 2-1 there are $n\geq0$ and $\brac{A_k}\in \Scal$ such that $A=\biguplus_{k=1}^n A_k$. Thus let $\bar{\mu}\brac{A}\defn \sum_{k=1}^n \mu\brac{A_k}$ for any $A\in \Ring{\Scal}$.

Let $A\in \Ring{\Scal}$ and suppose there exist $n,m\geq 0$ and $\brac{A_i}_{i=1}^n, \brac{B_j}_{j=1}^m\in \Scal$ such that $A = \biguplus_{i=1}^n A_i = \biguplus_{j=1}^m B_j$. Then $A_i = \biguplus_{j=1}^m A_i\cap B_j$ for each $i=1\ldots n$. Since $\mu$ is finitely additive on $\Scal$ and any semi-ring is closed under finite set intersection, $\mu\brac{A_i}=\sum_{j=1}^m \mu\brac{A_i\cap B_j}$ for each $i=1\ldots n$. Similarly, $\mu\brac{B_j}=\sum_{i=1}^n \mu\brac{A_i\cap B_j}$ for each $j=1\ldots m$. Since finite summation is commutative, it therefore must be true that \[\sum_{i=1}^n \mu\brac{A_i} = \sum_{i=1}^n \sum_{j=1}^m \mu\brac{A_i\cap B_j} = \sum_{j=1}^m \sum_{i=1}^n \mu\brac{A_i\cap B_j} = \sum_{j=1}^m\mu\brac{B_j}\] Thus by studying a finer partition of $A$, it has been shown that $\bar{\mu}\brac{A}$ is independent of the particular representation. Thus $\bar{\mu}:\Ring{\Scal}\to\Zinf$ is a well defined map.

For any $A\in \Scal$ the set $A$ is a finite representation of itself via elements of $\Scal$. Thus $\bar{\mu}\brac{A} = \mu\brac{A}$, whence $\induc{\bar{\mu}}_{\Scal} = \mu$.

Now, let $n\geq1$ and $A,\brac{A_k}_{k=1}^n\in \Ring{\Scal}$ be such that $A=\biguplus_{k=1}^n A_k$. Then by theorem 2-1 there exist $\brac{p_k}_{k=1}^n \geq 0$ and $\brac{A_k^i}_{i=1}^{p_k}\in \Scal$ such that $A_k = \uplus_{i=1}^{p_k} A_k^i$. Since $A=\uplus_{k=1}^n \uplus_{i=1}^{p_k} A_k^i$, the definition of $\bar{\mu}$ implies that \[\bar{\mu}\brac{A} = \sum_{k=1}^n \sum_{i=1}^{p_k} \mu\brac{A_k^i} = \sum_{k=1}^n \sum_{i=1}^{p_k} \mu\brac{A_k^i} = \sum_{k=1}^n \bar{\mu}\brac{A_k}\] Therefore $\bar{\mu}$ is a finitely additive map on $\Ring{\Scal}$.

Let $A,B\in \Ring{\Scal}$ be such that $A\subseteq B$. Since $B=B\setminus A \uplus A$ and $B\setminus A\in \Ring{\Scal}$, finite additivity and non-negativity of $\bar{\mu}$ imply that \[\bar{\mu}\brac{A}\leq \bar{\mu}\brac{A}+\bar{\mu}\brac{B\setminus A} = \bar{\mu}\brac{B}\] Thus $\bar{\mu}\brac{A}\leq\bar{\mu}\brac{B}$.

Now let $n\geq1$ and $A, \brac{A_k}_{k=1}^n\in \Ring{\Scal}$ be such that $A\subseteq \cup_{k=1}^n A_k$. Then for $B_1\defn A_1\cap A$ and \[B_{k+1}\defn \brac{A_{n+1}\cap A}\setminus \bigcup_{i=1}^k \brac{A_i\cap A}\] it is true that $\brac{B_k}_{k=1}^n\in \Ring{\Scal}$ are pairwise disjoint and are such that $A=\biguplus_{k=1}^n B_k$. Now $\bar{\mu}\brac{B_k}\leq\bar{\mu}\brac{A_k}$ for all $k=1\ldots n$ because $B_k\subseteq A_k$. Hence \[\bar{\mu}\brac{A} = \sum_{k=1}^n \bar{\mu}\brac{B_k}\leq \sum_{k=1}^n \bar{\mu}\brac{A_k}\] Therefore the map $\bar{\mu}$ is finitely sub-additive.

Now, let $n\geq1$ and $A, \brac{A_k}_{k=1}^n\in \Scal$ be such that $A\subseteq \bigcup_{k=1}^n A_k$. Since $\induc{\bar{\mu}}_{\Scal} = \mu$, \[\mu\brac{A}=\bar{\mu}\brac{A}\leq \sum_{k=1}\bar{\mu}\brac{A_k} = \sum_{k=1}\mu\brac{A_k}\] Therefore the original map $\mu$ is finitely sub-additive.\\

\label{thm:stieltjes_meas_const1} \noindent \textbf{Theorem} 3-3.
Let $F:\Real\to\Real$ be a right-continuous non-decreasing map and $\Scal$ be the semi-ring of right closed left open intervals on $\Real$. Then the map defined by $\mu\brac{\emptyset}=0$ and for $a\leq b\in \Real$ \[\mu\brac{\ploc{a,b}}\defn F\brac{b}-F\brac{a}\] is finitely additive and sub-additive on $\Scal$.

Let $a<b$ and $a_k<b_k$ for $k=1\ldots n$ and $n\geq1$ be such that $\ploc{a,b}=\biguplus_{k=1}^n \ploc{a_k,b_k}$. Therefore $a\leq a_k<b_k\leq b$ for all $k=1\ldots n$.

First there exists $k_1\in \obj{1\ldots n}$ such that $a_{k_1}\leq a_k$ for all $k=1\ldots n$. If $a<a_{k_1}$ then for any $a<x<a_{k_1}$ it is true that $x\in \ploc{a,b}$ yet $x\notin \ploc{a_k,b_k}$ for each $k=1\ldots n$, whence $x\notin \ploc{a,b}$. Hence, since the intervals are non-overlapping, it is true that $a=a_{k_1}<b_{k_1}\leq a_k<b_k\leq b$ for all $k\neq k_1$. Therefore \[\ploc{b_{k_1},b}=\biguplus_{k\neq k_1} \ploc{a_k,b_k}\]

Thus a repeated application of this argument yields a permutation $\brac{k_i}_{i=1}^n$ of $\obj{1\ldots n}$ such that $b_{k_i}=a_{k_{i+1}}$ for all $i=1\ldots {n-1}$ with $a=a_{k_1}$ and $b=b_{k_n}$.

Therefore, \[\mu\brac{\ploc{a,b}} = F\brac{b}-F\brac{a} = \sum_{i=1}^n F\brac{b_{k_i}} - F\brac{a_{k_i}} = \sum_{k=1}^n F\brac{b_k}-F\brac{a_k}\] which implies that $\mu$ well-defined and finitely additive on $\Scal$. Furthermore by theorem 3-2 $\mu$ is finitely sub-additive on $\Scal$.\\

\label{thm:stieltjes_meas_const2} \noindent \textbf{Theorem} 3-4.
Let $F:\Real\to\Real$ be a right-continuous non-decreasing map and $\Scal$ be the semi-ring of right closed left open intervals on $\Real$. Then the map defined by $\mu\brac{\emptyset}=0$ and for $a\leq b\in \Real$ \[\mu\brac{\ploc{a,b}}\defn F\brac{b}-F\brac{a}\] is a measure on $\Scal$.

Suppose $a<b$ and $\brac{a_k}_{k\geq1}, \brac{b_k}_{k\geq1}\in \Real$ are such that $a_k<b_k$ for $k\geq1$ and $\ploc{a,b}=\biguplus_{k\geq1} \ploc{a_k,b_k}$.

For any $N\geq1$ by a similar procedure there is a permutation $\brac{i_k}_{k=1}^N$ of $\obj{1\ldots N}$ such that $a\leq a_{i_k}$, $b_{i_k}\leq b$ and $b_{i_k}\leq a_{i_{k+1}}$ for all $k=1\ldots{N-1}$. Since $F$ is a non-decreasing function\[\sum_{k=1}^N F\brac{b_{i_k}} - F\brac{a_{i_k}} = F\brac{b_{i_N}} - F\brac{a_{i_1}} - \sum_{k=1}^{N-1} F\brac{a_{i_{k+1}}}-F\brac{b_{i_k}} \leq F\brac{b} - F\brac{a}\] Thus for every $N\geq1$ it is true that $\sum_{k=1}^N F\brac{b_k} - F\brac{a_k}\leq F\brac{b} - F\brac{a}$, whence by definition of infinite sums of non-negative numbers \[\sum_{k\geq1} \mu\brac{\ploc{a_k,b_k}}\leq \mu\brac{\ploc{a,b}}\]

By the right-continuity and non-decreasing nature of $F$ for any $\epsilon>0$ there is $\eta>0$ with $\eta<b-a$ such that $0\leq F\brac{a+\eta}-F\brac{a}<\epsilon$. Furthermore by right-continuity at each $b_k$ there exists $\brac{\eta_k}_{k\geq1}>0$ with $0\leq F\brac{b_k+\eta_k}-F\brac{b_k}<\frac{\epsilon}{2^k}$ for all $k\geq 1$.

Now if $x\in \clo{a+\eta,b}$, then $x\in\ploc{a,b}$, whence there is $n\geq1$ such that $x\in \ploc{a_k,b_k}\subseteq \brac{a_k,b_k+\eta_k}$. Therefore $\brac{\crab{a_n, b_n+\eta_n}}_{n\geq 1}$ is an open cover of $\clo{a+\eta, b}$.

Since $\clo{a+\eta,b}$ is a compact subset of $\Real$, by Heine-Borel theorem (theorem 34) this open cover has a finite sub-cover. Thus there is $p\geq 1$ and $\brac{n_k}_{k=1}^p$ with $\clo{a+\eta,b}\subseteq \bigcup_{k=1}^p \brac{a_{n_k}, b_{n_k}+\eta_{n_k}}$, whence \[\ploc{a+\eta,b} \subseteq \bigcup_{k=1}^p \ploc{a_{n_k},b_{n_k}+\eta_{n_k}}\] Thus, since $\mu$ is sub-additive on $\Scal$ by theorem 3-3, it must be true that \[F\brac{b}-F\brac{a+\eta} = \mu\brac{\ploc{a,b+\eta}}\leq \sum_{k=1}^p \mu\brac{\ploc{a_{n_k},b_{n_k}+\eta_{n_k}}} = \sum_{k=1}^p F\brac{b_{n_k}+\eta_{n_k}}-F\brac{a_{n_k}}\] Hence for all $\epsilon>0$ by definition of infinite sum of non-negative numbers \begin{align*}F\brac{b}-F\brac{a} &\leq \sum_{k=1}^p F\brac{b_{n_k}}-F\brac{a_{n_k}} + \sum_{k=1}^p F\brac{b_{n_k}+\eta_{n_k}}-F\brac{b_{n_k}} + F\brac{a+\eta}-F\brac{a} \\&< \sum_{k\geq1} F\brac{b_k}-F\brac{a_k} + 2\epsilon\end{align*} implying that $\mu\brac{\ploc{a,b}} = \sum_{k\geq1} \mu\brac{\ploc{a_k,b_k}}$.

Finally, $\ploc{a,b}=\emptyset$ if and only if $a\geq b$. Thus $\mu\brac{\emptyset}=\mu\brac{\ploc{a,a}}=F\brac{a}-F\brac{a}=0$. Therefore the map $\mu\brac{a,b}\defn F\brac{b}-F\brac{a}$ is a measure on $\Scal$.\\

\noindent \textbf{Definition} 13.
A topological space is an ordered pair $\brac{\Omega, \Tcal}$, where $\Omega$ is a set, and $\Tcal$ is a collection of subsets, called ``open'' in $\Omega$, such that \begin{enumerate}
	\item $\Omega, \emptyset \in \Tcal$.
	\item If $A,B\in \Tcal$, then $A\cap B\in \Tcal$.
	\item If $\brac{A_i}_{i\in I}\in \Tcal$, then $\bigcup_{i\in I}A_i \in \Tcal$.
\end{enumerate}

\noindent \textbf{Definition} 14.
The usual topology on $\Real$, denoted by $\Tcal_\Real$ is \[\Tcal_\Real = \obj{\induc{ U\subseteq \Real } \forall x\in U \exists \epsilon > 0,\,\text{s.t.}\,\brac{x-\epsilon,x+\epsilon}\subseteq U}\] As one can see it is nothing but a topology generated by a topological basis of open intervals in $\Real$. Whenever a subset $U$ of $\Real$ is said to be open in $\Real$, it is meant that it is open with respect to this usual topology on $\Real$, i.e. $U\in \Tcal_\Real$.

\label{thm:borel1} \noindent \textbf{Theorem} 6.
Let $\Scal = \obj{\induc{\ploc{a,b}}a,b\in\Real}$. Then $\Scal$ is a semi-ring on $\Real$ and the Borel $\sigma$-algebra $\borel{\Real}\defn\sigma\brac{\Tcal_\Real}$ is actually generated by $\Scal$.

Since $\ploc{a,b}=\bigcap_{k\geq1}\brac{a,b+\frac{1}{k}}$ for any $a<b$, $\Scal\subseteq \borel{\Real}$, whence $\sigma\brac{\Scal}\subseteq \borel{\Real}$.

Now, for any $U\in \Tcal_\Real$ and any $x\in U$ there is $\eta>0$ with $\brac{x-\eta,x+\eta}\subseteq U$. Furthermore, since $\mathbb{Q}$ is dense in $\Real$ there are $p,q\in\mathbb{Q}$ with $x-\eta<p<x$ and $x<q<x+\eta$. Therefore for every $x\in U$ there are $p_x, q_x\in \mathbb{Q}$ such that $x\in\ploc{p_x,q_x}\subseteq U$.

For $\Gamma_U\defn\obj{ \induc{\ploc{a,b}}\,a,b\in\mathbb{Q}\,\ploc{a,b}\subseteq U}$ it is true that $U=\bigcup_{V\in \Gamma_U} V$. Furthermore $\abs{\Gamma_U}\leq \abs{\mathbb{Q}\times \mathbb{Q}}$ meaning that $\Gamma_U$ is at most countable. Therefore, since $\Scal\subseteq \sigma\brac{\Scal}$ it must be true that $U\in \sigma\brac{\Scal}$, whence $\Tcal_\Real\subseteq \sigma\brac{\Scal}$. Thus $\borel{\Real}=\sigma\brac{\Tcal_\Real}\subseteq \sigma\brac{\Scal}$.

In fact since $\Real$ is a metric space with an everywhere dense countable subset $\mathbb{Q}$, $\brac{\Real, \Tcal_\Real}$ must have a countable base. Since $\Tcal_\Real$ is an order topology and $\Real$ is and ordered continuum, within each subset of the topological base it is possible to select a half open interval. These interval altogether form a structure analogous to a countable base, but more suited for the current goal.\\

\label{thm:meas_cont_up} \noindent \textbf{Theorem} 7.
Let $\brac{\Omega, \Fcal, \mu}$ be a measure space. If $\brac{A_n}_{n\geq1}\in \Fcal$ with $A_n\uparrow A$, then $\mu\brac{A_n}\uparrow \mu\brac{A}$.

Indeed, $A_n\uparrow A$ means that $A_n\subseteq A_{n+1}$ and $A=\bigcup_{n\geq1} A_n$. So define a decomposition of a sequence of expanding nested sets as follows: let $B_1\defn A_1$ and $B_{n+1}\defn A_{n+1}\setminus A_n$ for all $n\geq1$. Then $\biguplus_{k\geq1} B_k=\bigcap_{n\geq1} A_n = A$ and $\brac{B_n}_{n\geq1}\in \Fcal$ are pairwise disjoint. Furthermore $A_n=\biguplus_{k=1}^n B_k$ for all $n\geq1$.

Since $\mu$ is a measure on $\Fcal$ it is finitely additive, whence $\mu\brac{A_n}=\sum_{k=1}^n \mu\brac{B_k}$ for all $n\geq1$. Furthermore $\mu\brac{A_n}\subseteq \mu\brac{A_{n+1}}$ and \[\lim_{n\to\infty}\sum_{k=1}^n\mu\brac{B_k}=\sum_{n\geq1}\mu\brac{B_n}=\mu\brac{\biguplus_{k\geq1}B_k}\]whence $\lim_{n\to\infty} \mu\brac{A_n}=\mu\brac{A}$. Therefore $\mu\brac{A_n}\uparrow\mu\brac{A}$.\\

\label{thm:cont_up_is_meas} \noindent \textbf{Theorem} 3-5.
Let $\brac{\Omega, \Fcal}$ be a measurable space. If $\mu:\Fcal\to\Zinf$ is a finitely additive map, such that $\mu\brac{\emptyset}=0$ and for any $A,\brac{A_n}_{n\geq1}\in \Fcal$ with $A_n\uparrow A$ it is true that $\mu\brac{A_n}\uparrow \mu\brac{A}$, then $\mu$ is a measure on $\brac{\Omega, \Fcal}$.

Indeed, let $A,\brac{A_n}_{n\geq1}\in \Fcal$ be such that $A=\biguplus_{k\geq1} A_k$. Then $B_n\defn \biguplus_{k=1}^n A_k$ is such that $B_n\uparrow A$, whence $\mu\brac{B_n}\uparrow\mu\brac{A}$. However $\mu\brac{B_n}=\sum_{k=1}^n \mu\brac{A_n}$ for all $n\geq1$, which implies that $\sum_{k\geq1}\mu\brac{A_n}=\mu\brac{A}$. It is possible to regard measures as additive maps possessing of a certain ``continuity'' property.\\

\label{thm:meas_cont_down} \noindent \textbf{Theorem} 8.
Let $\brac{\Omega, \Fcal, \mu}$ be a measure space. If $\brac{A_n}_{n\geq 1}\in \Fcal$ with $A_n\downarrow A$ and there is $N\geq 1$ such that $\mu\brac{A_N}<+\infty$, then $\mu\brac{A_n}\downarrow\mu\brac{A}$.

This theorem crucially depends on the finiteness of $\mu\brac{A_n}$ for all sufficiently large $n\geq 1$ to enable subtraction. Indeed, if there exists $N\geq1$ such that $\mu\brac{A_N}<+\infty$, then $\mu\brac{A_n}<+\infty$ for all $n\geq N$, since $A_n\downarrow A$.

Define $B_n\defn A_N\setminus A_n$ for all $n\geq1$. Obviously $\brac{B_n}_{n\geq1}\in \Fcal$ and $B_n\uparrow A_N\setminus A$. However $A_N=B_n\uplus A_n$ and $\mu\brac{A_n}$ is finite for all $n\geq N$, whence $\mu\brac{A_N}-\mu\brac{B_n}=\mu\brac{A_n}$ for all $n\geq N$. Similarly $\mu\brac{A_N}-\mu\brac{A_N\setminus A} = \mu\brac{A}$. Therefore \[\mu\brac{A_N}-\mu\brac{A_n}\uparrow \mu\brac{A_N}-\mu\brac{A}\] which implies that $\mu\brac{A_n}\downarrow \mu\brac{A}$.\\

\label{thm:meas_subadd} \noindent \textbf{Theorem} 3-6.
Let $\brac{\Omega, \Fcal, \mu}$ be a measure space. If $\brac{A_n}_{n\geq1}\in \Fcal$, then \[\mu\brac{\bigcup_{n\geq1} A_n}\leq \sum_{n\geq1} \mu\brac{A_n}\]

Indeed, put $B_{n+1}\defn A_{n+1}\setminus\brac{ \bigcup_{k=1}^n A_k}$ with $B_1\defn A_1$. Then $\brac{B_n}_{n\geq1}\in \Fcal$, $\biguplus_{n\geq1} B_n = \bigcup_{n\geq1} A_n$ and $B_n\subseteq A_n$. Therefore \[\mu\brac{\bigcup_{n\geq1} A_n} = \mu\brac{\biguplus_{n\geq1} B_n} = \sum_{n\geq1} \mu\brac{B_n}\leq \sum_{n\geq1} \mu\brac{A_n}\]

\label{thm:stieltjes_meas} \noindent \textbf{Theorem} 9.
Let $F:\Real\to\Real$ ba a right continuous non-decreasing map. There exists a unique measure $\mu:\borel{\Real}\to \Zinf$, such that \[\forall a, b\in \Real, a\leq b, \mu\brac{\ploc{a, b}} = F\brac{b} - F\brac{a}\] Such measure is called the Stieltjes measure on $\borel{\Real}$ and is denoted by $dF$.

By theorem 3-4 thus defined map $\mu$ is a measure on $\Scal$. By a corollary to Caratheodory's extension theorems (theorem 5), this measure $\mu$ can be extended from a semi-ring of half-open intervals $\Scal$ to $\sigma\brac{\Scal}$, which by theorem 6 coincides with $\borel{\Real}$ -- the $\sigma$ algebra of open sets in $\Real$.

Let $F_n\defn \clop{-n,n}\in \Scal$ for all $n\geq1$. Then $\mu\brac{F_n}=F\brac{n}-F\brac{-n}<+\infty$ since $F:\Real\to\Real$. Since $F_n\uparrow \Real$, this means that $\mu$ is a $\sigma$-finite measure (definition ??) on $\brac{\Real,\borel{\Real}}$.

Suppose there are $\sigma$-finite measures $\mu_1,\mu_2$ with $\mu_1\brac{A}=\mu_2\brac{A}$ for all $A\in \Scal$ and there is $\brac{F_n}_{n\geq1}\in \Scal$ such that $\mu_1\brac{F_n}<+\infty$ while $F_n\uparrow \Real$. Define \[\Dcal_n\defn \obj{\induc{B\in \sigma\brac{\Scal}}\,\mu_1\brac{B\cap F_n}=\mu_2\brac{B\cap F_n}}\]

Since $\Scal$ is a $\pi$-system, $\mu_1\brac{A\cap F_n}=\mu_2\brac{A\cap F_n}$ for any $A\in \Scal$, whence $\Scal\subseteq \Dcal_n$ for all $n\geq1$.

First, $\Real\in \Dcal_n$ as $\Real\cap F_n=F_n\in\Scal$. Second, if $\brac{A_k}_{k\geq1}\in \Dcal_n$ with $A_k\subseteq A_{k+1}$, then $A_k\cap F_n\uparrow A\cap F_n$, where $A\defn \bigcup_{k\geq1} A_k$. Thus by theorem 7 $\mu_1\brac{A_k\cap F_n}\uparrow\mu_1\brac{A\cap F_n}$ and $\mu_2\brac{A_k\cap F_n}\uparrow\mu_2\brac{A\cap F_n}$, which implies that $\mu_1\brac{A\cap F_n}=\mu_2\brac{A\cap F_n}$ and $\bigcup_{k\geq1} A_k\in \Dcal_n$ for all $n\geq1$.

Finally, if $A,B\in\Dcal_n$ with $A\subseteq B$ then $B=A\uplus \brac{B\setminus A}$, which implies that \begin{align*}\mu_1\brac{B\cap F_n}&=\mu_1\brac{A\cap F_n}+\mu_1\brac{\brac{B\setminus A}\cap F_n}\\\mu_2\brac{B\cap F_n}&=\mu_2\brac{A\cap F_n}+\mu_2\brac{\brac{B\setminus A}\cap F_n}\\\end{align*} As $\mu_1\brac{\cdot\cap F_n}\leq \mu_1\brac{F_n}<+\infty$ and $\mu_2\brac{\cdot\cap F_n}\leq \mu_2\brac{F_n}<+\infty$, both sides of each expression of finite real numbers. Therefore $\mu_1\brac{\brac{B\setminus A}\cap F_n}=\mu_2\brac{\brac{B\setminus A}\cap F_n}$ whence $B\setminus A\in \Dcal_n$ for all $n\geq1$.

Thus the measurable sets, on which the measures truncated to $F_n$ coincide, constitute a Dynkin system containing a semi-ring $\Scal$, it is easily inferred from the Dynkin system theorem (theorem 1) that these truncated measures coincide on the whole $\sigma$-algebra generated by the semi-ring. Since $\sigma\brac{\Scal}=\borel{\Real}$ it is true that $\borel{\Real}\subseteq \Dcal_n$ for all $n\geq1$.

Now, $B\cap F_n\uparrow B$ for any $B\in \borel{\Real}$ and $\mu_1\brac{B\cap F_n}=\mu_2\brac{B\cap F_n}$ for all $n\geq1$. Thus by the upper continuity of measures (theorem 7), $\mu_1\brac{B\cap F_n}\uparrow\mu_1\brac{B}$ and $\mu_2\brac{B\cap F_n}\uparrow\mu_2\brac{B}$, whence $\mu_1\brac{B}=\mu_2\brac{B}$ for all $B\in \borel{\Real}$. Thus $\mu_1$ and $\mu_2$ coincide on $\borel{\Real}$, which proves the uniqueness of the extension of the measure $\mu$ from $\Scal$ to $\borel{\Real}$.\\

\label{thm:stieltjes_meas_atom} \noindent \textbf{Theorem} 3-7.
Let $F:\Real\to\Real$ be a right-continuous, non-decreasing map and $dF$ be the Stieltjes measure on $\Real$ associated with it. Then for any $x\in \Real$ \[dF\brac{\obj{x}} = F\brac{x}-F\brac{x-}\] where $F\brac{x-}\defn \lim_{y<x,\,y\to x} F\brac{y}$ -- the left-limit of $F$ at $x$.

Indeed, let $x\in \Real$. Since $F\brac{y}\leq F\brac{x}$ for all $y<x$, it is true that $\alpha\defn \sup_{y<x} F\brac{y}$ is not greater than $F\brac{x}$ which implies that $\alpha\in \Real$. Now for any $\epsilon>0$ there is $y<x$ such that $\alpha-\epsilon<F\brac{y}\leq \alpha$. Since $F$ is non-decreasing, it must be that $\abs{F\brac{z}-\alpha}<\epsilon$ for all $z\in \brac{y,x}$. Therefore $\alpha = \lim_{y<x,\,y\to x} F\brac{y}$, whence $F\brac{x-}\defn \alpha$ is a well-defined real number for any $x\in \Real$.

Since $\obj{x}=\bigcap_{n\geq1} A_n$, where $A_n\defn\ploc{x-\frac{1}{n},x}$ it is true that $\obj{x}\in \borel{\Real}$, whence it is legitimate to consider $dF\brac{\obj{x}}$ for any $x\in \Real$. Since $dF\brac{A_n}<\infty$ as $F$ is finite and $A_n\downarrow \obj{x}$ by definition, ``continuity'' in theorem 8 implies that $dF\brac{A_n}\downarrow dF\brac{\obj{x}}$, whence \[dF\brac{\obj{x}}=\lim_{n\to \infty} F\brac{x}-F\brac{x-\frac{1}{n}} = F\brac{x}-F\brac{x-}\]\\

\noindent \textbf{Definition} 20.
Let $F:\Real\to\Real$ be a right-continuous, non-decreasing map. The Stieltjes measure on $\Real$ associated with $F$ is the unique measure $dF\borel{\Real}\to\Zinf$ such that for all $a\leq b\in \Real$\[dF\brac{\ploc{a,b}}\defn F\brac{b}-F\brac{a}\] A Lebesgue measure on $\borel{\Real}$ denoted by $dx$ is the Stieltjes measure associated with $F\brac{x}=x$.

\label{thm:trace_sigma} \noindent \textbf{Theorem} 10.
Let $\Omega'\subseteq \Omega$ and $\Scal$ be a collection of subsets of $\Omega$. The trace on $\Omega'$ of the $\sigma$-algebra $\sigma\brac{\Scal}$ is equal to the $\sigma$-algebra on $\Omega'$ generated by the trace of $\Scal$ on $\Omega'$\[\induc{\sigma\brac{\Scal}}_{\Omega'} = \sigma\brac{\induc{\Scal}_{\Omega'}}\]

Let $\Gamma\defn \obj{\induc{A\in \sigma\brac{\Scal}}\,A\cap \Omega'\in \sigma\brac{\induc{\Scal}_{\Omega'}}}$. Obviously $\Scal\subseteq\Gamma$ and $\Gamma$ is a $\sigma$-algebra on $\Omega$, whence $\sigma\brac{\Scal}\subseteq\Gamma$. Thus \[\induc{\sigma\brac{\Scal}}_{\Omega'}\subseteq \sigma\brac{\induc{\Scal}_{\Omega'}}\]

Conversely, if $A'\in\induc{\Scal}_{\Omega'}$ then there exists $A\in \Scal$ such that $A'=A\cap \Omega'$. Since $A\cap \Omega'\in\induc{\sigma\brac{\Scal}}_{\Omega'}$, $\induc{\Scal}_{\Omega'}\subseteq \induc{\sigma\brac{\Scal}}_{\Omega'}$ whence \[\sigma\brac{\induc{\Scal}_{\Omega'}}\subseteq \induc{\sigma\brac{\Scal}}_{\Omega'}\] because $\induc{\sigma\brac{\Scal}}_{\Omega'}$ is a $\sigma$-algebra on $\Omega'$.\\

A corollary of this theorem is that $\borel{\Omega'} = \induc{\borel{\Omega}}_{\Omega'} \subseteq \borel{\Omega}$ if $\Omega'\in \borel{\Omega}$, as topologies are closed under finite intersection. In particular $\borel{\Real^+}\subseteq \borel{\Real}$ and $\borel{\Real^+}=\induc{\borel{\Real}}_{\Real^+}$.

\label{thm:stieltjes_meas_rplus} \noindent \textbf{Theorem} 11.
Let $F:\Real^+\to\Real$ be a right continuous non-decreasing map, with $F\brac{0}\geq 0$. Then there exists a unique measure $\mu:\borel{\Real^+}\to\Zinf$ such that $\mu\brac{\obj{0}} = F\brac{0}$ and if $0\leq a\leq b$ then $\mu\brac{\ploc{a, b}} = F\brac{b} - F\brac{a}$. Such measure is known as the Stieltjes measure on $\Real^+$ associated with $F$ and is denoted similarly by $dF$.\\

Indeed, put $\bar{F}\brac{x}\defn0$ for all $x<0$ and $\bar{F}\brac{x}\defn F\brac{x}$ for all $x\geq0$. Then $\bar{F}$ is non-decreasing and right-continuous, since for all $x<0$ there is $U$ open in $\Real$ with $x\in U$ such that $\bar{F}\brac{y}=0$ for all $y\in U$. By theorem 9 there exists a unique measure $d\bar{F}$ on $\brac{\Real, \borel{\Real}}$ such that $d\bar{F}\brac{\ploc{a,b}}=\bar{F}\brac{b}-\bar{F}\brac{a}$ for every $a\leq b\in \Real$.

Since $\borel{\Real^+}\subseteq \borel{\Real}$, the map $\mu\defn \induc{d\bar{F}}_{\borel{\Real^+}}$ is a measure on $\borel{\Real^+}$. Furthermore $\mu\brac{\ploc{a,b}}=F\brac{b}-F\brac{a}$ for all $0\leq a\leq b$ and by theorem 3-7 \[\mu\brac{\obj{0}}=\bar{F}\brac{0}-\bar{F}\brac{0-}=F\brac{0}\] since $\lim_{x<0,\,x\to0} \bar{F}\brac{x}=0$.

Now let $\Ccal\defn \obj{\induc{\ploc{a,b}}\,0\leq a\leq b}\uplus\obj{\obj{0}}$. Since $\Ccal=\induc{\Scal}_{\Real^+}$, theorem 10 implies that $\sigma{\Ccal}=\induc{\sigma{\Scal}}_{\Real^+}$ whence by theorem 6 and corollary to theorem 10 $\sigma{\Ccal}=\borel{\Real^+}$.

Suppose there are two measures $\mu_1, \mu_2$ on $\borel{\Real^+}$ such with \[\mu_1\brac{\ploc{a,b}}=\mu_2\brac{\ploc{a,b}}=F\brac{b}-F\brac{a}\] for all $0\leq a\leq b$ and $\mu_1\brac{\obj{0}}=\mu_2\brac{\obj{0}}=F\brac{0}$.

For any $n\geq 1$ define $E_n\defn \clo{0,n}\in\borel{\Real^+}$ and \[\Dcal_n\defn \obj{ \induc{B\in\borel{\Real^+}}\, \mu_1\brac{B\cap E_n}=\mu_2\brac{B\cap E_n} }\] Note that $\mu_i\brac{E_n}=F\brac{n}<+\infty$ because $E_n = \obj{0}\uplus\ploc{0,n}$. Thus $\mu_i\brac{B\cap E_n}<+\infty$ for any $B\in \borel{\Real^+}$ for all $n\geq1$.

First, $\Ccal\subseteq \Dcal_n$ and $\Real^+\in \Dcal_n$ for all $n\geq1$. Next if $A,B\in \Dcal_n$ and $A\subseteq B$, then $B=B\setminus A \uplus A$ and finiteness of $\mu_i\brac{\cdot\cap E_n}$ implies that \[\mu_1\brac{B\setminus A\cap E_n}=\mu_1\brac{B\cap E_n}-\mu_1\brac{A\cap E_n}=\mu_2\brac{B\cap E_n}-\mu_2\brac{A\cap E_n}=\mu_2\brac{B\setminus A\cap E_n}\] whence $B\setminus A\in \Dcal_n$. Finally, if $\brac{A_k}_{k\geq1}\in \Dcal_n$ with $A_k\subseteq A_{k+1}$ then by theorem 7 $\mu_1\brac{A\cap E_n}=\mu_2\brac{A\cap E_n}$, where $A\defn \bigcup_{k\geq1} A_k$. Therefore $\Dcal_n$ is a Dynkin system on $\Real^+$ for all $n\geq1$.

Thus by theorem 1 $\sigma\brac{\Ccal}\subseteq\Dcal_n$, whence $\mu_1\brac{\cdot\cap \clo{0,n}}=\mu_1\brac{\cdot\cap \clo{0,n}}$ for all $n\geq1$. By theorem 7 for all $B\in \borel{\Real^+}$ it is true that $\mu_i\brac{B\cap\clo{0,n}}\uparrow \mu_i\brac{B}$ as $B\cap\clo{0,n}\uparrow B$. Thus $\mu_1=\mu_2$ and the measure $\mu$ defined above is indeed unique.\\

% section tut_3 (end)

\section{Measurability} % (fold)
\label{sec:tut_4}
\url{http://probability.net/PRTmeasurability.pdf}

\label{thm:metric_trace} \noindent \textbf{Theorem} 12.
Let $\brac{E,d}$ be a metric space and $F\subseteq E$. Then the induced (subspace) topology on $F$ -- the trace on $F$ of the metric topology $\Tcal_E^d$, is equal to the metric topology on $F$ associated with the induced metric on $F$, $\Tcal_F^{\induc{d}_F}$.

This result rests on the fact that the metric topology is generated by the open balls in $\brac{E,d}$ \[B^d_E\brac{x,\epsilon}\defn\obj{ \induc{y\in E}\, d\brac{x,y}<\epsilon }\] and so every open set is in fact an arbitrary union of open balls.

Indeed, let $\Tcal_F'\defn\Tcal_F^{\induc{d}_F}$ and $\Tcal_F\defn\induc{\Tcal_E^d}_F$.

If $U'\in \Tcal_F$, then there is $U\in\Tcal_E^d$ with $U'=U\cap F$, whence for all $x\in U$ there is $\epsilon>0$ such that $B^d_E\brac{x,\epsilon}\cap F \subseteq U\cap F$. Since $B^d_E\cap F=B^{\induc{d}_F}_F$, $U'\in \Tcal_F'$.

Now, if $U\in \Tcal_F'$, then for each $x\in U$ there is $\epsilon_x>0$ such that $B^d_E\brac{x,\epsilon_x}\cap F=B^{\induc{d}_F}_F\brac{x,\epsilon_x}\subseteq U$. Therefore \[U = \brac{\bigcup_{x\in U} B^d_E\brac{x,\epsilon_x}}\cap F\] whence $U\in \Tcal_F$.\\

\noindent \textbf{Definition} 34.
The usual topology $\Tcal_\Rbar$ on the extended real line is defined as a topological space $\brac{\Rbar, \Tcal_\Rbar}$ homeomorphic to the topological subspace $\clo{-1, 1}$ with the usual (subspace) topology of $\brac{\Real, \Tcal_\Real}$. For an increasing homeomorphism $\phi:\Rbar\to\clo{-1, 1}$ \[\Tcal_\Rbar \defn \obj{\induc{ U\subseteq \Rbar } \phi\brac{U}\, \text{is open in}\,\clo{-1,1} }\]

\label{thm:rbar_topology} \noindent \textbf{Theorem} 4-1.
The topological space $\brac{\Real, \Tcal_\Real}$ is a topological subspace of $\brac{\Rbar, \Tcal_\Rbar}$ and $\Tcal_\Real\subseteq \Tcal_\Rbar$.

If $\phi\brac{x}\defn\frac{x}{1+\abs{x}}$, then $\phi:\Real\to\brac{-1,1}$ is a homeomorphism between $\brac{\Real, \Tcal_\Real}$ and $\brac{\brac{-1,1}, \Tcal_{\brac{-1,1}}}$ by the $\epsilon$-$\delta$ definition of continuity, where \[\Tcal_{\brac{-1,1}}\defn\induc{\Tcal_\Real}_{\brac{-1,1}}=\induc{\Tcal^d_\Real}_{\brac{-1,1}}=\Tcal^{\induc{d}_{\brac{-1,1}}}_{\brac{-1,1}}\] using theorem 12.

Next, define $\bar{\phi}\brac{+\infty}\defn+1$, $\bar{\phi}\brac{-\infty}\defn-1$ and $\bar{\phi}\brac{x}\defn\phi\brac{x}$ for all $x\in \Real$. The map $\bar{\phi}:\Rbar\to\clo{-1,1}$ is a continuous bijection, since $\lim_{x\to+\infty}\bar{\phi}\brac{x}=+1$ and $\lim_{x\to-\infty}\bar{\phi}\brac{x}=-1$.

Let \[\Tcal_{\clo{-1,1}}\defn\induc{\Tcal_\Real}_{\clo{-1,1}}=\induc{\Tcal^d_\Real}_{\clo{-1,1}}=\Tcal^{\induc{d}_{\clo{-1,1}}}_{\clo{-1,1}}\] and define $\Tcal_\Rbar\defn \obj{\induc{U\subseteq \Rbar}\,\bar{\phi}\brac{U}} \in \Tcal_{\clo{-1,1}}$. Then, since the pre-image of any map preserves set operations, $\Tcal_\Rbar$ must be a topology on $\Rbar$.

Since $\bar{\phi}$ is a bijection, $\brac{\bar{\phi}^{-1}}^{-1}\brac{U}=\bar{\phi}\brac{U}$  for every $U\in \Tcal_\Rbar$, whence $\bar{\phi}^{-1}$ is continuous. Conversely, let $U\in \Tcal_{\clo{-1,1}}$ and put $V\defn\bar{\phi}^{-1}\brac{U}$. As $\bar{\phi}$ is a bijection, $\bar{\phi}\brac{V} = U$, whence $V\in \Tcal_\Rbar$. Thus $\bar{\phi}$ is a homomorphism.

Now let $\Tcal'\defn \induc{\Tcal_\Rbar}_\Real$. Since $\phi\brac{U\cap\Real}=\bar{\phi}\brac{U}\cap\brac{-1,1}$ for all $U\subseteq \Rbar$ and $\Tcal_{\brac{-1,1}}=\induc{\Tcal_{\clo{-1,1}}}_{\brac{-1,1}}$, it must be true that $\phi\brac{U\cap\Real}$ is open in $\brac{\brac{-1,1}, \Tcal_{\brac{-1,1}}}$. For any $U'\in\Tcal'$ there is $U\in \Tcal_\Rbar$ with $U'=U\cap \Real$, whence \[\phi\brac{U'}=\bar{\phi}\brac{U}\cap \brac{-1,1}\in\induc{\Tcal_{\clo{-1,1}}}_{\brac{-1,1}}=\Tcal_{\brac{-1,1}}\] Since $\phi$ is a homeomorphism between $\Real$ and $\brac{-1,1}$, it is true that $\Tcal'\subseteq \Tcal_\Real$.

Conversely, $\bar{\phi}\brac{U}=\phi\brac{U}$ for any $U\in\Tcal_\Real$, whence $\bar{\phi}\brac{U}$ is open in $\brac{\brac{-1,1}, \Tcal_{\brac{-1,1}}}$ and in $\brac{\clo{-1,1}, \Tcal_{\clo{-1,1}}}$, because $\brac{-1,1}$ is open in $\clo{-1,1}$. Therefore $\Tcal_\Real\subseteq \Tcal_\Rbar$ and $\Tcal_\Real\subseteq\Tcal'$.\\

\label{thm:t_rbar_metr} \noindent \textbf{Theorem} 13.
The topological space $\brac{\Rbar, \Tcal_\Rbar}$ is metrizable.

Indeed, let $h:\Rbar\to \clo{-1,1}$ be a homeomorphism and $d_\Rbar\brac{x,y}\defn \abs{h\brac{x}-h\brac{y}}$. Since $\abs{\cdot}$ obey the triangle law and $h$ is bijective, $d_\Rbar$ is indeed a metric on $\Rbar$.

If $U\in \Tcal_\Rbar$ then $h\brac{U}$ is open in $\clo{-1, 1}$. If $x\in U$, then $h\brac{x}\in h\brac{U}$, whence there is $\epsilon>0$ with $B_\Real^{d_\Real}{h\brac{x}, \epsilon}\cap \clo{-1,1}\subseteq h\brac{U}$. Since $h$ is continuous and by theorem 12 the topological subspace space $\clo{-1,1}$ of $\Real$ is metrized by the induced metric $\induc{d_\Real}_{\clo{-1,1}}$, by the $\epsilon$-$\delta$ definition of continuity there is $\delta>0$ such that $h\brac{y}\in B_\Real^{d_\Real}{h\brac{x}, \epsilon}\cap \clo{-1,1}$ for all $y\in \Rbar$ with $d_\Rbar\brac{x,y}<\delta$. Therefore, $h\brac{y}\in h\brac{U}$ and $y\in U$. In conclusion, for any $x\in U$ there is $\delta>0$ such that $B_\Rbar^{d_\Rbar}\brac{x, \delta}\subseteq U$, whence $U\in \Tcal_\Rbar^{d_\Rbar}$.

Conversely, let $U\in \Tcal_\Rbar^{d_\Rbar}$ and consider $h\brac{U}$. For any $y\in h\brac{U}$ there is $x\in U$ such that $y=h\brac{x}$. Thus there is $\epsilon>0$ such that $B_\Rbar^{d_\Rbar}\brac{x, \epsilon}\subseteq U$. If $z\in \clo{-1,1}$ is such that $\abs{y-z}<\epsilon$, then by bijectivity of $h$ there is $\omega\in \Rbar$ such that $h\brac{\omega}=z$, whence $\abs{y-z}=\abs{h\brac{x}-h\brac{\omega}}<\epsilon$ and $\omega\in B_\Rbar^{d_\Rbar}\brac{x, \epsilon}$. Thus $\omega\in U$ and so $z\in h\brac{U}$, whence \[B_\Real^{d_\Real}{y, \epsilon}\cap \clo{-1,1}\subseteq h\brac{U}\] and for any $y\in h\brac{U}$ there is $V$ open in $\clo{-1,1}$ with $y\in V\subseteq h\brac{U}$. Therefore $h\brac{U}$ is open in $\clo{-1,1}$ and $U\in \Tcal_\Rbar$.\\

\label{thm:meas_generating} \noindent \textbf{Theorem} 4-2.
Let $\brac{\Omega, \Fcal}$ and $\brac{\Scal, \Sigma}$ be two measurable spaces. Let $S'\subseteq S$ and $f:\Omega\to S$ be a map such that $f\brac{\Omega}\subseteq S'$. Denote $\Sigma'\defn \induc{\Sigma}_{S'}$. Then the map $f:\brac{\Omega, \Fcal}\to\brac{S',\Sigma'}$ is measurable if and only if $f:\brac{\Omega, \Fcal}\to\brac{S, \Sigma}$ is measurable.

Indeed, for any $E\subseteq S$, if $x\in f^{-1}\brac{E}$ then $f\brac{x}\in E$, whence $f\brac{x}\in E\cap S'$, because $f\brac{\Omega}\subseteq S'$ and $x\inf^{-1}\brac{E\cap S'}$. The converse is true by definition on the inverse image $f^{-1}\brac{E\cap S'}\subseteq f^{-1}\brac{E}$.

If $f:\brac{\Omega, \Fcal}\to\brac{S, \Sigma}$ is measurable, then for any $E'\in \Sigma'$ exists $E\in \Sigma$ such that $E'=E\cap S'$, whence by the above $f^{-1}\brac{E'} = f^{-1}\brac{E} \in \Fcal$.

If $f:\brac{\Omega, \Fcal}\to\brac{S', \Sigma'}$ is measurable, then $f^{-1}\brac{E} = f^{-1}\brac{E\cap S'} \in \Fcal$ for every $E\in \Sigma$, since $E\cap S'\in \Sigma'$ by definition.\\

\label{thm:measurability1} \noindent \textbf{Theorem} 14.
Let $\brac{\Omega,\Fcal}$ and $\brac{\Scal,\Sigma}$ be two measurable spaces, and $\Scal$ be a collection which generates $\Sigma$. Then $f:\brac{\Omega, \Fcal}\to\brac{\Scal, \Sigma}$ is measurable if and only if $f^{-1}\brac{B}\in\Fcal$ for all $B\in\Scal$.

Recall that $f:\brac{\Omega, \Fcal}\to\brac{\Scal, \Sigma}$ is measurable if $f^{-1}\brac{B} \in \Fcal$ for all $B\in \Sigma$.

Let $\Gamma\defn\obj{\induc{B\in\Sigma}\,f^{-1}\brac{B}\in\Fcal}$. Then $S\in\Gamma$ since $f^{-1}\brac{S}=\Omega$. Next if $A\in\Gamma$, then $S\setminus\in\Gamma$, because $f^{-1}\brac{S\setminus A} = \Omega\setminus f^{-1}\brac{A}$. Finally if $\brac{A_n}_{n\geq1}\in\Gamma$ then $f^{-1}\brac{\bigcup_{n\geq1} A_n}=\bigcup_{n\geq1} f^{-1}\brac{A_n}$ by the basic properties of pre-images of maps. Thus $\Gamma$ is $\sigma$-algebra on $S$.

Now, if $f:\Omega\to S$ is a map, such that $f^{-1}\brac{A}\in \Fcal$for all $A\in \Scal$, then $\Scal\subseteq\Gamma$, whence $\Sigma=\sigma\brac{\Scal}\subseteq\Gamma$. Therefore $f^{-1}\brac{A}\in\Fcal$ for all $A\in \Sigma$, which implies that $f$ is measurable. 

Conversely, if $f:\brac{\Omega,\Fcal}\to\brac{S,\Sigma}$ is measurable, then, in particular, $f^{-1}\brac{A}\in\Fcal$ for all $A\in\Scal$. In fact, measurability, at least in this theorem, has many similarities with the continuity in the case when the topology is generated by a topological basis.\\

\label{thm:measurability2} \noindent \textbf{Theorem} 15.
Let $\brac{\Omega, \Fcal}$ be a measurable space, and $f:\Omega\to\Rbar$ be a map. Then the following are equivalent\begin{enumerate}
	\item $f:\brac{\Omega,\Fcal}\to\brac{\Rbar,\borel{\Rbar}}$ is measurable
	\item $\obj{f\in B}\in \Fcal$ for all $B\in \borel{\Rbar}$
	\item $\obj{f\leq c}\in \Fcal$ for all $c\in \Real$
	\item $\obj{f < c}\in \Fcal$ for all $c\in \Real$
	\item $\obj{c\leq f}\in \Fcal$ for all $c\in \Real$
	\item $\obj{c < f}\in \Fcal$ for all $c\in \Real$
\end{enumerate}

This theorem is made possible by exercise 14 and theorem 14, which shows that\begin{align*}\borel{\Rbar} &= \sigma\brac{\obj{ \induc{ \clo{-\infty, c} } c\in \Real}} \\&= \sigma\brac{\obj{ \induc{ \left [-\infty, c \right) } c\in \Real}} \\&= \sigma\brac{\obj{ \induc{ \clo{c, +\infty} } c\in \Real}} \\&= \sigma\brac{\obj{ \induc{ \ploc{c, +\infty} } c\in \Real}}\end{align*} which is due to existence of a countable topological basis in $\Real$, every open set in $\Real$ is a countable union of bounded open intervals in $\Real$.\\

\label{thm:meas_inf} \noindent \textbf{Theorem} 4-3.
Let $\brac{\Omega, \Fcal}$ be a measurable space and $\brac{f_n}_{n\geq1}:\brac{\Omega, \Fcal}\to\brac{\Rbar,\borel{\Rbar}}$ be a sequence of measurable maps. Then the maps $g\brac{\omega}\defn \inf_{n\geq1}f_n\brac{\omega}$ and $h\brac{\omega}\defn \sup_{n\geq1}f_n\brac{\omega}$ are $\Fcal$-$\borel{\Rbar}$ measurable.

Indeed, note that $\obj{c\leq g}=\bigcap_{n\geq1}\obj{c\leq f_n}$, while $\obj{h\leq c}=\bigcap_{n\geq1}\obj{f_n\leq c}$ for any $c\in \Real$ by definition of the greatest lower and the least upper bounds respectively. Therefore $h^{-1}\brac{\ploc{-\infty,c}}$ and $h^{-1}\brac{\clop{c,+\infty}}$ are $\Fcal$-measurable sets for all $c\in \Real$. Therefore by theorem 15 $f$ and $g$ are $\Fcal$-$\borel{\Rbar}$ measurable.\\

\label{thm:limsup_liminf} \noindent \textbf{Theorem} 16.
If $\brac{u_n}_{n\geq 1}\in\Rbar$, then $\lim_{n\to\infty} u_n$ exists in $\Rbar$ if and only if \[\liminf_{n\to\infty} u_n = \limsup_{n\to\infty} u_n\] in which case $\lim_{n\to\infty} u_n \defn \liminf_{n\to\infty} u_n = \limsup_{n\to\infty} u_n$.

Indeed, let $\brac{u_n}_{n\geq 1}\in \Rbar$, $v_n\defn\inf_{k\geq n} u_k$ and $w_n\defn\sup_{k\geq n} u_k$. Further put $v\defn\sup_{n\geq1}v_n$ and $w\defn\inf_{n\geq1}w_n$.

First, $v_n\leq v_{n+1}\leq v$ by definition of the greatest lower bound. Next if $v\neq-\infty$ then for all $L\in \Real$ with $L<v$ there must be $N\geq1$ with $L<v_N\leq v$, whence $L<v_n\leq v$ for all $n\geq N$, and if $v=-\infty$, then $v_n=-\infty$ for all $n\geq1$. Therefore $v_n\uparrow v$. Similar arguments imply that $w_n\downarrow w$.

Now, $v_n\leq u_k\leq w_n$ for all $k\geq n$, whence $v_n\leq u_n\leq w_n$ for all $n\geq 1$. Since $v_n\to v$ and $w_n\to w$ it must be that $v\leq w$.

Let $v=w\in \Real$. For all $\epsilon>0$ there are $M,K\geq 1$ with $v-\epsilon<v_m$ and $w_k<w+\epsilon$ for all $m\geq M$ and $k\geq K$, whence for $N\defn \max\obj{M,K}$ it is true that $v-\epsilon<u_n<w+\epsilon$ for all $n\geq N$. Since $v=w$ this means that for all $\epsilon>0$ there is $N\geq1$ such that $\abs{v-u_n}<\epsilon$for all $n\geq N$.

If $v=w$ and $v=+\infty$, then for all $L\in \Real$ there is $N\geq 1$ with $L<v_n$ for all $n\geq N$, whence $L<u_n$ for all $n\geq N$. Thus $u_n\to +\infty$. Finally, if $v=w$ and $w=-\infty$, similar arguments implies that $u_n\to-\infty$. Therefore $u_n\to u$ whenever $v=w$.

If $v<w$, then there are $a,b\in \Real$ such that $v<a<b<w$ ($\Real$ is a linear continuum). Since $v<a$ and $b<w$ for all $n\geq1$ there are $M,K\geq n$ with $v_n\leq u_M < a$ and $b < u_K \leq w_n$. This implies that there are $\brac{m_n}_{n\geq1}$ and $\brac{k_n}_{n\geq1}$ with $k_n,m_n\uparrow+\infty$ such that $u_{m_n}<a$ and $b<u_{k_n}$ for all $n\geq 1$.

Since $k_n,m_n\uparrow+\infty$ for every $N\geq1$ there are $K,M\geq 1$ with $N<k_n$ for all $n\geq K$ and $N<m_n$ for all $n\geq M$. Thus $u_n\not\to+\infty$ since for $a\in \Real$ there is no $N\geq1$ with $a<u_n$ for all $N\geq N$. Similarly $u_n\not\to-\infty$. Finally, suppose $u_n\to u$ with $u\in \Real$. If $u\leq a$ or $b\leq u$, let $\epsilon\defn b-a$ otherwise let $\epsilon\defn\min\obj{u-a,b-u}$. Then there is no $N\geq1$ with $u-\epsilon<u_n<u+\epsilon$ for all $n\geq N$. Therefore $u_n\not\to u$.

In conclusion, $\lim_{n\to\infty} u_n$ exists in $\Rbar$ if and only if $\liminf_{n\to\infty} u_n = \limsup_{n\to\infty} u_n$.\\

\label{thm:func_ineq_meas} \noindent \textbf{Theorem} 4-4.
Let $\brac{\Omega, \Fcal}$ be a measurable space and $f,g:\brac{\Omega, \Fcal}\to\brac{\Rbar,\borel{\Rbar}}$ be a sequence of measurable maps. Then \[\obj{f<g}, \obj{f\leq g}\in\Fcal\]

Since $\mathbb{Q}$ is an everywhere dense countable subset of $\Real$, \[\obj{f<g} = \bigcup_{q\in \mathbb{Q}} \brac{\obj{f<q}\cap\obj{q<g}}\] is a measurable subset of $\brac{\Omega, \Fcal}$. Similarly, $\obj{g<f}\in \Fcal$, whence $\obj{f\leq g}=\obj{g<f}^c\in \Fcal$. Furthermore, $\obj{g\leq f}\in \Fcal$ and $\obj{f=g}\in \Fcal$.\\

\label{thm:limsup_liminf_meas} \noindent \textbf{Theorem} 4-5.
Let $\brac{\Omega, \Fcal}$ be a measurable space and $\brac{f_n}_{n\geq1}:\brac{\Omega, \Fcal}\to\brac{\Rbar,\borel{\Rbar}}$ be a sequence of measurable maps.

First, the maps $g\defn\liminf_{n\to\infty}f_n$ and $h\defn\limsup_{n\to\infty}f_n$ are $\Fcal$-$\borel{\Rbar}$ measurable. Indeed, by theorem 4-3 the maps $g_n\defn\inf_{k\geq n}f_k$ and $h_n\defn\sup_{k\geq n}f_k$ are $\Fcal$-$\borel{\Rbar}$ measurable, whence $g=\sup_{n\geq1} g_n$ and $h=\inf_{n\geq1} h_n$ are measurable by the same theorem 4-3.

By theorem 16 $g\brac{\omega}\leq h\brac{\omega}$ for all $\omega\in \Omega$, and $\obj{g=h}\in\Fcal$ by theorem 4-4. By theorem 16 the set \[\Omega'\defn \obj{\induc{\omega\in \Omega}\,\lim_{n\to\infty} f_n\brac{\omega}\,\text{exists in}\,\Real}\] is exactly the same as the set $\obj{g=h}$, whence $\Omega'\in \Fcal$.

If $\Omega=\Omega'$ then the map $f\defn \lim_{n\to\infty}f_n$ is well-defined everywhere, and is $\Fcal$-$\borel{\Rbar}$ measurable, since in this case $f=g=h$ by theorem 16. 

If it so happens that $\Omega\neq\Omega'$ then the map $f:\brac{\Omega', \induc{\Fcal}_{\Omega'}}\to\brac{\Rbar,\borel{\Rbar}}$ defined as $f\defn\lim_{n\to\infty} f_n = \induc{g}_{\Omega'}$ is measurable. Another strategy is to re-define $f_n'\defn f_n 1_{\Omega'}$ and follow the same argument.\\

\label{thm:meas_compo} \noindent \textbf{Theorem} 4-6.
Let $\brac{\Omega,\Fcal}$, $\brac{G,\mathcal{G}}$ and $\brac{\Scal,\Sigma}$ are three measurable spaces. Let $f:\Omega\to G$ and $g:G\to \Scal$ be two maps. If $f$ is $\Fcal$-$\mathcal{G}$ measurable and $g$ is $\mathcal{G}$-$\Sigma$ measurable, then $h\defn g\circ f$ is $\Fcal$-$\Sigma$ measurable.

Indeed, if $B\in \Sigma$ then $A\defn g^{-1}\brac{B}\in \mathcal{G}$, whence $f^{-1}\brac{A}\in \Fcal$. Since $h^{-1}=f^{-1}\circ g^{-1}$, it must therefore be true that $h^{-1}\brac{B}\in \Fcal$, implying that $h$ is $\Fcal$-$\Sigma$ measurable.\\

\label{thm:sum_prod_meas} \noindent \textbf{Theorem} 4-7.
Let $\brac{\Omega, \Fcal}$ be a measurable space and $f,g:\brac{\Omega, \Fcal}\to\brac{\Rbar,\borel{\Rbar}}$ be two measurable maps. Then well-defined arithmetic operations produce measurable maps.

Let $\phi:\brac{\Rbar,\Tcal_\Rbar}\to\brac{\Rbar,\Tcal_\Rbar}$ be a continuous map. Since by theorem 14 $\phi:\brac{\Rbar,\borel{\Rbar}}\to\brac{\Rbar,\borel{\Rbar}}$ is measurable, theorem 4-6 implies that $\phi\circ f$ is $\Fcal$-$\borel{\Rbar}$ measurable. Thus $-f, \abs{f}$ as well as $f^+\defn \max\obj{f,0}$ and $f^-\defn \max\obj{-f,0}$ are $\Fcal$-$\borel{\Rbar}$ measurable.

If $a\in \Real$ the nthe sum $a+f$ is well defined every, which implies that $\obj{a+f\leq c} = \obj{f\leq c-a}\in \Fcal$ for all $c\in \Real$. Thus by theorem 14 $\omega\to a+f\brac{\omega}$ is $\Fcal$-$\borel{\Rbar}$ measurable.

If $a=0$ then $a\cdot f = 0$ is measurable. If $a\in\Real$, but $a>0$ then $\obj{a \cdot f < c}=\obj{f < \frac{c}{a}}\in \Fcal$ for all $c\in \Real$, implying that $a\cdot f$ is measurable. Finally, if $a=+\infty$, then $\brac{a\cdot f}\brac{\omega} = 0$ when $\omega\in\obj{f=0}$, and $\brac{a\cdot f}\brac{\omega} = \pm\infty$ when $\omega\in\obj{f\neq0}$. Thus \begin{align*}\obj{a\cdot f < c}&=\obj{a\cdot f < c}\cap \obj{f=0}\uplus \obj{a\cdot f < c}\cap\obj{f>0}\uplus \obj{a\cdot f < c}\cap\obj{f<0}\\&=\obj{0<c}\cap \obj{f=0}\uplus \obj{f<0}\in \Fcal\end{align*} Therefore the map $a\cdot f$ is $\Fcal$-$\borel{\Rbar}$ measurable for all $a\in \Zinf$. Now, for any $a\in \Rbar$ with $a\leq 0$, then $\brac{-a}\cdot f$ is $\Fcal$-$\borel{\Rbar}$ measurable, whence $a\cdot f = -\brac{\brac{-a}\cdot f}$ is $\Fcal$-$\borel{\Rbar}$ measurable.

If $f,g\geq 0$, then the sum $f+g$ is well defined everywhere on $\Omega$ and $\obj{f+g<c}=\obj{f<c-g}$ for all $c\in\Real$. Since for any $c\in\Real$ the map $c-g$ is well-defined and $\Fcal$-$\borel{\Rbar}$ measurable, $\obj{f\leq c-g}\in\Fcal$ for all $c\in\Real$, whence $f+g$ is $\Fcal$-$\borel{\Rbar}$ measurable.

For $F_-\defn \obj{f=-\infty}$, $F_+\defn \obj{f=+\infty}$ and $F\defn \obj{f\in \Real}$, $\Omega=F_-\uplus F\uplus F_+$ and $F_-,F,F_+\in\Fcal$. Similarly $\Omega=G_-\uplus G\uplus G_+$ with $G_-,G,G_+\in\Fcal$. Suppose that $f$ and $g$ are such that \[F_+\cap G_-\cup F_-\cap G_+=\emptyset\] then the sum $f+g$ is a well-defined number everywhere on $\Omega$. For any $c\in\Real$ by the above the maps $c-g$ and $c-f$ are well-defined and $\Fcal$-$\borel{\Rbar}$ measurable, whence $\obj{f\leq c-g},\obj{g\leq c-f}\in\Fcal$ for all $c\in\Real$. However, for all $c\in\Real$ \[\obj{f+g\leq c} = \obj{g\leq c-f}\cap F \cup \obj{f\leq c-g}\cap G \cup \brac{G_- \cap F_-}\] which implies that $\obj{f+g\leq c}\in\Fcal$ for all $c\in\Real$. Therefore $f+g$ is $\Fcal$-$\borel{\Rbar}$ measurable.

Suppose $f\brac{\Omega}\subseteq \Real\setminus \obj{0}$. Then for any $c\in\Real$ \[\obj{\frac{1}{f}\leq c} = \obj{c f \geq 1}\cap \obj{f>0} \cup \obj{c f \leq 1}\cap \obj{f<0}\] which implies that $\frac{1}{f}$ is $\Fcal$-$\borel{\Rbar}$ measurable.

Suppose $f\brac{\Omega}\subseteq\Real$ and define $\bar{f}\brac{\omega}\defn f\brac{\omega}$ if $f\brac{\omega}\neq0$ and $\bar{f}\brac{\omega}\defn 1$ otherwise. Then $\obj{\bar{f}\in B}=\obj{f\in B}\cap\obj{f\neq0}\uplus\obj{1\in B}\cap\obj{f=0}$, whence $\obj{\bar{f}\in B}\in\Fcal$ for all $B\in\borel{\Rbar}$. Now \begin{align*}\obj{fg<c} &= \obj{fg<c} \cap \obj{f<0} \uplus \obj{fg<c} \cap \obj{f=0} \uplus \obj{fg<c} \cap \obj{f>0}\\ &= \obj{g>\frac{c}{f}} \cap \obj{f<0} \uplus \obj{0<c} \cap \obj{f=0} \uplus \obj{g<\frac{c}{f}} \cap \obj{f>0}\\ &= \obj{g>\frac{c}{\bar{f}}} \cap \obj{f<0} \uplus \obj{f<c} \cap \obj{f=0} \uplus \obj{g<\frac{c}{\bar{f}}} \cap \obj{f>0} \end{align*} Since $\bar{f}\brac{\Omega}\subseteq \Real\setminus\obj{0}$, by the above $\frac{c}{f}$ is $\Fcal$-$\borel{\Rbar}$ measurable for any $c\in \Real$, whence $\obj{fg < c}\in\Fcal$ for all $c\in\Real$ as $g$ and $f$ are $\Fcal$-$\borel{\Rbar}$ measurable. Thus $f g:\brac{\Omega, \Fcal}\to\brac{\Rbar,\borel{\Rbar}}$ is measurable.

Now suppose that $f\brac{\Omega},g\brac{\Omega}\subseteq \Rbar$ and put $\bar{f}\brac{\omega}\defn f\brac{\omega}$ if $f\brac{\omega}\neq \pm\infty$ and $\bar{f}\brac{\omega}\defn 1$ otherwise. Definition $\bar{g}$ analogously. Since \begin{align*}\obj{\bar{f}\in B}&=\obj{\bar{f}\in B}\cap \obj{f\neq\pm\infty} \uplus \obj{\bar{f}\in B}\cap \obj{f=\pm\infty}\\&=\obj{f\in B}\cap \obj{f\neq\pm\infty} \uplus \obj{1\in B}\cap \obj{f=\pm\infty}\end{align*} is in $\Fcal$ for all $B\in\borel{\Rbar}$, the maps $\bar{f}$ and $\bar{g}$ are $\Fcal$-$\borel{\Rbar}$ measurable.

Next, define $F_1\defn \obj{f=-\infty}$, $F_2\defn \obj{f\in\brac{-\infty,0}}$, $F_3\defn \obj{f=0}$, $F_4\defn \obj{f\in\brac{0,+\infty}}$ and put $F_5\defn \obj{f=+\infty}$ and similarly for $g$. Since $\Omega=\uplus_{i=1}^5 F_i = \uplus_{j=1}^5 G_j$, for all $B\in\borel{\Rbar}$ it is true that \[\obj{fg\in B} = \biguplus_{i,j=1}^5 \obj{fg\in B} \cap \brac{F_i\cap G_j} = \biguplus_{i,j=1}^5 A_{ij} \cap \brac{F_i\cap G_j}\] By definition of $F_i$ and $G_j$ each set $A_{ij}$ in the partition is one of the following:
\begin{table*}[htb]
	\centering \begin{minipage}{0.9\textwidth}
	\centering \begin{tabular}{|c|ccccc|}
		\hline
		{i,j}& \multicolumn{1}{c|}{1} & \multicolumn{1}{c|}{2} & \multicolumn{1}{c|}{3} & \multicolumn{1}{c|}{4} & \multicolumn{1}{c|}{5}\\
		\hline
		{1} & {$\obj{+\infty\in B}$} & {$\obj{+\infty\in B}$} & {$\obj{0\in B}$} & {$\obj{-\infty\in B}$} & {$\obj{-\infty\in B}$} \\ \cline{1-1}
		{2} & {$\obj{+\infty\in B}$} & {$\obj{\bar{f}\bar{g}\in B}$} & {$\obj{\bar{f}\bar{g}\in B}$} & {$\obj{\bar{f}\bar{g}\in B}$} & {$\obj{-\infty\in B}$} \\ \cline{1-1}
		{3} & {$\obj{0\in B}$} & {$\obj{\bar{f}\bar{g}\in B}$} & {$\obj{\bar{f}\bar{g}\in B}$} & {$\obj{\bar{f}\bar{g}\in B}$} & {$\obj{0\in B}$} \\ \cline{1-1}
		{4} & {$\obj{-\infty\in B}$} & {$\obj{\bar{f}\bar{g}\in B}$} & {$\obj{\bar{f}\bar{g}\in B}$} & {$\obj{\bar{f}\bar{g}\in B}$} & {$\obj{+\infty\in B}$} \\ \cline{1-1}
		{5} & {$\obj{-\infty\in B}$} & {$\obj{-\infty\in B}$} & {$\obj{0\in B}$} & {$\obj{+\infty\in B}$} & {$\obj{+\infty\in B}$} \\ \hline
	\end{tabular}
	\caption{Sets $A_{ij}$ intersected with $F_i\cap G_j$ in the partition of $\obj{fg\in B}$.}
	\end{minipage}
\end{table*}
Since $\obj{c\in B}$ is either $\emptyset$ if $c\notin B$ or $\Omega$ when $c\in B$ for any $c\in\Rbar$ and $\bar{f}$ and $\bar{g}$ are $\Fcal$-$\borel{\Rbar}$ measurable, $\obj{fg\in B} \cap \brac{F_i\cap G_j}\in \Fcal$ for all $i,j=1\ldots 5$. Therefore $\obj{fg\in B}\in \Fcal$ for all $B\in\borel{\Rbar}$ and the map $fg:\brac{\Omega, \Fcal}\to\brac{\Rbar, \borel{\Rbar}}$ is measurable.\\

\label{thm:set_dist} \noindent \textbf{Theorem} 4-8.
Let $\brac{E,d}$ be a metric space and $A\subseteq E$. The map $\Phi_A:E\to \Zinf$ defined for any $x\in E$ as \[\Phi_A\brac{x}\defn d\brac{x,A}\defn \inf\obj{\induc{d\brac{x,y}}\,y\in A}\], where $\inf \emptyset = +\infty$, is $\Tcal_E^d$-$\Tcal_\Rbar$ continuous. Also if $A$ is closed in $\brac{E, d}$, then if and only if $A=\Phi_A^{-1}\brac{\obj{0}}$.

Indeed, since $A\subseteq \clo{A}$, $d\brac{x,\clo{A}}\leq d\brac{x,A}$ even if $A=\emptyset$. On the other hand if $d\brac{x,\clo{A}} < d\brac{x,A}$ for some $x\in E$, then there exists $z\in\clo{A}$ with $d\brac{x,z}<d\brac{x,A}$. For such $z$, since $\clo{A}$ is the closure of $A$, there is $y\in A$ with $d\brac{z,y}<d\brac{x,A}-d\brac{x,z}$, whence by the triangle inequality \[d\brac{x,A}\leq d\brac{x,y} < d\brac{x,z}+d\brac{z,y} < d\brac{x, A}\] Therefore, it has to be true that $d\brac{x,\clo{A}} \geq d\brac{x,A}$ for all $x\in E$.

Next, if $x\in \clo{A}$, then for every $\epsilon>0$ there exists $y\in A$ with $d\brac{x,y}<\epsilon$, whence $d\brac{x,A}<\epsilon$. Thus $d\brac{x,A}=0$. Conversely, if $d\brac{x,A}=0$, then for any $\epsilon>0$ there is $y\in A$ with $d\brac{x,y}<\epsilon$. Since for every $U\subseteq E$ open in $\brac{E,d}$ with $x\in U$ there is $\eta>0$ such that $B^d_E\brac{x,\eta}\subseteq U$, by the above there is $y\in A$ with $y\in B^d_E\brac{x,\eta}$, whence $U\cap A\neq \emptyset$. Therefore $x\in \clo{A}$.

Suppose $A\neq \emptyset$. Since $d\brac{x,A}\leq d\brac{x,z}<+\infty$ for all $z\in A$, $d\brac{x,A}\in \Real^+$. Furthermore, for every $x,y\in E$ and $z\in A$ the triangle inequality implies \[d\brac{x, A}\leq d\brac{x,z}\leq d\brac{x,y}+d\brac{y,z}\] whence because $d\brac{y,A}$ is the greatest lower bound and $d\brac{x,y}\in \Real^+$, $d\brac{x,A}-d\brac{x,y}\leq d\brac{y,A}$. Therefore $d\brac{x,A}\leq d\brac{y,A}+d\brac{x,y}$ and, similarly, $d\brac{y,A}\leq d\brac{x,A}+d\brac{x,y}$ for all $x,y\in E$. Hence for all $x,y\in E$ \[\abs{ d\brac{y,A}-d\brac{x,A}} \leq d\brac{x,y}\]

So the $\epsilon$-$\delta$ definition of continuity (in metric spaces) implies that for every $A\neq \emptyset$ the map $d\brac{\cdot,A}:E\to\Real^+$ is $\Tcal_E^d$-$\Tcal_{\Real^+}$ continuous. By theorem Sup-A-2 thus $d\brac{\cdot,A}:\brac{E, \Tcal_E^d}\to\brac{\Rbar, \Tcal_\Rbar}$ is continuous if $A\neq \emptyset$.

For $A=\emptyset$, the map $d\brac{x,A}=+\infty$ on $E$, which implies that $d\brac{\cdot, \emptyset}:\brac{E, \Tcal_E^d}\to\brac{\Rbar, \Tcal_\Rbar}$ is continuous.

Therefore, for any $A\subseteq E$ the map $\Phi_A$ is $\Tcal_E^d$-$\Tcal_{\Real^+}$ continuous, whence by theorem Sup-A-1 $\clo{A}=\Phi_A^{-1}\brac{\obj{0}}$ for any $A\subseteq E$, since $\obj{0}$ is closed in $\Rbar$. Hence if $A$ is closed in $\brac{E, d}$, then $\Phi_A^{-1}\brac{\obj{0}} = \clo{A} = A$ and vice versa.\\

\label{thm:metric_lim_meas} \noindent \textbf{Theorem} 17.
Let $\brac{\Omega, \Fcal}$ be a measurable space, and $\brac{f_n}_{n\geq 1}:\brac{\Omega, \Fcal}\to\brac{E, \borel{E}}$ be a sequence of measurable maps, where $\brac{E, d}$ is a metric space. Then if $f_n\to f$ on $\Omega$ in $\brac{E, d}$ for some $f:\Omega\to E$, then $f:\brac{\Omega, \Fcal}\to\brac{E, \borel{E}}$ is itself measurable.

This theorem is proven, by noting that for any $A\subseteq E$ the map $\brac{\Phi_A\circ f_n}:\brac{E, \Tcal_E^d}\to\brac{\Rbar, \borel{\Rbar}}$ is measurable and converges to $\Phi_A\circ f$ in $\Rbar$, because

Indeed, since by theorem 4-8 $\Phi_A:\brac{E, \Tcal_E^d}\to\brac{\Rbar, \Tcal_\Rbar}$ is continuous for any $A\subseteq E$, the map $\Phi_A:\brac{E, \borel{E}}\to\brac{\Rbar, \borel{\Rbar}}$ is measurable, whence by theorem 4-6 $\brac{\Phi_A\circ f_n}:\brac{E, \Tcal_E^d}\to\brac{\Rbar, \borel{\Rbar}}$ is measurable. Furthermore, continuous also implies that $\Phi_A\circ f_n\to \Phi_A\circ f$ in $\Rbar$ everywhere on $\Omega$.

By theorem 16, $\Phi_A\circ f$ is $\Fcal$-$\borel{\Rbar}$ measurable itself, whence by theorem 4-8 for any $A$ closed in $\brac{E,d}$ \[f^{-1}\brac{A} = f^{-1}\brac{\clo{A}} = \brac{\Phi_A\circ f}^{-1}\brac{\obj{0}} \in \Fcal\] because $\obj{0}\in\borel{\Rbar}$.

Now, since the pre-image by $f$ honours set operations, $f:\brac{\Omega, \Fcal}\to\brac{\Rbar, \borel{\Rbar}}$ is measurable, because \[\borel{E} = \sigma\brac{\Tcal_E^d} = \sigma\brac{\obj{ \induc{ C\subseteq E }\, C\,\text{-- closed in}\,\brac{E,d}}}\]

\noindent \textbf{Definition} 38.
The usual topology $\Tcal_\Cplx$ on the the set of complex numbers is defined as the metric topology associated with the complex modulus metric $d\brac{z, z'} = \abs{z-z'}$: $\Tcal_\Cplx = \Tcal_\Cplx^d$. Whenever it is said that some set $U$ is open in $\Cplx$, it is meant that it is an element of the usual topology on $\Cplx$: $U\in \Tcal_\Cplx$.

\label{thm:cplx_real_topology} \noindent \textbf{Theorem} 4-?9.
It is true that $\Tcal_\Real=\induc{\Tcal_\Cplx}_\Real$ whence $\borel{\Real} = \induc{\borel{\Cplx}}_\Real$.

Indeed, if $U\in \Tcal_\Real$, then for any $x\in U$ there is $\epsilon_x>o$ such that $B_\Real\brac{x, \epsilon_x}\subseteq U$. Let \[V\defn \bigcup_{x\in U} B_\Cplx\brac{x, \epsilon_x}\] Being a union of open sets, $V$ is itself open in $\Cplx$ with respect to the usual topology. Since $B_\Cplx\brac{x, \delta}\cap \Real = B_\Real\brac{x, \delta}$ for all $x\in \Cplx$ and $\delta\geq 0$, it is true that $U = V\cap \Real$, whence $U\in\induc{\Tcal_\Cplx}_\Real$. Conversely, if $U\in\induc{\Tcal_\Cplx}_\Real$, then there is $V\in\Tcal_\Cplx$ with $U = V\cap \Real$. For any $x\in U$ there is $\delta>0$ with $B_\Cplx\brac{x, \delta}\subseteq V$, whence \[B_\Real\brac{x, \delta} = B_\Cplx\brac{x, \delta}\cap \Real\subseteq V\cap \Real = U\]

Now, if $z\notin \Real$, then $\im z\neq 0$, whence for $\delta\defn \abs{\im z}>0$. Let $z'\in \Cplx$ be such that $\abs{z-z'}<\delta$. Since \[\abs{ \abs{\im z} - \abs{\im z'} } \leq \abs{\im z-\im z'} = \abs{\im\brac{z-z'}} \leq \abs{z-z'}\] it must be true that $\abs{\abs{\im z'}-\abs{\im z}}<\delta$. Therefore $\abs{\im z'}>\abs{\im z'}-\delta>0$, whence $z'\notin \Real$. Thus $B_\Cplx\brac{z, \delta}\cap \Real = \emptyset$, which implies that $z\notin \clo{\Real}$, where the closure is performed in $\brac{\Cplx, \Tcal_\Cplx}$. Hence $\clo{\Real}=\Real$, implying that $\Real$ is closed in $\Cplx$.

Since $\Tcal_\Real=\induc{\Tcal_\Cplx}_\Real$, theorem 10 implies that $\borel{\Real} = \induc{\borel{\Cplx}}_\Real$. In addition, the fact that $\Real$ is closed in $\Cplx$ means that $\Real\in\borel{\Cplx}$, whence $\borel{\Real}\subseteq \borel{\Cplx}$.\\

\label{thm:cplx_func_meas} \noindent \textbf{Theorem} 4-9.
Let $\brac{\Omega, \Fcal}$ be a measurable space and $f:\brac{\Omega, \Fcal}\to\brac{\Cplx,\borel{\Cplx}}$ be a measurable map. Then $\re f, \im f$ and $\abs{f}$ are $\Fcal$-$\borel{\Real}$ measurable maps.

With respect to this topology, the real $\re f$ and imaginary $\im f$ parts and the complex modulus $\abs{f}$ are continuous. Indeed, $\abs{\re \cdot}, \abs{\im \cdot}\leq \abs{\cdot}$ and $\abs{\abs{z}-\abs{z'}}\leq \abs{z-z'}$, whence these maps are $\Tcal_\Cplx$-$\Tcal_\Real$ continuous. Therefore they are measurable with respect to $\borel{\Cplx}$ and $\borel{\Real}$.\\

\label{thm:cplx_sigma_algebra} \noindent \textbf{Theorem} 4-10.
Let $\Ccal \defn \obj{ \induc{ \crab{a,b}\times\crab{c,d}\, } a,b,c,d \in \Real}$ be a collection of complex ``boxes'', where it is understood that \[\crab{a,b}\times\crab{c,d} \defn \obj{ \induc{ z = x+iy\in \Cplx\, } \brac{x,y} \in \brac{a,b}\times\brac{c,d} \subseteq \Real\times\Real }\] Then $\sigma\brac{\Ccal} = \borel{\Cplx}$.

First, every $U\in \Ccal$ is open in $\Cplx$, because for any $z\in U$ it is possible to fit a disc centred at $z$ that is entirely contained within $U$. Indeed, $z\in U$ implies that $\re z\in\brac{a,b}$ and $\im z\in\brac{c,d}$, whence for any \[0<\delta < \frac{1}{2}\min\obj{\re z-a, b-\re z, \im z - c, d-\im z}\] it is true that $z'\in U$ for any $z'\in \Cplx$ with $\abs{z-z'}<\delta$. Thus $\Ccal\in \Tcal_\Cplx$, whence $\sigma\brac{\Ccal}\subseteq \borel{\Cplx}$.

Conversely, for every $U$ open in $\Cplx$, it is true that for any $z\in U$ there is $\epsilon>0$ such that $z'\in U$ for all $z'\in\Cplx$ with $\abs{z-z'}<\epsilon$. For $\eta\defn \frac{\epsilon}{2}>0$ the above result implies that for any $z'\in\Cplx$ with $\abs{\re z-\re z'}<\eta$ and $\abs{\im z-\im z'}<\eta$ it is true that $\abs{z-z'}<\sqrt{2}\eta<\epsilon$, whence $z'\in U$. Hence there exists $V_z\in \Ccal$ such that $z\in V_z$ and $V_z\subseteq U$.

Since $\mathbb{Q}$ is countably dense in $\Real$, there are $a,b,c,d\in \mathbb{Q}$ such that $z\in \crab{a,b}\times\crab{c,d} \subseteq V_z$. Therefore \[\Gamma\defn \obj{\induc{\crab{a,b}\times\crab{c,d}}\,\crab{a,b}\times\crab{c,d}\subseteq U,\,a,b,c,d\in \mathbb{Q}}\] Thus by the above for any $z\in U$ there is $A\in \Gamma$ such that $z\in A$, whence $U=\bigcup_{A\in\Gamma}A$. Since $\mathbb{Q}^4$ is countable, it is true that $\Gamma$ is countable as well. Furthermore $\Gamma\subseteq \Ccal$, which implies that for any $U\in \Tcal_\Cplx$ there is $\brac{A_n}_{n\geq1}\in \Ccal$ such that $U=\bigcup_{n\geq1} A_n$. Therefore $\Tcal_\Cplx\subseteq \sigma\brac{\Ccal}$ and $\borel{\Cplx}\subseteq\sigma\brac{\Ccal}$.\\

\label{thm:cplx_func_part_meas} \noindent \textbf{Theorem} 4-11.
Let $\brac{\Omega, \Fcal}$ be a measurable space and $u,v:\brac{\Omega, \Fcal}\to\brac{\Real,\borel{\Real}}$ be two measurable maps. Then $f\defn u+iv$ is a $\Fcal$-$\borel{\Cplx}$ measurable map.

Indeed, for every complex ``box'' $\crab{a,b}\times\crab{c,d}\subseteq \Cplx$ with $a,b,c,d \in \Real$ \[f^{-1}\brac{\crab{a,b}\times\crab{c,d}} = \obj{\re f\in \brac{a,b}}\cap \obj{\im \in \brac{c,d}} = \obj{u\in \brac{a,b}}\cap \obj{v \in \brac{c,d}} \in \Fcal\], whence $f:\brac{\Omega, \Fcal}\to\brac{\Cplx, \borel{\Cplx}}$ is a measurable map.\\

% section tut_4 (end)

\section{Lebesgue Integration} % (fold)
\label{sec:tut_5}
\url{http://probability.net/PRTintegral.pdf}

\noindent \textbf{Definition} 39.
Let $A\subseteq \Omega$, defined the map $1_A\brac{\omega}\defn 1$ if $
\omega\in A$ and $!_A\brac{\omega}\defn 0$ otherwise.

\label{thm:indicator} \noindent \textbf{Theorem} 5-1.
Let $\brac{\Omega, \Fcal}$ be a measurable space and $A\subseteq \Omega$. Then $1_A$ is $\Fcal$-$\borel{\Rbar}$ measurable if and only if $A\in \Fcal$.

Indeed, if $1_A$ is measuralbe then $\obj{1_A\leq c}\in \Fcal$ for all $c\in \Real$. Hence $A^c=\obj{1_A\leq 0}\in \Fcal$, whence $A\in \Fcal$. Conversely, $\obj{1_A \leq c}=\emptyset$ if $c<0$, $\obj{1_A < c}=A^c$ if $c\in \clop{0,1}$,  and $\obj{1_A\leq c}=\Omega$ if $c\geq 1$. Since $A\in \Fcal$, $A^c\in \Fcal$ whence $\obj{1_A\leq c}\in\Fcal$ for all $c\in \Real$. Therefore by theorem 15 $1_A$ is $\Fcal$-$\borel{\Rbar}$ measurable.\\

\noindent \textbf{Definition} 40.
Let $\brac{\Omega, \Fcal}$ be a measurable space. A map $s:\Omega\to\Real^+$ is said to be simple if it is of the from: \[s = \sum_{k=1}^n \alpha_k 1_{A_k}\] where $n\geq 1$, $\brac{\alpha_k}_{k=1}^n \in \Real^+$ and $\brac{A_k}_{k=1}^n \in \Fcal$.

\label{thm:simple_meas} \noindent \textbf{Theorem} 5-2.
Let $\brac{\Omega, \Fcal}$ be a measurable space and $s$ be a simple function on $\brac{\Omega, \Fcal}$. Then $s$ is $\Fcal$-$\borel{\Real^+}$ measurable.

Indeed, there is $n\geq1$, $\brac{A_k}_{k=1}^n\in \Fcal$ and $\brac{\alpha_k}_{k=1}^n\in \Real^+$ such that $s=\sum_{k=1}^n \alpha_k 1_{A_k}$. By theorem 4-7 $h_k\defn \alpha_k 1_{A_k}$ is $\Fcal$-$\borel{\Rbar}$ measurable for each $k=1\ldots n$. Since $h_k\brac{\Omega}\subseteq \Real^+$, by inductive application of theorem 4-7 their sum $s=\sum_{k=1}^n h_k$ id $\Fcal$-$\borel{\Rbar}$ measurable. Furthermore $s\brac{\Omega}\subseteq \Real^+$, whence $s$ is $\Fcal$-$\borel{\Real^+}$ measurable.\\

\label{thm:simple_partition} \noindent \textbf{Theorem} 5-3.
Let $\brac{\Omega, \Fcal}$ be a measurable space and $s$ be a simple function on $\brac{\Omega, \Fcal}$. Then there exists $n\geq 1$, $\brac{\alpha_k}_{k=1}^n\in\Real^+$ and $\brac{A_k}_{k=1}^n\in \Fcal$ with $\Omega=\biguplus_{k=1}^n A_k$ such that $s=\sum_{k=1}^n \alpha_k 1_{A_k}$. Such representation of a simple function $s$ is called a partition.

Indeed, let $S\defn s\brac{\Omega}$ be the range of $s$ (the image of the entire domain by $s$). Therefore for any $y\in S$ there is at least one $\omega\in \Omega$ such that $s\brac{\omega}=y$ ($\omega\in s^{-1}\brac{\obj{y}}$). For every $y\in S$ pick just one fixed $\omega_y\in \Omega$ with $s\brac{\omega_y}=y$. 

Define a map $\phi:\Omega\to\obj{0,1}^n$ by $\phi\brac{\omega}\defn \brac{1_{A_k}\brac{\omega}}_{k=1}^n$. Then put $\psi\brac{y}\defn \phi\brac{\omega_y}$ for any $y\in S$.

First the map is well-defined, since the correspondence $y\to\omega_y$ is fixed. Next, if $\psi\brac{y_1}=\psi\brac{y_2}$, then $\phi\brac{\omega_1}=\phi\brac{\omega_2}$ where $\omega_i\defn \omega_{y_i}$, whence $1_{A_k}\brac{\omega_1}=1_{A_k}\brac{\omega_2}$ for all $k=1\ldots n$. Thus \[y_1=s\brac{\omega_1} = \sum_{k=1}^n \alpha_k 1_{A_k}\brac{\omega_1} = \sum_{k=1}^n \alpha_k 1_{A_k}\brac{\omega_2} = s\brac{\omega_2} = y_2\] which implies that $\psi:S\to\obj{0,1}^n$ is injective. Therefore $\abs{S}\leq \abs{\obj{0,1}^n}$ which implies that $S$ is finite.

Consider $\brac{F_\alpha}_{\alpha\in S}$ defined as $F_\alpha\defn \obj{s=\alpha}$. First, if $F_{\alpha_1}\cap F_{\alpha_2}\neq \emptyset$ then there is $\omega\in \Omega$ such that $s\brac{\omega}=\alpha_1$ and $s\brac{\omega}=\alpha_2$ whence $\alpha_1=\alpha_2$. Hence $\brac{F_\alpha}_{\alpha\in S}$ are pairwise disjoint. Furthermore $\alpha\defn s\brac{\omega}\in S$ for any $\omega\in\Omega$, whence $\omega\in F_\alpha$. Therefore $\Omega\subseteq \biguplus_{\alpha\in S} F_\alpha$. Finally, $F_\alpha\in \Fcal$ since $s$ is measurable by theorem 5-2.

Put $t\defn \sum_{\alpha\in S} \alpha 1_{F_\alpha}$. Then for any $\omega\in \Omega$ there exists a unique $\alpha_\omega\in S$ with $\omega\in F_{\alpha_\omega}$, which implies that \[t\brac{\omega} = \sum_{\alpha\in S} \alpha 1_{F_\alpha} = \alpha_\omega 1_{F_{\alpha_\omega}} = s\brac{\omega}\] since $F_{\alpha_\omega}=s^{-1}\brac{\obj{\alpha_\omega}}$.

Thus since $S$ is finite the above implies that there exist $N\geq1$, $\brac{F_k}_{k=1}^N\in \Fcal$ and $\brac{\alpha_k}_{k=1}^N$ with $\Omega=\biguplus_{k=1}^n A_k$ such that $s=\sum_{k=1}^N \alpha_k 1_{F_k}$.\\


\noindent \textbf{Definition} 42.
Let $\brac{\Omega, \Fcal, \mu}$ be a measure space and $s$ be a simple function on $\brac{\Omega, \Fcal}$. The integral of $s$ with respect to $\mu$ is defined as \[I^\mu\brac{s} \defn \sum_{i=1}^n \alpha_i \mu\brac{A_i}\] where $s = \sum_{i=1}^n \alpha_i 1_{A_i}$ is any partition of $s$.

\label{thm:simple_int} \noindent \textbf{Theorem} 5-4.
Let $\brac{\Omega, \Fcal, \mu}$ be a measure space. Then the integral sum of any simple function is a well defined linear functional. Also for any simple functions $s,t$ on $\brac{\Omega, \Fcal}$, if $s\leq t$ then $I^\mu\brac{s}\leq I^\mu\brac{t}$.

The fact that $I^\mu$ is well-defined stems from the fact that such sum is invariant under different partitions of $s$, which can be shown, by forming a finer partition of $s$ from the two partitions and showing that the value of the integral sum over this new partition coincides with both original integral sums for any finitely additive map $\mu:\Omega\to\Zinf$.

Similarly, by considering a finer partition, formed from the partitions of some simple functions $s$ and $t$, it is possible to show that $I^\mu$ is a linear functional, such that if $s\leq t$, then $I^\mu\brac{s}\leq I^\mu\brac{t}$.

Indeed, let $s$ be a simple function on $\brac{\Omega, \Fcal}$ and $n,m\geq1$, $\brac{A_k}_{k=1}^n, \brac{B_k}_{k=1}^m\in \Fcal$ and $\brac{\alpha_k}_{k=1}^n$, $\brac{\beta_k}_{k=1}^m\in \Real^+$ be two partitions of $s$.

Since a product of finitely many finite sets is itself finite, $A_i\cap B_j$ defines a finite collection of measurable sets, which satisfy $A_i\cap B_j\cap A_k\cap B_l=\emptyset$ whenever $i\neq k$ or $j\neq l$ and are such that $\Omega = \biguplus_{i=1,j=1}^{n,m} A_i\cap B_j$.

Now, for any $\omega\in\Omega$ there exists $i=1\ldots n$ such that $\omega\in A_i$, whence $s\brac{\omega}=\alpha_i$. Furthermore there is $j=1\ldots m$ with $\omega\in B_j$, which implies $\omega\in A_i\cap B_j$. Thus $\sum_{k,l} \alpha_k 1_{A_k\cap B_l}\brac{\omega}=\alpha_i$, meaning that $s=\sum_{i,j} \alpha_i 1_{A_i\cap B_j}$ is a partition of $s$. Similarly, $s=\sum_{i,j} \beta_j 1_{A_i\cap B_j}$ is a partition of $s$.

For any $i=1\ldots n$ and $j=1\ldots m$, $A_i\cap B_j = \emptyset$ implies $\mu\brac{A_i\cap B_j} = 0$, whence $\alpha_i\mu\brac{A_i\cap B_j}=\beta_j\mu\brac{A_i\cap B_j}$. If $A_i\cap B_j\neq \emptyset$ then $s\brac{\omega}=\alpha_i$ and simultaneously $s\brac{\omega}=\beta_j$, whence $\alpha_i\mu\brac{A_i\cap B_j}=\beta_j\mu\brac{A_i\cap B_j}$ again. Therefore, since $\mu$ is a measure on $\brac{\Omega, \Fcal}$, $\mu\brac{A_i}=\sum_{j=1}^m \mu\brac{A_i\cap B_j}$ and $\mu\brac{B_j}=\sum_{i=1}^n \mu\brac{A_i\cap B_j}$ for all $i,j$, whence \[\sum_{i=1}^n \alpha_i \mu\brac{A_i} = \sum_{i=1}^n \sum_{j=1}^m \alpha_i \mu\brac{A_i\cap B_j} = \sum_{j=1}^m \sum_{i=1}^n \beta_j \mu\brac{A_i\cap B_j} = \sum_{j=1}^m\beta_j \mu\brac{B_j}\] Thus $I^\mu\brac{s}$ is invariant under different partitions of $s$.

Let $s$ an $t$ be two simple functions on $\brac{\Omega, \Fcal}$ with partitions $s=\sum_{i=1}^n \alpha_i 1_{A_i}$ and $t=\sum_{j=1}^m \beta_j 1_{B_j}$. Then $\sum_{i=1}^n \sum_{j=1}^m \alpha_i 1_{A_i\cap B_j}$ and $\sum_{j=1}^m \sum_{i=1}^n \beta_j 1_{A_i\cap B_j}$ are another partitions of $s$ and $t$ respectively. Then $\sum_{i,j} \brac{\alpha_i+\beta_j} 1_{A_i\cap B_j}$ is a partition of $s+t$, whence \begin{align*}I^\mu\brac{s+t} &= \sum_{i,j} \brac{\alpha_i+\beta_j} \mu\brac{A_i\cap B_j} = \sum_{i=1}^n \sum_{j=1}^m \alpha_i \mu\brac{A_i\cap B_j} + \sum_{j=1}^m \sum_{i=1}^n \beta_j \mu\brac{A_i\cap B_j} \\&= \sum_{i=1}^n \alpha_i \mu\brac{A_i} + \sum_{j=1}^m \beta_j \mu\brac{B_j} \\&= I^\mu\brac{s} + I^\mu\brac{t}\end{align*}

If $\alpha\in \Real^+$, then $\alpha s = \sum_{i=1}^n \alpha \alpha_i 1_{A_i}$, which is a simple function on $\brac{\Omega, \Fcal}$. Therefore \[I^\mu\brac{\alpha s} = \sum_{i=1}^n \alpha \alpha_i \mu\brac{A_i} = \alpha \sum_{i=1}^n \alpha_i \mu\brac{A_i} = \alpha I^\mu\brac{s}\]  Since the value of $I^\mu\brac{\cdot}$ is invariant under different partitions of the same simple map, this proves that $I^\mu\brac{\cdot}$ is a linear functional.

Finally, let $s=\sum_{i=1}^n \alpha_i 1_{A_i}$ and $t=\sum_{j=1}^m \beta_j 1_{B_j}$ be partitions of $s$ and $t$ respectively. Then $\sum_{i=1}^n \sum_{j=1}^m \alpha_i 1_{A_i\cap B_j}$ and $\sum_{j=1}^m \sum_{i=1}^n \beta_j 1_{A_i\cap B_j}$ are another partitions of $s$ and $t$. Therefore $\alpha_i\leq \beta_j$ whenever $A_i\cap B_j\neq \emptyset$, whence \[I^\mu\brac{s} = \sum_{i=1}^n \sum_{j=1}^m \alpha_i \mu\brac{A_i\cap B_j}\leq \sum_{j=1}^m \sum_{i=1}^n \beta_j \mu\brac{A_i\cap B_j} = I^\mu\brac{t}\]

\label{thm:simp_seq} \noindent \textbf{Theorem} 18.
For any non-negative measurable map $f:\brac{\Omega, \Fcal}\to\Zinf$, there is sequence of simple functions $\brac{s_n}_{n\geq 1}$ on $\brac{\Omega, \Fcal}$ such that $s_n\uparrow f$ in $\Zinf$ everywhere on $\Omega$.

A particular example is the following clever sequence \[s_n \defn \sum_{k=0}^{n 2^n - 1} \frac{k}{2^n} 1_{B_{nk}} + n 1_{\obj{n\leq f}} \] where $B_{nk} \defn \obj{ \sfrac{k}{2^n} \leq f < \sfrac{k+1}{2^n}}$, convergence of which to $f$ stems from the fact that the set $\bigcup_{n\geq 1} \obj{\induc{\frac{k}{2^n}} k = 0\ldots n 2^n}$ is everywhere dense in $\Real$.\\

\noindent \textbf{Definition} 43.
Let $f:\brac{\Omega, \Fcal}\to\Zinf$ be a non-negative and measurable map, where $\brac{\Omega, \Fcal, \mu}$ is a measure space. The (usual) Lebesgue integral of $f$ with respect to $\mu$ is defined as \[\int f d\mu\defn\sup\obj{\induc{I^\mu\brac{s}}\,s\,\text{-- simple on}\,\brac{\Omega,\Fcal},\, s\leq f}\]

\label{thm:usual_lebesgue_properties} \noindent \textbf{Theorem} 5-5.
The usual Lebesgue integral has some very nice properties, since it is the least upper bound on integral sums of simple functions.

If $f$ is a simple function on $\brac{\Omega, \Fcal}$, then by definition $I^\mu\brac{s} \leq I^\mu\brac{f}$ for any simple function $s\leq f$, whence $\int f d\mu \leq I^\mu\brac{f}$. However, $I^\mu\brac{f}\leq \int f d\mu$ since $f$ is itself simple.

Let $f,g:\brac{\Omega,\Fcal}\to \Zinf$ be two measurable non-negative maps. If $g\leq f$, then  $s\leq f$ and $I^\mu\brac{s}\leq \int f d\mu$for any simple function $s\leq g$, whence by definition $\int g d\mu\leq \int f d\mu$.

If $\mu\brac{\obj{f>0}}=0$, then for any simple function $s:\brac{\Omega, \Fcal}\to \Real^+$ with $s=\sum_{i=1}^n \alpha_i 1_{A_i} \leq f$ it is true that $\alpha_i \mu\brac{A_i} = 0$ for all $i=1\ldots n$, because if $\alpha_i>0$ then $f\brac{\omega}\leq \alpha_i>0$ for any $\omega\in A_i$, which implies $A_i\subseteq \obj{f>0}$. In any case $I^\mu\brac{s} = 0$, whence $\int f d\mu = 0$. Conversely, suppose $\int f d\mu = 0$. If $A_n\defn\obj{f>\frac{1}{n}}$ for all $n\geq1$, then $A_n \uparrow \obj{f>0}$, whence $\mu\brac{A_n}\uparrow \mu\brac{\obj{f>0}}$ by theorem 7. For all $n\geq1$ setting $s_n\defn \frac{1}{n} 1_{A_n}$ gives simple on $\brac{\Omega, \Fcal}$ with $s_n\leq f$. Thus $I^\mu\brac{s_n} = 0$, whence $\frac{1}{n}\mu\brac{A_n}=0$ for all $n\geq1$. Therefore $\mu\brac{\obj{f > 0}} = 0$.

If $c\in \Real^+$ then simple fiddling with various simple functions yields $\int \brac{cf}d\mu = c\int fd\mu$. Indeed, if $s\leq f$, then $cs$ is simple and $cs\leq cf$, implying that $c I^\mu{s}=I^\mu{c s}\leq \int cf d\mu$. Thus $c \int f d\mu \leq \int cf d\mu$. Now, if $c=0$ then $cf = 0$, whence $\int cf d\mu = 0 = c\int f d \mu$. If $c>0$ and $s\leq cf$, then $\frac{s}{c} \leq f$ implies that $\frac{1}{c} I^\mu\brac{s}\leq \int f d\mu$, whence $\int cf d\mu \leq c \int f d\mu$.

Proving for $c=+\infty$ it a little trickier. First $\brac{+\infty} f$ is $\Fcal$-$\borel{\Rbar}$ measurable by theorem 4-7, whence $\int \brac{+\infty} f d\mu$ is legitimate. If $\int f d\mu = 0$, then $\mu\brac{\obj{f>0}}=0$. Since $\obj{f>0} = \obj{c f>0}$, this implies that $\int \brac{+\infty} f d\mu = 0$, whence $\int \brac{+\infty} f d\mu =0= \brac{+\infty} \int f d\mu$.

Next, for $A_n\defn\obj{f>\frac{1}{n}}$, it is true that $A_n\uparrow\obj{f>0}$, which implies $\mu\brac{A_n}\uparrow \mu\brac{\obj{f>0}}$ by theorem 7. If $\int f d\mu > 0$, then there is $N\geq1$ such that $0<\mu\brac{A_N}\leq \mu\brac{A_n}$ for all $n\geq N$. Now, $s_n\defn n 1_{A_n}$ is a collection of simple functions on $\brac{\Omega, \Fcal}$ with $s_n\leq \brac{+\infty} f$. Thus for all $n\geq N$ \[n \mu\brac{A_N} \leq n \mu\brac{A_n} = I^\mu{s_n}\leq \int \brac{+\infty} f d\mu\], whence $\int \brac{+\infty} f d\mu=+\infty = \brac{+\infty} \int f d\mu$.

For any $f:\brac{\Omega, \Fcal}\to\Zinf$ the map $h\defn\brac{+\infty} 1_{\obj{f=+\infty}}$ is $\Fcal$-$\borel{\Rbar}$ measurable and such that $h\leq f$. If $\int f d\mu < +\infty$, then $\mu\brac{\obj{f=+\infty}}=0$ since \[ \brac{+\infty} \mu\brac{\obj{f=+\infty}} = \brac{+\infty} \int 1_{\obj{f=+\infty}} d\mu = \int h d\mu \leq \int f d\mu\]

\label{thm:simple_int_meas} \noindent \textbf{Theorem} 5-6.
Let $s$ be a simple map on $\brac{\Omega, \Fcal}$. Then the map $\nu:\Fcal\to\Zinf$ defined as $\nu\brac{E}\defn I^\mu\brac{s 1_E}$ is a measure on $\brac{\Omega, \Fcal}$.

Indeed, for any $E\in\Fcal$, the map $s 1_E$ is a simple function, since if $s = \sum_{k=1}^n \alpha_k 1_{A_k}$ is a partition of $s$, then \[s 1_E =  \sum_{k=1}^n \alpha_k 1_{A_k} 1_E = \sum_{k=1}^n \alpha_k 1_{A_k\cap E}\] Thus $I^\mu\brac{s 1_E}$ is well defined for every $E\in \Fcal$. Furthermore since $\Omega= E \uplus E^c$ \[I^\mu\brac{s 1_E} = \sum_{k=1}^n \alpha_k \mu\brac{A_k\cap E} + 0 \mu\brac{E^c}\] Hence, $\nu\brac{\emptyset} = I^\mu\brac{s 1_\emptyset} = 0$.

If $\brac{E_n}_{n\geq1}\in \Fcal$ is a collection of pairwise disjoint measurable sets and $E\defn \biguplus_{n\geq1} E_n$, then $\mu\brac{A_k\cap E} = \sum_{n\geq1} \mu\brac{A_k\cap E_n}$, whence by theorem Sup-B-3 \begin{align*}\nu\brac{E} &= I^\mu\brac{s 1_E} = \sum_{k=1}^n \alpha_k \mu\brac{A_k\cap E} = \sum_{k=1}^n \sum_{n\geq1} \alpha_k \mu\brac{A_k\cap E_n} \\ &= \sum_{n\geq1} \sum_{k=1}^n \alpha_k \mu\brac{A_k\cap E_n} = \sum_{n\geq1} I^\mu\brac{s 1_{E_n}}\\ &= \sum_{n\geq1} \nu\brac{E_n}\end{align*} Therefore $\nu$ is a measure on $\brac{\Omega, \Fcal}$.

If $E,\brac{E_n}_{n\geq 1}\in \Fcal$ and $E_n\uparrow E$, then $\nu\brac{E_n}\uparrow\nu\brac{E}$ by theorem 7.\\

\label{thm:MCT} \noindent \textbf{Theorem} 19 (the MCT).
Let $\brac{\Omega, \Fcal, \mu}$ be a measure space and $\brac{f_n}_{n\geq 1}:\brac{\Omega, \Fcal}\to\Zinf$ be such measurable and non-negative maps, that $f_n\uparrow f$ point-wise on $\Omega$ in $\Zinf$ for some $f:\brac{\Omega, \Fcal}\to\Zinf$. Then $\int f_n d\mu \uparrow \int f d\mu$.

Since $f_n\leq f_{n+1}\leq f$, by theorem 5-5 $\int f_n d\mu \leq \int f_{n+1} d\mu \leq \int f d\mu$. Therefore, if $\alpha\defn\sup_{n\geq 1} \int f_n d\mu$, then $\int f_n d\mu \uparrow \alpha \leq \int f d\mu$.

Consider any simple function $s$ on $\brac{\Omega, \Fcal}$ with $s\leq f$, and let $c\in \brac{0,1}$. The collection $\brac{A_n}_{n\geq1}\in \Fcal$ defined as $A_n\defn\obj{c s\leq f_n}$ is such that $A_n\uparrow\Omega$. Indeed, $\omega\in \Omega$ is such that $s\brac{\omega}>0$, then $\omega\in A_n$ for all $n\geq1$. If $s\brac{\omega}>0$ then $f\brac{\omega}>0$ and \[c s\brac{\omega} \leq c f \brac{\omega} < f\brac{\omega}\] Since $f\brac{\omega}=\sup_{n\geq1} f_n\brac{\omega}$ there is $N\geq 1$ with $c s\brac{\omega} < f_N\brac{\omega}$, whence $\omega\in A_N$. Hence $\Omega=\bigcup_{n\geq1} A_n$ and $A_n\subseteq A_{n+1}$ for all $n\geq1$.

Therefore for all $n\geq 1$ \[c I^\mu\brac{s 1_{A_n}} = I^\mu\brac{c s 1_{A_n}} = \int c s 1_{A_n} d\mu\leq \int f_n 1_{A_n} d\mu\leq \int f_n d\mu\leq \alpha\] By theorem 5-6 $I^\mu\brac{s 1_{A_n}}\uparrow I^\mu\brac{s 1_\Omega}$, whence $c I^\mu\brac{s} \leq \alpha$ for any $c\in \brac{0,1}$. Thus $I^\mu\brac{s} \leq \alpha$ for every simple function $s$ on $\brac{\Omega, \Fcal}$ with $s\leq f$, which implies that $\int f d\mu \leq \alpha$.\\

\label{thm:usu_int_linear} \noindent \textbf{Theorem} 5-7.
Let $\brac{\Omega, \Fcal, \mu}$ be a measure space, $f,g:\brac{\Omega, \Fcal}\to\Zinf$ be measurable and non-negative maps and $a,b\in\Zinf$. Then \[\int \brac{a f + b g} d\mu = a\int f d\mu + b\int g d\mu\]

Indeed, let $\brac{f_n}_{n\geq1}$ and $\brac{g_n}_{n\geq1}$ be non-negative and measurable maps on $\brac{\Omega, \Fcal}$ with $f_n\uparrow f$ and $g_n\uparrow g$. Then since the sequence $\brac{f_n+g_n}_{n\geq1}$ is non-decreasing on $\Omega$, it must be true that $f_n+g_n\uparrow f+g$ over $\Omega$.

By theorem 18 there are $\brac{f_n}_{n\geq1}$ and $\brac{g_n}_{n\geq1}$ simple maps on $\brac{\Omega, \Fcal}$ such that $f_n\uparrow f$ and $g_n\uparrow g$, whence by the above $f_n+g_n\uparrow f+g$. By theorems 5-4 and 5-5 \[\int f_n + g_n d\mu = I^\mu\brac{f_n+g_n}=I^\mu\brac{f_n}+I^\mu\brac{g_n} = \int f_n d\mu + \int g_n d\mu \] whence by theorem 19 (MCT) $\int f_n d\mu \uparrow \int f d\mu$, $\int g_n d\mu \uparrow \int g d\mu$ and $\int f_n+g_n d\mu \uparrow \int f+g d\mu$. Therefore since the limit of non-decreasing non-negative sequence always exists in $\Rbar$ and honours summation over the extended real numbers, it must be true that $\int f+g d\mu = \int f d\mu+\int g d\mu$.

Finally, by theorem 5-5 and the just shown linearity property \[\int \brac{a f + b g} d\mu = \int a f d\mu + \int b g d\mu = a \int f d\mu + b \int g d\mu\]

\label{thm:usu_int_inf_sum} \noindent \textbf{Theorem} 5-8.
Let $\brac{\Omega, \Fcal, \mu}$ be a measure space, $\brac{f_n}_{n\geq1}:\brac{\Omega, \Fcal}\to\Zinf$ be measurable and non-negative maps. Then \[\int \sum_{n\geq1}f_n d\mu = \sum_{n\geq1} \int f_n d\mu\]

For any $\omega\in\Omega$, the non-negative of $f_n\brac{\omega}$ implies that $h\brac{\omega}\defn\sum_{n\geq1} f_n\brac{\omega}$ is in $\Zinf$. Thus the map $h\defn\sum_{n\geq1} f_n$ is such that $h_n\uparrow h$, where $h_n\defn\sum_{k=1}^n f_k$ are non-negative and $\Fcal$-$\borel{\Rbar}$ measurable by theorem 4-7. By theorem 4-3 $h:\brac{\Omega, \Fcal}\to\brac{\Rbar, \borel{\Rbar}}$ is measurable as well.

Since $h_n\uparrow h$, theorem 19 (MCT) implies that $\int h_n d\mu\uparrow \int h d\mu$. However by theorem 5-7 $\int h_n d\mu = \sum_{k=1}^n \int f_k d\mu$ for all $n\geq1$, whence $\sum_{n\geq1}\int f_k d\mu = \int \sum_{n\geq1} f_n d\mu$, as $\int f_n d\mu$ are non-negative for all $n\geq1$.\\

\noindent \textbf{Definition} 44.
Let $\brac{\Omega, \Fcal, \mu}$ be a measure space and $\pwr{\omega}$ be a property, which depends on $\omega\in \Omega$. The property $\pwr{\omega}$ holds $\mu$-almost surely if $\exists N\in \Fcal$ with $\mu\brac{N}=0$ such that $\pwr{\omega}$ holds for all $\omega\in N^c$.

%%% OVERFULL HERE
A collection of properties $\brac{\mathcal{P}_n\brac{\omega}}_{n\geq1}$ with each holding $\mu$-almost surely is equivalent to $\obj{\mathcal{P}_n\brac{\omega},\,\text{holds}\,\forall n\geq 1}$ holds $\mu$-a.s., since $\mu\brac{\bigcup_{n\geq 1} A_n} \leq \sum_{n\geq 1} \mu\brac{A_n}$ by theorem 3-6.\\

Indeed, since $f_n\leq f_{n+1}$ $\mu$-a.s. for each $n\geq1$ there is $\brac{N_n}_{n\geq1}\in\Fcal$ with $\mu\brac{N_n}=0$, such that $f_n\leq f_{n+1}$ everywhere outside of $N_n$ for every $n\geq1$. If $\bar{N}\defn\bigcup_{n\geq1} N_n$ then $f_n\brac{\omega}\leq f_{n+1}\brac{\omega}$ for all $n\geq 1$ for every $\omega\in\bar{N}^c$. By theorem 3-6 $\mu\brac{N}\leq \sum_{n\geq1} \mu\brac{N_n}=0$ ,which implies that $f_n\leq f_{n+1}$ for all $n\geq1$ altogether $\mu$-almost surely.

Furthermore $f_n\to f$ in $\Zinf$ $\mu$-a.s., which implies that there is $N\in\Fcal$ with $\mu\brac{N}=0$ such that $f_n\brac{\omega}\to f\brac{\omega}$ for all $\omega\in N^c$. Therefore for $N'\defn \bar{N}\cup N$, $N'\in \Fcal$, $\mu\brac{N'}=0$ and $N'$ is such that $f_n\brac{\omega}\uparrow f\brac{\omega}$ for all $\omega\notin N'$. Hence $f_n\uparrow f$ $\mu$-a.s.

\label{thm:usu_int_muas} \noindent \textbf{Theorem} 5-9.
Let $\brac{\Omega, \Fcal, \mu}$ be a measure space. If $f,g:\brac{\Omega,\Fcal}\to\Zinf$ are such that $f=g$ $\mu$-a.s, then $\int f d\mu = \in g d\mu$.

Since $f=g$ $\mu$-a.s. there is $N\in \Fcal$ with $\mu\brac{N}=0$ such that $f\brac{\omega}=g\brac{\omega}$ for all $\omega\in N^c$. Since $\Omega=N \uplus N^c$, it must be true that $f = f 1_{N^c}+ f 1_N$. By theorem 4-7 and theorem 5-1 $f 1_A$ is $\Fcal$-$\borel{\Rbar}$ measurable and non-negative for any $A\in \Fcal$. By theorem 5-7 $\int f d\mu = \int f 1_{N^c} d\mu + \int f 1_N d\mu$ and similarly $\int g d\mu = \int g 1_{N^c} d\mu + \int g 1_N d\mu$. By the basic properties of the usual Lebesgue integral established in theorem 5-5 $\int f 1_N d\mu=0$ because $\obj{f 1_N>0} \subseteq N$ and $\mu\brac{N}=0$. Since the same holds for $\int g 1_N d\mu$, and $f 1_{N^c} = g 1_{N^c}$ it is therefore true that \[\int f d\mu = \int f 1_{N^c} d\mu = \int g 1_{N^c} d\mu = \int g d\mu\]

\label{thm:muas_mct} \noindent \textbf{Theorem} 5-10 (MCT $\mu$-a.s.).
Let $\brac{\Omega, \Fcal, \mu}$ be a measure space and $f,\brac{f_n}_{n\geq 1}:\brac{\Omega, \Fcal}\to\Zinf$ be measurable and non-negative maps. If $f_n \uparrow f$ $\mu$-a.s. then $\int f_n d\mu \uparrow \int f d\mu$.

Since $f_n\uparrow f$ $\mu$-a.s. there is $N\in \Fcal$ wit h$\mu\brac{N}=0$ and such that $f_n\brac{\omega}\uparrow f\brac{\omega}$ for all $\omega\in N^c$. Hence, let $\bar{f}_n\defn f_n 1_{N^c}$ and $\bar{f}\defn f 1_{N^c}$. Obviously, $\bar{f_n}\uparrow \bar{f}$ everywhere on $\Omega$ and $f_n=\bar{f_n}$ $\mu$-a.s for all $n\geq1$ and $\bar{f}=f$ $\mu$-a.s. Thus by theorems 5-9 and 19 (MCT) it is true that \[\int f_n d\mu=\int \bar{f_n} d\mu\uparrow \int \bar{f} d\mu=\int f d\mu\] Therefore the requirements of the MCT can be relaxed a little.\\

\label{thm:fatou} \noindent \textbf{Theorem} 20.
Let $\brac{\Omega, \Fcal, \mu}$ be a measure space and $\brac{f_n}_{n\geq 1}:\brac{\Omega, \Fcal}\to\Zinf$ be measurable and non-negative maps. Then \[\int \liminf_{n\to+\infty} f_n d\mu \leq \liminf_{n\to+\infty} \int f_n d\mu\]

If $g_n\defn \inf_{k\geq n} f_n$ and $g\defn \sup_{n\geq1} g_n = \liminf f_n$, then $g$ and $\brac{g_n}_{m\geq1}$ are non-negative and $\Fcal $-$\borel{\Rbar}$ measurable by theorem 4-3. By theorem 5-5 $\int g_n d\mu\leq \int f_n d\mu$ for all $n\geq 1$, whence $\liminf \int g_n d\mu\leq \liminf \int f_n d\mu$ by the basic properties of the lower limit. However $g_n\uparrow g$, whence by theorem 19 (MCT) $\int g_n d\mu \uparrow \int g d\mu$. Therefore \[\int \liminf f_n d\mu = \int g d\mu = \liminf \int g_n d\mu \leq \liminf \int f_n d\mu\]

\noindent \textbf{Definition} 45.
Let $f:\brac{\Omega, \Fcal}\to\Zinf$ be a non-negative and measurable map, where $\brac{\Omega, \Fcal, \mu}$ is a measure space. For any $E\in \Fcal$, the partial Lebesgue integral of $f$ with respect to $\mu$ over $E$ is defined as \[\int_E fd\mu \defn \int \brac{f 1_E} d\mu = \int f d\mu^E = \int \induc{f}_E d\induc{\mu}_E\] The first integral is taken over the measure space $\brac{\Omega, \Fcal, \mu}$, the second -- over $\brac{\Omega, \Fcal, \mu^E}$, where $\mu^E \defn \mu\brac{\bullet\cap E}$ -- the measure on $\brac{\Omega, \Fcal}$ of measurable sets truncated by $E$. The third is over $\brac{E, \induc{\Fcal}_E, \induc{\mu}_E}$ with $\induc{\mu}_E$ being the restriction of the measure $\mu$ to the subspace $\sigma$-algebra $\induc{\Fcal}_E$ on $E$, the trace of $\Fcal$ on $E$. 

For every $E\in \Fcal$, the equivalence of definitions is established, first, for every indicator $1_A$, $A\in \Fcal$. Indeed, $\int 1_A 1_E d\mu = \mu\brac{A\cap E}$ and $\int 1_A d\mu^E = \mu^E\brac{A} = \mu\brac{A\cap E}$. Since $\induc{1_A}_E = \induc{1_{A\cap E}}_E$ and the latter is the indicator of $A\cap E\in \induc{\Fcal}_E$ and a map $E\to \Zinf$, it must be true that \[\int \induc{1_A}_E d\induc{\mu}_E = \int \induc{1_{A\cap E}}_E d\induc{\mu}_E = \induc{\mu}_E\brac{A\cap E} = \mu\brac{A \cap E}\] Then, by linearity, for every simple function on $\brac{\Omega, \Fcal}$, because the restriction of a simple function on $\brac{\Omega, \Fcal}$ to $E\subseteq \Omega$ is a simple function on $\brac{E, \induc{\Fcal}_E}$. Finally theorem 19 (MCT) proves the equivalence for any non-negative and measurable map $f:\brac{\Omega, \Fcal}\to\Zinf$, since by theorem 18 there are $\brac{s_n}_{n\geq1}$ simple maps on $\brac{\Omega, \Fcal}$ such that $s_n\uparrow f$.\\

\label{thm:int_meas} \noindent \textbf{Theorem} 21.
Let $f:\brac{\Omega, \Fcal}\to\Zinf$ be a non-negative and measurable map, where $\brac{\Omega, \Fcal, \mu}$ is a measure space. Let $\nu:\Fcal\to\Zinf$ be defined for all $A\in \Fcal$ as $\nu\brac{A} \defn \int_A f d\mu$. Then $\nu$ is a measure on $\brac{\Omega, \Fcal}$, and for every non-negative and measurable map $g:\brac{\Omega, \Fcal}\to\Zinf$ \[\int g d\nu =\int g f d\mu\]

This is established similarly to theorem 5-6. Indeed, for every non-negative and measurable map $f:\brac{\Omega, \Fcal}\to\Zinf$, $\nu\brac{E} \defn \int_E f d\mu$ defines a map on $\Fcal$, which is a measure courtesy of linearity and theorem 19. Indeed, $\nu\brac{\emptyset}=\int f 1_\emptyset d\mu = 0$ since $\mu\brac{\obj{f 1_\emptyset>0}}=0$. Next, let $\brac{E_n}_{n\geq1}\in\Fcal$ are pairwise disjoint and define $B_n\defn \biguplus_{k=1}^n E_k$. Note that $B_n\uparrow E\defn\biguplus_{n\geq1} E_n$ implies $1_{B_n}=\sum_{k=1}^n 1_{A_k}$ for all $n\geq1$ and $1_{B_n}\uparrow 1_E$, whence $f 1_{B_n}\uparrow f 1_E$.

By theorem 19 (MCT) $\int f 1_{B_n} d\mu\uparrow \int f 1_E d\mu$ and by theorem 5-7 $\int f 1_{B_n} d\mu = \sum_{k=1}^n \int f 1_{E_k} d\mu$. Therefore \[\nu\brac{\biguplus_{n\geq1} E_n} = \int f 1_E d\mu = \sum_{n\geq1} \int f 1_{E_n} d\mu = \sum_{n\geq1} \nu\brac{E_n}\]

The standard argument, which proceeds from indicators $1_A$, $A\in \Fcal$, through simple functions on $\brac{\Omega, \Fcal}$ to general non-negative and measurable maps using theorem 18.

Indeed, for any $g\defn 1_A$ with $A\in \Fcal$ it is true that \[\int g d\nu = \int 1_A d\nu = \nu\brac{A} = \int_A f d\mu = \int 1_A f d\mu = \int g f d\mu\] If $g\defn \sum_{i=1}^n \alpha_i 1_{A_i}$ is a simple function on $\brac{\Omega,\Fcal}$, then by linearity of the usual Lebesgue integral (theorem 5-7) \[\int g d\nu = \sum_{i=1}^n \int 1_{A_i} d\nu = \sum_{i=1}^n \int 1_{A_i} f d\mu = \int g f d\mu\] Finally, by theorem 18 for any $g:\brac{\Omega,\Fcal}\to\Zinf$ there exist simple functions $\brac{s_n}_{n\geq1}$ on $\brac{\Omega, \Fcal}$ with $s_n\uparrow g$. Therefore $\int s_n d\nu \uparrow \int g d\nu$ by theorem 19 (MCT) on $\brac{\Omega,\Fcal,\nu}$ and $\int f s_n d\mu \uparrow \int f g d\mu$ by theorem 19 (MCT) on $\brac{\Omega,\Fcal,\mu}$, since $f s_n \uparrow f g$ everywhere on $\Omega$ and $\brac{f s_n}_{n\geq1}$ and $f g$ are non-negative and $\Fcal$-$\borel{\Rbar}$ measurable by theorem 4-7. Since $\int s_n d\nu = \int f s_n d\mu$ for all $n\geq1$, for all $g:\brac{\Omega, \Fcal}\to \Zinf$ this establishes $\int g d\nu = \int g f d\mu$.

\noindent \textbf{Definition} 46.
Let $\brac{\Omega, \Fcal, \mu}$ be a measure space. The $L^1$-space over a field $\mathbf{K} = \Cplx$ or $\Real$, is defined as \[ L^1_\mathbf{K}\brac{\Omega, \Fcal, \mu} \defn \obj{ f:\brac{ \Omega, \Fcal }\to \brac{\mathbf{K}, \borel{\mathbf{K}}\,\text{-- measurable} },\,\int \abs{f} d\mu < +\infty } \]

\label{thm:l_1_space_linear} \noindent \textbf{Theorem} 5-11.
Let $\brac{\Omega, \Fcal, \mu}$ be a measure space and $K=\Real$ or $K=\Cplx$. Then the $L^1$-spaces are linear, because the modulus is $\Tcal_\mathbf{K}$-$\Tcal_\Real$ is continuous, all arithmetic operations in a field $\mathbf{K}$ are defined, maps are measurable by theorem 4-7 and by theorems 5-7 and 5-5 $\int \abs{f+\alpha g} d\mu\leq \int \abs{f} d\mu + \abs{\alpha} \int \abs{g} d\mu$. In addition $L^1_\Real \subseteq L^1_\Cplx$ and $L^1_\Real = \obj{ \induc{ f\in L^1_\Cplx } f\brac{\Omega}\subseteq \Real }$.

Indeed, let $f:\brac{\Omega,\Fcal}\to\brac{\Cplx, \borel{\Cplx}}$ be measurable and $f\brac{\Omega}\subseteq \Real$. Since by theorem 4-?9 $\borel{\Real}\subseteq \borel{\Cplx}$ and the range of $f$ is $\Real$, $\Fcal$-$\borel{\Real}$ measurable. Thus $\obj{\induc{f\in L^1_\Cplx}\,f\brac{\Omega}\subseteq \Real}\subseteq L^1_\Real$, because $\int \abs{f} d\mu<+\infty$.

Conversely, let $f\in L^1_\Real$. By theorem 4-?9 $A\cap \Real\in \borel{\Real}$ for any $A\in\borel{\Cplx}$. Since $f\brac{\Omega}\subseteq \Real$ implies $f^{-1}\brac{B} = f^{-1}\brac{B\cap \Real}$ for any $B\subseteq \Cplx$, it must be true that $f^{-1}\brac{B} = f^{-1}\brac{A}\in \Fcal$ for all $B\in \borel{\Cplx}$, whence $f$ is $\Fcal$-$\borel{\Real}$ measurable. Thus $f\in L^1_\Cplx$ and $f\brac{\Omega}\subseteq \Real$.

For any $f,g\in L^1_\Real$ and $\alpha\in \Real$ theorem 4-7 implies that $f+\alpha g$ is $\Fcal$-$\borel{\Real}$ measurable. Furthermore, since by the triangle inequality $\abs{f+\alpha g}\leq \abs{f}+\abs{\alpha}\abs{g}$, it must be true that \[\int \abs{f+\alpha g} d\mu\leq \int \int \abs{f}+\abs{\alpha} \abs{g} d\mu = \abs{f} d\mu + \abs{\alpha}\int \abs{g} d\mu<+\infty\] by theorem 4-7 and linearity of the usual Lebesgue integral (theorem 5-7). Therefore $L^1_\Real$ is closed under $\Real$-linear combinations.

Let $f\in L^1_\Cplx$. By theorem 4-9 $\re f$ and $\im f$ are $\Fcal$-$\borel{\Real}$ measurable and, since $\abs{\re \cdot}, \abs{\im \cdot }\leq \abs{\cdot}$, theorem 5-5 implies \[\int \abs{\re f}d\mu,\,\int \abs{\im f}d\mu \leq \int \abs{f}d\mu < +\infty\] whence $\re f, \im f\in L^1_\Real$.

Now if $f,g\in L^1_\Cplx$, then by the above $\re f + \re g$ and $\im f + \im g$ are in $L^1_\Real$. By theorem 4-10 therefore the map $h\defn \re f + \re g + i\brac{\im f + \im g}$ is $\Fcal$-$\borel{\Cplx}$ measurable. Furthermore since $\abs{\cdot}\leq \abs{\re \cdot}+\abs{\im \cdot}$, by theorem 5-5 it must be true that \[\int \abs{h}d\mu \leq \int \abs{\re \brac{f+g}}d\mu+\int \abs{\im \brac{f+g}}d\mu < +\infty\] whence $f+g\in L^1_\Cplx$.

If $f=u+iv\in L^1_\Cplx$ and $\alpha=a+ib\in \Cplx$, then by the above abd theorem 4-7 $au-bv$ and $av+bu$ are in $L^1_\Real$. Thus by theorem 4-10 and since $L^1_\Real\subseteq L^1_\Cplx$, it must be true that $\alpha f = au-bv + i\brac{av+bu}\in L^1_\Cplx$. Therefore $L^1_\Cplx$ is closed under $\Cplx$-linear combinations.\\

If $f\in L^1_\Real$, then by theorems 4-9 and 4-7 $f^+, f^-, \abs{f}\in L^1_\Real$, $\abs{f}=f^++f^-$ and $f = f^+-f^-$. And for any $f\in L^1_\Cplx$, $\re f, \im f\in L^1_\Real$ by theorem 5-11. Hence the integrals $\int u^+ d\mu, \int u^- d\mu, \int v^+ d\mu$ and $\int v^- d\mu$ are well-defined non-negative real numbers. Therefore the following is a well-defined complex number \[\int fd\mu \defn \int u^+ d\mu - \int u^- d\mu + i\brac{ \int v^+ d\mu - \int v^- d\mu }\]

If $f\in L^1_\Cplx$ with $f\brac{\Omega}\subseteq \Cplx\cap \Zinf$, then $f$ is non-negative and measurable, whence $\int f d\mu$ is a well-defined usual Lebesgue integral (definition  43). Furthermore $f\geq 0$ on $\Omega$, whence $\brac{\re f}^+ = f$ and $\brac{\re f}^+=\brac{\im f}^+=\brac{\im f}^- = 0$. Thus by theorem 5-5 the complex Lebesgue integral is consistent with the usual Lebesgue integral \[\int^\Cplx f d\mu = \int \brac{\re f}^+ d\mu + 0 = \int^{\Zinf} f d\mu\] Finally, for any $g\in L^1_\Real$, since $\im g = 0$ it must be true that \[\int g d\mu = \int \brac{\re g}^+ d\mu - \int \brac{\re g}^- d\mu = \int g^+ d\mu - \int g^- d\mu\]

\noindent \textbf{Definition} 48.
Let $f = u + iv \in L^1_\Cplx\brac{\Omega, \Fcal, \mu}$. The Lebesgue integral of a complex valued $f$ is defined as \[\int f d\mu \defn \int u^+ d\mu - \int u^- d\mu + i\brac{\int v^+ d\mu - \int v^- d\mu } \] This definition is consistent with the original, because if $f\brac{\Omega}\subseteq \Cplx\cap \Zinf = \Real^+$ then $f \equiv f^+$ and integrals of other ``parts'' are equal to zero. Moreover the same reasons imply that $\int f d\mu = \int u d\mu + i\int v d\mu$.\\

\label{thm:lin_cpls_int} \noindent \textbf{Theorem} 22.
For all $f,g\in L^1_\Cplx\brac{\Omega, \Fcal, \mu}$ and $\alpha \in \Cplx$, \[\int \brac{\alpha f + g} d\mu = \alpha \int f d\mu + \int g d\mu\]

Indeed, if $f,g\in L^1_\Real$, then $h = f+g \in L^1_\Real$ and $h^+ + f^- + g^- = h^- + f^+ + g^+$. So by theorem 5-7, the definition of the signed Lebesgue integral and the finiteness of integrals of partial functions of $f,g$ and $h$ it is true that \begin{align*}\int h d\mu &= \int h^+ d\mu - \int h^- d\mu \\ &= \int f^+ d\mu - \int f^- d\mu + \int g^+ d\mu -  \int g^- d\mu \\&= \int f d\mu + \int g d\mu\end{align*}

Further, since $\brac{-f}^+ = f^-$ and $\brac{-f}^- = f^+$, $\int \brac{-f} d\mu = -\int f d\mu$. Based on that $\int \brac{\alpha f} d\mu = \alpha \int f d\mu$ for any $\alpha \in \Real$. If $f\leq g$, then $f^+ + g^- \leq f^- + g^+$, and so, since $f$ and $g$ are integrable, \[\int f d\mu = \int f^+ d\mu-\int f^- d\mu \leq \int g^+ d\mu - \int g^- d\mu = \int g d\mu\] Straight from the definition of the complex Lebesgue integral follows that it is $\Cplx$-linear.\\

\noindent \textbf{Definition} 49.
Let $f\in L^1_\Cplx\brac{\Omega, \Fcal, \mu}$ where $\brac{\Omega, \Fcal, \mu}$ is a measure space. Let $E\in \Fcal$. The partial Lebesgue integral of $f$ with respect to $\mu$ over $E$ is defined as \[\int_E f d\mu \defn \int f 1_E d\mu = \int f d\mu^E = \int \induc{f}_E d\induc{\mu}_E\]

For any $f\in L^1_\Cplx\brac{\Omega, \Fcal, \mu}$ and all $E\in \Fcal$, $f 1_E \in L^1_\Cplx\brac{\Omega, \Fcal, \mu}$, $f \in L^1_\Cplx\brac{\Omega, \Fcal, \mu^E}$ and $\induc{f}_E \in L^1_\Cplx\brac{E, \induc{\Fcal}_E, \induc{\mu}_E}$, because $\abs{\induc{f}_E} = \induc{\brac{\abs{f}}}_E$ and the equivalence of definitions of the partial usual Lebesgue integral for non-negative measurable maps.\\

\label{thm:DCT} \noindent \textbf{Theorem} 23.
Let $\brac{f_n}_{n\geq 1}:\brac{\Omega, \Fcal}\to \brac{\Cplx, \borel{\Cplx}}$ be a sequence, such that $f_n\to f$ in $\Cplx$ everywhere on $\Omega$. If there exists $g\in L^1_\Real\brac{\Omega, \Fcal, \mu}$, such that $\abs{f_n}\leq g$, then $f, f_n\in L^1_\Cplx\brac{\Omega, \Fcal, \mu}$ for all $n\geq 1$ and \[\int \abs{f_n - f} d\mu \to 0\]

First, $u_n\to 0$ in $\Cplx$ is equivalent to $\limsup_{n\to \infty} \abs{u_n} = 0$. Indeed, if $\limsup_{n\to \infty} \abs{u_n} = 0$, then for any $\epsilon>0$ there is $N\geq 1$ such that $\inf_{k\geq N} \abs{u_k}<\epsilon$, whence $\abs{u_n}<\epsilon$ for all $n\geq N$. If $u_n\to 0$ in $\Cplx$, then $\abs{u_n}\to 0$ in $\Real$, whence by theorem 16 $\limsup_{n\to\infty}\abs{u_n}=0$.

Second, $\liminf \brac{\alpha - v_n} = \alpha + \limsup v_n$ for all $\alpha\in \Real$ and $\brac{v_n}_{n\geq 1}\in \Real$. Indeed, it is directly obvious that $\liminf -v_n = - \limsup v_n$ and that $\liminf \alpha + v_n = \alpha + \liminf v_n$ since $\alpha\in \Real$.

Now, since $\abs{\cdot}:\Cplx\to\Real^+$ is continuous and $\abs{f_n}\leq g$, $\abs{f_n}\to \abs{f}$ in $\Real^+$ and $\abs{f}\leq g$. By theorem 17 $f$ is $\Fcal$-$\borel{\Cplx}$ measurable, since $\Cplx$ is a metrizable topological space. Hence $f\in L^1_\Cplx$.

For $h_n = 2 g - \abs{f_n - f}$, $\brac{h_n}_{n \geq 1}:\brac{\Omega, \Fcal}\to \Zinf$ is non-negative and measurable with \[\liminf_{n\to \infty} h_n = 2 g - \limsup_{n\to \infty} \abs{f_n - f} = 2 g - 0\] Thus $\int \liminf h_n d\mu = \int 2g d\mu$ and furthermore by the above \[\liminf_{n\to \infty} \int h_n d\mu = \int 2 g d\mu - \limsup_{n\to \infty} \int \abs{f_n - f} d\mu\] However by theorem 20 (Fatou) \[\int \liminf h_n d\mu \leq \liminf \int h_n d\mu\] implying $\limsup_{n\to \infty} \int \abs{f_n - f} d\mu=0$ since $\int \abs{f_n - f} d\mu\leq \int 2g d\mu < +\infty$. Finally, by the observation above $\int \abs{f_n - f} d\mu \to 0$.\\

\label{thm:int_modulus} \noindent \textbf{Theorem} 24.
Let $f\in L^1_\Cplx\brac{\Omega, \Fcal, \mu}$, where $\brac{\Omega, \Fcal, \mu}$ is a measure space. Then \[\abs{ \int f d\mu }\leq \int \abs{f} d\mu\]

Indeed, let $z = \int f d\mu$ and $\alpha\in \Cplx$ be such that $\abs{\alpha} = 1$ and $\alpha z = \abs{z}$. First, $u = \re\brac{\alpha f} \leq \abs{\alpha f} = \abs{f}$ on $\Omega$ and so $u \in L^1_\Real$. Second, $\abs{\int f d\mu} = \abs{z} = \alpha z = \alpha \int f d\mu = \int \brac{\alpha f} d\mu$. Third, $\int \brac{\alpha f} d\mu = \int u d\mu$, as $\im\brac{\alpha f} = 0$. Finally $\abs{\int f d\mu} = \int \brac{\alpha f} d\mu = \int u d\mu \leq \int \abs{f} d\mu$.\\

\label{thm:int_mu_as} \noindent \textbf{Theorem} 5-11
Suppose $f,g\in L^1_\Cplx$ are such that $f=g$ $\mu$-a.s. Then \[\int f d\mu = \int g d\mu\]

Indeed, there is $N\in \Fcal$ with $\mu\brac{N}=0$ such that $1_{N^c} f = 1_{N^c} g$. By theorem 24 and the basic properties of the usual Lebesgue integral $\abs{\int_\Omega 1_N f d\mu}\leq \int_\Omega 1_N \abs{h}d\mu = 0$, whence by linearity of the complex Lebesgue integral \[\int_\Omega f d\mu = \int_\Omega 1_N f + 1_{N^c}f d\mu = \int_\Omega 1_{N^c}f d\mu = \int_\Omega 1_{N^c}g d\mu = \int_\Omega 1_N g + 1_{N^c}g d\mu = \int_\Omega g d\mu\]\\

% section tut_5 (end)

\section{Product spaces} % (fold)
\label{sec:tut_6}
\url{http://probability.net/PRTproduct.pdf}

\noindent \textbf{Definition} 50.
Let $\brac{\Omega_i}_{i\in I}$ be a family of sets, indexed by a non-empty $I$. The Cartesian Product (generalised) of the family $\brac{\Omega_i}_{i\in I}$ is the set of all maps $\omega$ defined on $I$ such that $\omega\brac{i}\in \Omega_i$:\[\prod_{i\in I} \Omega_i \defn = \obj{ \induc{ \omega:I\to \bigcup_{i\in I} \Omega_i }\, \omega\brac{i}\in \Omega_i\, \forall i\in I }\] Note, that any map can be treated as a collection of its values indexed by elements of its domain.\\

\label{thm:AOChoice} \noindent \textbf{Theorem} 25.
Let $\brac{\Omega_i}_{i\in I}$ be a family of sets, indexed by a non-empty $I$. Then $\prod_{i\in I} \Omega_i$ is non-empty if and only if $\Omega_i\neq \emptyset$ for all $i\in I$.

\noindent \textbf{Definition} 52.
Let $\brac{\Omega_i}_{i\in I}$ be a family of sets, indexed by a non-empty $I$. Let $\Ecal_i$ be a collection of subsets of $\Omega_i$ for all $i\in I$. A rectangle of the family $\brac{\Ecal_i}_{i\in I}$ is any subset of the form $\prod_{i\in I} A_i$ with $A_i\in \Ecal_i \cap \obj{\Omega_i}$ for all $i\in I$ and such that the set $\obj{\induc{i\in I} A_i\neq \Omega_i}$ is finite. The collection of all rectangles of the family $\brac{\Ecal_i}_{i\in I}$ is \[\coprod_{i\in I} \Ecal_i \defn \obj{ \induc{ \prod_{i\in I} A_i } A_i \in \Ecal_i \cap \obj{\Omega_i},\, A_i\neq \Omega_i\,\text{for finitely many}\, i\in I }\]

For any $A\in \coprod_{i\in I} \Ecal_i$, there is a finite $J\subseteq I$such that $A = \obj{ \induc{ \omega\in \prod_{i\in I} \Omega_i } \omega\brac{i} \in A_i\, \forall i\in J}$. When $I$ is itself finite, the generic rectangle of $\brac{\Ecal_i}_{i\in I}$ is of the form $A_1\times \ldots \times A_n$ with $A_i\in \Ecal_i \cap \obj{\Omega_i}$ -- the ``finiteness'' condition is dropped due to finiteness of $I$.

For any non-empty rectangle $A = \prod_{i\in I} A_i$ of $\brac{\Ecal_i}_{i\in I}$, $J_A\defn \obj{ \induc{i\in I} A_i \neq \Omega_i }$ is a well-defined finite set, because the representation of $A$ as a Cartesian product is unique. Indeed, if $\prod_{i\in I} A_i = \prod_{i\in I} B_i$ and $\exists j\in I$ such that $A_j\setminus B_j\neq \emptyset$, then $\omega\in \prod_{i\in I} A_i$ with $\omega_j \in A_j\setminus B_j$, must still be in $\prod_{i\in I} B_i$ as well, implying that $\omega_j\in B_j$. Thus there can be no $j\in I$ such that $A_j\neq B_j$.\\

\label{thm:rectangle_pi_sys} \noindent \textbf{Theorem} 6-1.
If $\brac{\Ecal_i}_{i\in I}$ is a family of $\pi$-systems, then $\coprod_{i\in I} \Ecal_i$ is a $\pi$-system.

Indeed, let $A \defn \prod_{i\in I} A_i, B \defn \prod_{i\in I} B_i \in \coprod_{i\in I} \Ecal_i$. Obviously $A\cap B = \prod_{i\in I} A_i \cap B_i$. Now $J_{A\cap B} = J_A \cap J_B$ since $A_i \cap B_i \neq \Omega_i$ if and only if neither $A_i$ nor $B_i$ are $\Omega_i$. Hence $\obj{ \induc{ i\in I } A_i \cap B_i \neq \Omega_i }$ is finite and $A\cap B \in \coprod_{i\in I} \Ecal_i$, because $A_i\cap B_i \in \Ecal_i$ for all $i\in I$.\\

% If $\brac{\Ecal_i}_{i\in I}$ is a collection of semi-rings on $\brac{\Omega_i}_{i\in I}$, then $\coprod_{i\in I} \Ecal_i$ is a semi-ring on $\prod_{i\in I} \Omega_i$. Indeed, $\coprod_{i\in I} \Ecal_i$ is a closed under finite intersections and definitely contains the empty set. Further, let $A = \prod_{i\in I} A_i, B = \prod_{i\in I} B_i \in \coprod_{i\in I} \Ecal_i$, then $A\setminus B = \biguplus_{j=0}^n \Delta_j$, where $n = \abs{J_B}$, $\Delta_k = \brac{\prod_{i\notin J_B} A_i} \times \brac{\prod_{i\in J_B\setminus J_k} A_i\setminus B_i} \times \prod_{i\in J_k} B_i$, with $J_k = \obj{ j_1\ldots j_k }$ and $J_n \equiv J_B$.

\label{thm:rectangle_trace} \noindent \textbf{Theorem} 6-2.
If $\brac{\Ecal_i}_{i\in I}$ is a family of $\sigma$-algebras with each $\Ecal_i$ being a $\sigma$-algebra on $\Omega_i$ and let $F\defn \prod_{i\in I} F_i \subseteq \prod_{i\in I} \Omega_i$. Then \[\induc{\brac{\coprod_{i\in I} \Ecal_i}}_F = \coprod_{i\in I} \induc{\Ecal_i}_{F_i}\]

Indeed, for any $A \in \induc{\brac{\coprod_{i\in I} \Ecal_i}}_F$, by definition of a trace, there exits $B\defn \prod_{i\in I} B_i\in \coprod_{i\in I} \Ecal_i$ such that $A = B \cap F$. With $B$ being a measurable rectangle the set $J_B\defn \obj{ \induc{ i\in I } B_i \neq \Omega_i }$ is, by definition, finite. Now, since $F$ is structured like a generalised Cartesian product, \[A = \brac{\prod_{i\in I} B_i} \cap \prod_{i\in I} F_i = \prod_{i\in I} \brac{B_i\cap F_i}\] Let $A_i\defn B_i\cap F_i$ for all $i\in I$ and $J_A\defn \obj{ \induc{ i\in I } A_i \neq F_i }$. For any $i\in I$ by definition of a trace of $\Ecal_i$ on $F_i$, $A_i\in \induc{\Ecal_i}_{F_i}$. Furthermore if $i\notin J_B$ then $B_i = \Omega_i$ and so $A_i = B_i\cap F_i = F_i$, whence $i\notin J_A$. Thus $J_A$ is finite because $J_B$ is finite. Therefore $A\in \coprod_{i\in I} \induc{\Ecal_i}_{F_i}$.

Conversely, if $\prod_{i\in I} A_i \in \coprod_{i\in I} \induc{\Ecal_i}_{F_i}$ then $A_i\in \induc{\Ecal_i}_{F_i}$ and $J_A\defn \obj{ \induc{i\in I} A_i\neq F_i}$ is finite. By definition of a trace, there exists $B_i\in \Ecal_i$ with $A_i \equiv B_i \cap F_i$ for every $i\in I$. So for $i\notin J_A$ pick $B_i\defn \Omega_i$, and for $i\in J_A$ pick whatever $B_i$ the above mentioned trace property provides. Then $A_i = B_i\cap F_i$ for all $i\in I$ since $A_i = F_i$ for every $i\notin J_A$, and \[\prod_{i\in I} A_i \equiv \brac{\prod_{i\in I} B_i} \cap \prod_{i\in I} F_i \] Let $B\defn \prod_{i\in I} B_i$ and put $J_B\defn \obj{ \induc{ i\in I } B_i\neq \Omega_i }$. For this specifically designed product $B$ $J_B\subseteq J_A$, whence $J_B$ is finite, and so $B\in \coprod_{i\in I} \Ecal_i$. with $A \equiv B\cap F$. Therefore $A\in \induc{\brac{\coprod_{i\in I} \Ecal_i}}_F$.\\

\noindent \textbf{Definition} 54.
Let $\brac{\Omega_i, \Fcal_i}_{i\in I}$ be a family of measurable spaces, indexed by a non-empty set $I$. The product $\sigma$-algebra $\bigotimes_{i\in I} \Fcal_i$ on $\prod_{i\in I} \Omega_i$ is generated by all measurable rectangles \[\bigotimes_{i\in I} \Fcal_i \defn \sigma\brac{\coprod_{i\in I} \Fcal_i}\]

\label{thm:sigma_coprod} \noindent \textbf{Theorem} 26.
Let $\brac{\Omega_i}_{i\in I}$ be a family of non-empty sets, indexed by a non-empty $I$, and $\Ecal_i\subseteq \pwr{\Omega_i}$ for all $i\in I$. Then the product $\sigma$-algebra $\bigotimes_{i\in I} \sigma\brac{\Ecal_i}$ on $\prod_{i\in I}\Omega_i$ is generated by rectangles of $\brac{\Ecal_i}_{i\in I}$: \[\sigma\brac{\coprod_{i\in I} \Ecal_i} \equiv \bigotimes_{i\in I} \sigma\brac{\Ecal_i} \]

Indeed, since  $\Ecal_i \subseteq \sigma\brac{\Ecal_i}$ for all $i\in I$: \[\coprod_{i\in I} \Ecal_i \subseteq \coprod_{i\in I} \sigma\brac{\Ecal_i} \subseteq \bigotimes_{i\in I} \sigma\brac{\Ecal_i} \]

Let $A \in \coprod_{i\in I} \sigma\brac{\Ecal_i}$. If $A = \emptyset$, then obviously $A\in \sigma\brac{ \coprod_{i\in I} \Ecal_i }$. If $A \neq \emptyset$ and $J_A = \emptyset$, then $A = \prod_{i\in I} \Omega_i$ and again $A\in \sigma\brac{ \coprod_{i\in I} \Ecal_i }$. Suppose that for all $A\in \coprod_{i\in I} \sigma\brac{ \Ecal_i }$, $A\neq \emptyset$ with $\abs{J_A} = n$ it is true that $A\in \sigma\brac{ \coprod_{i\in I} \Ecal_i }$. This is the induction hypothesis.

Let $A\in \coprod_{i\in I} \sigma\brac{ \Ecal_i }$ be non-empty and $\abs{J_A} = n + 1$. Pick any $j\in J_A$, let \[A^B \defn \prod_{i\in I} \bar{A}_i \equiv B \times \prod_{i\in I, i\neq j} A_i\] where $\bar{A}_j = B$ and $\bar{A}_i = A_i$ for all $i\neq j$. Further, let \[\Gamma \defn \obj{ \induc{ B\subseteq \Omega_j } A^B \in \sigma\brac{ \coprod_{i\in I} \Ecal_i } }\]

Since $A\neq \emptyset$, $A^{\Omega_j} \neq \emptyset$. By definition and uniqueness of the Cartesian product representation of $A^{\Omega_j}$, $J_{A^{\Omega_j}} = J_A\setminus \obj{j}$, whence $\abs{ J_{A^{\Omega_j}} } = n$. Also $A_i\in \sigma\brac{\Ecal_i}$ for all $i\neq j$ and obviously $\Omega_j\in \sigma\brac{\Ecal_j}$. Thus $A^{\Omega_j} \in \sigma\brac{ \coprod_{i\in I} \Ecal_i }$, implying that $\Omega_j \in \Gamma$.

For all $B\subseteq \Omega_j$ and any $\brac{B_n}_{n\geq 1}\subseteq \Omega_j$, direct checking demonstrates that $A^{\bigcup_{n\geq 1} B_n} \equiv \bigcup_{n\geq 1} A^{B_n}$ and $A^{\Omega_j\setminus B} \equiv A^{\Omega_j}\setminus A^B$. Therefore if $B\in\Gamma$, then $\Omega_j\setminus B\in \Gamma$ and if $\brac{B_n}_{n\geq 1}\in \Gamma$, then $\bigcup_{n\geq 1} B_n\in \Gamma$, and $\Gamma$ is a $\sigma$-algebra on $\Omega_j$.

For any $B\in \Ecal_j$, direct checking implies that $A^B \equiv A^{\Omega_j} \cap \bar{B}$, where $\bar{B} = B_j \times \prod_{i\in I,\, i\neq j} \Omega_i$. As $\bar{B} \in \coprod_{i\in I} \Ecal_i$ and $A^{\Omega_j} \in \sigma\brac{\coprod_{i\in I} \Ecal_i}$, it must be true that $A^B \in \sigma\brac{\coprod_{i\in I} \Ecal_i}$, implying that $\Ecal_j\subseteq \Gamma$ and  $\sigma\brac{\Ecal_j} \subseteq \Gamma$. Since $A \equiv A^{A_j}$ and $A_j\in \sigma\brac{\Ecal_j}\subseteq \Gamma$, it has to be true that $A\in \sigma\brac{\coprod_{i\in I} \Ecal_i}$, thereby proving the inductive hypothesis. Therefore $\coprod_{i\in I} \sigma\brac{ \Ecal_i } \subseteq \sigma\brac{ \coprod_{i\in I} \Ecal_i }$.\\

\noindent \textbf{Definition} 51.
Let $I$ be a non-empty set. A family of sets $\brac{I_\lambda}_{\lambda \in \Lambda}$ is a partition of $I$ if $I = \uplus_{\lambda\in \Lambda} I_\lambda$ and $I_\lambda \neq \emptyset$ for all $\lambda\in \Lambda$.

Since $\Lambda$ is a partition of $I$, for any $i \in I$ there exists \emph{only one} $\lambda \in \Lambda$ such that $i \in I_\lambda$. Therefore there exists a well-defined map $j: I\to \Lambda$ with $i\in I_{j\brac{i}}$ for every $i\in I$. If $\lambda\in \Lambda$, then for any $i\in I_\lambda$, $I_\lambda \cap I_{j\brac{i}}\neq \emptyset$, whence $j\brac{i} = \lambda$, because $\brac{I_\lambda}_{\lambda\in \Lambda}$ are pairwise disjoint. Thus $j\brac{i} = \lambda$ if and only if $i\in I_\lambda$. Denote $\lambda_i\defn j\brac{i}$.\\

\label{thm:product_partition} \noindent \textbf{Theorem} 6-3.
Let $\brac{\Omega_i, \Fcal_i}_{i\in I}$ be a family of measurable spaces, and $\brac{I_\lambda}_{\lambda\in \Lambda}$ be a partition of $I$. Then $\prod_{i\in I} \Omega_i$ and $\prod_{\lambda\in \Lambda}\brac{ \prod_{i\in I_\lambda} } \Omega_i$ are equivalent in the sense that there exists a natural bijection between these sets. Furthermore the same bijection preserves the structure and maps in a one-to-one fashion rectangles $\coprod_{i\in I} \Fcal_i$ and $\coprod_{\lambda\in \Lambda} \brac{ \coprod_{i\in I_\lambda} \Fcal_i }$ between one another. Consequently, the following $\sigma$-algebras are equivalent\[ \bigotimes_{i\in I} \Fcal_i\,\text{and}\, \bigotimes_{\lambda\in \Lambda} \brac{ \bigotimes_{i\in I_\lambda} \Fcal_i }\]

Let $\Omega' \defn \prod_{\lambda \in \Lambda} \brac{ \prod_{i \in I_\lambda}\Omega_i }$ and $\Omega \defn \prod_{i \in I}\Omega_i$, where $\brac{I_\lambda}_{\lambda \in \Lambda}$ is a partition of $I$. Consider a map $\Phi$ defined by: $\Phi\brac{\omega} = \brac{\induc{\omega}_{I_\lambda}}_{\lambda \in \Lambda}$, where $\induc{\omega}_{i \in I}$ is the restriction of the map $\omega$ to domain $I_\lambda$. This map $\Phi$ is defined well, because restriction of the domain of any map is a well defined operation. For any $\omega \in \Omega$ and any $\lambda\in \Lambda$ \[\Phi\brac{\omega}\brac{\lambda} = \brac{\Phi\brac{\omega}}_\lambda = \induc{ \omega }_{I_\lambda} \in \prod_{i\in I_\lambda} \Omega_i\] whence $\Phi\brac{\omega}\in \Omega'$. Thus $\Phi$ is a well-defined map $\Omega \to \Omega'$.

First, $\Phi$ is a surjective map. Indeed, let $\theta \in \Omega'$, then $\theta\brac{\lambda} \in \prod_{i \in I_\lambda} \Omega_i$ for all $\lambda \in \Lambda$, whence $\theta\brac{\lambda}\brac{i} \in \Omega_i$ for all $i \in I$ since $I=\bigcup_{\lambda\in \Lambda} I_\lambda$. If $\omega_i \defn \theta\brac{\lambda_i}\brac{i}$, then $\omega_i \in \Omega_i$ for any $i \in I$ and $\omega\defn \brac{\omega_i}_{i\in I}$ is a map from $I$ to $\bigcup_{i \in I}\Omega_i$, whence $\omega \in \Omega$.

If $\lambda\in \Lambda$ is fixed, then for all $i\in I_\lambda$ \[\Phi\brac{\omega}\brac{\lambda}\brac{i} = \induc{\omega}_{I_\lambda}\brac{i} = \induc{\omega}_{I_{\lambda_i}}\brac{i} = \omega_i = \theta\brac{\lambda_i}\brac{i} = \theta\brac{\lambda}\brac{i}\] since $\lambda_i = \lambda$ for every $i \in I_\lambda$. Hence $\Phi\brac{\omega}\brac{\lambda} = \theta\brac{\lambda}$ for any $\lambda \in \Lambda$, and so $\Phi\brac{\omega} = \theta$.

Suppose there are $\alpha, \beta \in \Omega$ such that $\Phi$ maps both into the same element of $\Omega'$. Then $\induc{\alpha}_{I_\lambda}\brac{i}=\induc{\beta}_{I_\lambda}\brac{i}$ for every $\lambda \in \Lambda$ and all $i\in I_\lambda$. Thus for any $i\in I$ since restriction is well-defined \[\alpha_i = \induc{\alpha}_{I_{\lambda_i}}\brac{i} = \induc{\beta}_{I_{\lambda_i}}\brac{i} = \beta_i\] whence $\Phi$ is and injective projection. This bijection is natural given a partition of the index set $I$, since it just re-groups the domain of the elements of $\Omega$ according to the given non-overlapping sub-domains. 

This point-wise bijection $\Phi$ can be extended to a set-mapping. Indeed for any subset $A\subseteq \Omega$ the direct image of $A$ by $\Phi$ is a subset of $\Omega'$.

By definitions of the direct and the inverse image by $\Phi$, $A \subseteq \Phi^{-1}\brac{\Phi\brac{A}}$. If $\omega\in \Phi^{-1}\brac{\Phi\brac{A}}$ then $\Phi\brac{\omega} \in \Phi(A)$ and there is some $z \in A$, such that $\Phi\brac{z} = \Phi\brac{\omega}$. If $\omega \notin A$, then $z$ and $\omega$ are completely different, yet are mapped into the same value by $\Phi$, which contradicts the fact the $\Phi$ is an injection. Therefore $\omega \in A$, and $\Phi^{-1}\brac{\Phi\brac{A}} \subseteq A$.

If $A, B \subseteq \Omega$ are such that $\Phi\brac{A} = \Phi\brac{B}$, then $A = \Phi^{-1}\brac{\Phi\brac{A}}$ and $B = \Phi^{-1}\brac{\Phi\brac{B}}$, whence $A = B$. Therefore the direct image by $\Phi$ is an injective set-map.

If $A' \subseteq \Omega'$, then $\Phi\brac{\Phi^{-1}\brac{A'}} \subseteq A'$, since for a general map not all elements from the co-domain may have a corresponding element in the domain. Now, let $\omega' \in A'$. Since $\Phi$ is surjective, there exists some $\omega \in \Omega$ such that $\Phi\brac{\omega} = \omega'$, whence $\omega \in \Phi^{-1}\brac{A'}$ and $\omega' \in \Phi\brac{\Phi^{-1}\brac{A'}}$. Therefore $A' \subseteq \Phi\brac{\Phi^{-1}\brac{A'}}$. So for any $A' \subseteq \Omega'$ the set $A = \Phi^{-1}\brac{A'}$ is such that $\Phi\brac{ A } = A'$, whence the direct image by $\Phi$ is a surjective map. Consequently, the direct image by $\Phi$ is a natural bijective map between $\Omega$ and $\Omega'$.

In addition the map $\Phi$ is natural because it preserves the generalised Cartesian product structure. Indeed, if $\theta \in \Phi\brac{\prod_{i\in I} A_i}$ then there is $\omega \in \prod_{i\in I} A_i$ with $\theta = \Phi\brac{\omega}$. If $\lambda\in \Lambda$, then for all $i\in I_\lambda$ \[\theta\brac{\lambda}\brac{i} = \theta\brac{\lambda_i}\brac{i} = \omega_i\] whence $\theta\brac{\lambda}\in \prod_{i\in I_\lambda} A_i$ for every $\lambda\in \Lambda$. Thus $\Phi\brac{\prod_{i\in I} A_i} \subseteq \prod_{\lambda\in\Lambda} \brac{\prod_{i\in I_\lambda} A_i}$.

Conversely, for any $\theta \in \prod_{\lambda\in\Lambda} \brac{ \prod_{i\in I_\lambda} A_i }$, there exists $\omega\in \Omega$ such that $\Phi\brac{\omega} = \theta$. Since $\theta\brac{\lambda}\in \prod_{i\in I_\lambda} A_i$ for any $\lambda\in \Lambda$, $\theta\brac{\lambda_i}\brac{i} \in A_i$ for every $i \in I$. However $\omega_i = \theta\brac{\lambda_i}\brac{i}$ for all $i\in I$, whence $\omega \defn \brac{\omega_i}_{i\in I}\in \prod_{i\in I} A_i$. Therefore $\theta\in \Phi{\prod_{i\in I} A_i}$ and \[\Phi\brac{\prod_{i\in I} A_i} = \prod_{\lambda\in\Lambda} \brac{\prod_{i\in I_\lambda} A_i}\]

If $A = \prod_{i\in I} A_i\in \coprod_{i\in I} \Fcal_i$, then for any $\lambda\in \Lambda$, $J_A\cap I_\lambda$ is finite whence $\prod_{i\in I_\lambda} A_i \in \coprod_{i\in I_\lambda} \Fcal_i$. Furthermore the set \[J_\Lambda\defn \obj{ \induc{ \lambda\in \Lambda } J_A \cap I_\lambda \neq \emptyset}\] is finite since $\brac{I_\lambda}_{\lambda\in \Lambda}$ is a partition of $I$ and $J_A$ is finite. Hence the set \[J'\defn \obj{ \induc{ \lambda \in \Lambda } \prod_{i\in I_\lambda} A_i \neq \prod_{i\in I_\lambda} \Omega_i }\] is at most finite since it is a subset of $J_\Lambda$. Consequently $\Phi\brac{A} \in \coprod_{\lambda\in \Lambda}\brac{ \coprod_{i\in I_\lambda} \Fcal_i }$.

If $A = \prod_{\lambda\in \Lambda}\brac{ \prod_{i\in I_\lambda} A_i } \in \coprod_{\lambda\in \Lambda}\brac{ \coprod_{i\in I_\lambda} \Fcal_i }$, then $A = \Phi\brac{\prod_{i\in I} A_i}$, as $\Phi$ honours the product structure. But $A$ is such that $\prod_{i\in I_\lambda} A_i\in \coprod_{i\in I_\lambda} \Fcal_i$ for any $\lambda\in \Lambda$ and \[J_A\defn \obj{ \induc{\lambda \in \Lambda } \brac{\prod_{i\in I_\lambda} A_i} \neq \prod_{i\in I_\lambda} \Omega_i}\] is finite. Therefore the set\[J'_A \defn \obj{ \induc{ i\in I } A_i \neq \Omega_i } = \biguplus_{\lambda\in \Lambda} \obj{ \induc{ i\in I_\lambda } A_i \neq \Omega_i } = \biguplus_{\lambda\in J_A} \obj{ \induc{ i\in I_\lambda } A_i \neq \Omega_i } \] is at most finite, since $\obj{ \induc{ i\in I_\lambda } A_i \neq \Omega_i }$ is at most finite for every $\lambda \in \Lambda$, whence $\prod_{i\in I} A_i\in \coprod_{i\in I} \Fcal_i$. In conclusion, the natural bijection $\Phi$ identifies $\coprod_{i\in I} \Fcal_i$ and $\coprod_{\lambda\in \Lambda} \brac{\coprod_{i\in I_\lambda} \Fcal_i}$ as well.

Consequently \[\coprod_{i\in I} \Fcal_i \equiv \coprod_{\lambda\in \Lambda} \brac{ \coprod_{i\in I_\lambda} \Fcal_i } \subseteq \coprod_{\lambda\in \Lambda} \brac{ \bigotimes_{i\in I_\lambda} \Fcal_i } \subseteq \bigotimes_{\lambda\in \Lambda}\brac{ \bigotimes_{i\in I_\lambda} \Fcal_i }\] where the second inclusion is obvious. Finally by virtue of to theorem 26 \[ \bigotimes_{\lambda\in \Lambda} \brac{ \bigotimes_{i\in I_\lambda} \Fcal_i } = \sigma\brac{\coprod_{\lambda\in \Lambda} \sigma\brac{ \coprod_{i\in I_\lambda} \Fcal_i }} = \sigma\brac{ \coprod_{\lambda\in \Lambda} \coprod_{i\in I_\lambda} \Fcal_i } \subseteq \sigma\brac{\coprod_{i\in I} \Fcal_i }\]\\

\noindent \textbf{Definition} 56.
Let $\brac{\Omega_i, \Tcal_i}$ be a family of topological spaces, indexed by a non-empty set $I$. The product topology of $\brac{\Tcal_i}_{i\in I}$ is a topology on $\prod_{i\in I} \Omega_i$, generated by all rectangles of $\brac{\Tcal_i}_{i\in I}$: \[\bigodot_{i\in I} \defn \Tcal\brac{ \coprod_{i\in I} \Tcal_i }\]

In fact $\coprod_{i\in I} \Tcal_i$ is a basis of the topology $\bigodot_{i\in I} \Tcal_i$. Indeed $\prod_{i\in I}\Omega_i\in \coprod_{i\in I} \Tcal_i$, so for any $\omega\in \prod_{i\in I}\Omega_i$ there is $V\in \coprod_{i\in I} \Tcal_i$ such that $\omega\in V$. Since a topology is closed under finite intersections, $U\cap V \coprod_{i\in I} \Tcal_i$ for any $U, V \coprod_{i\in I} \Tcal_i$. Hence if $\omega\in U\cap V$, then $\exists W\in \coprod_{i\in I} \Tcal_i$ with $\omega\in W\subseteq U\cap V$.

Therefore \[\Tcal \defn \obj{ \induc{ U\subseteq \prod_{i\in I} \Omega_i } \forall x\in U,\, \exists V\in \coprod_{i\in I}\Tcal_i\;\text{s.t.}\; x\in V\subseteq U } \] is a topology on $\prod_{i\in I} \Omega_i$ such that $\coprod_{i\in I} \Tcal_i \subseteq \Tcal$. Conversely, for any $U\in \Tcal$ there is $\Gamma \subseteq \coprod_{i\in I} \Tcal_i$, such that $U \equiv \bigcup_{V\in \Gamma} V$, and thus $U\in \bigodot_{i\in I} \Tcal_i$, because $\Gamma \subseteq \Tcal\brac{\coprod_{i\in I} \Tcal_i}$. Now \[\bigotimes_{i\in I} \borel{\Omega_i} \equiv \sigma\brac{\coprod_{i\in I} \Tcal_i}\subseteq \sigma\brac{\bigodot_{i\in I} \Tcal_i} \equiv \borel{\prod_{i\in I} \Omega_i}\]

\label{thm:fin_prod_space_metr} \noindent \textbf{Theorem} 6-3-1.
Let $\brac{\Omega_k, \Tcal_k}_{k=1}^n$ be a finite collection of metrizable topological spaces, and let $d_k$ be the metric on $\Omega_k$ such that $\Tcal_k \equiv \Tcal_{\Omega_k}^{d_k}$. Then there exists a metric which induces the product topology on $\prod_{k=1}^n\Omega_k$. %% \noindent \textbf{Exercise} 14. Let $\brac{\Omega_k, \Tcal_k}_{k=1}^n$ be a finite collection of metrizable topological spaces, and let $d_k$ be the metric on $\Omega_k$ such that $\Tcal_k \equiv \Tcal_{\Omega_k}^{d_k}$.

Let $\Omega \defn \prod_{k=1}^n\Omega_k$ and define the metric $d$ on $\Omega$ \[d\brac{x,y} \defn \sqrt{ \sum_{k=1}^n d^2_k\brac{x_k, y_k} }\] This is indeed a metric, since each $d_k$ is a metric and $\abs{\sum_k a_k b_k} \leq \sqrt{\sum_k a_k^2 }\sqrt{\sum_k b_k^2}$.

Since $\coprod_{k=1}^n \Tcal_k$ is the topological basis of $\bigodot_{k=1}^n \Tcal_k$, for any $U$ -- open in $\Omega$ and any $x\in U$, there are $U_k\in \Tcal_k$ for all $k=1\ldots n$ such that $x\in U_1\times \ldots \times U_n\subseteq U$. Since $\Tcal_k$ is induced by $d_k$, for each $k=1\ldots n$ $\exists \epsilon_k>0$ such that $\forall y_k\in \Omega_k$ with $d_k\brac{x_k, y_k} < \epsilon_k$, $y_k\in U_k$. Hence, for $\epsilon = \min_{k=1\ldots n} \epsilon_k > 0$ it is true that for all $y\in \prod_{k=1}^n \Omega_k$ with $d\brac{x, y} < \epsilon$, for all $k=1\ldots n$ \[d_k\brac{x_k, y_k} \leq d\brac{x, y} < \epsilon \leq \epsilon_k\] Hence $y\in U_1\times \ldots \times U_n\subseteq U$. Thus for every $x\in U$ there is $\epsilon>0$ such that $\forall y\in \Omega$ if $d\brac{x,y} < \epsilon$ then $y\in U$, whence $U\in \Tcal_\Omega^d$. Therefore, $\bigodot_{k=1}^n \Tcal_k \subseteq \Tcal_\Omega^d$ -- the metric topology on $\prod_{k=1}^n \Omega_k$ induced by $d$.

Conversely, if $U\in \Tcal_\Omega^d$, then for all $x\in U$ $\exists \delta>0$ such that for all $y\in \Omega$ with $d\brac{x,y}<\delta$ $y\in U$. Letting $\epsilon = \frac{\delta}{\sqrt{n}}>0$ implies that for all $k=1\ldots n$ if $y_k\in \Omega_k$ is such that $d_k\brac{x_k, y_k} < \epsilon$ then $y\in U$, where $y=\brac{y_k}_{k=1}^n$, since \[d\brac{x,y} < \sqrt{ \sum_{k=1}^n \frac{\delta^2}{n} } < \delta\] Thus setting $B_k\brac{x_k, \epsilon}\defn \obj{ \induc{ y_k\in \Omega_k } d_k\brac{ x_k, y_k } < \epsilon }$ for $k=1\ldots n$, the prior observation is equivalent to $B_1\brac{x_1, \epsilon}\times\ldots \times B_n\brac{x_n, \epsilon} \subseteq U$. Therefore for every $x\in U$ there exists $W\in \coprod_{k=1}^n \Tcal_k$ with $x\in W$ such that $W\subseteq U$, implying that $U\in \bigodot_{k=1}^n \Tcal_k$.

Various different metrics are ``equivalent'' in that they induce the same metric topology on $\prod_{k=1}^n \Omega_k$. First, due to Cauchy-Schwartz inequality \[\sum_{k=1}^n d_k\brac{ x_k, y_k } \leq \sqrt{ \sum_{k=1}^n d^2_k\brac{ x_k, y_k } } \sqrt{ \sum_{k=1}^n 1 }\] Second, $\sum_{k=1}^n d^2_k\brac{ x_k, y_k } \leq \brac{ \sum_{k=1}^n d_k\brac{ x_k, y_k } }^2$. Therefore \[\frac{1}{\sqrt{n}} \sum_{k=1}^n d_k\brac{ x_k, y_k }\leq d\brac{ x, y }\leq \sum_{k=1}^n d_k\brac{ x_k, y_k }\]

Now, $\brac{d_k\brac{ x_k, y_k }}^2 \leq \sum_{k=1}^n d^2_k\brac{ x_k, y_k }$ for all $k=1\ldots n$ since metrics take non-negative values. Second, since power function is non-decreasing \[\sum_{k=1}^n d^2_k\brac{ x_k, y_k }\leq \sum_{i=1}^n \max_{k=1\ldots n} d^2_k\brac{ x_k, y_k } = \brac{ \max_{k=1\ldots n} d_k\brac{ x_k, y_k } }^2 \sum_{k=1}^n 1 \] Thus \[\max_{k=1\ldots n} d_k\brac{ x_k, y_k }\leq d\brac{x, y}\leq \sqrt{n} \max_{k=1\ldots n} d_k\brac{ x_k, y_k }\]\\

\label{thm:count_prod_space_metr} \noindent \textbf{Theorem} 6-3-2.
Let $\brac{\Omega_n, \Tcal_n}_{n\geq 1}$ be a countable collection of metrizable topological spaces, where $d_n$ is a metric inducing $\Tcal_n$, $n\geq 1$. Then the product space $\brac{\Omega, \Tcal}$ is metrizable, where $\Omega\defn \prod_{n = 1}^\infty \Omega_n$ and $\Tcal\defn \bigodot_{n=1}^\infty \Tcal_n$.%%\textbf{Exercise} 15. Let $\brac{\Omega_n, \Tcal_n}_{n\geq 1}$ be a countable collection of metrizable topological spaces, where $d_n$ is a metric inducing $\Tcal_n$, $n\geq 1$. Let $\Omega\defn \prod_{n = 1}^\infty \Omega_n$ and $\Tcal\defn \bigodot_{n=1}^\infty \Tcal_n$.

First, for all $a,b\geq1$, $a+b>1$ and $1\wedge \brac{a+b} = 1 < 1 + 1 = 1\wedge a + 1\wedge b$. If $0\leq a<1\leq b$ then $a+b\geq 1$ and $1\wedge \brac{a+b} = 1 \leq a + 1 = 1\wedge a + 1\wedge b$. If $0\leq a,b<1$ but $a+b\geq1$ then $1\wedge \brac{a+b} = 1 \leq a + b = 1\wedge a + 1\wedge b$. Finally, if $0\leq a,b<1$ and $a+b\leq1$ then $1\wedge \brac{a+b} = a + b = 1\wedge a + 1\wedge b$. Therefore $1\wedge \brac{a+b} \leq 1\wedge a + 1\wedge b$. for all $a,b\in \Real^+$.

For all $x,y\in \Omega$ define \[d\brac{x, y}\defn \sum_{n=1}^\infty \frac{1}{2^n} \brac{ 1\wedge d_n\brac{x_k, y_k} }\] This $d$ is a metric, since for all $x,y,z\in \Omega$ $d_k\brac{x_k, y_k}\leq d_k\brac{x_k, z_k} + d_k\brac{z_k, y_k}$ for all $n\geq 1$ and thus \[1\wedge d_n\brac{x_k, y_k} \leq 1\wedge \brac{d_k\brac{x_k, z_k} + d_k\brac{z_k, y_k}}\leq 1 \wedge d_k\brac{x_k, z_k} + 1 \wedge d_k\brac{z_k, y_k}\] and any sum of non-negative numbers is zero if and only if each member is zero.

First, $\sfrac{1}{2^k} 1\wedge d_k\brac{x_k,y_k} \leq d\brac{x,y}$ for all $k\geq 1$. If $d\brac{x, y}<\sfrac{1}{2^n}$, then, in particular, $1\wedge d_k\brac{x_k,y_k} < 2^{k-n} \leq 1$ for every $n\geq k$, whence $d_k\brac{x_k,y_k} < 1$. Therefore $\frac{1}{2^k}d_k\brac{x_k,y_k} \leq d\brac{x, y}$ for all $n\geq k$.

Since $\coprod_{n=1}^\infty \Tcal_n \equiv \obj{ \induc{U_1\times\ldots \times U_N \times \prod_{n=N+1}^\infty \Omega_n } N\geq 1, U_k\in \Tcal_k \forall k=1\ldots N }$, for any $U\in \Tcal$ and any $x\in U$, there is $N\geq 1$ and $\brac{U_k}_{k=1}^N$ -- open in $\brac{\Omega_k, \Tcal_k}_{k=1}^N$ such that $x\in U_1\times\ldots \times U_N \times \prod_{n=N+1}^\infty \Omega_n\subseteq U$. Since $U_k\in \Tcal_k$ and $x_k\in U_k$, there exists $\epsilon_k>0$ with $B_k\brac{x_k, \epsilon_k} \subseteq U_k$ for every $k=1\ldots N$. Letting $\delta = \frac{1}{2^N} \wedge \min_{k=1\ldots N} \frac{\epsilon_k}{2^k} > 0$, for all $y\in \Omega$ with $d\brac{x, y} < \delta$, $d\brac{x, y}<\sfrac{1}{2^N}$ so $d_k\brac{x_k,y_k} \leq 2^k d\brac{x, y}< 2^k \delta\leq 2^k \frac{\epsilon_k}{2^k}$ for all $N \geq k$. Therefore $y_k\in U_k$ for all $k=1 \ldots N$ whence $y\in U_1\times\ldots \times U_N \times \prod_{n=N+1}^\infty \Omega_n\subseteq U$. Therefore for every $x\in U$ there is $\delta>0$ such that if $y\in \Omega$ with $d\brac{x, y}<\delta$ then $y\in U$, so $U\in T_\Omega^d$.

Suppose $U\in \Tcal_\Omega^d$ and $x\in U$, then $\exists \epsilon>0$ such that if $d\brac{x,y}<\epsilon$ then $y\in U$. Letting $N\defn\left \lfloor -\log_2 \epsilon\right \rfloor+1$, gives $\delta = \epsilon - \frac{1}{2^N}>0$. Then \[d\brac{ x, y } \leq \sum_{k=1}^N \frac{1}{2^k} \brac{ 1\wedge d_k\brac{ x_k, y_K } } + \sum_{k=N+1}^\infty \frac{1}{2^k} < \delta + \frac{1}{2^N} = \epsilon \] for all $y\in \Omega$ with $\sum_{k=1}^N 2^{-k} \brac{ 1\wedge d_k\brac{ x_k, y_K } } < \delta$. Hence for all $U\in \Tcal_\Omega^d$ and $x\in U$, there is $N\geq 1$ and $\epsilon>0$ such that $y\in U$ for every $y\in \Omega$ with $y_k\in B_k\brac{x_k, \epsilon}$ for $k=1\ldots N$, because \[\sum_{k=1}^N 2^{-k} \brac{ 1\wedge d_k\brac{ x_k, y_K } } < \frac{\epsilon}{2} \brac{ 1-\frac{1}{2^N} }\] Therefore $\exists N\geq 1$ and $\exists \brac{U_k}_{k=1}^N$ -- open in $\brac{\Omega_k, \Tcal_k}_{k=1}^N$ such that $x\in U_1\times\ldots \times U_N \times \prod_{k=N+1}^\infty \Omega_k\subseteq U$, whence $U\in \Tcal = \bigodot_{k=1}^\infty \Tcal_k$. In summary, the topological space $\brac{\prod_{k=1}^\infty \Omega_k, \bigodot_{k=1}^\infty \Tcal_k}$ is metrizable.\\

\noindent \textbf{Definition} 57.
Let $\brac{\Omega, \Tcal}$ be a topological space. A collection $\mathcal{H}\subseteq \Tcal$ is called a countable base of $\brac{\Omega, \Tcal}$ if $\mathcal{H}$ is countable and $\Tcal = \Tcal\brac{\mathcal{H}}$: \[\Tcal = \obj{ \induc{ \bigcup_{ V\in \Gamma } V } \Gamma \subseteq \mathcal{H} }\]

The last equivalence implies that any topological subspace $\brac{\Omega', \induc{\Tcal}_{\Omega'}}$ with $\Omega'\subseteq \Omega$ has a countable base if $\brac{\Omega, \Tcal}$ has a countable base. If $f:\brac{\Omega, \Tcal}\to \brac{\Scal, \Tcal_\Scal}$ is a homomorphism, then for a countable base $\mathcal{H}$ of $\brac{\Omega, \Tcal}$, $\brac{ f\brac{V} }_{V\in \mathcal{H}}$ is a countable base of $f\brac{\Scal, \Tcal_\Scal}$. Therefore since $\brac{\Real, \Tcal_\Real}$ has a countable base $\mathcal{H} \defn \obj{ \induc{ \brac{r,q} } r,q\in \mathcal{Q} }$, $\brac{ \clo{-1, 1 }, \Tcal_{ \clo{ -1, 1 } } }$ has a countable base and so does $\brac{\Rbar, \Tcal_\Rbar}$.\\

\label{thm:topo_count_base} \noindent \textbf{Theorem} 27.
Let $\brac{\Omega_n,\Tcal_n}_{n\geq 1}$ be a sequence of topological spaces with countable base. The product space $\brac{ \prod_{k=1}^\infty \Omega_k, \bigodot_{k=1}^\infty \Tcal_k}$ has a countable base and \[\borel{\prod_{k=1}^\infty} \Omega_k \equiv \bigotimes_{k=1}^\infty \borel{\Omega_k}\]

Indeed, for $n\geq 1$, let $\obj{ \induc{V_n^k} k\in I_n }$ be a countable base $\brac{\Omega_n, \Tcal_n}$ where $I_n$ is at most countable. Let $\Omega\defn \prod_{k=1}^\infty \Omega_k$ and $\Tcal \defn \bigodot_{k=1}^\infty \Tcal_k$. For all $p\geq 1$ let\[ \mathcal{H}^p \defn \obj{ \induc{ V_1^{k_1} \times \ldots \times V_p^{k_p} \times \prod_{k=p+1}^\infty \Omega_k } \brac{k_1, \ldots,  k_p}\in I_1\times \ldots \times I_p }\] and put $\mathcal{H} \defn \bigcup_{p\geq 1} \mathcal{H}^p$.

For any $p\geq 1$ the generic element of $\mathcal{H}^p$ is an element of the $\coprod_{n=1}^\infty \Tcal_n\subseteq \Tcal$, whence $\mathcal{H}\subseteq \Tcal$. For any $p\geq 1$ the map $j_p:\mathcal{H}^p\to \mathbb{N}^p$ defined for every $V\in \mathcal{H}^p$ as $j_p\brac{V}\defn \brac{k_i}_{i=1}^p\in \mathbb{N}^p$ such that $V \equiv V_1^{k_1}\times \ldots \times V_1^{k_1}\times \prod_{k=p+1}^\infty \Omega_k$ is an injective map.

The map $\phi_2:\mathbb{N}\times \mathbb{N}\to \mathbb{N}$ defined as $\phi_2\brac{n,m}\defn h\brac{n+m} + n$, with $h\brac{k}\defn \sum_{i=1}^k i$ representing the number of pairs on all ``diagonals'' before the $k$-th. This map is a bijection, since for all $k\geq 0$ and $n=0\ldots k$ \[0 \leq n-h\brac{k} \leq k < k + 1 = h\brac{k+1}-h\brac{k}\] Based on this map it is possible to construct a bijective map $\phi_k:\mathbb{N}^k\to \mathbb{N}$ inductively for all $k\geq 1$ as $\phi_k\brac{\brac{n_i}_{i=1}^k} \defn \phi_2\brac{ \phi_{k-1}\brac{ \brac{n_i}_{i=1}^{k-1} }, n_k }$, where $\phi_1\brac{n} \defn n$.

For every $p\geq 1$ $\psi_p\defn \phi_p\circ j_p$ is an injective map, that enumerates every element (open rectangle) of $\mathcal{H}^p$. The map $j:\mathcal{H}\to \mathbb{N}$ defined for $V\in \mathcal{H}$ as $j\brac{V}\defn \brac{p, \psi_p\brac{V}}$, with $p\geq 1$ such that $V\in \mathcal{H}^p$, is injective. Therefore the map $\psi \defn \phi_2\circ j$ is injective as well, whence $\abs{ \mathcal{ H } } \leq \abs{ \mathbb{ N } }$ and $\mathcal{H}$ is at most countable.

For every $V\in \coprod_{k=1}^\infty \Tcal_k$, $J_V\defn \obj{ \induc{ k\geq 1 } V_k\neq \Omega_k }$ is finite, so $\exists p\geq 1$ such that $k\leq p$ for all $k\in J_V$. Thus $V\equiv W_1\times \ldots \times W_N \times \prod_{k=N+1}^\infty \Omega_k$ for some $N\geq 1$ and some $W_k$ -- open in $\brac{ \Omega_k, \Tcal_k}$ for all $k=1\ldots N$.

For any $U\in \Tcal$ and $x\in U$ there is $V\in \coprod_{k=1}^\infty \Tcal_k$ such that $x\in V\subseteq U$. Hence there is $p\geq 1$ and $\brac{U_k}_{k=1}^p$ -- open in $\brac{\Omega_k, \Tcal_k}_{k=1}^p$ such that \[x\in U_1\times \ldots \times U_p\times \prod_{k=p+1}^\infty \Omega_k \subseteq U\] Thus for all $U\in \Tcal$ and $x\in U$ there is $V_x\in \mathcal{H}$ with $x\in V_x\subseteq U$

Hence for every $U\in \Tcal$ $\exists \Gamma\subseteq \mathcal{H}$ such that $U\equiv \bigcup_{V\in \Gamma} V$ and $\Gamma$ is at most countable. Since $\mathcal{H}\subseteq \coprod_{k=1}^\infty \Tcal_k$ \[\Tcal \subseteq \bigotimes_{k=1}^\infty \borel{\Omega_k}\]

If $\brac{\Omega_k,\Tcal_k}_{k=1}^n$ is a finite family of topological spaces with countable base, then analogous argument implies that $\brac{\prod_{k=1}^n \Omega_k, \bigodot_{k=1}^n \Tcal_k}$ has a countable base and $\borel{\prod_{k=1}^n \Omega_k} \equiv \bigotimes_{k=1}^n \borel{\Omega_k}$. Moreover $\borel{ \Omega^n\times \Omega^m} \equiv \bigotimes_{k=1}^n \borel{\Omega} \otimes \bigotimes_{k=1}^m \borel{\Omega}$. Thus if $\brac{\Omega, \Tcal}$ has a countable base $\borel{\Omega^n} \equiv \bigotimes_{k=1}^n\borel{\Omega}$, and further $\borel{\Real^n} \equiv \bigotimes_{k=1}^n \borel{\Real}$. Also $\borel{\Cplx} \equiv \borel{\Real\times\Real} \equiv \borel{\Real}\otimes \borel{\Real}$.\\

\noindent \textbf{Definition} 58.
A metric space $\brac{E, d}$ is separable if there exists an at most countable dense subset of $E$.

\label{thm:metric_separab}\textbf{Theorem} 6-4.
A metric space $\brac{E, d}$ is separable if and only if the topological space $\brac{E, \Tcal}$ has a countable base and $\Tcal$ is metrizable by $d$.

If a metric space $\brac{E,d}$ is separable then there exists $X\defn \obj{ \induc{ x_n } n\geq 1 }$ such that $X \equiv \clo{E}$. Then $\mathcal{H} \defn \obj{ \induc{ B^d_E\brac{x_n, \sfrac{1}{p} } } n,p\geq 1 }$ is an at most countable collection of open sets in $\brac{E,d}$. For every $U\in \Tcal_E^d$ and $x\in U$ $\exists \epsilon>0$ such that $B_E^d\brac{x, \epsilon}\subseteq U$. Hence there is $p\geq 1$ with $\sfrac{1}{p} \leq \sfrac{ \epsilon }{2}$ and $\exists n\geq 1$ with $x_n\in B_E^d\brac{x, \sfrac{1}{p}}$. Thus \[x\in _E^d\brac{x_n, \sfrac{1}{p}}\subseteq B_E^d\brac{x, \sfrac{2}{p}} \subseteq B_E^d\brac{x, \epsilon}\] whence there is $V_x\in \mathcal{H}$ such that $x\in V_x\subseteq U$. Therefore $\mathcal{H}$ is a countable base of the metric topological space $\brac{E, \Tcal_E^d}$.

Conversely, suppose $\brac{E, \Tcal_E^d}$ has a countable base $\mathcal{H}$. Let $g:\pwr{E}\setminus \obj{\emptyset }\to E$ be a choice function (axiom of choice theorem 25) and let $X\defn \obj{ \induc{ g\brac{V} } V\in \mathcal{H} }$. Then $\forall x\in E$ and every $U\in \Tcal_E^d$ with $x\in V$, there is $V\in \mathcal{H}$ such that $x\in V\subseteq U$ and $V\cap X\neq \emptyset$, since $g\brac{V}\in X$ and $g\brac{M}\in M$ for every non-empty $M\subseteq E$. Therefore $\clo{X} = E$ is a dense subset and by construction is at most countable.

If $\brac{E_n, d_n}_{n\geq 1}$ be a sequence of separable metric spaces, then each $\brac{E_n, \Tcal_{E_n}^{d_n}}$ has a countable base and is metrizable. Hence $\brac{ \prod_{n=1}^\infty E_n, \bigodot_{n=1}^\infty \Tcal_n }$ is metrizable by some metric $d$ and has a countable base. As such $\brac{\prod_{n=1}^\infty E_n, d}$ must be separable.\\

\label{thm:meas_concat} \noindent \textbf{Theorem} 28.
Let $\brac{\Omega_i, \Fcal_i}_{i\in I}$ be a family of measurable spaces and $\brac{\Omega, \Fcal}$ -- a measurable space. Then $\brac{f_i}_{i\in I}:\brac{\Omega, \Fcal}\to\brac{\Omega_i, \Fcal_i}$ are measurable if and only if the map $f:\Omega\to \prod_{i\in I} \Omega_i$ defined by $f\brac{\omega}\defn \brac{f_i\brac{\omega}}_{i\in I}$ is $\Fcal$--$\bigotimes_{i\in I} \Fcal_i$ measurable.

Indeed, for any $A = \prod_{i\in I} A_i\in \coprod_{i\in I} \Fcal_i$, $f^{-1}\brac{\prod_{i\in I} A_i} = \bigcap_{i\in I} f^{-1}_i\brac{A_i} = \bigcap_{i\in J_A} f^{-1}_i\brac{A_i}$, because $f^{-1}_i\brac{\Omega_i} = \Omega$ for every $i\in I$, where $J_A\defn \obj{ \induc{i\in I} A_i\neq \Omega_i }$. If $\brac{f_i}_{i\in I}:\brac{\Omega, \Fcal}\to\brac{\Omega_i, \Fcal_i}$ are measurable then this implies that $f^{-1}\brac{A}\in \Fcal$, since $J_A\subseteq I$ is finite.

Conversely, if such $f:\brac{\Omega, \Fcal}\to \brac{\prod_{i\in I} \Omega_i, \bigotimes_{i\in I} \Fcal_i}$ is measurable then for all $j \in I$ and every $B_j\in \Fcal_j$, $B\defn B_j\times \prod_{i\neq j,\,i\in I} \Omega_i \in \coprod_{i\in I} \Fcal_i$ and so $f_j^{-1}\brac{B_j} = f_j^{-1}\brac{B_j} \cap \bigcap_{i\neq j,\,i\in I} f_i\brac{\Omega_i} = f^{-1}\brac{B} \in \Fcal$.\\

In fact let $\brac{\Omega_i, \Fcal_i}_{i\in I}$ be a family of measurable spaces and $\Omega\defn \prod_{i\in I}\Omega_i$. For every $j\in I$ the map $\pi_j:\Omega\to\Omega_j$ defined as $\pi_j\brac{\omega} = \omega\brac{j}$ for any $\omega\in \Omega$ is $\bigotimes_{i\in I} \Fcal_i$--$\Fcal_j$ measurable. Indeed, for any $B_j\in \Fcal_j$, the set $\pi_j^{-1}\brac{ B_j }\equiv \prod_{i\in I} B_i$ with $B_i=\Omega_i$ for $i\neq j$ is a measurable rectangle of the family $\brac{\Fcal_i}_{i\in I}$. Therefore, for any $B\in \Fcal_j$, $\pi_j^{-1}\brac{B}\in \bigotimes_{i\in I} \Fcal_i$. Therefore for any non-empty subset $J\subseteq I$ the map $\omega\to \brac{\pi_j\brac{\omega}}_{j\in J}$ is $\bigotimes_{i\in I} \Fcal_i$--$\bigotimes_{j\in J} \Fcal_j$ measurable by theorem 28.\\

\label{thm:metric_product_contin} \noindent \textbf{Theorem} 6-5.
If $\brac{E,d}$ is a metric space, then the metric $d$ is $\Tcal_{E\times E}$-$\Tcal_{\Real^+}$ continuous and $\borel{E}\otimes\borel{E}$-$\borel{\Rbar}$.measurable.

Indeed, by the triangle law $d\brac{x,y} \leq d\brac{x,y'} + d\brac{y',y} \leq d\brac{x,x'} + d\brac{x',y'} + d\brac{y',y}$ for any $x,y,x',y'\in E$, whence \[ \abs{ d\brac{x,y} - d\brac{x',y'}} \leq d\brac{x,x'} + d\brac{y,y'} \] Via the $\epsilon$-$\delta$ criteria of continuity on metric spaces, this implies that the metric $d$ is a continuous map from the product space $\brac{E\times E, \Tcal_{E\times E}}$ to $\brac{\Real^+, \Tcal_{\Real^+}}$, where the product topology $\Tcal_{E\times E} = \Tcal_E^d \odot \Tcal_E^d$ is metrizable by theorem 6- 3-1 by a metric $\delta{\brac{x,y}, \brac{x',y'}} \defn d\brac{x, x'} + d\brac{y,y'}$.

Therefore $d:\brac{E\times E, \borel{E\times E} }\to \brac{\Rbar, \borel{\Rbar}}$ is measurable, since continuous maps are measurable with respec to Borel $\sigma$-algebras. If $\brac{E, d}$ is separable, then $\brac{E, \Tcal_E^d}$ has a countable base by theorem 6-4, whence $\Tcal_{E\times E}$ has a countable base made of open rectangles and by theorem 27 $\borel{E\times E} = \borel{E} \otimes \borel{E}$. Therefore $d:\brac{E\times E, \borel{E}\otimes \borel{E}}\to \brac{\Rbar, \borel{\Rbar} }$ is measurable when $\brac{E, d}$ is separable.\\

If $f,g:\brac{\Omega, \Fcal}\to \brac{ E, \borel{E} }$ are measurable, then by theorem 28 $\Psi\brac{\omega}\defn\brac{f\brac{\omega}, g\brac{\omega}}$ is a $\Fcal$-$\borel{E}\otimes \borel{E}$ measurable map. Therefore $d\circ \Psi:\brac{\Omega, \Fcal}\to \brac{\Rbar, \borel{\Rbar}}$ is measurable, and hence $\obj{ f = g } = \obj{ d\brac{f\brac{\omega}, g\brac{\omega} } \in \obj{ 0 } } \in \Fcal$.\\

\label{thm:count_sepa_metric_spaces} \noindent \textbf{Theorem} 6-6.
Let $\brac{E_n, d_n}_{n\geq1}$ be a sequence of separable metric spaces. Then the product space $E\defn \prod_{n\geq1} E_n$ is metrizable and separable.

Indeed, for each $n\geq1$ define $\Tcal_{E_n}$ as the natural topology on the metric space $\brac{E_n, d_n}$, the metric topology $\Tcal_{E_n}^{d_n}$. Put $\Tcal\defn \bigotimes_{n\geq1} \Tcal_{E_n}$. By theorem 6-3-2 the product space $\brac{E, \Tcal}$ is metrizable by some metric $d$.

Now by theorem 6-4 each topological space $\brac{E_n, \Tcal_{E_n}}$ has a countable base, whence by theorem 27 the space $\brac{E, \Tcal}$ is also endowed with a countable base. Finally, by theorem 6-4 the facts that $\brac{E, \Tcal}$ is metrizable and has a countable base imply that $\brac{E,d}$ is separable.\\

\label{thm:alt_produc_and_sum_measurable} \noindent \textbf{Theorem} 6-7.
Let $\brac{\Omega, \Fcal}$ be a measurable space and $f,g:\brac{\Omega, \Fcal}\to \brac{\Real, \borel{\Real}}$ are measurable maps. Then $f\cdot g$ and $f+g$ are $\Fcal$-$\borel{\Real}$ measurable.

Indeed, if $\phi:\Real^2\to \Real$ is defined as $\phi\brac{x,y}\defn x+y$, then $\abs{\phi\brac{x,y}-\phi\brac{x',y'}}\leq \abs{x-x'}+\abs{y-y'}$, whence $\phi$ is continuous and therefore $\borel{\Real^2}$--$\borel{\Real}$ measurable. As $\brac{\Real, \Tcal_\Real}$ has a countable base, this $\phi$ is $\borel{\Real}\otimes \borel{\Real}$-$\borel{\Real}$ measurable as well.

If $\psi:\Real^2\to \Real$ is defined as $\psi\brac{x,y}\defn x\cdot y$, then for any $\brac{x,y}, \brac{x',y'}\in \Real^2$: \[ \abs{ xy - x'y' } = \abs{ x\brac{y - y'} + y'\brac{x - x'} } \leq \abs{ x } \abs{y - y'} + \abs{y}\abs{x - x'} + \abs{y - y'}\abs{x - x'}\] Letting $M \defn \abs{x} \vee \abs{y}\geq 0$ yields \[\abs{ xy - x'y' }\leq M \brac{ \abs{x - x'} + \abs{y - y'} } + \frac{1}{2}\brac{ \abs{x - x'} + \abs{y - y'} }^2\] $M=0$ if and only if $\brac{x,y} = \brac{0,0}$. Taking $\delta \defn \sqrt{2\epsilon+M^2}-M>0$ implies that for all $\brac{x',y'}\in \Real^2$ with $\brac{ \abs{x - x'} + \abs{y - y'} } < \delta$ \[ \abs{ xy - x'y' } < M\delta +\frac{\delta^2}{2} = \epsilon\] Therefore by the $\epsilon-\delta$ definition of continuity in metric spaces the map $\psi$ is $\Tcal_{\Real\times \Real}$--$\Tcal_\Real$ continuous, whence it is also $\borel{\Real}\otimes \borel{\Real}$--$\borel{\Real}$ measurable, since $\brac{\Real, \Tcal_\Real}$ has a countable base.

Another more elegant and radical--free method, is to take $\brac{ \abs{x - x'} + \abs{y - y'} } < \delta\leq 1$ and note that in this case $\abs{y'}\leq \abs{y}+\abs{y - y'}\leq \abs{y}+1$. Taking $M\defn \max\brac{ \abs{x}, 1+\abs{y} }> 0$ yields $\abs{xy-x'y'} \leq M\abs{x - x'} + \abs{y - y'}$. Letting $\delta \defn 1 \wedge \frac{\epsilon}{M}>0$ gives the same result.

By theorem 28, the map $\chi\brac{\omega}\defn\brac{f\brac{\omega}, g\brac{\omega}}$ is measurable if and only if $f,g:\brac{\Omega, \Fcal}\to \brac{\Real, \borel{\Real}}$ are measurable. Hence \begin{align*}f\brac{\omega}+g\brac{\omega} &\defn \brac{f+g}\brac{\omega} = \brac{\phi\circ\chi}\brac{\omega}\\ f\brac{\omega}\cdot g\brac{\omega} &\defn \brac{f\cdot g}\brac{\omega} = \brac{\psi\circ\chi}\brac{\omega}\end{align*} are measurable maps with respect to $\Fcal$ and $\borel{\Real}$.\\

% section tut_6 (end)

\section{Fubini Theorem} % (fold)
\label{sec:tut_7}
\url{http://probability.net/PRTfubini.pdf}

\label{thm:fun_section} \noindent \textbf{Theorem} 29.
Let $\brac{S,\Sigma}$, $\brac{\Omega_1, \Fcal_1}$ and $\brac{\Omega_2, \Fcal_2}$ be measurable spaces and $f:\brac{\Omega_1\times \Omega_2, \Fcal_1 \otimes \Fcal_2 }\to \brac{ S, \Sigma }$ be a measurable map. For all $\brac{\omega_1, \omega_2} \in \Omega_1\times \Omega_2$ the map $\omega\to f\brac{\omega_1, \omega}$ is measurable with respect to $\Fcal_2$ and $\Sigma$, and the map $\omega\to f\brac{\omega, \omega_2}$ is $\Fcal_1$--$\Sigma$ measurable.

Fix any arbitrary $\omega_1\in \Omega_1$. The $\omega_1$--section of $E$ is \[E^{\omega_1} \defn \obj{ \induc{ x\in \Omega_2 } \brac{ \omega_1, x } \in E }\] For any $\omega_1\in \Omega_1$ define \[\Gamma^{\omega_1} \defn \obj{ \induc{ E\subseteq \Omega_1\times \Omega_2 } E^{\omega_1}\in \Fcal_2 }\]

Further pick any $E = A_1\times A_2 \in \Fcal_1 \coprod \Fcal_2$. If $\omega_1\in A$, then $E^{\omega_1} \equiv A_2\in \Fcal_2$, otherwise $E^{\omega_1} \equiv \emptyset\in \Fcal_2$. Hence $E\subseteq \Gamma^{\omega_1}$. Now, $\emptyset^{\omega_1} \equiv \emptyset\in \Fcal_2$ and, obviously, $\brac{\Omega_1\times \Omega_2}^{\omega_1} \equiv \Omega_2\in \Fcal_2$. Moreover for any $E\in \Gamma^{\omega_1}$, $\brac{\Omega_1\times \Omega_2\setminus E}^{\omega_1} \equiv \Omega_2\setminus E^{\omega_1} \in \Fcal_2$. Also if $\brac{E_n}_{n\geq 1}\in \Gamma^{\omega_1}$, then $\brac{ \bigcup_{n\geq 1} E }^{\omega_1} \equiv \bigcup_{n\geq 1} E^{\omega_1} \in \Fcal_2$. Therefore $\Gamma^{\omega_1}$ is a $\sigma$-algebra on $\Omega_1\times \Omega_2$, and, consequently, $\Fcal_1 \otimes \Fcal_2\subseteq \Gamma^{\omega_1}$. Thus $E^{\omega_1}\in \Fcal_2$ for all $E\in \Fcal_1 \otimes \Fcal_2$.

Let $h\brac{\omega}\defn f\brac{\omega_1, \omega}$. Then for any $B\in \Sigma$ \[h^{-1}\brac{ B } \equiv \obj{ \induc{ \omega\in \Omega_2 } f\brac{\omega_1, \omega} \in B } \equiv \brac{f^{-1}\brac{B}}^{\omega_1} \in \Fcal_2\] which implies that given $\omega_1\in \Omega_1$ the map $\omega\to f\brac{\omega_1, \omega}$ is measurable. By symmetry the map $\omega\to f\brac{\omega, \omega_2}$ is $\Fcal_1$--$\Sigma$ measurable.

Let $\brac{S, \Sigma}$ and $\brac{\Omega_i, \Fcal_i}_{i\in I}$ be measurable spaces and $f:\brac{\prod_{i\in I}\Omega_i, \bigotimes_{i\in I}\Fcal_i}\to\brac{S,\Sigma}$ be measurable. For any $j\in I$ define $\Omega_{-j} \defn \prod_{i\neq j,\,i\in I}\Omega_i$ and $\Fcal_{-j} \defn\bigotimes_{i\neq j,\,\in I} \Fcal_i$. Then $\prod_{i\in I}\Omega_i$ and $\Omega_j\times \Omega_{-j}$ are, in essence, equivalent, as are $\bigotimes_{i\in I} \Fcal_i$ and $\Fcal_j\otimes \Fcal_{-j}$. Therefore for every $\omega_j\in \Omega_j$ the map $\omega_{-j}\to f\brac{\omega_j, \omega_{-j}}$ defined on $\Omega_{-j}$ is measurable with respect to $\Fcal_{-j}$ and $\Sigma$.\\

\noindent \textbf{Definition} 60, 61.
Let $\brac{\Omega, \Fcal, \mu}$ be a measure space. $\mu$ is a finite measure if $\mu\brac{\Omega}<+\infty$ and $\mu$ is a $\sigma$-finite measure if $\exists \brac{\Omega_n}_{n\geq 1}\in \Fcal$ such that $\Omega_n\uparrow \Omega$ and $\mu\brac{\Omega_n}<+\infty$ for all $n\geq 1$.

For any $F:\Real\to\Real$ non-decreasing and right continuous function, the Lebesgue-Stieltjes measure $dF$ is $\sigma$-finite, since $\ploc{-n, n}\uparrow \Real$ and $dF\brac{\ploc{-n, n}} = n - \brac{-n}<+\infty$ for all $n\geq 1$.\\

\label{thm:fubini1} \noindent \textbf{Theorem} 30.
Let $\brac{\Omega_1, \Fcal_1}$ be a measurable space and $\brac{\Omega_2, \Fcal_2, \mu_2}$ be a $\sigma$-finite measure space. The for all non-negative and measurable maps $f:\brac{\Omega_1\times\Omega_2, \Fcal_1\otimes \Fcal_2}\to \Zinf$, the map:\[\omega \to \int_{\Omega_2} f\brac{\omega, x} d\mu_2\brac{x}\] is measurable with respect to $\Fcal_2$ and $\borel{\Rbar}$.

Indeed, for every $E\in \Fcal_1\otimes \Fcal_2$ and $\omega_1\in \Omega_1$ define \[\Phi_E\brac{\omega_1} \defn \int_{\Omega_2} 1_E\brac{\omega_1, x} d\mu_2\brac{x}\] Further let \[\Dcal\defn \obj{ \induc{ E\in \Fcal_1\otimes \Fcal_2 } \Phi_E:\brac{\Omega_1, \Fcal_1}\to \brac{ \Rbar, \borel{ \Rbar } }\, \text{-- measurable} } \]

By theorem 29 for every $f:\brac{\Omega_1\times\Omega_2, \Fcal_1\otimes \Fcal_2}\to \Zinf$ and any $\omega_1\in \Omega$ the map $\omega\to f\brac{\omega_1, \omega}$ is $\Fcal_2$ is non-negative and measurable, whence the integral $\int_{\Omega_2} f\brac{\omega_1, x} d\mu_2\brac{x}$ over $\brac{\Omega_2, \Fcal_2, \mu_2}$ is well defined. But for every $E\in \Fcal_1 \otimes \Fcal_2$, $1_E:\brac{\Omega_1\times\Omega_2, \Fcal_1\otimes \Fcal_2}\to \Zinf$, whence $\Phi_E$ is well-defined.

If $A\times B\in \Fcal_1 \coprod \Fcal_2$ then $1_{A\times B}\brac{\omega_1, \omega_2} \equiv 1_A\brac{\omega_1} \cdot 1_B\brac{\omega_2}$ and \[\Phi_E\brac{\omega_1} \equiv \int_{\Omega_2} 1_A\brac{\omega_1} 1_B\brac{x} d\mu_2\brac{x} = 1_A\brac{\omega_1} \mu_2\brac{B}\] Since $\mu_2\brac{B}\in \Rbar$ and $A\in \Fcal_1$, $\Phi_E \equiv \mu_2\brac{B} 1_A$ is $\Fcal_1$-$\borel{\Rbar}$ measurable. Therefore $\Fcal_1\coprod \Fcal_2\subseteq \Dcal$ and, in particular, $\Omega_1\times \Omega_2 \in \Dcal$.

Let $\brac{E_n}_{n\geq 1}\in \Fcal_1\otimes \Fcal_2$ be such that $E_n\uparrow E$, then $E\in \Fcal_1\otimes \Fcal_2$ and $1_{E_n}\uparrow 1_E$ in $\Rbar$ point-wise on $\Omega_1\times \Omega_2$. However $\brac{1_{E_n}\brac{\omega_1, \cdot}}_{n\geq 1}, 1_E\brac{\omega_1, \cdot}:\brac{\Omega_2, \Fcal_2}\to\Zinf$ are measurable for every $\omega_1\in \Omega_1$ by theorem 28 and $1_{E_n}\brac{\omega_1, \cdot}\uparrow 1_E\brac{\omega_1, \cdot}$. Therefore \[\Phi_{E_n}\brac{\omega_1} = \int_{\Omega_2} 1_{E_n}\brac{\omega_1, x} d\mu_2\brac{x}\uparrow \int_{\Omega_2} 1_E\brac{\omega_1, x} d\mu_2\brac{x} = \Phi_E\brac{\omega_1}\] by the MCT on $\brac{\Omega_2, \Fcal_2, \mu_2}$.

Suppose $\mu_2$ is a finite measure and $A,B\in \Dcal$ with $A\subseteq B$. If $E\in \Dcal$ then $1_E\leq 1_{\Omega_1\times\Omega_2} = 1_{\Omega_1} 1_{\Omega_2}$ and $\Phi_E\brac{\omega_1} \leq 1_{\Omega_1} \mu_2\brac{\Omega_2} < +\infty$. Since $A\uplus B\setminus A \equiv B$, $1_A+ 1_{B\setminus A} \equiv 1_B$  and so for all $\omega_1\in \Omega_1$ \[\Phi_B\brac{\omega_1} = \Phi_A\brac{\omega_1} + \Phi_{B\setminus A}\brac{\omega_1} \Rightarrow \Phi_{B\setminus A}\brac{\omega_1} = \Phi_B\brac{\omega_1} - \Phi_A\brac{\omega_1} \geq 0\] Hence if $\mu_2$ is finite, then $\Dcal$ is a Dynkin-system on $\omega_1\times\Omega_2$ which contains a $\pi$-system $\Fcal_1\coprod \Fcal_2$. Therefore $\Fcal_1\otimes \Fcal_2\subseteq \Dcal$ and $\Phi_E:\brac{\Omega_1, \Fcal_1}\to\brac{\Rbar, \borel{\Rbar}}$ is a non-negative measurable map for all $E\in \Fcal_1\otimes \Fcal_2$.

If $\brac{\Omega_2^n}_{n\geq 1}\in \Fcal_2$ is such that $\Omega_2^n \uparrow \Omega_2$ and $\mu_2\brac{\Omega_2^n}<+\infty$, then for every $n\geq 1$ the measure space $\brac{\Omega_2, \Fcal_2, \mu_2^n}$ with $\mu_2^n \defn \mu_2\brac{\bullet \cap \Omega_2^n}$ is a finite measure space. If $\Phi_E^n\brac{\omega_1}\defn \int_{\Omega_2} 1_E\brac{\omega_1, x} d\mu_2^n\brac{x}$, then $\Phi_E^n:\brac{\Omega_1, \Fcal_1}\to\brac{\Rbar, \borel{\Rbar}}$ is measurable. However $\Phi_E^n\brac{\omega_1} = \int_{\Omega_2} 1_{\Omega_2^n}\brac{x}1_E\brac{\omega_1, x} d\mu_2\brac{x}$ by definition of a partial Lebesgue integral over $\brac{\Omega_2, \Fcal_2, \mu_2}$, because $\mu_2^n \equiv \mu_2^{\Omega_2^n}$. As $1_{\Omega_2^n} \uparrow 1_{\Omega_2}$ and $1_{\Omega_2^n}\brac{\cdot} 1_E\brac{\omega_1, \cdot}$ is non-negative and measurable, by the MCT on $\brac{\Omega_2, \Fcal_2, \mu_2}$ it is true that $\Phi_E^n \uparrow \Phi_E$ in $\Rbar$ everywhere on $\Omega_1$. Therefore $\Phi_E = \sup_{n\geq 1} \Phi_E^n$, whence by theorem 16, $\Phi_E:\brac{\Omega_1, \Fcal_1}\to \brac{\Rbar, \borel{\Rbar}}$ is measurable for every $E\in \Fcal_1\otimes \Fcal_2$ (or by theorem 17 since as $\Rbar$ has a natural metric defined through a increasing homeomorphism with $\clo{-1,1}$).

If $s:\brac{\Omega_1\times \Omega_2, \Fcal_1\otimes \Fcal_2}\to \Real^+$ is a simple map, then $s\equiv \sum_{i=1}^n \alpha_i 1_{E_i}$ for some $\brac{E_i}_{i=1}^n\in \Fcal_1$. By theorem 29 $s\brac{\omega_1, \cdot}:\brac{\Omega_2, \Fcal_2}\to \Zinf$ is measurable, whence $\omega_1\to\int_{\Omega_2} s\brac{\omega_1, x}d\mu_2\brac{x}$ is well defined $\omega_1\in \Omega_1$. Since the Lebesgue integral is linear \[\int_{\Omega_2} s\brac{\omega_1, x}d\mu_2\brac{x} = \sum_{i=1}^n \alpha_i \int_{\Omega_2} 1_{E_i}\brac{\omega_1, x}d\mu_2\brac{x}\] Hence $\omega_1\to\int_{\Omega_2} s\brac{\omega_1, x}d\mu_2\brac{x}$ is a measurable function $\Fcal_1$--$\borel{\Rbar}$, being a sum of measurable maps.

If $\brac{s_n}_{n\geq 1}$ is a collection of simple maps on $\brac{\Omega_1\times \Omega_2, \Fcal_1\otimes \Fcal_2}$ such that $s_n\uparrow f$ in $\Zinf$ everywhere on $\Omega_1\times \Omega_2$, then $s\brac{\omega_1, \cdot}\uparrow f\brac{\omega_1, \cdot}$ for all $\omega_1\in \Omega_1$. In this case \[\int_{\Omega_2} s_n\brac{\omega_1, x} d\mu_2\brac{x} \uparrow \int_{\Omega_2} f\brac{\omega_1, x} d\mu_2\brac{x}\] for every $\omega_1\in \Omega_1$ by the MCT over $\brac{\Omega_2, \Fcal_2, \mu_2}$. By theorem 16 then $\omega_1\to\int_{\Omega_2} f\brac{\omega_1, x}d\mu_1\brac{x}$ is measurable $\Fcal_1$--$\borel{\Rbar}$.

Let $\brac{\Omega_i, \Fcal_i}_{i\in I}$ be measurable spaces and $f:\brac{\prod_{i\in I}\Omega_i, \bigotimes_{i\in I}\Fcal_i}\to\Zinf$ be measurable. Let $j\in I$ and suppose $\mu_j$ is a $\sigma$-finite measure on $\brac{\Omega_j, \Fcal_j}$. Define $\Omega_{-j} \defn \prod_{i\neq j,\,i\in I}\Omega_i$ and $\Fcal_{-j} \defn\bigotimes_{i\neq j,\,\in I} \Fcal_i$. Then $\prod_{i\in I}\Omega_i$ and $\Omega_{-j} \times \Omega_j$ are, in essence, equivalent, as are $\bigotimes_{i\in I} \Fcal_i$ and $\Fcal_{-j}\otimes \Fcal_j$. Therefore for every $\omega_{-j}\in \Omega_{-j}$ the map $f\brac{\omega_{-j}, \cdot}$ defined on $\brac{\Omega_j, \Fcal_j}$ is non-negative and measurable. This by theorem 30, the map\[\omega_{-j}\to\int_{\Omega_j} f\brac{\omega_{-j}, x} d\mu_j\brac{x}\] is measurable with respect to $\Fcal_{-j}$ and $\borel{\Rbar}$.\\

\label{thm:product_meas} \noindent \textbf{Theorem} 7-1.
Let $\brac{\Omega_i, \Fcal_i, \mu_i}_{i=1}^n$ be a collection of $\sigma$-finite measure spaces with $n\geq 2$. There exists a measure $\mu$ on $\bigotimes_{i=1}^n \Fcal_i$ such that for all measurable rectangles $\prod_{i=1}^n A_i \in \coprod_{i=1}^n \Fcal_i$ \[\mu\brac{ \prod_{i=1}^n A_i } = \prod_{i=1}^n \mu_i\brac{A_i}\] This unique measure, called the product measure of $\brac{\mu_i}_{i=1}^n$, is $\sigma$-finite and denoted by $\bigotimes_{i=1}^n \mu_i$. Obviously, the product measure is the same regardless of how the product is composed.

Let $\brac{\Omega_1, \Fcal_1, \mu_1}$ and $\brac{\Omega_2, \Fcal_2, \mu_2}$ be two $\sigma$-finite measure spaces. Define for all $E\in \Fcal_1\otimes\Fcal_2$ \[\mu_1\otimes\mu_2\brac{E} \defn \int_{\Omega_1}\brac{ \int_{\Omega_2}1_E\brac{\omega_1, \omega_2} d\mu_2\brac{\omega_2} }d\mu_1\brac{\omega_1}\]
The map $\mu_1\otimes\mu_2\brac{E}$ is well-defined for all $E\in \Fcal_1\otimes\Fcal_2$ by theorem 30.

If $E = \emptyset\in \Fcal_1\otimes \Fcal_2$ then $\mu_1\otimes\mu_2\brac{E} = 0$, since $1_E = 0$ on $\Omega_1\times \Omega_2$. If $\brac{E_n}_{n\geq 1}\in \Fcal_1\otimes \Fcal_2$ is a measurable partition of some $E = \biguplus_{n\geq 1} E_n$, then $\sum_{k=1}^n 1_{E_k} = 1_{\uplus_{k=1}^n E_k}\uparrow 1_E$, whence \[\sum_{k=1}^n \int_{\Omega_2}1_{E_k}\brac{\omega_1, \omega_2} d\mu_2\brac{\omega_2}\uparrow \int_{\Omega_2}1_E\brac{\omega_1, \omega_2} d\mu_2\brac{\omega_2}\] by linearity of the Lebesgue integral and by the MCT on $\brac{\Omega_2, \Fcal_2, \mu_2}$ for every $\omega_1\in \Omega_1$. By theorem 30 $\omega_1\to\int_{\Omega_2}1_F\brac{\omega_1, \omega_2} d\mu_2\brac{\omega_2}$ is well-defined, non-negative and measurable on $\brac{\Omega_1, \Fcal_1, \mu_1}$ for every $F\in \Fcal_1\otimes \Fcal_2$. Thus \[\sum_{k=1}^n \int_{\Omega_1}\brac{ \int_{\Omega_2}1_{E_n}\brac{\omega_1, \omega_2} d\mu_2\brac{\omega_2} } d\mu_1\brac{\omega_1}\uparrow \int_{\Omega_1}\brac{ \int_{\Omega_2}1_E\brac{\omega_1, \omega_2} d\mu_2\brac{\omega_2} } d\mu_1\brac{\omega_1}\] by linearity of the Lebesgue integral and by the MCT on $\brac{\Omega_1, \Fcal_1, \mu_1}$. Hence $\mu_1\otimes\mu_2\brac{E} = \sum_{k=1}^\infty \mu_1\otimes\mu_2\brac{E_n}$, and therefore $\mu_1\otimes\mu_2$ is a measure on $\brac{\Omega_1\times \Omega_2, \Fcal_1\otimes \Fcal_2}$.

If $E = A\times B\in \Fcal_1 \coprod \Fcal_2$, then $1_E \brac{\omega_1, \omega_2} \equiv 1_A \brac{\omega_1} 1_B \brac{\omega_2}$, so \[\mu_1\otimes\mu_2\brac{E} = \int_{\Omega_1} 1_A\brac{\omega_1} \brac{ \int_{\Omega_2} 1_B\brac{\omega_2} d\mu_2\brac{\omega_2} } d\mu_1\brac{\omega_1} = \mu_1\brac{A} \mu_2\brac{B}\] again by linearity of the Lebesgue integral over non-negative measurable maps with values in $\Zinf$.

Since both measure spaces are $\sigma$-finite there exist $\brac{\Omega_1^n}_{n\geq 1}\in \Fcal_1$ and $\brac{\Omega_2^n}_{n\geq 1}\in \Fcal_2$ such that $\Omega_k^n\uparrow \Omega_k$ and $\mu_k\brac{\Omega_k^n}< +\infty$ for $k=1,2$. For all $n\geq 1$ obviously $\Omega^n \defn \Omega_1^n \times \Omega_2^n$ is a measurable rectangle, so $\mu_1\otimes\mu_2\brac{\Omega^n} = \mu_1\brac{\Omega_1^n}\mu_2\brac{\Omega_2^n} < +\infty$ for every $n\geq 1$. Therefore $\mu_1\otimes\mu_2$ is a $\sigma$-finite measure, because $\Omega_1^n\times \Omega_2^n\uparrow \Omega_1\times \Omega_2$.

Let $\mu$ be another measure on $\Fcal_1\otimes\Fcal_2$ with $\mu\brac{A\times B} = \mu_1\brac{A}\mu_2\brac{B}$ for all measurable rectangles $A\times B$. Define $\lambda^n\defn \mu_1\otimes \mu_2\brac{ \bullet \cap \Omega_1^n\times\Omega_2^n }$ and $\mu^n \defn \mu\brac{\bullet \cap  \Omega_1^n\times\Omega_2^n }$ and let \[\Dcal_n \defn \obj{ \induc{ E\in \Fcal_1\coprod \Fcal_2 } \mu^n\brac{E} = \lambda^n\brac{E} }\] Note that $\lambda^n$ and $\mu^n$ are finite measures for every $n\geq 1$. Indeed, $\mu^n\brac{E}\leq \mu\brac{\Omega_1^n \times \Omega_2^n} = \mu_1\brac{\Omega_1^n}\mu_2\brac{\Omega_2^n} < +\infty$ and $\lambda^n\brac{E}\leq \lambda\brac{\Omega_1^n \times \Omega_2^n} = \mu_1\brac{\Omega_1^n}\mu_2\brac{\Omega_2^n} < +\infty$ for all $n\geq 1$ and for every $E\in \Fcal_1\otimes \Fcal_2$ because $\mu_1\otimes\mu_1$ and $\mu$ are a measures on $\Fcal_1\otimes \Fcal_2$.

For any $A_1\times A_2\in \Fcal_1\coprod \Fcal_2$ it is true that $\lambda^n\brac{A_1\times A_2} = \mu^n\brac{A_1\times A_2}$ because $\brac{\Omega_1^n \times \Omega_2^n} \in \Fcal_1\coprod \Fcal_2$ -- a $\pi$-system. Thus $\Fcal_1\coprod \Fcal_2\subseteq \Dcal_n$ for all $n\geq 1$.

Let $n\geq 1$. If $A, B\in\Dcal_n$ with $A\subseteq B$, then $\lambda^n\brac{B\setminus A} = \mu^n\brac{B\setminus A}$, since $B = \brac{B\setminus A}  \uplus A$ and both measures are finite and so their values on $A$ and $B$ admit arithmetical manipulation. Therefore $B\setminus A\in \Dcal_n$ for every $n\geq 1$.

If $\brac{E_m}_{m\geq 1}\in \Dcal_n$ are such that $E_m\subseteq E_{m+1}$ for all $m\geq 1$, then $E_m\cap \brac{\Omega_1^n \times \Omega_2^n}\uparrow E\cap \brac{\Omega_1^n \times \Omega_2^n}$, where $E=\bigcup_{m\geq 1}E_m\in \Fcal_1\otimes\Fcal_2$. Since $\mu^n$ and $\lambda^n$ are measures, theorem 7 implies that $\mu^n\brac{E_m}\uparrow \mu^n\brac{E}$ and $\lambda^n\brac{E_m} \uparrow \lambda^n\brac{E}$. Thus $\mu^n\brac{E} = \lambda^n\brac{E}$, since $\brac{E_m}_{m\geq 1}\in \Dcal_n$, and therefore $\Dcal_n$ is a Dynkin system on $\Omega_1\times\Omega_2$ which contains $\Fcal_1\coprod\Fcal_2$, whence $\Fcal_1\otimes \Fcal_2 = \sigma\brac{\Fcal_1\coprod \Fcal_2}\subseteq \Dcal_n$. Next, if $E\in \Fcal_1\otimes\Fcal_2$, then $E\cap \Omega_1^n\times \Omega_2^n \uparrow E$, whence $\mu^n\brac{E}\uparrow \mu\brac{E}$ and $\lambda^n\brac{E}\uparrow \lambda\brac{E}$ by theorem 7. Thus $\mu = \lambda$, since $\mu^n\brac{E}=\lambda^n\brac{E}$ for every $n\geq 1$ and therefore $\mu_1\otimes \mu_2$ is a unique measure with such ``product'' property.

Suppose for all $E\in \Fcal_1\otimes\Fcal_2$ \[\mu \defn \int_{\Omega_2} \brac{ \int_{\Omega_1} 1_E\brac{\omega_1, \omega_2 } d\mu_1\brac{\omega_1} } d\mu_2\brac{\omega_2}\] By theorem 29 and 30, the map $\mu$ is a well defined non-negative number. By the linearity of the Lebesgue integral and the MCT, $\mu$ is a measure on $\Fcal_1\otimes\Fcal_2$. For every $A\times B\in \Fcal_1\coprod \Fcal_2$ by linearity of the Lebesgue integral \[\mu\brac{A\times B} = \int_{\Omega_2} \mu_1\brac{A} 1_B\brac{\omega_2} d\mu_2\brac{\omega_2} = \mu_1\brac{A} \mu_2\brac{B}\] Therefore by the previous result $\mu_2\otimes\mu_1 \equiv \mu = \mu_1\otimes\mu_2$.\\

Let $\brac{\Omega_k, \Fcal_k, \mu_k}_{k=1}^n$ be $\sigma$-finite measure spaces, $n\geq 2$, and let $j=1\ldots n$. Put $\Omega_{-j} \defn \prod_{k\neq j} \Omega_k$ and $\Fcal_{-j} \defn \bigotimes_{k\neq j} \Fcal_k$. Suppose $\nu$ is a $\sigma$-finite measure on $\brac{\Omega_{-j}, \Fcal_{-j}}$ such that $\nu\brac{ \prod_{k\neq j} A_k } = \prod_{k\neq j} \mu_k\brac{A_k}$ for every measurable rectangle $\prod_{k\neq j} A_k \in \coprod_{k\neq j} \Fcal_k$. By the previous result $\mu_j\otimes\nu$ is a measure on $\brac{\prod_{k=1}^n\Omega_k,\otimes_{k=1}^n\Fcal_k}$.

Since $\nu$ is $\sigma$-finite, $\exists \brac{\Omega_{-j}^m}_{m\geq 1}\in \Fcal_{-j}$ with $\nu\brac{\Omega_{-j}^m}<+\infty$ such that $\Omega_{-j}^m\uparrow \Omega_{-j}$. As $\mu_j$ is $\sigma$-finite, there is $\brac{\Omega_j^m}_{m\geq 1}\in \Fcal_j$ with $\mu_j\brac{\Omega_j^m}<+\infty$ and such that $\Omega_j^m\uparrow \Omega_j$. Therefore $\Omega_j^m\times\Omega_{-j}^m\uparrow \Omega_j\times\Omega_{-j}$, where sets are measurable rectangles of $\Fcal_j$ and $\Fcal_{-j}$, and $\mu_j\otimes\nu\brac{\Omega_j^m\times\Omega_{-j}^m} = \mu_j\brac{\Omega_j^m} \nu\brac{\Omega_{-j}^m}<+\infty$. Thus, since the measurable spaces $\brac{\prod_{k=1}^n\Omega_k,\otimes_{k=1}^n\Fcal_k}$ and $\brac{\Omega_j\times\Omega_{-j}, \Fcal_j\otimes\Fcal_{-j}}$ are identical, 
$\mu_j\otimes \nu$ is a $\sigma$-finite measure. If $\prod_{k=1}^n A_k$ is a measurable rectangle, then $A_j\times \prod_{k\neq j} A_k \in \Fcal_j\coprod \Fcal_{-j}$, whence $\mu_j\otimes\nu\brac{\prod_{k=1}^n A_k} = \mu_j\brac{A_j} \nu\brac{\prod_{k\neq j} A_k} = \prod_{k=1}^n \mu_k\brac{A_k}$.

Therefore a $\sigma$-finite measure $\mu =\bigotimes_{i=1}^n \mu_i$ on $\brac{\prod_{k=1}^n\Omega_k,\otimes_{k=1}^n\Fcal_k}$ can be constructed inductively, which has the property $\mu\brac{\prod_{k=1}^n A_k} = \prod_{k=1}^n \mu_k\brac{A_k}$ for every $\prod_{k=1}^n A_k\in \coprod_{k=1}^n \Fcal_k$.

Let $\nu$ be another measure with such property. For each $k=1\ldots n$ there are $\brac{\Omega_k^m}_{m\geq 1}\in \Fcal_k$ such that $\mu_k\brac{\Omega_k^m}<+\infty$ and $\Omega_k^m\uparrow \Omega_k$. Therefore $\brac{\Omega^m}_{m\geq1} \in \coprod_{k=1}^n\Fcal_k$, $\Omega^m\uparrow \Omega$ and $\mu\brac{\Omega^m} = \nu\brac{\Omega^m} = \prod_{k=1}^n \mu_k\brac{\Omega_k^m}<+\infty$, where $\Omega^m \defn \prod_{k=1}^n \Omega_k^m$ and $\Omega \defn \prod_{k=1}^n \Omega_k$. Put $\mu^m = \mu\brac{ \bullet \cap \Omega^m }$ and $\nu^m = \nu\brac{ \bullet \cap \Omega^m }$. Let \[\Dcal^m\defn \obj{ \induc{ E\in \bigotimes_{k=1}^n \Fcal_k } \mu^m\brac{E} = \nu^m\brac{E} }\] Note that $\mu^m$ and $\nu^m$ are finite measures for every $m\geq 1$.

First, $\coprod_{k=1}^n \Fcal_k \subseteq \Dcal^m$ for every $m\geq 1$, since the collection of measurable rectangles is a $\pi$-system. Second,  for every $m\geq 1$ if $A,B \in \Dcal^m$ with $A\subseteq B$, then $B\setminus A \in \Dcal^m$ because $B \equiv A\uplus B\setminus A$ and $\mu^m$ and $\nu^m$ are finite. Third, if $\brac{E_p}_{p\geq 1}\in \Dcal^m$ is such that $E_p\subseteq E_{p+1}$, then $\mu^m\brac{E_p} = \nu^m\brac{E_p}$ for every $p\geq 1$ and $E_p\uparrow E = \bigcup_{p\geq 1} E_p \in \bigotimes_{k=1}^n \Fcal_k$. This implies $\mu^m\brac{E_p}\uparrow\mu^m\brac{E}$ and $\nu^m\brac{E_p}\uparrow\nu^m\brac{E}$ by theorem 7, whence $\mu^m\brac{E} = \nu^m\brac{E}$. Therefore $\Dcal^m$ is a Dynkin system on $\Omega$ for every $m\geq 1$, which also contains $\bigotimes_{k=1}^n \Fcal_k$ by virtue of theorem 1.

If $E\in \bigotimes_{k=1}^n \Fcal_k$, then $\mu^m\brac{E} = \nu^m\brac{E}$ for every $m\geq 1$. However $E\cap \Omega^m\uparrow E\cap \Omega$, whence $\mu^m\brac{E} \uparrow \mu\brac{E}$ and $\nu^m\brac{E} \uparrow \nu\brac{E}$, due to theorem 7 and definitions of $\mu^m$ and $\nu^m$. Hence $\nu$ and $\mu$ are identical on $\bigotimes_{k=1}^n \Fcal_k$.

Therefore this inductively constructed $\sigma$-finite measure $\mu_1\otimes\ldots\otimes\mu_n$ is, in fact, unique. Moreover for any $j=1\ldots n$, $\mu_1\otimes\ldots\otimes\mu_n = \mu_j\otimes\brac{\bigotimes_{k\neq j} \mu_k}$ since the latter measure takes same values on the measurable rectangles of $\brac{\Fcal_k}_{k=1}^n$.\\

\label{thm:lebesgue_stieltjes_prod_meas} \noindent \textbf{Theorem} 7-2.
Let $\brac{F_k}_{k=1}^n:\Real\to \Real$ be non-decreasing right-continuous functions. Then there is a unique measure on $\brac{\Real^n, \borel{\Real^n}}$ denoted by $d\bar{F}$ such that for all $a_i\leq b_i\in \Real$, $i=1\ldots n$ we have:\[d\bar{F}\brac{ \ploc{a_1, b_1}\times \ldots \times \ploc{a_n, b_n} } = \prod_{k=1}^n \brac{ F_k\brac{b_k} - F_k\brac{a_k} }\] This measure is $\sigma$-finite.

Indeed, each function from $\brac{F_k}_{k=1}^n$ defines a unique $\sigma$-finite Stieltjes measure $dF_k$ on $\brac{\Real, \borel{\Real}}$. By theorem 7-1 the product measure $d\bar{F}\defn dF_1\otimes\ldots\otimes dF_n$ is the unique measure on $\bigotimes_{k=1}^n \borel{\Real}$, such that for all $\prod_{i=1}^n A_i \in \coprod_{i=1}^n \borel{\Real}$ \[d\bar{F}\brac{\prod_{i=1}^n A_i} = \prod_{i=1}^n dF_i\brac{A_i}\] Such measure is also $\sigma$-finite. Since $\Real^n$ has a countable topological basis, $\borel{\Real^n} \equiv \bigotimes_{k=1}^n \borel{\Real}$, whence in this case $d\bar{F}$ is also a measure on $\brac{\Real^n, \borel{\Real^n}}$.

Let $\Scal\defn \obj{\induc{ \ploc{a, b} } a, b\in \Real}$ and $\Scal_n\defn \obj{ \induc{ \ploc{a_1, b_1}\times\ldots\times \ploc{a_n, b_n} } a_i, b_i\in \Real }$. First, $\borel{\Real} = \sigma\brac{\Scal}$, since $\Real$ has a countably dense subset of rational numbers $\mathbb{Q}$. Second, $\Scal_n \subset \coprod_{i=1}^n \Scal$, whence $\sigma\brac{\Scal_n}\subseteq \bigotimes_{k=1}\borel{\Real}$ by theorem 26. Third, every rectangle of $\coprod_{i=1}^n \Scal$ is a countable union of boxes from $\Scal_n$. Thus $\coprod_{i=1}^n \Scal \subseteq \sigma\brac{\Scal_n}$ and $\bigotimes_{k=1}\borel{\Real} \equiv \sigma\brac{\Scal_n}$. Furthermore $\Scal_n$ is a $\pi$-system, since $C\equiv \obj{\induc{ \prod_{i=1}^n A_i } \brac{A_i}_{i=1}^n \in S }$ and $\Scal$ is a $\pi$-systems itself.

Since $\Scal_n \subset \coprod_{i=1}^n \Scal \subseteq \coprod_{i=1}^n \borel{\Real}$, for all $a_i\leq b_i\in \Real$, $i=1\ldots n$ \[dF_1\otimes\ldots\otimes dF_n\brac{ \prod_{i=1}^n \ploc{a_i, b_i} } = \prod_{i=1}^n dF_i\brac{ \ploc{a_i, b_i} } = \prod_{i=1}^n \brac{F_i\brac{b_i} - F_i\brac{a_i}}\] 

Let $\lambda \defn \bigotimes_{i=1}^n dF_i$ and $\mu$ be another measure on $\brac{\Real, \borel{\Real}}$, which coincides with $\lambda$ on $\Scal_n$, and put $M_k \defn \prod_{i=1}^n\ploc{-k, k}\in \Scal_n$ for $k\geq 1$. Then $M_k\uparrow \Real^n$ and byt the prior observation $\mu\brac{M_k} = \otimes_{i=1}^n dF_i \brac{ M_k } < +\infty$, whence $\mu$ is also $\sigma$-finite. Define $\lambda^k\defn \lambda\brac{\bullet \cap M_k}$ and $\mu^k\defn \mu\brac{\bullet \cap M_k}$ and let \[\Dcal_k\defn \obj{ \induc{ E\in \borel{\Real} } \lambda^k\brac{E} = \mu^k\brac{E} } \] Since $\Scal_n$ is a $\pi$-system, for every $E\in \Scal_n$, $E\cap M_k\in \Scal_n$, whence $\Scal_n\subseteq \Dcal_k$ for all $k\geq 1$. As both $\lambda^k$ and $\mu^k$ are finite measures on $\brac{\Real, \borel{\Real}}$, $\Real^n\in \Dcal_k$ and $\Dcal_k$ is a Dynkin system on $\Real^n$ for all $k\geq 1$. Therefore $\borel{\Real^n} = \sigma\brac{\Scal_n} \subseteq \Dcal_k$ for every $k\geq 1$ by the Dynkin system theorem.

If $E\in \borel{\Real^n}$ then $\lambda\brac{E\cap M_k} = \mu\brac{E\cap M_k}$ and $E\cap M_k\uparrow E$. Since $\lambda$ and $\mu$ are measures, theorem 7 implies that $\lambda\brac{E\cap M_k}\uparrow\lambda\brac{E}$ and $\mu\brac{E\cap M_k}\uparrow\mu\brac{E}$, whence $\mu\brac{E} = \lambda\brac{E}$. Therefore the Stieltjes product measure $dF_1\otimes\ldots\otimes dF_n$ is the unique $\sigma$-finite measure on $\borel{\Real^n}$ such that for all $a_i\leq b_i\in \Real$, $i=1\ldots n$ \[dF_1\otimes\ldots\otimes dF_n\brac{ \prod_{i=1}^n \ploc{a_i, b_i} } = \prod_{i=1}^n dF_i\brac{\ploc{a_i, b_i}} = \prod_{i=1}^n \brac{F_i\brac{b_i}-F_i\brac{a_i}}\]

On a collection of $\sigma$-finite measure spaces $\brac{\Real, \borel{\Real}, dx}$ it is possible to construct a product measure $dx^n$ on $\brac{\Real^n, \bigotimes_{i=1}^n \borel{\Real} } = \brac{\Real^n, \borel{\Real^n}}$ , which will be $\sigma$-finite and unique up to measurable rectangles $\coprod_{i=1}^n \borel{\Real}$, where it is decomposed into a product. Since $\borel{\Real}$ is also generated by the collection of closed intervals $\mathcal{K} \defn \obj{\induc{\clo{a, b}} a, b \in \Real}$, for \[\mathcal{K}_n \defn \obj{ \induc{\prod_{i=1}^n A_i} \brac{A_i}_{i=1}^n \in \mathcal{K} }\] a similar argument gives $ \bigotimes_{i=1}^n \borel{\Real} = \sigma\brac{\coprod_{i=1}^n \mathcal{K}} = \sigma\brac{\mathcal{K}_n}$. Any other measure on $\brac{\Real^n, \borel{\Real^n}}$, which equals $dx^n$ on $\mathcal{K}_n$, by virtue of truncation and the Dynkin system theorem coincides with $dx^n$ on $\borel{\Real^n}$. Therefore the Lebesgue measure on $\Real^n$, $n\geq 1$ is the unique $\sigma$-finite product measure $dx^n$, such that for all $a_i\leq b_i$, $i=1\ldots n$ \[dx\brac{\clo{a_1, b_1}\times \ldots \times \clo{a_n, b_n}} = \prod_{i=1}^n \brac{b_i-a_i}\] Moreover the product measure $dx^n \otimes dx^p$ on $\brac{\Real^n\times\Real^p, \borel{\Real^n}\otimes \borel{\Real^p}}$, is also a measure on $\brac{\Real^{n+p}, \borel{\Real^{n+p}}}$ identical to $dx^{n+p}$ due to uniqueness.\\

\label{thm:fubini2} \noindent \textbf{Theorem} 31 (Fubini).
Let $\brac{\Omega_1, \Fcal_1, \mu_1}$ and $\brac{\Omega_2, \Fcal_2, \mu_2}$ be $\sigma$-finite measure spaces and let $f:\brac{\Omega_1\times \Omega, \Fcal_1\otimes \Fcal_2}\to\Zinf$ be a measurable map. Then \[\int_{\Omega_1\times\Omega_2} f d\mu_1\otimes\mu_2 = \int_{\Omega_1}\brac{\int_{\Omega_2} f d\mu_2} d\mu_1 = \int_{\Omega_2}\brac{\int_{\Omega_1} f d\mu_1} d\mu_2\]

Indeed, the previous efforts show that for every $E\in \Fcal_1\otimes\Fcal_2$\[\int_{\Omega_1\times\Omega_2} 1_E d\mu_1\otimes\mu_2 = \mu_1\otimes\mu_2\brac{E} = \int_{\Omega_1}\brac{\int_{\Omega_2} 1_E d\mu_2} d\mu_1 = \int_{\Omega_2}\brac{\int_{\Omega_1} 1_E d\mu_1} d\mu_2\] If $s$ is a simple function then $s = \sum_{i=1}^n \alpha_i 1_{E_i}$ and by linearity of the Lebesgue integral on each particular measure space as well as on $\brac{\Omega_1\times \Omega, \Fcal_1\otimes \Fcal_2, \mu_1\otimes\mu_2}$ \[\int_{\Omega_1\times\Omega_2} s d\mu_1\otimes\mu_2 = \int_{\Omega_1}\brac{\int_{\Omega_2} s d\mu_2} d\mu_1 = \int_{\Omega_2}\brac{\int_{\Omega_1} s d\mu_1} d\mu_2\]

By theorem 18 there exists a collection of simple functions $\brac{s_n}_{n\geq 1}$ such that $s_n\uparrow f$ everywhere on $\Omega_1\times\Omega_2$ in $\Zinf$, whence by the MCT on the product space \[\int_{\Omega_1\times\Omega_2} s_n d\mu_1\otimes\mu_2\uparrow \int_{\Omega_1\times\Omega_2} f d\mu_1\otimes\mu_2\] However by theorems 29 and 17 $s_n\brac{\cdot, \omega_2}\uparrow f\brac{\cdot, \omega_2}$ and $s_n\brac{\omega_1, \cdot}\uparrow f\brac{\omega_1, \cdot}$ for every $\omega_1\in \Omega_1$ and $\omega_2\in\Omega_2$. Therefore applying the MCT over $\brac{\Omega_2, \Fcal_2, \mu_2}$ and $\brac{\Omega_1, \Fcal_1, \mu_1}$ one gets \[\int_{\Omega_2} s_n\brac{\omega_1, \cdot} d\mu_2\uparrow \int_{\Omega_2} f\brac{\omega_1, \cdot} d\mu_2\;\text{and}\;\int_{\Omega_1} s_n\brac{\cdot, \omega_2} d\mu_1\uparrow \int_{\Omega_1} f\brac{\cdot, \omega_2} d\mu_1\] respectively. Since by theorem 30 the involved once-integrated functions are respectively $\brac{\Omega_1, \Fcal_1, \mu_1}$ and $\brac{\Omega_2, \Fcal_2, \mu_2}$ measurable, yet another invocation of the MCT on the mentioned spaces respectively yields the desired theorem.\\

\label{thm:fubini3} \noindent \textbf{Theorem} 32.
Let $\brac{\Omega_k, \Fcal_k, \mu_k}_{k=1}^n$ be $\sigma$-finite measure spaces with $n\geq 2$, and $\sigma$ be a permutation of $1\ldots n$. Let $f:\brac{\prod_{k=1}^n \Omega_k, \bigotimes_{k=1}^n \Fcal_k } \to \Zinf$ be a measurable map. Then \[\int_{\Omega_1\times \ldots \times\Omega_n} f d\mu_1\otimes \ldots \otimes\mu_n = \int_{\Omega_{\sigma\brac{ n } } } \ldots  \int_{\Omega_{\sigma\brac{ 1 } } } f d\mu_{\sigma\brac{ 1 } } \ldots d\mu_{\sigma\brac{ n } } \]

Let $J_0 \defn f$ and for $k=1\ldots n-1$ put \[J_k\brac{\omega} \defn \int_{\Omega_{\sigma\brac{k}}} J_{k-1}\brac{\omega, x} d\mu_{\sigma\brac{k}}\brac{x}\] Each $J_k:\brac{ \prod_{i\notin \sigma\brac{1\ldots k}}\Omega_i, \bigotimes_{i\notin \sigma\brac{1\ldots k}}\Fcal_i }\to\Zinf$ is well-defined, non-negative and measurable by virtue of theorem 30 provided $J_{k-1}:\brac{ \prod_{i\notin \sigma\brac{1\ldots k-1}}\Omega_i, \bigotimes_{i\notin \sigma\brac{1\ldots k-1}}\Fcal_i }\to\Zinf$ is non-negative and measurable. However $J_0 = f$ and $f:\brac{\prod_{k=1}^n \Omega_k, \bigotimes_{k=1}^n \Fcal_k } \to \Zinf$ is a non-negative measurable map. The nested integral can thus be defined as \[\int_{\Omega_{\sigma\brac{ n } } } \ldots  \int_{\Omega_{\sigma\brac{ 1 } } } f d\mu_{\sigma\brac{ 1 } } \ldots d\mu_{\sigma\brac{ n } } \defn \int_{\Omega_{\sigma\brac{n}}} J_{n-1}\brac{x} d\mu_{\sigma\brac{n}}\brac{x}\]

Let $\brac{f_p}_{p\geq 1}:\brac{\prod_{k=1}^n \Omega_k, \bigotimes_{k=1}^n \Fcal_k}\to\Zinf$ be such that $f_p\uparrow f$. Let $J^p_0 \defn f_p$ and for $k=1\ldots n-1$ put \[J^p_k\brac{\omega} \defn \int_{\Omega_{\sigma\brac{k}}} J_{k-1}\brac{\omega, x} d\mu_{\sigma\brac{k}}\brac{x}\] Each $J^p_k$ is well-defined, non-negative and measurable on $\brac{ \prod_{i\notin \sigma\brac{1\ldots k}}\Omega_i, \bigotimes_{i\notin \sigma\brac{1\ldots k}}\Fcal_i }$ by virtue of theorem 30 if $J^p_{k-1}$ is $\brac{ \prod_{i\notin \sigma\brac{1\ldots k-1}}\Omega_i, \bigotimes_{i\notin \sigma\brac{1\ldots k-1}}\Fcal_i }$ is such as well.

Furthermore, $J^p_k\uparrow J_k$ in $\Zinf$ by the MCT on $\brac{\Omega_{\sigma\brac{k}}, \Fcal_{\sigma\brac{k}}, \mu_{\sigma\brac{k}}}$ provided $J^p_{k-1}\uparrow J_{k-1}$ everywhere on $\brac{ \prod_{i\notin \sigma\brac{1\ldots k-1}}\Omega_i, \bigotimes_{i\notin \sigma\brac{1\ldots k-1}}\Fcal_i }$ in $\Zinf$. Since $J^p_0\uparrow f$ on $\brac{ \prod_{i=1}^n\Omega_i, \bigotimes_{i=1}^n\Fcal_i}$ in $\Zinf$, $J^p_k\uparrow J_k$ for all $k=1\ldots n-1$. Therefore $\int_{\Omega_{\sigma\brac{n}}} J^p_{n-1}\brac{x} d\mu_{\sigma\brac{n}}\brac{x} \uparrow \int_{\Omega_{\sigma\brac{n}}} J_{n-1}\brac{x} d\mu_{\sigma\brac{n}}\brac{x}$ by the last application of the MCT over $\brac{\Omega_{\sigma\brac{n}}, \Fcal_{\sigma\brac{n}}, \mu_{\sigma\brac{n}}}$, whence \[\int_{\Omega_{\sigma\brac{ n } } } \ldots  \int_{\Omega_{\sigma\brac{ 1 } } } f_p d\mu_{\sigma\brac{ 1 } } \ldots d\mu_{\sigma\brac{ n } } \uparrow \int_{\Omega_{\sigma\brac{ n } } } \ldots \int_{\Omega_{\sigma\brac{ 1 } } } f d\mu_{\sigma\brac{ 1 } } \ldots d\mu_{\sigma\brac{ n } }\]

Let $\mu:\bigotimes_{i=1}^n \Fcal_i\to\Zinf$ be defined as \[\mu\brac{E} \defn \int_{\Omega_{\sigma\brac{ n } } } \ldots \int_{\Omega_{\sigma\brac{ 1 } } } 1_E d\mu_{\sigma\brac{ 1 } } \ldots d\mu_{\sigma\brac{ n } }\] If $E\in \bigotimes_{i=1}^n \Fcal_i$ is the empty set, then $1_E\equiv 0$, whence $J_k = 0$ for $k=1\ldots n-1$, and so $\mu\brac{\emptyset} = 0$. Now, if $\brac{E_p}_{p\geq 1}\in \bigotimes_{i=1}^n \Fcal_i$ are pairwise disjoint, then $\sum_{p=1}^m 1_{E_p} = 1_{\uplus_{p=1}^m E_p} \uparrow 1_{\uplus_{p\geq 1} E_p}$ and so $\mu\brac{\uplus_{p=1}^m E_p}\uparrow \mu\brac{\uplus_{p\geq 1} E_p}$ by the previous result. Since each nested Lebesgue integral is linear $\mu\brac{\uplus_{p=1}^m E_p} = \sum_{p=1}^m \mu{E_p}$ for any $m\geq 1$. Therefore $\mu$ is a measure on $\brac{\prod_{i=1}^n \Omega_i, \bigotimes_{i=1}^n \Fcal_i}$.

If $E\in \coprod_{i=1}^n \Fcal_i$, then $E = \prod_{i=1}^n E_i$ with $E_i\in \Fcal_i$ for $i=1\ldots n$. For every $x \in \prod_{i=1}^n \Omega_i$ the indicator $1_E\brac{x}$ is identical to $\prod_{i=1}^n 1_{E_i}\brac{x_i}$, from where $J_1\brac{\omega} = \prod_{i\neq \sigma\brac{1} } 1_{E_i}\brac{\omega_i} \mu_{\sigma\brac{1}}\brac{E_{\sigma\brac{1}}}$. Therefore $J_k\brac{\omega} = \prod_{i\notin \sigma\brac{1\ldots k} } 1_{E_i}\brac{\omega_i} \prod_{i=1}^k \mu_{\sigma\brac{i}}\brac{E_{\sigma\brac{i}}}$ and finally \[\mu\brac{E} = \prod_{i=1}^n \mu_{\sigma\brac{i}}\brac{E_{\sigma\brac{i}}} = \prod_{i=1}^n \mu_i\brac{E_i} = \bigotimes_{i=1}^n \mu_i \brac{E}\] Hence this measure, defined through the nested integral of an indicator of a measurable set, coincides with the product measure on $\bigotimes_{i=1}^n \Fcal_i$.

If $s$ is a simple function on $\brac{\prod_{i=1}^n \Omega_i, \bigotimes_{i=1}^n \Fcal_i}$, then the linearity of each of the nested Lebesgue integrals and the linearity on the integral over the product space imply that \[\int_{\Omega_1\times \ldots \times\Omega_n} s d\mu_1\otimes \ldots \otimes\mu_n = \int_{\Omega_{\sigma\brac{ n } } } \ldots  \int_{\Omega_{\sigma\brac{ 1 } } } s d\mu_{\sigma\brac{ 1 } } \ldots d\mu_{\sigma\brac{ n } }\] whence the MCT over the product space and the monotone convergence noted a couple of paragraphs before yield the desired result.\\

%% I just don't know how to format the discussion below.
\noindent\textbf{Exercise} 14, 15.
Let $\brac{\Omega, \Fcal, \mu}$ be a measure space and put \[L^1 \defn \obj{ \induc{f:\Omega\to \Rbar}  \exists g\in L^1_\Real\brac{\Omega, \Fcal, \mu},\,f=g\,\mu\text{-a.s}}\] Obviously not all maps in $L^1$ are measurable, so Lebesgue integrals of such maps are undefined. If $f\in L^1$, then there is $g\in L^1_\Real\brac{\Omega, \Fcal, \mu}$ with $f=g$ $\mu$-a.s., whence $\abs{f} < +\infty$ $\mu$-a.s. since $\int \abs{g} d\mu < +\infty$ implies that $\mu\brac{\obj{\abs{g} = +\infty}} = 0$. If $f:\brac{\Omega, \Fcal}\to\brac{\Rbar, \borel{\Rbar}}$ is measurable and $\int \abs{f} d\mu < +\infty$, then $\abs{f} < +\infty$ $\mu$-a.s., and so there is $N\in \Fcal$ with $\mu\brac{N} = 0$ such that $\abs{f}\in \Real$ everywhere on $N$. For $g \defn f 1_N$ then $g\in L^1_\Real\brac{\Omega, \Fcal, \mu}$ and $g=f$ $\mu$-a.s., and thus $f\in L^1$. It is thus trivial to see that $L^1_\Real\brac{\Omega, \Fcal, \mu}\subseteq L^1$. Next, obviously if $f\in L^1$ and $f=h$ $\mu$-a.s. then $h\in L^1$. Finally if for $f\in L^1$ there are $g_1, g_2\in L^1_\Real\brac{\Omega, \Fcal, \mu}$ such that $f=g_1$ $\mu$-a.s. and $f=g_2$ $\mu$-a.s., then $g_1 = g_2$ $\mu$-a.s. and $\int g_1 d\mu = \int g_2 d\mu$. Therefore it is possible to extend in a well-defined manner the usual Lebesgue integral defined on $L^1_\Real\brac{\Omega, \Fcal, \mu}$ to the maps in $L^1$ by integrating the almost surely equal integrable counterpart. If $f \in L^1 \cap L^1_\Real\brac{\Omega, \Fcal, \mu}$ then the extended integral coincides with the usual signed Lebesgue integral.

Let $\brac{f_n}_{n\geq 1}, f, h\in L^1$ be such that $f_n\to f$ $\mu$-a.s. and $\abs{f_n}\leq h$ $\mu$-a.s. for all $n\geq 1$. Since there are countably many $\mu$ almost sure properties, there is $N_1\in \Fcal$ with $\mu\brac{N_1}=0$ such that $f_n\brac{\omega}\to f\brac{\omega}$ and $\abs{f_n\brac{\omega}}\leq h\brac{\omega}$ for all $n\geq 1$ for any $\omega\notin N_1$. There also exist $\brac{g_n}_{n\geq 1}, g, h_1 \in L^1_\Real\brac{\Omega, \Fcal, \mu}$ such that $f_n=g_n$ $\mu$-a.s. for $n\geq 1$, $f=g$ $\mu$-a.s. and , $h=h_1$ $\mu$-a.s. By similar reasoning there is $N_2\in \Fcal$ with $\mu\brac{N_2}=0$ and such that $f_n\brac{\omega}=g_n\brac{\omega}$ for all $n\geq 1$, $f\brac{\omega}=g\brac{\omega}$ and $h\brac{\omega}=h_1\brac{\omega}$ for any $\omega\notin N_2$. Therefore there is $N\in \Fcal$ with $\mu\brac{N}=0$ and such that $g_n\brac{\omega}\to g\brac{\omega}$ pointwise and $\abs{g_n\brac{\omega}}\leq h_1\brac{\omega}$ for every $n\geq 1$ and all $\omega\notin N$. So applying the DCT to $g_n 1_{N^c}$, $g 1_{N^c}$ and $h_1 1_{N^c}$, yields $\int \abs{ g_n 1_{N^c} - g 1_{N^c} } d\mu \to 0$ and $\abs{g 1_{N^c}} \leq h_1 1_{N^c}$. Since $\abs{f_n-f} = \abs{g_n 1_{N^c}-g 1_{N^c}}$ $\mu$-a.s., $\int \abs{ f_n - f } d\mu \defn \int \abs{ g_n 1_{N^c} - g 1_{N^c} } d\mu$, whence $\int \abs{ f_n - f } d\mu \to 0$. Though it can only be asserted that $\abs{f}\leq h$ holds $\mu$ almost surely and not pointwise.\\

\label{thm:fubini4} \noindent \textbf{Theorem} 33.
Let $\brac{\Omega_1, \Fcal_1, \mu_1}$ and $\brac{\Omega_2, \Fcal_2, \mu_2}$ be $\sigma$-finite measure spaces. For every $\Cplx$-valued integrable map $f\in L^1_\Cplx\brac{\Omega_1\times\Omega_2, \Fcal_1\otimes\Fcal_2, \mu_1\otimes\mu_2}$, the map \[\omega_1\to\int_{\Omega_2} f\brac{\omega_1, x} d\mu_2\brac{x}\] is $\mu_1$-almost surely equal to an element of $L^1_\Cplx\brac{\Omega_1,\Fcal_1, \mu_1}$ and \[\int_{\Omega_1} \brac{ \int_{\Omega_2} f\brac{x, y} d\mu_2\brac{y} } d\mu_1\brac{x} = \int_{\Omega_1\times\Omega_2} f d\mu_1\otimes\mu_2\] Similarly, the map\[\omega_2\to\int_{\Omega_1} f\brac{x, \omega_2} d\mu_1\brac{x}\] is $\mu_2$-almost surely equal to an element of $L^1_\Cplx\brac{\Omega_2,\Fcal_2, \mu_2}x$ and \[\int_{\Omega_2} \brac{ \int_{\Omega_1} f\brac{x, y} d\mu_1\brac{x} } d\mu_2\brac{y} = \int_{\Omega_1\times\Omega_2} f d\mu_1\otimes\mu_2\]

Indeed, suppose  $f\in L^1_\Real\brac{\Omega_1\times\Omega_2, \Fcal_1\otimes\Fcal_2, \mu_1\otimes\mu_2}$. By theorem 29 $f\brac{\omega_1, \cdot}$ is $\brac{\Omega_2, \Fcal_2}$ measurable for every $\omega_1\in\Omega_1$. Since the measure space $\brac{\Omega_2, \Fcal_2, \mu_2}$ is $\sigma$-finite and the map $\abs{f}$ is non-negative and measurable with respect to $\Fcal_1\otimes \Fcal_2$, theorem 30 implies that the map \[\omega_1\to\int_{\Omega_2} \abs{f\brac{\omega_1, x}} d\mu_2\brac{x}\] is measurable with respect to $\Fcal_1$ and $\borel{\Rbar}$. Hence \[A\defn \obj{ \induc{ \omega_1\in \Omega_1 } \int_{\Omega_2} \abs{ f\brac{\omega_1, y } } d\mu_2\brac{y} < +\infty }\] is $\Fcal_1$-measurable.

By theorem 31 (Fubini) for $\abs{f}$ over these measure spaces \[\int_{\Omega_1} \brac{ \int_{\Omega_2} \abs{f\brac{x, y}} d\mu_2\brac{y} } d\mu_1\brac{x} = \int_{\Omega_1\times\Omega_2} \abs{f} d\mu_1\otimes\mu_2\] Since $f\in L^1_\Real$, $\int_{\Omega_1\times\Omega_2} \abs{f} d\mu_1\otimes\mu_2 < +\infty$ implies that $\mu_1\brac{ A^c } = 0$. Therefore for every $\omega_1\in A$, $f\brac{\omega_1, \cdot}\in L^1_\Real\brac{\Omega_2, \Fcal_2, \mu_2}$ and the signed Lebesgue integral $\int_{\Omega_2} f\brac{\omega_1, y} d\mu_2\brac{y}$ over $\brac{\Omega_2, \Fcal_2, \mu_2}$ is well defined.

Since maps $f^+, f^-$ are non-negative and measurable with respect to $\Fcal_1\otimes \Fcal_2$, for any $u = f^+,f^-$ the map $\omega_1\to\int_{\Omega_2} u \brac{\omega_1, y} d\mu_2\brac{y}$ is $\Fcal_1$-$\borel{\Rbar}$ measurable by theorem 30 and $\Real^+$-valued on $A$, because $0\geq u\leq \abs{f}$. Hence $\omega_1\to 1_A\brac{\omega_1} \int_{\Omega_2} u\brac{\omega_1, y} d\mu_2\brac{y}$ is $L^1_\Real\brac{\Omega_1, \Fcal_1, \mu_1}$. Due to theorem 31 (Fubini) and the fact that $\mu_1\brac{A^c} = 0$ it is true that \[\int_{\Omega_1} \brac{ 1_A\brac{\omega_1}\int_{\Omega_2} u\brac{\omega_1, \omega_2} d\mu_2 } d\mu_1 = \int_{\Omega_1} \brac{ \int_{\Omega_2} u\brac{\omega_1, \omega_2} d\mu_2 } d\mu_1 = \int_{\Omega_1\times\Omega_2} u d\mu_1\otimes\mu_2\]

Put $\bar{I}\brac{\omega_1} \defn \int_{\Omega_2} f\brac{\omega_1, y} d\mu_2\brac{y}$ and let $I:\Omega_1\to\Rbar$ be an arbitrary extension of $\bar{I}$, i.e. $\induc{I}_A = \bar{I}$, which is not required to be measurable at all. If $J\defn I 1_A$, then the map \[\omega_1\to 1_A\brac{\omega_1} \int_{\Omega_2} f^+\brac{\omega_1, \omega_2} d\mu_2 - 1_A\brac{\omega_1} \int_{\Omega_2} f^-\brac{\omega_1, \omega_2} d\mu_2 \] is identical to $J\brac{\omega_1}$ since $J\brac{\omega_1} = 0 I\brac{\omega_1} = 0$ for any $\omega_1\notin A$ while $J\brac{\omega_1} = \bar{I}\brac{\omega_1}$ and the signed Lebesgue integral is legitimate for all $\omega_1\in A$. Therefore $J$ is well-defined, $\Real$-valued and $\Fcal_1$ measurable. Furthermore, by the triangle inequality for Lebesgue integral over $\brac{\Omega_1, \Fcal_1, \mu_1}$ and the above observation \[\int_{\Omega_1} \abs{J} d\mu_1 \leq \int_{\Omega_1} \brac{ 1_A\brac{x} \int_{\Omega_2} \abs{f\brac{x, y}} d\mu_2\brac{y} } d\mu_1\brac{x} = \int_{\Omega_1\times\Omega_2} \abs{f} d\mu_1\otimes\mu_2\] from where it follows that $J\in L^1_\Real\brac{\Omega_1, \Fcal_1, \mu_1}$. Moreover $J = I$ $\mu_1$-a.s, since there is $N \defn A^c\in \Fcal_1$ with $\mu_1\brac{N} = 0$ such that $J = I$ over $N^c$.

As in exercises 14, 15 above, $I\in L^1$ and so $\int_{\Omega_1} I d\mu_1\brac{x} \defn \int_{\Omega_1} J d\mu_1$. Therefore by the definition of the signed Lebesgue integral, its general linearity and by the prior observation \[\int_{\Omega_1} J d\mu_1 = \int_{\Omega_1} \brac{ 1_A\int_{\Omega_2} f^+ d\mu_2 - 1_A\int_{\Omega_2} f^- d\mu_2 } d\mu_1 = \int_{\Omega_1\times\Omega_2} f^+ d\mu_1\otimes\mu_2 - \int_{\Omega_1\times\Omega_2} f^- d\mu_1\otimes\mu_2\] Thus for every $f\in L^1_\Real$ the map $\omega_1\to \int_{\Omega_2} f\brac{\omega_1, y} d\mu_2\brac{y}$ is $\mu_1$-almost surely equal to $J\in L^1_\Real\brac{\Omega_1,\Fcal_1, \mu_1}$ with $\int_{\Omega_1} J d\mu_1 = \int_{\Omega_1\times\Omega_2} f d\mu_1\otimes\mu_2$.

If $f\in L^1_\Cplx\brac{\Omega_1\times\Omega_2, \Fcal_1\otimes\Fcal_2, \mu_1\otimes\mu_2}$ then the maps $\re{f}, \im{f} \in L^1_\Real$, whence there are $J_r, J_i\in L^1_\Real\brac{\Omega_1, \Fcal_1, \mu_1}$ such that $J_r = \int_{\Omega_2} \re{f} d\mu_2$ $\mu_1$-a.s. and $J_i = \int_{\Omega_2} \im{f} d\mu_2$ $\mu_1$-a.s. Furthermore \[\int_{\Omega_1} J_r d\mu_1 = \int_{\Omega_1\times\Omega_2} \re{f} d\mu_1\otimes\mu_2\,\text{and}\,\int_{\Omega_1} J_i d\mu_1 = \int_{\Omega_1\times\Omega_2} \im{f} d\mu_1\otimes\mu_2\] By definition of the complex Lebesgue integral \[\int_{\Omega_1\times\Omega_2} f d\mu_1\otimes\mu_2 = \int_{\Omega_1} J_r d\mu_1 + i \int_{\Omega_1} J_i d\mu_1 = \int_{\Omega_1} J d\mu_1 \] where $J\defn J_1 + i J_2$ is an element of $L^1_\Cplx\brac{\Omega_1,\Fcal_1, \mu_1}$. For this $f$ there is $M\in \Fcal_1$ of $\mu_1$-measure zero, such that the map \[\omega_1\to \int_{\Omega_2} \re{f}\brac{\omega_1, \omega_2} d\mu_2 + i \int_{\Omega_2} \im{f}\brac{\omega_1, \omega_2} d\mu_2 \] is a well-defined complex number outside of $M$. On the whole $\Omega_1$ this map is understood as an arbitrary function $\Omega_1\to \Cplx$, the restriction of which to $M^c$ is this sum of Lebesgue integrals.

By theorem 29 $f\brac{\omega_1, \cdot}$ is measurable with respect to $\Fcal_2$ and $\borel{\Cplx}$, and by theorem 30 the map $\omega_1\to\int_{\Omega_2}\abs{f\brac{\omega_1, \omega_2}} d\mu_2$ is $\Fcal_1$-$\borel{\Rbar}$ measurable, whence $A \defn \obj{\int_{\Omega_2}\abs{f\brac{\cdot, \omega_2}}d\mu_2 < +\infty} \in \Fcal_1$. From theorem 31 (Fubini) it follows that $\mu_1\brac{A^c} = 0$, whence $f\brac{\omega_1, \cdot} \in L^1_\Cplx\brac{\Omega_2, \Fcal_2, \mu_2}$ for every $\omega_1\in A$. Therefore the marginal complex Lebesgue integral $\bar{I}\brac{\omega_1}\defn\int_{\Omega_2} f\brac{\omega_1, x} d\mu_2\brac{x}$ is well-defined for every $\omega_1 \in A$.

If the map $I$ is a an arbitrary extension of $\bar{I}$ to the whole $\Omega_1$, then for $J'\brac{\omega_1} \defn 1_A\brac{\omega_1} I\brac{\omega_1}$ it is true that $J' = I$ $\mu_1$-a.s., since $A^c$ is meagre with respect to $\mu_1$.Furthermore $J$ is $\Fcal_1$-$\borel{\Cplx}$ measurable, because \[ J' = 1_A \int_{\Omega_2} u^+ d\mu_2 - 1_A \int_{\Omega_2} u^- d\mu_2 + i\brac{ 1_A \int_{\Omega_2} v^+ d\mu_2 - 1_A \int_{\Omega_2} v^- d\mu_2 } \] where $u=\re{f}$ and $v=\im{f}$, each integral is measurable due to theorem 30 and the sum is meaningful, since $u^+, u^-, v^+, v^-\leq \abs{f}$ and $\int_{\Omega_2} \abs{f\brac{\omega_1, \omega_2}} d\mu_2 < +\infty$ over $A$. Due to the triangle inequality for complex Lebesgue integrals $J'\in L^1_\Cplx\brac{\Omega_1, \Fcal_1, \mu_1}$.

By definition of the complex Lebesgue integral the maps $J'$, $I$ and $J$ coincide on $A$ and so all three are $\mu_1$-almost surely equal. Thus there exists $J\in L^1_\Cplx\brac{\Omega_1, \Fcal_1, \mu_1}$ such that $\omega_1\to\int_{\Omega_2} f\brac{\omega_1, x} d\mu_2\brac{x}$ and $J$ are $\mu_1$-almost surely equal. Furthermore by extending the definition of the complex Lebesgue integral to $L^1$-like maps one gets \[\int_{\Omega_1}\brac{ \int_{\Omega_2} f\brac{\omega_1, \omega_2} d\mu_2 }d\mu_1 \defn \int_{\Omega_1} J d\mu_1 = \int_{\Omega_1\times\Omega_2} f d\mu_1\otimes\mu_2\]

If $\theta:\brac{\Omega_2\times\Omega_1, \Fcal_2\otimes\Fcal_1} \to \brac{\Omega_1\times\Omega_2, \Fcal_1\otimes\Fcal_2}$ is defined as $\theta\brac{\omega_2, \omega_1}=\brac{\omega_1, \omega_2}$, then by theorem 28 $\theta$ is measurable. For  $f:\brac{\Omega_1\times\Omega_2, \Fcal_1\otimes\Fcal_2}\to \Zinf$ the map $f\circ \theta$ is $\Fcal_2\otimes\Fcal_1$-$\borel{\Rbar}$ measurable. So by theorem 31 (Fubini) for $f\circ \theta$ \[\int_{\Omega_2\times\Omega_1} f\circ \theta d\mu_2\otimes\mu_1 = \int_{\Omega_1} \brac{ \int_{\Omega_2} \brac{f\circ \theta}\brac{\omega_2, \omega_1} d\mu_2 }d\mu_1 = \int_{\Omega_1} \brac{ \int_{\Omega_2} f\brac{\omega_1, \omega_2} d\mu_2 }d\mu_1\] whence by theorem 31 for $f$ it holds that \[\int_{\Omega_1} \brac{ \int_{\Omega_2} f\brac{\omega_1, \omega_2} d\mu_2 }d\mu_1 = \int_{\Omega_1\times\Omega_2} f d\mu_1\otimes\mu_2\] Thus if $f\in L^1_\Cplx$ then $f\circ \theta \in L^1_\Cplx\brac{\Omega_2\times\Omega_1, \Fcal_2\otimes\Fcal_1, \mu_2\otimes\mu_1}$, because $\abs{f \circ \theta} = \abs{f}$. Also since $\re{f\circ \theta} =\re{f}\circ \theta$ and $\brac{u\circ \theta}^+ = u^+ \circ \theta$ the above result for non-negative maps and the definition of the complex Lebesgue integral imply \[\int_{\Omega_2\times\Omega_1} f\circ \theta d\mu_2\otimes\mu_1 = \int_{\Omega_1\times\Omega_2} f d\mu_1\otimes\mu_2\]

Finally, for $h \defn f\circ \theta \in L^1_\Cplx\brac{\Omega_2\times\Omega_1, \Fcal_2\otimes\Fcal_1, \mu_2\otimes\mu_1}$ due to symmetry of $\brac{\Omega_1, \Fcal_1, \mu_1}$ and $\brac{\Omega_2, \Fcal_2, \mu_2}$ the prior results yield the existence of a map $J\in L^1_\Cplx\brac{\Omega_2, \Fcal_2, \mu_2}$ which is $\mu_2$-almost surely equal to $\omega_2\to\int_{\Omega_1} h\brac{\omega_2, x} d\mu_1\brac{x}$ such that \[\int_{\Omega_2} J d\mu_2 = \int_{\Omega_2\times\Omega_1} h d\mu_2\otimes\mu_1 \equiv \int_{\Omega_1\times\Omega_2} f d\mu_1\otimes\mu_2 \] Furthermore by extending the complex Lebesgue integral to $L^1$ $\mu_2$-a.s. integrable maps \[\int_{\Omega_2} \brac{ \int_{\Omega_1} f\brac{ x, \omega_2 } d\mu_1\brac{x} } d\mu_2\brac{\omega_2} \equiv \int_{\Omega_2} \brac{ \int_{\Omega_1} h\brac{ \omega_2, x } d\mu_1\brac{x} } d\mu_2\brac{\omega_2}\defn \int_{\Omega_2} J d\mu_2\] one obtains the symmetrical result, stated in the theorem.\\

%% Fubini extension

Let $\brac{\Omega_i, \Fcal_i, \mu_i}_{i=1}^n$ be a collection of $\sigma$-finite measure spaces, $n\geq 2$. Let $\sigma$ be a permutation of $\obj{1\ldots n}$ and $f\in L^1_\Cplx\brac{ \prod_{i=1}^n \Omega_i, \bigotimes_{i=1}^n \Fcal_i, \bigotimes_{i=1}^n \mu_i}$. Denote $\Omega \defn \prod_{i=1}^n \Omega_i$, $\Fcal \defn \bigotimes_{i=1}^n \Fcal_i$ and $\mu \defn \bigotimes_{i=1}^n \mu_i$. Let's introduce the following notation for sake of brevity. For every $k = 0\ldots n-1$ put $E_{-k} \defn \prod_{i\notin \sigma\brac{1\ldots k}} \Omega_i$, $\Ecal_{-k} \defn \bigotimes_{i\notin \sigma\brac{1\ldots k}} \Fcal_i$ and $\nu_{-k}\defn \bigotimes_{i\notin \sigma\brac{1\ldots k}} \mu_i$. Similarly for every $k=1\ldots n$ let $E_k \defn \Omega_{\sigma\brac{k}}$, $\Ecal_k \defn \Fcal_{\sigma\brac{k}}$ and $\nu_k\defn \mu_{\sigma\brac{k}}$.

For $k=0$ put $J_0 = \bar{J}_0 \defn f$ and for $k=1\ldots n-2$ let $J_k:E_{-k}\to \Cplx$ be $\nu_{-k}$-almost surely equal to some $\bar{J}_k\in L^1_\Cplx\brac{ E_{-k}, \Ecal_{-k}, \nu_{-k} }$ with \[\int_{E_{-k}} \bar{J}_k d\nu_{-k} = \int_\Omega f d\mu\]

For any $k=1\ldots n-1$, the spaces $\brac{E_{-k}, \Ecal_{-k}, \nu_{-k}}$ and $\brac{E_k, \Ecal_k, \nu_k}$ are $\sigma$-finite, since $\nu_{-k}$ is the product measure of $\sigma$-finite measures. By theorem 6-3 the measurable space $\brac{E_{-k}\times E_k, \Ecal_{-k}\otimes \Ecal_k}$ is equivalent to $\brac{ E_{-\brac{k-1}}, \Ecal_{-\brac{k-1}} }$, and by theorem 7-1 the product measure $\nu_{-k}\otimes \nu_k$ is identical to $\nu_{-\brac{k-1}}$ since both coincide on the measurable rectangles spawning $\Ecal_{-\brac{k-1}}$.

Therefore for $\bar{J}_{k-1}\in L^1_\Cplx\brac{ E_{-\brac{k-1}}, \Ecal_{-\brac{k-1}}, \nu_{-\brac{k-1}} }$ theorem 33 implies that the map $J_k:E_{-k}\to \Cplx$ defined as \[J_k\brac{\omega_{-k}}\defn \int_{E_k} \bar{J}_{k-1}\brac{\omega_{-k}, \omega_k} d\nu_k\brac{\omega_k}\] is $\nu_{-k}$-almost surely equal to $\bar{J}_k \in L^1_\Cplx\brac{ E_{-k}, \Ecal_{-k}, \nu_{-k} }$ and \[ \int_{E_{-k}} J_k d\nu_{-k} = \int_{E_{-k}} \brac{ \int_{E_k} \bar{J}_{k-1}\brac{\omega_{-k}, \omega_k} d\nu_k } d\nu_{-k} = \int_{E_{-\brac{k-1}}} \bar{J}_{k-1} d\nu_{-\brac{k-1}} = \int_\Omega f d\mu\] Where the complex Lebesgue integral is extended to $\Cplx$-valued functions, which are only almost surely integrable, in the manner suggested in exercise 14: $\int_{E_{-k}} J_k d\nu_{-k} \defn \int_{E_{-k}} \bar{J}_k d\nu_{-k}$.

Thus, by finite induction until $k=n-1$, the map $J_{n-1}:E_{-n}\to \Cplx$ has a $\nu_{-n}$-almost surely identical twin $\bar{J}_{n-1}\in L^1_\Cplx\brac{ E_{-n}, \Ecal_{-n}, \nu_{-n} }$ with $\int_{E_{-n}} \bar{J}_{n-1} d\nu_{-n} = \int_\Omega f d\mu$. In the original notation, $J_{n-1}:\Omega_{\sigma\brac{n}}\to \Cplx$ is $\mu_{\sigma\brac{n}}$-a.s. equal to some $\bar{J}_{n-1}\in L^1_\Cplx\brac{ \Omega_{\sigma\brac{n}}, \Fcal_{\sigma\brac{n}}, \mu_{\sigma\brac{n}}}$ with \[\int_{\Omega_{\sigma\brac{n}}} \bar{J}_{n-1} d\mu_{\sigma\brac{n}} = \int_{\prod_{i=1}^n \Omega_i} f d\bigotimes_{i=1}^n \mu_i\] Therefore, very much like it has been done in exercise 14, it is possible to define the multiple integral as \[\int_{\Omega_{\sigma\brac{n}}} \ldots \int_{\Omega_{\sigma\brac{2}}} \brac{ \int_{\Omega_{\sigma\brac{1}}} f d\mu_{\sigma\brac{1}} } d\mu_{\sigma\brac{2}} \ldots d\mu_{\sigma\brac{n}} \defn \int_{\Omega_{\sigma\brac{n}}} J_{n-1} d\mu_{\sigma\brac{n}}\] where the right-hand side integral is understood as the extended complex Lebesgue integral. Under this definition, bearing in mind the property of $\bar{J}_{n-1}$, it is true that \[\int_{\Omega_{\sigma\brac{n}}} \ldots \int_{\Omega_{\sigma\brac{2}}} \brac{ \int_{\Omega_{\sigma\brac{1}}} f d\mu_{\sigma\brac{1}} } d\mu_{\sigma\brac{2}} \ldots d\mu_{\sigma\brac{n}} = \int_{\prod_{i=1}^n \Omega_i} f d\bigotimes_{i=1}^n \mu_i \]


% section tut_7 (end)

\section{Jensen inequality} % (fold)
\label{sec:tut_8}
\url{http://probability.net/PRTjensen.pdf}

\noindent \textbf{Definition} 64.
Let $a < b\in \Real$. The function $\phi:\brac{a, b}\to \Real$ is a convex function if for all $x, y\in \brac{a,b}$ and $t\in \clo{0,1}$ \[\phi\brac{tx+\brac{1-t}y} \leq t\phi\brac{x}+\brac{1-t}\phi\brac{y} \]

\label{thm:convex_func1} \noindent \textbf{Theorem} 8-1.
Let $a<b\in \Rbar$ and $\phi:\brac{a, b}\to \Real$ be a map. Then $\phi$ is convex if and only if for any $\brac{x_k}_{k=1}^n\in \brac{a,b}$ and any $\brac{\alpha_k}_{k=1}^n\in \Real+$ with $\sum_{k=1}^n \alpha_k = 1$ \[\phi\brac{\sum_{k=1}^n \alpha_k x_k}\leq \sum_{k=1}^n \alpha_k \phi\brac{x_k}\]

Indeed, let $\phi$ be convex. Obviously, the statement is true for $n=1$, so consider any $n\geq 2$. Assume that for any $\brac{x_k}_{k=1}^{n-1}\in \brac{a,b}$ and any $\brac{\alpha_k}_{k=1}^{n-1}\in \Real+$ with $\sum_{k=1}^{n-1} \alpha_k = 1$ \[\phi\brac{\sum_{k=1}^{n-1} \alpha_k x_k}\leq \sum_{k=1}^{n-1} \alpha_k \phi\brac{x_k}\]

Consider any $\brac{x_k}_{k=1}^n\in \brac{a,b}$ and any $\brac{\alpha_k}_{k=1}^n\in \Real+$ with $\sum_{k=1}^n \alpha_k = 1$. Put $\alpha_{-n} \defn \sum_{k=1}^{n-1} \alpha_k$, and note that $\alpha_{-n}\in \Real+$ and $\alpha_n + \alpha_{-n} = 1$. If $\alpha_{-n} = 0$, then $\alpha_n = 1$ due to the constraint and $\alpha_k = 0$ for every $k=1\ldots n-1$ since $\alpha_k\geq 0$. Furthermore $\sum_{k=1}^n \alpha_k x_k = x_n$ and so $ \phi\brac{\sum_{k=1}^n \alpha_k x_k} = \phi\brac{x_n} \leq \phi\brac{x_n} = \sum_{k=1}^n \alpha_k \phi\brac{x_k}$. If $\alpha_{-n} \neq 0$ then $\alpha_{-n} \in \ploc{0, 1}$, since $\brac{\alpha_k}_{k=1}^n\in \Real^+$. For $x_{-n}\defn \sum_{k=1}^{n-1} \frac{\alpha_k}{\alpha_{-n}} x_k$ and $t\defn \alpha_{-n}$ it is true that $x_{-n}\in \brac{a, b}$ and $t \in \clo{0,1}$. Then since $\phi$ is convex for $x_{-n}, x_n$ and $t$ \[ \phi\brac{\sum_{k=1}^n \alpha_k x_k} = \phi\brac{t x_{-n} + \brac{1-t} x_n } \leq t \phi\brac{ x_{-n} } + \brac{1-t} \phi\brac{x_n} = \alpha_{-n} \phi\brac{ x_{-n} } + \alpha_n \phi\brac{x_n}\] However, the inductive hypothesis implies that \[\phi\brac{ x_{-n} } = \phi\brac{ \sum_{k=1}^{n-1} \frac{\alpha_k}{\alpha_{-n}} x_k } \leq \sum_{k=1}^{n-1} \frac{\alpha_k}{\alpha_{-n}} \phi\brac{ x_k }\] whence $\phi\brac{\sum_{k=1}^n \alpha_k x_k} \leq \sum_{k=1}^n \alpha_k \phi\brac{x_k}$. Thus by induction the statement is true for every $n\geq 1$.

The converse is almost immediate. For the statement holds in particular for $n=2$, so for $x, y\in \brac{a,b}$ and $t\in \clo{0,1}$, we have $\phi\brac{t x + \brac{1-t} y} \leq t \phi\brac{x} + \brac{1-t} \phi\brac{y}$.\\

\label{thm:convex_func2} \noindent \textbf{Theorem} 8-2.
Let $a<b\in \Rbar$ and $\phi:\brac{a, b}\to \Real$ be a map. Then $\phi$ is convex if and only if for all $x,y,z\in \brac{a,b}$ with $x<y<z$ it is true that \[\phi\brac{y} \leq \frac{z-y}{z-x}\phi\brac{x} + \frac{y-x}{z-x}\phi\brac{z}\]

Indeed, if $a<x<y<z<b$, then $t\defn \frac{z-y}{z-x} \in \brac{0,1}$ and \[t x + \brac{1-t} z = \frac{z-y}{z-x} x + \frac{y-x}{z-x} z = y \] from where it follows that \[\phi\brac{y} = \phi\brac{t x + \brac{1-t} z } \leq t \phi\brac{x} + \brac{1-t} \phi\brac{z} = \frac{z-y}{z-x} \phi\brac{x} + \frac{y-x}{z-x} \phi\brac{z}\]

The converse is obvious, since for any $x,z\in \brac{a,b}$ and $t\in \clo{0,1}$, setting $y\defn t x \brac{1-t} z$ gives $\frac{z-y}{z-x} = t$, since $y = z - t \brac{z-x}$.\\

\label{thm:convex_func3} \noindent \textbf{Theorem} 8-3.
Let $a<b\in \Rbar$ and $\phi:\brac{a, b}\to \Real$ be a map. Then $\phi$ is convex if and only if for all $x,y,z\in \brac{a,b}$ with $x<y<z$ it is true that \[\frac{\brac{\phi\brac{y}-\phi\brac{x}}}{x-y} \leq \frac{\brac{\phi\brac{z}-\phi\brac{y}}}{z-y}\]

Indeed, let $\phi$ is convex and $a<x<y<z<b$. By theorem 8-2 and the relationship between $x$,$y$ and $z$ \[\frac{z-y}{z-x} \phi\brac{y} + \frac{y-x}{z-x} \phi\brac{y} = \phi\brac{y} \leq \frac{z-y}{z-x}\phi\brac{x} + \frac{y-x}{z-x}\phi\brac{z}\] Rearranging both sides of the inequality gives \[\frac{z-y}{z-x}\brac{ \phi\brac{y} -\phi\brac{x} } \leq \frac{y-x}{z-x}\brac{ \phi\brac{z} -\phi\brac{y} } \] which after cancelling the denominators and dividing by $\brac{y-x}\brac{z-y}$ yields the desired result. The converse holds again by theorem 8-2 and since the employed arithmetic transformations are non-degenerate.\\

\label{thm:convex_func4} \noindent \textbf{Theorem} 8-4.
Let $a<b\in \Rbar$ and $\phi:\brac{a, b}\to \Real$ be a convex function. Then $\phi$ is $\brac{a, b}$-$\Real$ continuous.

First, let's show a useful auxiliary result. Let $x_0\in \brac{a,b}$ and suppose there exist $u, u', v, v' \in \brac{a,b}$ such that $u < u' < x_0 < v < v'$. Then, for any $x\in \brac{x_0, v}$ theorem 8-3 applied sequentially to the chain $u<u'< x_0 < x < v < v'$ yields \[\frac{\phi\brac{u'} - \phi\brac{u}}{u'-u} \leq \frac{\phi\brac{x_0} - \phi\brac{u'}}{x_0-u'} \leq \frac{\phi\brac{x} - \phi\brac{x_0}}{x-x_0} \leq \frac{\phi\brac{v} - \phi\brac{x}}{v-x} \leq \frac{\phi\brac{v'} - \phi\brac{v}}{v'-v}\] Similarly, for any $x\in \brac{u', x_0}$ applying theorem 8-3 to the chain $u<u'<x<x_0<v<v'$ yields \[\frac{\phi\brac{u'} - \phi\brac{u}}{u'-u} \leq \frac{\phi\brac{x_0} - \phi\brac{x}}{x_0-x} \leq \frac{\phi\brac{v'} - \phi\brac{v}}{v'-v}\] Therefore, for any $x_0\in \brac{a, b}$, there is $\brac{u',v}\subseteq \brac{a,b}$, with $u<x_0<v$, and $\alpha\leq \beta \in \Real$ such that for every $x\in \brac{u',v}$, $x\neq x_0$ \[\alpha \leq \frac{\phi\brac{x} - \phi\brac{x_0}}{x-x_0} \leq \beta\] If $M\defn \abs{\alpha}\wedge\abs{\beta}$, then $-M \leq -\abs{\alpha}\leq \alpha$ and $\beta\leq \abs{\beta} \leq M$ implies \[\abs{\frac{\phi\brac{x} - \phi\brac{x_0}}{x-x_0}} \leq M \Leftrightarrow \abs{\phi\brac{x} - \phi\brac{x_0}}\leq M \abs{x-x_0} \]

Since the topological space on $\brac{\brac{a,b}, \Tcal_{\brac{a,b}} }$ is the induced topological space of $\Real$, endowed with the usual topology $\Tcal_\Real$, which is metrizable, theorem 12 implies that $\brac{a,b}$ is metrizable by the induced metric. Therefore it is legitimate to employ the $\epsilon$-$\delta$ definition of continuity.

Let $x_0\in \brac{a,b}$ then there exist $u, u', v, v' \in \brac{a,b}$ such that $u < u' < x_0 < v < v'$. Indeed, since $\brac{a,b}$ is open in $\Real$ for this $x_0$ there is $\eta>0$ with $\brac{ x_0-\eta, x_0+\eta }\subseteq \brac{a, b}$, whence it is possible to pick $u,u'$ from $\brac{x_0-\eta, x_0}$ and $v,v'\in\brac{x_0, x_0+\eta}$. For such $u', v$ and $\delta_0 \defn \min\obj{ x_0 - u', v - x_0} > 0$. It is extremely important that $x_0$ be in the interior of the support of $\phi$, since in this case convexity guarantees that there are two-sided finite bounds on $\frac{\phi\brac{x}-\phi\brac{x_0}}{x-x_0}$. The fact that the topological interior of the support of $\phi$ coincides with the support it later used to prove continuity of $\phi$.

So, pick any $x_0\in \brac{a, b}$ and let $\epsilon>0$. If, on the one hand, $M$ for this $x_0$ is zero, then $\abs{\phi\brac{x}-\phi\brac{x_0}} \leq 0$ for any $x\in\brac{a,b}$ with $\abs{x-x_0}<\delta_0$, whence $\phi$ is constant in some open neighbourhood of $x_0$ and thus continuous. On the other hand, if $M>0$ then taking $\delta\defn \min\obj{ \frac{ \epsilon }{ M }, \delta_0 }$ implies $\abs{\phi\brac{x} - \phi\brac{x_0}} \leq M \abs{x-x_0} < \epsilon$ for any $x\in \brac{a,b}$ with $\abs{x-x_0}<\delta$. Therefore the map $\phi$ is continuous at $x_0$.

For any $U$ open in $\Real$, if $x_0 \in \phi^{-1}\brac{U}$, then there is $\epsilon>0$ such that $B\brac{\phi\brac{x_0}, \epsilon}\subseteq U$. Therefore by continuity of $\phi$ at $x_0$ there is $\delta>0$ such that $\phi\brac{x}\in B\brac{\phi\brac{x_0}, \epsilon}$ for all $x\in B\brac{x_0, \delta}$, where $B$ is the open metric ball in $\Real$ and $\brac{a,b}$ respectively. Thus $B\brac{x_0, \delta} \subseteq \phi^{-1}\brac{ B\brac{\phi\brac{x_0}, \epsilon} } \subseteq \phi^{-1}\brac{U}$ and so for $x_0 \in \phi^{-1}\brac{U}$ there is $\delta>0$ such that $B\brac{x_0, \delta}\subseteq \phi^{-1}\brac{U}$, whence $\phi^{-1}\brac{U}$ is open in $\brac{a,b}$.\\

\noindent \textbf{Definition} 65.
A topological space $\brac{\Omega, \Tcal}$ is compact if for any $\brac{U_i}_{i\in I}$ open in $\brac{\Omega, \Tcal}$ with $\Omega = \bigcup_{i\in I} U_i$, there exists $F\subseteq I$ - finite such that $\Omega = \bigcup_{i\in F} U_i$. In short $\brac{\Omega, \Tcal}$ is compact if from any open covering of $\Omega$ it is possible to extract a finite sub-covering. Since one cannot cover anything with nothing, the set $F$ is automatically non-empty.

\noindent \textbf{Definition} 66.
Let $\brac{\Omega, \Tcal}$ be a topological space. A set $K\subseteq \Omega$ is a compact subset of $\Omega$ if the induced topological space $\brac{K, \induc{\Tcal}_K}$ is a compact topological space.

First, $\induc{\Tcal}_\Omega = \Tcal$, so if $\brac{\Omega, \Tcal}$ is compact, $\Omega$ is a compact subset of itself. Second, if $K$ is a compact subset of $\Omega'\subseteq \Omega$, then $K\subseteq \Omega$ and $\induc{\induc{\Tcal}_{\Omega'}}_K = \induc{\Tcal}_K$, whence $K$ is a compact subset of $\Omega$.

Let $K$ be a compact subset of $\Omega$ and $\brac{U_i}_{I\in I}$ be any collection of open sets in $\brac{\Omega, \Tcal}$ with $K\subseteq \cup_{i\in I} U_i$. Then $V_i\defn U_i\cap K$ are open in $\brac{K, \induc{\Tcal}_K}$ and $\brac{V_i}_{i\in I}$ covers $K$, whence $K$'s compactness implies the existence of a finite $F\subseteq I$ with $K=\cup_{i\in F} V_i$. Thus $K\subseteq \cup_{i\in F} U_i$ for some $F\subseteq I$. Conversely, let $\brac{V_i}_{i\in I}$ be an open in $\brac{K, \induc{\Tcal}_K}$ cover of $K$. Then for every $i\in I$ there is $U_i\in \Tcal$ with $V_i = U_i \cap K$ and $K\subseteq \cup_{i\in I} U_i$. Then there is a finite set $F\subseteq I$ such that $K\subseteq \cup_{i\in F} U_i$, whence $K = \cup_{i\in I} V_i$, implying compactness of $K$ as a subset of $\Omega$.

\label{thm:clo_subset_compact} \noindent \textbf{Theorem} 8-5.
Any closed subset $K$ of a compact topological space $\brac{\Omega, \Tcal}$ is compact itself.

Indeed, if $K$ is non-empty and closed, then $U\defn K^c$ is open in $\brac{\Omega, \Tcal}$ and if $\Gamma\subseteq\Tcal$ is an open covering of $K$, then $\Gamma'\defn \Gamma\cup\obj{U}$ is a covering of $\Omega$. Compactness of $\Omega$ implies that there is a non-empty $F'\subseteq \Gamma'$ with $\Omega \subseteq \cup_{V\in F'} V$. Put $F\defn F'\cap \Gamma$ and observe that $K\subseteq \cup_{V\in F} V$, since for any $x\in K\subseteq \Omega$ there is $V\in F'$ with $x\in V$ such that $V\neq U$, since $x\notin K^c$, whence $V\in F$ and $x\in F\cup_{V\in F} V$. If $K$ is empty, then $K\subseteq \Omega$. If $\brac{U_i}_{i\in I}$ is any collection of open subsets of $\Omega$, each individual $U_i$ cannot but cover $K$ because $\emptyset\subseteq W$ for any $W\subseteq \Omega$. Therefore any closed subset of a compact space is compact itself.\\

%% FIP <-> Compactness
%% Compactness -> Limit point compactness
%% Limit point compactness + T_1 -> Compactness

\label{thm:hiene_borel} \noindent \textbf{Theorem} 34.
For any $a<b\in \Real$ the closed interval $\clo{a,b}$ is a compact subset of $\Real$.

Indeed, let $a<b\in \Real$ and $\brac{U_i}_{i\in I}$ be open in $\brac{\Real, \Tcal_\Real}$ with $\clo{a,b}\subseteq \bigcup_{i\in I} U_i$. Define $A$ as the set of all $x\in \clo{a,b}$ such that there is $F\in I$ with $\clo{a, x}\subseteq\bigcup_{i\in F} U_i$, i.e. the sub-interval can be covered by finitely many $U_i$'s. Let $c\defn \sup A$.

Since $a\in \clo{a,b}\subseteq \cup_{i\in F} U_i$ there is $i_0\in I$ such that $a\in U_{i_0}$, whence the degenerate closed interval $\clo{a, a} = \obj{a}$ can be covered by a finite sub-cover $\brac{U_i}_{i\in F_a}$ where $F_a\defn \obj{i_0}\subseteq I$ and so $a\in A$. However since $U_{i_0}$ is open in $\Real$ for $a\in U_{i_0}$ there is $\delta > 0$ such that $\brac{a-\delta, a+\delta} \subseteq U_{i_0}$. For $\eta\defn \min\obj{\frac{\delta}{2}, b-a} > 0$, $\clo{a, a+\eta}\subseteq\brac{a-\delta, a+\delta}$ and $a+\eta \leq b$, whence $a+\eta \in A$ because $\brac{U_i}_{i\in F}$ is still a finite open covering. Thus $a < a+\eta \leq c\leq b$, where the last inequality stems form the fact that had $b$ been less than $c$ by definition of $\sup$ there would have existed $x\leq b$ with $b < x \leq c$.

As $c\in \clo{a,b}$, there is $i_0\in I$ such that $c\in U_{i_c}$, which implies that there is $\epsilon>0$ with $\brac{c-\epsilon, c+\epsilon} \subseteq U_{i_c}$, since $U_i\in \Tcal_\Real$ for any $i\in I$. If $c'' \in \brac{c, c+\epsilon}$ and $c'\in \brac{\max\obj{a, c-\epsilon}, c}$, then $\ploc{c', c''}\subseteq \brac{c-\epsilon, c+\epsilon} \subseteq U_{i_c}$ and $a < c'< c < c''$.

As $c'\in \brac{a, c}$, the definition of $\sup$ implies that there is $x\in \clo{a,b}$ such that $c'< x \leq c$ and there is $F'\subseteq I$ finite with $\clo{a,x}\subseteq \bigcup_{i\in F'} U_i$. Since $F''\defn F'\cup \obj{i_c}$ is at most finite, $\clo{a, c'}\subseteq \clo{a,x}$ and $\ploc{c', c''}\subseteq U_{i_c}$ it must be true that both $\clo{a, c'}$ and $\clo{a, c''}$ have finite sub-covers . If $z\defn b\wedge c''$, then $z\in \clo{a,b}$ and $\clo{a,z}\subseteq \clo{a,c''}$, whence $z\in A$, since by the above results there is finite $F\subseteq I$ with $\clo{a,z}\subseteq \bigcup_{i\in F} U_i$. Therefore $b\wedge c''\leq c$ by definition of the least upper bound, yet $c\leq b\wedge c''$. Thus $b\wedge c'' = c$, which implies that $c\in A$ and $b = c$, since $c < c''$, whence $\clo{a,b}$ can be covered by finitely many $U_i$'s. Therefore from any open covering $\brac{U_i}_{i\in I}$ of $\clo{a,b}$ it is possible to extract a finite sub-covering $F\subseteq I$ with $\clo{a,b}\subseteq \bigcup_{i\in F} U_i$, and so $\clo{a,b}$ is a compact subset of $\Real$.\\

\noindent \textbf{Definition} 67.
A topological space $\brac{\Omega, \Tcal}$ is Hausdorff if for any $x, y\in \Omega$ with $x\neq y$ there exist $U,V\in \Tcal$ such that $x\in U$, $y\in V$ and $x\in U\cap V = \emptyset$.

Let $\brac{\Omega_i, \Tcal_i}_{i \in I}$ be a family of Hausdorff topological spaces. For any $x\neq y\in \prod_{i\in I}\Omega_i$, there is $j\in I$ with $x_j\neq y_j\in \Omega_j$, whence there are $U_j, V_j$ open in $\brac{\Omega_j, \Tcal_j}$ with $x_j\in U_j$ and $y_j\in V_j$ such that $U_j\cap V_j = \emptyset$. Then $U\defn U_j\times \prod_{i\neq j} \Omega_i$ and $V\defn V_j\times \prod_{i\neq j} \Omega_i$ are such that $x\in U$, $y\in V$, $U,V\in \coprod_{i\in I}\Tcal_i\subseteq \bigodot_{i\in I}\Tcal_i$ and $U\cap V=\emptyset$.\\

\label{thm:hausdorf_metric} \noindent \textbf{Theorem} 8-6.
Any subset of a Hausdorff topological space is itself Hausdorff. If the topological space $\brac{\Omega, \Tcal}$ is metrizable, then it is Hausdorff. For example, any subset of $\Rbar$ is Hausdorff.

Let $\Omega'\subseteq \Omega$ and consider the induced topological space $\brac{\Omega', \induc{\Tcal}_{\Omega'}}$. If $x\neq y\in \Omega'$, then $x, y\in\Omega$ and $\exists U, V\in \Tcal$ with $x\in U$ and $y\in V$ such that $U\cap V = \emptyset$. Then $U'\defn U\cap \Omega'$ and $V'\defn V\cap \Omega'$ are such that $x\in U'$, $y\in V'$ and $U'\cap V' = \emptyset$. Thus the induced topological space of a Hausdorff space is itself Hausdorff.

Indeed, for a metrizable topological space there is a metric $d$ on $\Omega$ such that the metric topology $\Tcal_\Omega^d$ coincides with $\Tcal$. For any $x\neq y\in \Omega$, $\epsilon\defn \frac{1}{2} d\brac{x,y} > 0$ by the basic properties of a metric. Then $U_x\defn B_\Omega^d\brac{x, \epsilon}$ and $U_y\defn B_\Omega^d\brac{y, \epsilon}$ are such that $U_x, U_y$ are open, $x\in U_x$, $y\in u_y$ and $U_x\cap U_y = \emptyset$.

By theorem 13 $\brac{\Rbar, \Tcal_\Rbar}$ is metrizable and thus is Hausdorff. By the first part of this theorem any subset of $\Rbar$ is Hausdorff, and moreover by theorem 12 every induced topological space of $\Rbar$ is metrizable.\\

\label{thm:hausdorf_compact_closed} \noindent \textbf{Theorem} 35.
In a Hausdorff topological space $\brac{\Omega, \Tcal}$ any compact subset $K\subseteq \Omega$ is closed in $\brac{\Omega, \Tcal}$.

For empty $K$ the result is trivial, so let $K$ be a non-empty compact subset of $\Omega$ and pick any $y\in K^c$. Then for any $x\in K$, $x\neq y$ and so by the Hausdorff property there are $O_x, V_x\Tcal$ with $x\in O_x$ and $y\in V_x$ such that $O_x\cap V_x = \emptyset$. Since the collection $\brac{O_x}_{x\in K}$ is an open covering of $K$, there is finite $F\subseteq K$, such that $K\subseteq \cup_{x\in F} O_x$. The set $V\defn \cap_{x\in F} V_x$ is open in $\brac{\Omega, \Tcal}$, since it is an intersection of finite many open sets, $y\in V$ and $O_x\cap V \subseteq O_x\cap V_x = \emptyset$ for every $x\in F$ as $O_x\cap V_x = \emptyset$ for every $x\in K$. Now $O\defn \cup_{x\in F} O_x$ is open, $K\subseteq O$ and $O\cap V = \emptyset$. Hence in a Hausdorff topological space $\brac{\Omega, \Tcal}$ for any $K$ non-empty compact subset and any $y\notin K$ there are $O,V\in \Tcal$ with $K\subseteq O$ and $y\subseteq V$ such that $O\cap V = \emptyset$. This implies that for any $y\in K^c$ there is $V_y\in \Tcal$ such that $y\in V_y\subseteq O^c\subseteq K^c$, whence $K^c = \bigcap_{y\in K^c} V_y$. Therefore $K^c$ is open, hence $K$ is closed in $\brac{\Omega, \Tcal}$.\\

Now, since any closed subset of a compact space is itself compact, the last result also implies that in a compact Hausdorff space any two non-intersecting closed subsets are compact and have non-overlapping open neighbourhoods, meaning that such space is normal.\\

\noindent \textbf{Definition} 68, 69.
Let $\brac{E, d}$ be a metric space. For all $A\subseteq E$ the diameter of $A$ with respect to $d$ is $\delta\brac{A}\defn \sup\obj{ \induc{ d\brac{x, y} } x, y\in A }$ with the convention $\delta\brac{\emptyset}\defn -\infty$. A set $A\subseteq E$ is bounded if $\delta\brac{A}<+\infty$.

Suppose $A\subseteq E$ is such that $\delta\brac{A}=0$, If $x\in A$, then since $d\brac{x,y}\leq \delta\brac{x,y} = 0$ for any pair $y\in A$, it must be that $x=y$, whence $A = \obj{x}$. If $A=\obj{x}$ then for any $x,y\in A$ is such that $d\brac{x,y}=0$, and so $0\leq \delta\brac{A}\leq 0$.

\label{thm:compact_metric_bounded} \noindent \textbf{Theorem} 8-7.
Any compact subset $K$ of a metrizable topological space $\brac{\Omega, \Tcal}$ is bounded with respect to any metric inducing the topology $\Tcal$.

Indeed, let $d$ be any metric inducing the topology $\Tcal$: $\Tcal = \Tcal_\Omega^d$ and $K$ be a compact subset of $\Omega$. If $K$ is empty it must be bounded, since $\delta\brac{K}=-\infty<+\infty$. Otherwise, if $K$ is not empty, for any $\epsilon>0$ the collection $\brac{B\brac{x, \epsilon}}_{x\in K}$ of open balls in $\brac{\Omega, \Tcal_\Omega^d}$ is a covering of $K$, for which, due to compactness there is a finite $F\subseteq K$ with $K\subseteq \bigcup_{x\in F} B\brac{x, \epsilon}$. Since $F$ is finite, there are $\bar{x},\bar{y}\in F$ with $d\brac{x,y}\leq d\brac{\bar{x},\bar{y}} < +\infty$ for any $x,y\in F$. Put $M\defn d\brac{\bar{x},\bar{y}}$. For any $x, y \in K$ there are $x', y'\in F$ such that $x\in B\brac{x', \epsilon}$ and $y\in B\brac{y', \epsilon}$ and \[d\brac{x, y}\leq d\brac{x, x'} + d\brac{x', y'} + d\brac{y', y} < 2 \epsilon + d\brac{\bar{x}, \bar{y}}\] Therefore for $\epsilon=1$ the set $F$ is finite by compactness and $\delta\brac{K}\leq 2 + M < +\infty$, whence $K$ must be bounded.\\

For any increasing homeomorphism $\phi:\Real\to \brac{-1,1}$ let $d'\brac{x,y}\defn \abs{\phi\brac{x}-\phi\brac{y}}$ and in addition put $d\brac{x,y}\defn\abs{x-y}$.
First, note that $d$ is the usual metric on $\Real$ and actually induces the usual topology $\Tcal_\Real$ on $\Real$. Furthermore, the natural topology of $\brac{-1,1}$ is the induced topology $\Tcal_{\brac{-1,1}}\defn\induc{\Tcal_\Real}_{\brac{-1,1}}$. Since $\brac{-1,1}$ is open in $\Real$, the space $\brac{F, \Tcal_F}$ is a topological subspace of $\brac{\Real, \Tcal_\Real}$, where $F\defn\brac{-1,1}$. By theorem 12 $\Tcal_F = \induc{\Tcal_\Real^d}_F = \Tcal_F^{\induc{d}_F}$.

Now, if $U\in \Tcal_\Real$, then $\brac{\phi^{-1}}^{-1}\brac{U}=\phi\brac{U}\in \Tcal_{\brac{-1,1}}$ as $\phi$ has a continuous inverse. If $x\in U$ then $\phi\brac{x}\in \phi\brac{U}$ and so there is $\eta>0$ such that $z\in \phi\brac{U}$ whenever $z\in F$ with $\abs{z-\phi\brac{x}}<\eta$. But for any $y\in\Real$ with $d'\brac{x,y}<\eta$, by definition $\abs{\phi\brac{y}-\phi\brac{x}}<\eta$, whence $\phi\brac{y}\in \phi\brac{U}$. Such $y\in U$, since $\phi$ is injective. Therefore for any $x\in U$ $\exists \eta>0$ with $B^{d'}_\Real\brac{x, \epsilon}\subseteq \phi\brac{U}$, which implies that $U\in\Tcal_\Real^{d'}$.

Conversely, if $U\in\Tcal_\Real^{d'}$ and $x\in U$, then $\exists \epsilon>0$ such that $y\in U$ for every $y\in \Real$ with $\abs{\phi\brac{x}-\phi\brac{y}}<\epsilon$. By the $\epsilon$-$\delta$ definition of continuity, there must be $\eta>0$ such that $\abs{\phi\brac{x}-\phi\brac{y}}<\epsilon$ for every $z\in \Real$ with $\abs{z-x}<\eta$. Hence for any $x\in U$ there is $\eta>0$ such that $B_\Real^d\brac{x,\eta}\subseteq B_\Real^{d'}\brac{x,\epsilon}\subseteq U$. Therefore $U\in \Tcal_\Real$, and metrics $d$ and $d'$ are topologically equivalent.

The set $\Real$ is unbounded with respect to usual metric $d$, since for any $L>0$ one could chose $x\defn L$ and $y\defn -L$, which are in $\Real$ but such that $L < 2 L = d\brac{x,y}$, whence $\delta\brac{\Real}=+\infty$. Quite opposite in this respect is the topologically equivalent metric $d'$, which is bounded from above by 2, since $\abs{\phi\brac{ x }} < 1$  for any $x\in \Real$. Indeed, in this case $\delta\brac{\Real}\leq 2<+\infty$.

\label{thm:compact_clo_bound} \noindent \textbf{Theorem} 36.
A subset of $\Real$ is compact if and only if it is closed and bounded with respect to the usual metric on $\Real$. 

The topological space $\brac{\Real, \Tcal_\Real}$ is usually denoted by $\Real$, because most commonly the topological properties of the set $\Real$ are explored with respect to its usual topology $\Tcal_\Real$.

Let $d$ be any metric that induces the usual topology on $\Real$ and $K$ be a compact subset of $\Real$. By theorem 8-7 $K$ is bounded with respect to $d$ and, since $\brac{\Real, \Tcal_\Real^d}$ is Hausdorff by theorem 8-6, theorem 35 implies that $K$ is closed in $\brac{\Real, \Tcal_\Real}$. Therefore compact subsets in $\Real$ are closed and bounded.

Let $d$ be the usual metric on $\Real$, $d\brac{x,y}\defn \abs{x-y}$, and suppose a set $K$ is non-empty, closed in $\Real$ and bounded with respect to $d$. Pick any fixed $x_0\in \Real$. Since $K$ is bounded with respect to $d$, $0\leq\delta\brac{K}<+\infty$ and so for any $x\in K$ \[\abs{x}\leq \abs{x_0}+\abs{x-x_0}\leq \abs{x_0}+\delta\brac{K}\] Thus there is $M\in \Real^+$ such that $\abs{y}\leq M$ and so $K\subseteq \clo{-M,M}$. Furthermore as $K$ is closed, $K^c$ is open in $\Real$, so $K^c\cap \clo{-M,M}$ is open in $\brac{\clo{-M, M}, \Tcal_{\clo{-M, M}}}$. However $\clo{-M,M}\setminus \brac{K^c\cap \clo{-M,M}} = K$ because $K\subseteq \clo{-M,M}$, whence $K$ is closed in $\clo{-M, M}$. By theorem 34 $\clo{-M,M}$ is compact and therefore $K$ is a compact subset by theorem 8-5, implying that $K$ is a compact subset of $\Real$. If $K$ is empty, then it is obviously compact.\\

\label{thm:compact_image} \noindent \textbf{Theorem} 8-8.
The image of any continuous map with a compact domain is compact itself.

Indeed, let $\brac{\Omega, \Tcal}$ and $\brac{S, \Tcal_S}$ be topological spaces and $f\brac{\Omega, \Tcal}\to\brac{S, \Tcal_S}$ be a continuous map. If $\brac{W_i}_{i\in I}$ is a collection of open sets in $\brac{S, \Tcal_S}$ covering $f\brac{\Omega}$ then for any $x\in \Omega$ there is $j\in I$ such that $f\brac{x}\in W_j$ implying $x\in f^{-1}\brac{W_j}$. Hence due to $f$'s continuity $\brac{f^{-1}\brac{W_i}}_{i\in I}\in \Tcal$ is an open cover of $\Omega$. Given that $\brac{\Omega, \Tcal}$ is compact, there exists a finite $F\subseteq I$ such that $\Omega\subseteq\bigcup_{i\in F} f^{-1}\brac{W_i}$. In turn, the preservation of set operations the inverse image implies that $\Omega\subseteq f^{-1}\brac{\bigcup_{i\in F} W_i}$ and so $f\brac{\Omega}\subseteq \bigcup_{i\in F} W_i$. Therefore $f\brac{\Omega}$ is a compact subset of $\brac{S, \Tcal_S}$.\\

\label{thm:rbar_lpc} \noindent \textbf{Theorem} 8-9.
Every closed subset of $\Rbar$ contains both its lower and its upper bounds.

The usual topological space on $\Rbar$ is defined as the space homeomorphic to $\clo{-1, 1}$. Since any homeomorphism can be used to construct the space $\brac{\Rbar, \Tcal_\Rbar}$, let $\phi$ be an increasing homeomorphism. The topology on $\clo{-1,1}$ is induced by the usual topology on $\Real$ which in also induced by the usual metric on $\Real$, whence by theorem 12 that $\clo{-1,1}$ is a metric-topological space with respect to the restricted usual metric on $\Real$.

By theorem 34 $\clo{-1,1}$ is a compact subset of $\Real$ and, since $\phi^{-1}:\clo{-1,1}\to\Rbar$ exists and is continuous, theorem 8-8 implies that $\phi^{-1}\brac{\clo{-1,1}}$ is compact in $\Rbar$, whence $\brac{\Rbar, \Tcal_\Rbar}$ is a compact topological space due to $\phi$ being a bijection. Therefore any compact $K\subseteq \Real$ is also compact in $\Rbar$. Furthermore any closed subset of $\Rbar$ is compact by theorem 8-5. However, as $\Rbar$ is metrizable by theorem 13, it is Hausdorff by theorem 8-6, whence any compact subset of $\Rbar$ must be closed in $\Rbar$ according to theorem 35. A subset of $\Rbar$ is compact if and only if it is closed in $\Rbar$.

Let $A$ be a non-empty subset of $\Rbar$ and $\alpha\defn \sup A$. Suppose $\alpha\neq-\infty$ and pick any $U\in \Tcal_\Rbar$ with $\alpha\in U$. Since $\phi$ has continuous inverse, $\phi\brac{U}$ is open in $\clo{-1,1}$. Put $a\defn \phi\brac{\alpha} \in \phi\brac{U}$ and note that $-\infty<\alpha\leq+\infty$ implies that $a\neq-1$, because $\phi$ is increasing. Since $\Tcal_{\clo{-1,1}}$ is a metric topology there exists $\eta>0$ small enough with $-1<a-\eta$ and $\ploc{b,a} \subseteq \phi\brac{U}$, with $b\defn a-\eta$. Bijectivity of $\phi$ guarantees existence of $\beta\in \Rbar$ with $b=\phi\brac{\beta}$, and monotonicity gives $-\infty<\beta<\alpha$. Furthermore for any $x\in \ploc{\beta,\alpha}$ monotonicity implies $b<\phi\brac{x}\leq a$ and so $\phi\brac{x}\in\phi\brac{U}$. By bijectivity of $\phi$ $x\in U$, and therefore for any $U\in \Tcal_\Rbar$ with $\alpha\in U$ there is $\brac\in \Real$ with $\ploc{\beta, \alpha}\subseteq U$.

By definition of $\sup$ any $x\in A$ satisfies $x\leq \alpha$. Suppose there is no $x\in A$ with $\beta < x$. Then $x\leq \beta$ for all $x\in A$, whence by definition $\alpha\leq \beta$, which contradicts the choice of $\beta$. Therefore $A\cap\ploc{\beta,\alpha}\neq \emptyset$ and therefore $\alpha\in \clo{A}$. In the case when $\alpha=-\infty$ with $A\neq \emptyset$, it must be true that any $x\in A$ is such that $x\leq -\infty$ which being a number from $\Rbar$. Therefore $A\equiv \obj{-\infty}$ and $\alpha\in A\subseteq\clo{A}$.

If $A$ is non-empty and closed in $\Rbar$, then $A=\clo{A}$ and by the above $\sup A\in A$. Arguments, similar to the ones above, allow one to deduce that $\inf A\in A$ as well. Indeed, the signs, inequalities and ordering relations should be switched to the opposite.

Suppose $A$ is non-empty closed and bounded in $\Real$ with respect to the usual metric. Then by theorem 36 $A$ is non-empty compact in $\Real$ and thus non-empty and compact in $\Rbar$ by a prior argument. Therefore $A$ is closed in $\Rbar$, whence $\sup A, \inf A\in A$.\\

\label{thm:cont_extrema} \noindent \textbf{Theorem} 37.
Let $f:\brac{\Omega, \Tcal}\to \brac{\Rbar,\Tcal_\Rbar}$ be a continuous map where $\brac{\Omega, \Tcal}$ is a non-empty topological space. If $\brac{\Omega, \Tcal}$ is compact, then $f$ attains its maximum and minimum, i.e. there exist $x_m, x_M\in \Omega$ such that \[f\brac{x_m} = \inf_{x\in \Omega} f\brac{x}, f\brac{x_M} = \sup_{x\in \Omega} f\brac{x}\]

Indeed, by theorem 8-8 $A\defn f\brac{\Omega}$ is non-empty and compact in $\Rbar$ and by theorem 8-9 $\sup A, \inf A\in A$. As $\sup A = \sup \obj{\induc{f\brac{x}} x\in \Omega}$, whence $\sup A = \sup_{x\in \Omega} f\brac{x}$. Therefore there exist $x_m, x_M\in \Omega$ such that $f\brac{x_m} = \inf_{x\in \Omega} f\brac{x}$ and $f\brac{x_M} = \sup_{x\in \Omega} f\brac{x}$.

Recall that $\Real$ is open in $\Rbar$ and $\Tcal_\Real = \induc{\Tcal_\Rbar}_\Real$. Also if $f\brac{\Omega}\in \Real$ then $f^{-1}\brac{U} = f^{-1}\brac{U\cap \Real}$ for any $U\subseteq \Rbar$. Therefore when $f:\brac{\Omega, \Tcal}\to\brac{\Real, \Tcal_\Real}$ is continuous, $f^{-1}\brac{U} = f^{-1}\brac{U\cap \Real} \in \Tcal$ for any $U\in \Tcal_\Rbar$, since $U\cap \Real\in \Tcal_\Real$. If $f:\brac{\Omega, \Tcal}\to\brac{\Rbar, \Tcal_\Rbar}$ is continuous, then for any $U\in \Tcal_\Real$ there exists $V\in \Tcal_\Rbar$ such that $U = V\cap \Real$, whence $f^{-1}\brac{U} = f^{-1}\brac{V} \in \Tcal$. Consequently, the theorem applies equally well to finite continuous $\Real$-valued maps.\\

%% Maybe something about lower-/upper-semi continuous maps?

\label{thm:rolle} \noindent \textbf{Theorem} 38 (Rolle).
Let $a<b\in \Real$ and $f:\clo{a,b}\to\Real$ be continuous and differentiable on $\brac{a,b}$ with $f\brac{a} = f\brac{b}$. Then there exists $c\in \brac{a,b}$ with $f'\brac{c}=0$.

Recall that differentiability of $f$ at some $x\in \brac{a,b}$ means that \[f'\brac{x-}\defn\lim_{\Delta\to 0-}\frac{f\brac{x+\Delta} - f\brac{x} }{\Delta}\,\text{and}\,f'\brac{x+}\defn \lim_{\Delta\to 0+}\frac{f\brac{x+\Delta} - f\brac{x}}{\Delta}\] are such that $f'\brac{x+} = f'\brac{x-}$. Then $f'\brac{x}\defn f'\brac{x+}=f'\brac{x-}$. In other words $f'\brac{x}\in \Real$ is such that fro any $\epsilon>0$ there exists $\delta>0$ such that for every $0<\abs{y-x}<\delta$ \[\abs{ \frac{f\brac{x}-f\brac{y}}{x-y} - f'\brac{x} }<\epsilon\]

Indeed, by virtue of theorem 37 the continuity of $f$ implies that there are $\bar{c}, \underline{c} \in \clo{a,b}$ with $f\brac{\bar{c}} = \sup_{x\in \clo{a,b}} f\brac{x}$ and $f\brac{\underline{c}} = \inf_{x\in \clo{a,b}} f\brac{x}$. If $f\brac{\bar{c}}=f\brac{\underline{c}}$ then $f\brac{x}\leq f\brac{\bar{c}} = f\brac{\underline{c}}\leq f\brac{x}$ and $f\brac{x} = f\brac{a}$ for any $x\in \clo{a,b}$, whence $f$ is constant and so $f' = 0$ everywhere on $\clo{a,b}$. On the other hand if $f\brac{\underline{c}} < f\brac{\bar{c}}$ then it cannot be that both $\bar{c}$ and $\underline{c}$ are in $\clo{a,b}$ but not in $\brac{a,b}$, since $f\brac{a} = f\brac{b}$. Therefore suppose that $\bar{c} \in \clo{a,b}$ with $f\brac{\bar{c}} = \sup_{x\in \clo{a,b}} f\brac{x}$. Hence If $\bar{c}\in \brac{a,b}$ then for any $x\in \brac{a,\bar{c}}$ it must be $\frac{f\brac{\bar{c}} - f\brac{x} }{\bar{c}-x} \geq 0$ and $\frac{f\brac{\bar{c}} - f\brac{x} }{\bar{c}-x} \leq 0$ for every $x\in \brac{\bar{c},b}$. Therefore $0\leq f'\brac{x-} = f'\brac{x+}\leq 0$, whence $f'\brac{\bar{c}}=0$. If instead $\underline{c}\in \brac{a,b}$ then the result is similar. In any case under these conditions there exists $c\in \brac{a,b}$ such that $f'\brac{c}=0$.\\


\label{thm:taylor_lagrange} \noindent \textbf{Theorem} 39 (Taylor-Lagrange).
Let $a<b\in \Real$ and $n\geq 0$. If $f$ is a map of class $C^n\brac{\clo{a,b}\to\Real}$ such that $f^{\brac{n+1}}$ exists on $\brac{a,b}$, then there exists $c\in \brac{a,b}$ with \[f\brac{b}-f\brac{a} = \sum_{k=1}^n \frac{\brac{b-a}^k}{k!} f^{\brac{k}}\brac{a} + \frac{\brac{b-a}^{n+1}}{\brac{n+1}!} f^{\brac{n+1}}\brac{c}\]

First, the base $n=0$ case. Let $a<b\in \Real$ and $f:\clo{a,b}\to\Real$ be continuous and differentiable on $\brac{a,b}$. Since $a<b$ the map \[h\brac{x}\defn f\brac{x} - \brac{x-a}\frac{f\brac{b}-f\brac{a}}{b-a}\] is well defined and continuous and differentiable, as it is a sum of continuous and differentiable maps. In addition, $h$ is such that $h\brac{a}=h\brac{b} = f\brac{a}$. Therefore by theorem 38 (Rolle) there exists $c\in \brac{a,b}$ such that $h'\brac{c}=0$. However the derivative of $h$ on $\brac{a,b}$ is \[h'\brac{x} = f'\brac{x} - \frac{f\brac{b}-f\brac{a}}{b-a} \] whence $f\brac{b}-f\brac{a} = f'\brac{c} \brac{b-a}$, which coincides with the needed polynomial with residual for $n=0$.

If $n\geq 1$ and $f$ is a map of class $C^n\brac{\clo{a,b}\to\Real}$ such that $f^{\brac{n+1}}$ exists on $\brac{a,b}$, then the map \[h\brac{x}\defn f\brac{b} - f\brac{x} - \sum_{k=1}^n \frac{\brac{b-x}^k}{k!} f^{\brac{k}}\brac{x} - \alpha \frac{\brac{b-x}^{n+1}}{\brac{n+1}!} \] for any $\alpha\in \Real$ is a sum of continuous on $\clo{a,b}$ and differentiable on $\brac{a,b}$ functions, since $f^{\brac{k}}$ is continuous on $\clo{a,b}$ for all $k=1\ldots n$ by the definition of the class $C^n\brac{\clo{a,b}\to\Real}$. The derivative of $h$ is \[h'\brac{x} = -f'\brac{x} + \sum_{k=1}^n \brac{\frac{\brac{b-x}^{k-1}}{\brac{k-1}!} f^{\brac{k}}\brac{x} - \frac{\brac{b-x}^k}{k!} f^{\brac{k+1}}\brac{x}} - \alpha \frac{-\brac{b-x}^n}{n!} \] whence after some cancellation the derivative simplifies to \[h'\brac{x} = \brac{ \alpha - f^{\brac{n+1}}\brac{x} } \frac{\brac{b-x}^n}{n!} \]

Since $\alpha$ is a free parameter, suppose $\alpha$ is such that $h\brac{a}=0$. Thus using the result for $n=0$ there is $c\in \brac{a,b}$ with $h'\brac{c}\brac{b-a}=h\brac{b}-h\brac{a}$, whence \[h\brac{a} = \brac{ f^{\brac{n+1}}\brac{c} - \alpha } \frac{\brac{b-c}^n}{n!}\brac{b-a}\] and so $c$ is such that $f^{\brac{n+1}}\brac{c} = \alpha$. Finally, \[ f\brac{b} - f\brac{a} = \sum_{k=1}^n \frac{\brac{b-a}^k}{k!} f^{\brac{k}}\brac{a} + \frac{\brac{b-a}^{n+1}}{\brac{n+1}!} f^{\brac{n+1}}\brac{c} \] so for this $c\in \brac{a,b}$ the difference $f\brac{b}-f\brac{a}$ has the form suggested in the statement of the theorem.\\

\label{thm:convex_deriv} \noindent \textbf{Theorem} 8-10.
Let $a<b\in \Rbar$ and $\phi:\brac{a,b}\to\Real$ be differentiable. Then $\phi$ is convex if and only if $\phi'$ is non-decreasing.

Indeed, for any $x<y \in \brac{a,b}$ there are $u<v \in \brac{a,b}$ with $x < u < v < y$. Then by theorem 8-3, convexity of $\phi$ implies that for any $\abs{\Delta}<\min\obj{u-x,x-a}$ \[\frac{\phi\brac{x}-\phi\brac{x+\Delta}}{\Delta} \leq \frac{\phi\brac{v}-\phi\brac{u}}{v-u}\] Since $\phi$ is differentiable in $\brac{a,b}$, the values of $\phi'\brac{x}$ is the limit of the left-hand side as $\Delta\to 0$, and therefore is not greater than the right-hand side. Analogously for any $\abs{\Delta}<\min\obj{b-y,y-v}$ \[\frac{\phi\brac{v}-\phi\brac{u}}{v-u}\leq \frac{\phi\brac{y}-\phi\brac{y+\Delta}}{\Delta}\] whence $\frac{\phi\brac{v}-\phi\brac{u}}{v-u} \leq \phi'\brac{y}$. Therefore $\phi'\brac{x}\leq \phi'\brac{y}$ for any $x<y\in \brac{a,b}$.

Conversely, suppose $\phi'$ is non-decreasing and pick any $x<y<z \in \brac{a,b}$. The restriction $h\defn \induc{\phi}_{\clo{x,y}}$ is continuous and differentiable (since $\phi$ is differentiable on $\brac{a,b}$), whence by theorem 39 for $n=0$ there is $c\in \brac{x,y}$ with $h'\brac{c}\brac{y-x} = h\brac{y}-h\brac{x}$. Therefore there exists $c_1\in \brac{x,y}$ with $\phi\brac{y} - \phi\brac{x} = \phi'\brac{c_1}\brac{y-x}$. Similarly, for $\induc{\phi}_{\clo{y,z}}$ there is $c_2\in \brac{y,z}$ with $\phi\brac{z}-\phi\brac{y} = \phi'\brac{c_2}\brac{z-y}$ and $c_1<c_2$. Therefore since $\phi'\brac{c_1}\leq \phi'\brac{c_2}$ \[\frac{\phi\brac{y}-\phi\brac{x}}{y-x}\leq \frac{\phi\brac{z}-\phi\brac{y}}{z-y}\] whence follows the convexity of $\phi$ by virtue of theorem 8-3.\\

For instance, theorem b-10 implies that the map $x\to e^x$ is convex in $\Real$ because $\brac{e^x}' = e^x$ and $e^x\leq e^y$ for every $x\leq y\in \Real$. Also, by this theorem the map $x\to -\ln\brac{x}$ is convex in $\brac{0,+\infty}$ since $\brac{-\ln\brac{x}}' = -\frac{1}{x}$ and $\frac{1}{y}\leq\frac{1}{x}$ whenever $0<x\leq y$.

\noindent \textbf{Definition} 70.
A finite measure space $\brac{\Omega, \Fcal, \Pr}$ is a probability space if $\Pr{\Omega} = 1$.

\noindent \textbf{Definition} 71.
Let $\brac{\Omega, \Fcal, \Pr}$ be a probability space and $\brac{S,\Sigma}$ a measure space. A random variable with respect to $\brac{S,\Sigma}$ is any measurable map $X:\brac{\Omega, \Fcal}\to \brac{S, \Sigma}$.

\noindent \textbf{Definition} 72.
Let $\brac{\Omega, \Fcal, \Pr}$ be a probability space and let $X\in L^1_\Cplx\brac{\Omega, \Fcal, \Pr}$ or a non-negative random variable. The expectation of $X$, denoted $E\brac{X}$ is the complex Lebesgue integral \[E\brac{X}\defn \int_\Omega X d\Pr\]

\label{thm:jensen} \noindent \textbf{Theorem} 40.
Let $\brac{\Omega, \Fcal, \Pr}$ be a probability space, $a<b\in \Rbar$ and $\phi:\brac{a,b}\to \Real$ be a convex map. If $X\in L^1_\Real\brac{\Omega, \Fcal, \Pr}$ is such that $X\brac{\Omega}\subseteq \brac{a,b}$ and $\phi\circ X\in L^1_\Real\brac{\Omega, \Fcal, \Pr}$, then \[\phi\brac{\int_\Omega X d\Pr} \leq \int_\Omega \phi\circ X d\Pr\]

Let $a<b\in \Rbar$ and suppose $X\in L^1_\Real$ is such that $X\brac{\Omega}\subseteq \brac{a,b}$. Then since $\Tcal_{\brac{a,b}} = \induc{\Tcal_\Real}_{\brac{a,b}}$ (usual topology) and $\borel{\brac{a,b}} = \induc{\borel{\Real}}_{\brac{a,b}}$ (theorem 10), $X$ is measurable with respect to $\Fcal$ and $\borel{\brac{a,b}}$. If $\phi:\brac{a,b}\to\Real$ is convex then theorem 8-4 implies that it is continuous, whence $\phi:\brac{\brac{a,b},\borel{\brac{a,b}}} \to \brac{\Real, \borel{\Real}}$ is measurable. Thus $\phi\circ X$ is $\Fcal$-$\borel{\Real}$ is measurable. If $\int_\Omega \abs{\phi\circ X} d\Pr < +\infty$ then $\phi\circ L^1_\Real\brac{\Omega, \Fcal, \Pr}$, and obviously the converse holds as well.

Since $\Pr$ is a finite measure on $\brac{\Omega, \Fcal}$ for any $a\in \Real$ the constant function is in $L^1_\Real$, whence $X-a$ and $b-X$ are in $L^1_\Real$ and so their signed Lebesgue integrals are defined. If $m\defn \int_\Omega X d\Pr$ then $m\in \clo{a,b}$ since $a\leq X\leq b$ on $\Omega$. If $m=a$ then $\int_\Omega X d\Pr = a$ and so $\int_\Omega \brac{X - a} d\Pr = 0$. Therefore $\Pr\brac{\obj{X>a}} = 0$, because $X-a$ is non-negative  and measurable on $\brac{\Omega, \Fcal}$, whence $X=a$ $\Pr$-a.s. Thus $1 = \Pr\brac{\Omega} = \Pr\brac{\obj{X\leq a}} = \Pr\brac{\emptyset} = 0$, since $a<X$ on $\Omega$. Similarly, if $m=b$, then measurablility and non-negativity of $b-X$ implies that $\Pr\brac{\obj{X<b}} = 0$, whence $\Pr\brac{\Omega} = \Pr\brac{\obj{X>b}} = 0$.

Next, since $m\in \brac{a,b}$, it is possible to define \[\beta\defn \sup_{x\in \brac{a,m}}\frac{\phi\brac{m}-\phi\brac{x}}{m-x}\] Note that $\brac{a,m} \neq \emptyset$ and for any $x<y\in \brac{a,m}$ convexity implies \[\frac{\phi\brac{m}-\phi\brac{x}}{m-x}\leq \frac{\phi\brac{m}-\phi\brac{y}}{m-y}\] whence $\phi'\brac{m-}=\beta$. Further for any $y\in \brac{m, b}$ convexity implies that for all $x\in \brac{a,m}$ \[\frac{\phi\brac{m}-\phi\brac{x}}{m-x}\leq \frac{\phi\brac{y}-\phi\brac{m}}{y-m}<+\infty\] whence $\beta\in \Real$, because $\phi$ is $\Real$-valued. By definition of the least upper bound, for all $z\in \brac{m,b}$ \[\beta \leq \frac{\phi\brac{z}-\phi\brac{m}}{z-m} \]

Now if $x\in \brac{a,m}$ then $\phi\brac{m}-\phi\brac{x}\leq \beta\brac{m-x}$ and therefore $\phi\brac{m} + \beta\brac{x-m}\leq\phi\brac{x}$. If $x\in \brac{m,b}$ then $\phi\brac{m}+\beta\brac{x-m}\leq \phi\brac{x}$ by the fact upper bound of $\beta$. Finally, for $x=m$ $\phi\brac{m}+0\leq \phi\brac{x}$.

In particular for all $\omega\in \Omega$ it is true that $\phi\brac{m} + \beta\brac{ x\brac{\omega} - m}\leq \phi\brac{X\brac{\omega}}$, whence in the case when $\phi\circ X\in L^1_\Real$ certain property of the signed Lebesgue integral implies \[\int_\Omega \phi\brac{m} + \beta \brac{X-m} d\Pr \leq \int_\Omega \phi\brac{X} d\Pr\] however $m=\int_\Omega X d\Pr$ and so by linearity of the Lebesgue	integral \[ \phi\brac{\int_\Omega X d\Pr}\leq \int_\Omega \phi\brac{X} d\Pr\]\\

\label{thm:cont_ivt} \noindent \textbf{Theorem} 8-11 (The intermediate value theorem).
Let $a<b\in \Real$ and $f:\clo{a,b}\to \Rbar$ be continuous function and $f\brac{a}<f\brac{b}$. Then for any $u\in \brac{f\brac{a}, f\brac{b}}$ there exists $c\in \brac{a,b}$ such that $f\brac{c} = u$.

Note that $u\in \brac{f\brac{a}, f\brac{b}}$ implies that $u\in \Real$. So, let $S\defn \obj{ \induc{ x\in \clo{a,b} } f\brac{x}\leq u}$. By definition, $S\subseteq \clo{a,b}$, and $a\in S$ because $f\brac{a}\leq u$. Therefore, $c\defn \sup S\in \clo{a,b}$, whence $f$ is defined at $c$ and it is continuous at $c$, because $f$ is continuous on $\clo{a,b}$.

If $f\brac{c} = +\infty$, then for every $\epsilon>0$ there is $\delta>0$ such that $f\brac{x}>\epsilon$ for all $x\in \clo{a,b}$ with $\abs{x-c}<\delta$. If $c\neq a$, then $\brac{c-\delta,c}\cap \clo{a,b}\neq \emptyset$ and so there is $x<c$ such that $f\brac{x}>\epsilon$. As $x<c$, $x$ is not an upper bound on $S$, and so there is $y\in S$ with $x<y\leq c$. Thus $\epsilon<f\brac{y}\leq u$ for all $\epsilon>0$, whence $u=+\infty$, which contradicts $u\in \Real$. Therefore $c=a$ and so $f\brac{b} \leq +\infty = f\brac{a}$ which contradicts $f\brac{a}<f\brac{b}$.

If $f\brac{c}=-\infty$, then for every $\epsilon>0$ there is $\delta>0$ such that $f\brac{x}<-\epsilon$ for all $x\in \clo{a,b}$ with $\abs{x-c}<\delta$. If $c<b$ then there are $x\in \clo{a,b}$ with $c<x<c+\delta$. For such $x$, first, $u<f\brac{x}$ and, second, $f\brac{x}<-\epsilon$, from where it follows that $u<-\epsilon$ for all $\epsilon>0$ contradicting $u\in \Real$. Thus $c=b$ and so $f\brac{b}=-\infty\leq f\brac{a}$ which again contradicts $f\brac{a}<f\brac{b}$.

So $f\brac{c}\in \Real$, whence for any $\epsilon>0$ there is $\delta>0$ such that $\abs{f\brac{x}-f\brac{c}}<\epsilon$ for all $x\in \clo{a,b}$ with $\abs{x-c}<\delta$.

For any $x\in \ploc{c,b}$ with $x<c+\delta$, if any exist, it must be $f\brac{x}>u$ because otherwise would imply that $x\in S$ and $c$ is not an upper of $S$. Thus $f\brac{c}>f\brac{x}-\epsilon>u-\epsilon$.

On the other hand for all $x\in \clop{a,c}$ with $c-\delta<x$, again if any exist, the fact that $x<c$ implies that $x$ is not an upper bound for $S$, whence there exists $y\in S$ with $y\in \ploc{x,c}$. Therefore $c-\delta<y$ and so $f\brac{c}<f\brac{y}+\epsilon\leq u+\epsilon$.

Now, if $c=b$, then $\clop{a,c}$ is non-empty, whence $f\brac{c}<u+\epsilon$ for any $\epsilon>0$. In particular, for $\epsilon \defn f\brac{b}-u>0$ it has to be true that $f\brac{c}<f\brac{b}$, which is contradicts $c=b$. If $c=a$, then $\ploc{c,b}$ is not empty, and, again, $u-\epsilon<f\brac{c}$ for any $\epsilon>0$. So for $\epsilon \defn u-f\brac{a}>0$ the assumption implies that $f\brac{a}<f\brac{c}$, which contradicts $c=a$.

Thus continuity implies that $c\in\brac{a,b}$, whence $\abs{f\brac{c}-u}<\epsilon$ for any $\epsilon>0$. Thus $c$ is such, that $f\brac{c}=u$. Note that finiteness of $a$,$b$ or $u$ affects the definition of neighbourhoods and, slightly, the logic of the proof. So this result is extensible to the case of a continuous $f:\clo{a,b}\to \Rbar$ with $a<b\in \Rbar$.\\

% 1) mon: con => sur
Let $f:\clo{a,b}\to\Real$ be a continuous strictly monotonous map. Then $f\brac{a}<f\brac{b}$ and by theorem 8-11 for any $u \in \brac{f\brac{a}, f\brac{b}}$ there is $c\in \brac{a,b}$ with $f\brac{c}=u$. Hence $f\brac{\clo{a,b}}=\clo{f\brac{a},f\brac{b}}$.

% 3) con: inj => mon
% 4)    : mon => inj
Let $a<b\in \Rbar$ and $f:\clo{a,b}\to\Rbar$ be continuous. If $f$ is not monotonous, then there are $x<y<z\in \clo{a,b}$ with $f\brac{x}, f\brac{z} < f\brac{y}$. The restrictions $\induc{f}_{\clo{x,y}}$ and $\induc{f}_{\clo{y,z}}$ are continuous with respect to the induced topologies on each restricted domain. Since $f\brac{z}<f\brac{y}$ and $f\brac{x}<f\brac{y}$, for any $w\in \brac{ \max\obj{ f\brac{z}, f\brac{x} }, f\brac{y} }$ theorem 8-11 (the intermediate value theorem) applied to each restricted map implies that there are $u\in \brac{x,y}$ and $v\in \brac{y,z}$ with $f\brac{u}=f\brac{v}=w$, whence $f$ cannot be an injection. Therefore continuity ensures that injectivity implies monotonicity, whereas monotonicity alone is sufficient for injectivity.


2) mon: sur => con
5) hom:

Suppose $f:\Rbar\to\clo{a,b}$ is a homeomorphism, for which 


 Therefore, a homeomorphism is necessarily monotonous.

Since strictly monotonous functions are injective, $f$ must be bijective.


Conversely, suppose $f:\clo{a,b}\to\clo{\alpha, \beta}$ is strictly increasing and surjective. Let $W$ be open in $\clo{\alpha, \beta}$ and consider any $x\in f^{-1}\brac{W}$. Since $f\brac{x}\in W$ and the topology on $\clo{\alpha, \beta}$ is induced by the natural topology on $\Real$, there exists $\eta>0$ with $\brac{f\brac{x}-\eta,f\brac{x}+\eta}\cap \clo{\alpha,\beta} \subseteq W$.

If $x=b$ then monotonicity implies that 



Is the inverse of a continuous bijection continuous?\\

Let $f:\clo{a,b}\to\Real$ be a continuous monotonously increasing map. Let $\alpha\defn f\brac{a}$ and $\beta\defn f\brac{b}$ and pick any $\gamma \in \brac{\alpha, \beta}$. For $A\defn \obj{\induc{ x\in \clo{a,b} } f\brac{x}\leq \gamma }$, by definition, $x\leq b$ for any $x\in A$. Since $f$ is increasing and $f\brac{a}<\gamma$, it must be $a\in A$, whence $A$ is non-empty. Let $c\defn \sup A$. For any $x\in \clo{a,b}$ with $x<c$ the definition of the least upper bound implies that there is $x'> x$ such that $f\brac{x'}\leq \gamma$, whence $f\brac{x}\leq f\brac{x'}\leq \gamma$ and so $x\in A$. If $c=b$, then $f\brac{x}\leq \gamma$ for any $x<b$, whence by continuity of $f$ at $b$ it follows that $\beta = f\brac{b}\leq \gamma$. On the other hand $c=a$ implies that $\gamma<f\brac{x}$ for every $x>a$, whence $f$'s continuity implies that $\gamma\leq f\brac{a}=\alpha$. Indeed, for any $x\in \clo{a,b}$ with $c<x$, $f\brac{x}\leq \gamma$ would violate $c$'s being an upper bound of $A$. Therefore $c\in \brac{a,b}$ and is such that $f\brac{x}\leq \gamma$ for any $a<x<c$, and $\gamma\leq f\brac{y}$ for all $c<y<b$, whence continuity of $f$ at $c$ would imply that $f\brac{c}=\gamma$.

Conversely, suppose $f\clo{a,b}\to\clo{\alpha, \beta}$ is increasing and surjective. Let $U$ be open in $\clo{\alpha, \beta}$ and consider any $c\in f^{-1}\brac{U}$. Then $\gamma \defn f\brac{c}\in U$ and so there is $\epsilon>0$ with $\brac{\gamma-\epsilon, \gamma+\epsilon}\cap \clo{\alpha, \beta} \subseteq U$. Surjectivity implies that $\exists u,v \in \clo{a,b}$ such that $f\brac{u} = \min\obj{\alpha, \gamma-\epsilon}$ and $f\brac{v} = \min\obj{\beta, \gamma+\epsilon}$. Since $f$ is non-decreasing and $f\brac{u}< \gamma < f\brac{v}$ it must be $u<c<v$, and further $f\brac{u}\leq f\brac{x}\leq f\brac{v}$ for $x\in \brac{u,v}$, whence $f\brac{\brac{u,v}}\in U$. Therefore, since for every $c\in f^{-1}\brac{U}$ there is $V$ open in $\clo{a,b}$ with $c\in V$ such that $V\subseteq f^{-1}\brac{U}$, $f$ must be continuous.

\label{thm:cont_ivt2} \noindent \textbf{Theorem} 8-11 (The intermediate value theorem).
Let $f:\brac{a,b}\to \Real$ be continuous function and $f\brac{a}<f\brac{b}$. Then for any $u\in \brac{f\brac{a}, f\brac{b}}$ there exists $c\in \brac{a,b}$ such that $f\brac{c}=u$.

Indeed, consider $A\defn \obj{\induc{x\in \clo{a,b}} f\brac{x}\leq u}$. The set $A$ is non-empty, because $f\brac{a}<u$ and $a\in A$. And $A$ is bounded above by $b$. Then $c\defn \sup A$ exists and $c\in \clo{a,b}$. Since $f$ is continuous, for every $\epsilon>0$ there is $\delta>0$ such that $\abs{f\brac{x}-f\brac{c}}<\epsilon$ for all $\abs{x-c}<\delta$. By the properties of the supremum for $\delta>0$ there is $x\in A$ with $c-\delta<x\leq c$, whence $f\brac{c}<f\brac{x}+\epsilon\leq u+\epsilon$. For all $c<x<c+\delta$ it has to be true that $f\brac{c}>f\brac{x}-\epsilon > u-\epsilon$. Therefore $\abs{f\brac{c}-u}<\epsilon$ for all $\epsilon>0$, whence $f\brac{c}=u$.\\


% section tut_8 (end)

\section{$L^p$-spaces, $p\in \clo{1;+\infty}$} % (fold)
\label{sec:tut_9}
\url{http://probability.net/PRTspaces.pdf}

For all $\alpha>0$, define $\phi^\alpha:\Zinf\to\Zinf$ by $\phi^\alpha\brac{x}\defn x^\alpha$ if $x\in \Real^+$ and $\phi^\alpha\brac{x} = +\infty$ if $x=+\infty$.

This map is increasing for every $\alpha>0$, since $\frac{x}{y}<1$ implies $x^\alpha<y^\alpha$ for $x,y\in \Real^+$, and at $x<+\infty$, $\phi^\alpha\brac{x} = x^\alpha<+\infty = \phi^\alpha\brac{+\infty}$.

Pick any $x\in \Zinf$ and let $\epsilon>0$. When $x=0$, letting $\delta\defn\epsilon^{\sfrac{1}{\alpha}}$ gives $\phi^\alpha\brac{y} = y^\alpha < \epsilon$ for all $y\in\Zinf$ with $y<\delta$. If $x=+\infty$, then for every $y\in \Real^+$ with $y>\delta$, $\epsilon< y^\alpha=\phi^\alpha\brac{y}$, whence $\lim_{x\to+\infty} \phi^\alpha\brac{x} = +\infty = \phi^\alpha\brac{+\infty}$.

For $x\in \brac{0,+\infty}$ define $\eta\defn \min\obj{\epsilon, x^\alpha}$, which is positive since $x>0$ and not greater than $\epsilon$, and \[\delta\defn \min\obj{ x-\brac{x^\alpha-\eta}^{\sfrac{1}{\alpha}}, \brac{x^\alpha+\eta}^{\sfrac{1}{\alpha}}-x }\] The fact that $x\to x^\alpha$ is strictly increasing for any $\alpha>0$ implies that $\delta>0$. If $\abs{y-x}<\delta$, then \[\brac{x^\alpha-\eta}^{\sfrac{1}{\alpha}}-x<y-x<\brac{x^\alpha+\eta}^{\sfrac{1}{\alpha}}-x\] whence after cancellation and potentiating $-\eta<y^\alpha-x^\alpha<\eta$. Thus there is $\delta>0$ such that $\abs{x^\alpha-y^\alpha}<\epsilon$ for all $y\in\brac{0,+\infty}$ with $\abs{x-y}<\delta$. Therefore by $\epsilon$-$\delta$ definition on continuity, $\phi^\alpha:\Zinf\to\Zinf$ is continuous.

\label{thm:holder} \noindent \textbf{Theorem} 41 (Holder inequality).
Let $\brac{\Omega, \Fcal, \mu}$ be a measure space and $f,g:\brac{\Omega, \Fcal}\to \Zinf$ be two non-negative measurable maps. Let $p,q\in \Real^+$ be such that $\frac{1}{p}+\frac{1}{q}=1$. Then\[\int f g d\mu \leq \brac{ \int f^p d\mu }^\frac{1}{p} \brac{ \int g^q d\mu }^\frac{1}{q}\]

If $p<1$, then $q<0$, because $\frac{1}{p}+\frac{1}{q}=1$ implies $q=\frac{p}{p-1}$. If $p=1$ then no $q\in \Real^+$ can satisfy the constraint. Therefore $p,q>1$.

For any $\alpha>0$ the map $\phi^\alpha$ is $\borel{\Zinf}$ measurable because it is continuous with respect to $\Tcal_{\Zinf}$. Hence $f^p,g^q:\brac{\Omega, \Fcal}\to\Zinf$ are non-negative and measurable. Therefore $A\defn \brac{\int f^p d\mu}^\frac{1}{p}$ and $B\defn \brac{\int g^q d\mu}^\frac{1}{q}$ are well-defined elements of $\Zinf$. As for $C\defn \int f g d\mu$, it is well-defined since the product of measurable maps is measurable.

If $A=0$, then $\int f^p d\mu = 0$ whence $\mu\brac{\obj{f>0}}=0$ since $\obj{f^p>0}=\obj{f>0}$. Thus $\mu\brac{\obj{f g > 0 }} = 0$ and $C=0$. Similarly, $B=0$ implies $C=0$, whence $C = 0\leq 0 = A B$. If $A=+\infty$ and $B>0$, or $B=+\infty$ and $A>0$, then $C\leq +\infty = A B$ for any $C$. 

Assume $A,B\in \brac{0,+\infty}$. For $F\defn \frac{f}{A}$, the definition of $A$ implies that $\int F^p d\mu = \frac{1}{A^p} \int f^p d\mu = 1$. Similarly for $G\defn \frac{g}{B}$.

The map $x\to-\ln{x}$ is convex on $\brac{0,+\infty}$, since its derivative is $-\frac{1}{x}$, which is a non-decreasing map. Since $a^p, b^q\in \brac{0.+\infty}$ for $a,b\in \brac{0.+\infty}$ and $p,q>1$ are such that $\frac{1}{p}+\frac{1}{q}=1$, convexity implies that \[-\ln\brac{ \frac{1}{p}a^p + \brac{1-\frac{1}{p}}b^q }\leq -\brac{\frac{1}{p}\ln\brac{ a^p } + \frac{1}{q}\ln\brac{ b^q }}\] whence $a b \leq \frac{a^p}{p} + \frac{b^q}{q}$. If either $a=0$ or $b=0$, or if $a=+\infty$, but $b>0$, $b=+\infty$ with $a>0$, the last inequality still holds. Hence for all $\omega\in \Omega$\[F\brac{\omega} G\brac{\omega}\leq \frac{1}{p} F^p\brac{\omega} + \frac{1}{q} G^q\brac{\omega}\] Therefore integrating the inequality over $\brac{\Omega, \Fcal, \mu}$ gives $\int F G d\mu \leq \frac{1}{p}+\frac{1}{q}=1$, whence $C = \int f g d\mu \leq A B$.\\

\label{thm:cauchy_schwartz} \noindent \textbf{Theorem} 42 (Cauchy-Schwartz inequality).
Let $\brac{\Omega, \Fcal, \mu}$ be a measure space and $f,g:\brac{\Omega, \Fcal}\to \Zinf$ be two non-negative measurable maps. Then\[\int f g d\mu \leq \brac{ \int f^2 d\mu }^{\frac{1}{2}} \brac{ \int g^2 d\mu }^{\frac{1}{2}}\]

Indeed, apply theorem 41 to any non-negative measurable $f,g$ with $p=q=\frac{1}{2}$ to get the desired inequality.\\

\label{thm:minkowski} \noindent \textbf{Theorem} 43 (Minkowski inequality).
Let $\brac{\Omega, \Fcal, \mu}$ be a measure space and $f,g:\brac{\Omega, \Fcal}\to \Zinf$ be two non-negative measurable maps. Let $p\geq 1$. Then\[\brac{\int \brac{f+g}^p d\mu}^\frac{1}{p} \leq \brac{ \int f^p d\mu }^\frac{1}{p} + \brac{ \int g^p d\mu }^\frac{1}{p}\]

When $\alpha\geq 1$, $\phi^\alpha$ is convex on $\brac{0,+\infty}$, since its derivative is non-decreasing. Thus for any $a,b\in \brac{0, +\infty}$, \[\brac{\frac{1}{2} a + \frac{1}{2}b}^p \leq \frac{1}{2} a^p + \frac{1}{2}b^p\] whence $\brac{a+b}^p\leq 2^{p-1} \brac{a^p+b^p}$. If $a=0$ or $b=0$, then the inequality holds due to $2^{p-1}\geq 1$. If $a=+\infty$ or $b=+\infty$, then still $\brac{a+b}^p\leq 2^{p-1}\brac{a^p+b^p}$.

So, put $A\defn \brac{\int f^p d\mu}^{\sfrac{1}{p}}$ and $B\defn \brac{\int f^p d\mu}^{\sfrac{1}{p}}$ and $C\defn \brac{\int \brac{f+g}^p d\mu}^{\sfrac{1}{p}}$. These values are well-defined elements of $\Zinf$ due to $\phi^\alpha$ being non-negative, continuous, and measurable for any $\alpha>0$.

For $p=1$ the linearity of the usual Lebesgue integral implies that $C=A+B$. So from now it can be assumed that $p>1$. Since $\brac{f+g}^p \leq 2^{p-1} \brac{ f^p + g^p }$ everywhere on $\Omega$ \[C^p = \int \brac{f+g}^p d\mu \leq 2^{p-1}\brac{ \int f^p d\mu + \int g^p d\mu } = 2^{p-1}\brac{ A^p + B^p } \] Therefore either $A=+\infty$, $B=+\infty$ or $C=0$ imply $C\leq A+B$, because $0\leq x\leq +\infty$ for every $x\in \Zinf$. However if $A,B<+\infty$ then $C<+\infty$.

Suppose $A,B,C\in \Real^+$ with $C>0$. For any $p>1$ $q\defn\frac{p}{p-1}$ is such that $q\in \Real^+$ and $\frac{1}{p}+\frac{1}{q}=1$, with $p = q \brac{p-1}$.

In the case of $a,b\in \Real^+$, $\brac{a+b}^p=\brac{a+b}\brac{a+b}^{p-1}$. However, if $a=+\infty$, then $a+b = +\infty$ regardless of $b$, and thus $\brac{a+b}^p=+\infty=\brac{a+b}\brac{a+b}^{p-1}$, since $p>1$.

By theorem 40 applied to non-negative and measurable $f$ and $\brac{f+g}^{p-1}$ \[\int f \brac{f+g}^{p-1} d\mu \leq \brac{\int f^p d\mu}^\frac{1}{p} \brac{ \int \brac{ f+g }^{\brac{q \brac{p-1}}} d\mu }^\frac{1}{q} \] whence $\int f\brac{f+g}^{p-1} d\mu \leq A C^\frac{p}{q}$. Similarly $\int g\brac{f+g}^{p-1} d\mu \leq B C^\frac{p}{q}$, whence \[ \int \brac{f+g}^p d\mu \leq \brac{A+B} C^\frac{p}{q} \] and so $C^p \leq \brac{A+B} C^\frac{p}{q}$. Therefore $C\leq A+B$, since $C>0$ and $C^\frac{p}{q} = C^{p-1}$.\\

\noindent \textbf{Definition} 9-1.
The $L^p$ norm, $p\in\clo{1, +\infty}$, is defined as follows. When $p\in \clop{1, +\infty}$, for all measurable $f\brac{\Omega, \Fcal}\to \brac{\Cplx, \borel{\Cplx}}$ put \[\nrm{f}_p\defn \brac{ \int_\Omega \abs{f}^p d\mu }^\frac{1}{p}\] When $p=+\infty$, for all measurable $f:\brac{\Omega, \Fcal}\to\brac{\Cplx, \borel{\Cplx}}$ let \[\nrm{f}_\infty\defn \inf \obj{ \induc{ M\in \Real^+ }\,\abs{f} \leq M\,\mu\text{-a.s.} } \]

The $L^p$-space on $\brac{\Omega, \Fcal, \mu}$ is a space of measurable maps over a field $K$, where $K=\Real$ or $K=\Cplx$, defined as \[L^p_K\brac{\Omega, \Fcal, \mu} \defn \obj{ f:\brac{\Omega, \Fcal}\to\brac{K, \borel{K}}\,\text{measurable},\,\nrm{f}_p < +\infty }\]

Let $f,g:\brac{\Omega, \Fcal}\to\brac{\Real, \borel{\Real}}$ be measurable and $\alpha\in \Real$. Since the combination $f+\alpha g$ is well-defined on $\Omega$, tutorial 4 (exercise 20) implies that it is $\Fcal$-$\borel{\Real}$ measurable. Further, let $f,g:\brac{\Omega, \Fcal}\to\brac{\Cplx, \borel{\Cplx}}$ be measurable. Then by tutorial 4 (exercise 25) there are unique $x, y, u, v:\brac{\Omega, \Fcal}\to\brac{\Real, \borel{\Real}}$ such that $f=x + i y$ and $g = u + i v$. For $\alpha = a + i b\in \Cplx$, the linear combination $f+\alpha g = \brac{ x + a u - b v } + i \brac{ y + a v + b u }$ is a well-defined complex number, because $a,b\in \Real$ and $x,y,u,v:\Omega\to \Real$ are finite. Thus, each combination in brackets is well-defined and $\Fcal$--$\borel{\Real}$ measurable, whence by tutorial 4 (exercise 25) the map $f+\alpha g$ is $\Fcal$-$\borel{\Cplx}$ is measurable.

\label{thm:l_infty_norm_muas_prop} \noindent \textbf{Theorem} 9-1.
Let $\brac{\Omega, \Fcal, \mu}$ be a measure space and $f:\brac{\Omega, \Fcal}\to\brac{\Cplx, \borel{\Cplx}}$ be a measurable map. Then $\abs{f}\leq \nrm{f}_\infty$.

Indeed, if $f$ is such that $\nrm{f}_\infty=+\infty$, then, trivially,, $\abs{f}\leq \nrm{f}_\infty$ $\mu$-a.s. since $\abs{f}\leq +\infty$ everywhere on $\Omega$. If $\nrm{f}_\infty < +\infty$, then any $U>\nrm{f}_\infty$ is not a lower bound and there is $\nrm{f}_\infty\leq M < U$ such that $\abs{f}\leq M < U$ $\mu$-a.s. As a matter of fact, $\abs{f} \leq U$ $\mu$-a.s. for all $U>\nrm{f}_\infty$. Thus, if $U_n\defn \nrm{f}_\infty + \frac{1}{2^n}$, then $U_{n+1}<U_n$ for all $n\geq 1$ and $U_n\to \nrm{f}_\infty$. Since $f$ is $\Fcal$-$\borel{\Cplx}$ measurable, $\abs{f}$ is non-negative and measurable, whence $N_n\defn \obj{ \abs{f} > U_n}$ are such that $N_n\in \Fcal$, $N_n\subseteq N_{n+1}$ for all $n\geq 1$ and $N\defn \bigcup_{n\geq 1} N_n = \obj{ \abs{f} > \nrm{f}_\infty }$ due to $U_n\downarrow \nrm{f}_\infty$. Then, according to theorem 7 $N_n\uparrow N$ implies that $\mu\brac{N_n}\uparrow \mu\brac{N}$. However $\mu\brac{N_n}=0$, because $\abs{f}\leq U_n$ $\mu$-a.s. for all $n\geq 1$. Therefore $\mu\brac{N} = 0$, whence $\abs{f} \leq \nrm{f}_\infty$ $\mu$-a.s. Without relying on theorem 7 one could argue from $\sigma$-additivity of $\mu$: $\mu\brac{N}\leq \sum_{n\geq 1} \mu\brac{N_n} = 0$. In this case $N_n$ could be arbitrary $\Fcal$-measurable sets with $\mu\brac{N_n}=0$ outside of which $\abs{f\brac{\omega}}\leq U_n$.\\

\label{thm:l_p_norm_scaling} \noindent \textbf{Theorem} 9-2.
Let $\brac{\Omega, \Fcal, \mu}$ be a measure space and $f:\brac{\Omega, \Fcal}\to\brac{\Cplx, \borel{\Cplx}}$ be a measurable map. Then $\nrm{\alpha f}_p = \abs{\alpha} \nrm{f}_p$ for any $\alpha \in \Cplx$ and all $p\in \clo{1,+\infty}$.

If $p\in \clop{1, +\infty}$, then $\abs{\alpha f} = \abs{\alpha} \abs{f}$ everywhere on $\Omega$, whence by linearity on the usual Lebesgue integral \[\nrm{\alpha f}_p = \brac{\int \abs{\alpha f}^p d\mu}^\frac{1}{p} = \brac{\abs{\alpha}^p \int \abs{f}^p d\mu}^\frac{1}{p} = \abs{\alpha} \nrm{f}_p\]

For $p=+\infty$ consider two cases. When $\abs{\alpha}=0$, then $\abs{\alpha f} = 0$ everywhere on $\Omega$ implying $\nrm{\alpha f}_\infty = 0 = \abs{\alpha} \nrm{f}_\infty$. If $\abs{\alpha}>0$, then $\abs{\alpha} \abs{f} = \abs{\alpha f}\leq \nrm{\alpha f}_\infty$ $\mu$-a.s., whence $\nrm{f}_\infty \leq \frac{\nrm{\alpha f}_\infty}{\abs{\alpha}}$. Conversely, $\abs{f}\leq \nrm{f}_\infty$ $\mu$-a.s. implies that $\abs{\alpha f}=\abs{\alpha} \abs{f}\leq \abs{\alpha} \nrm{f}_\infty$ $\mu$-a.s., whence $\nrm{\alpha f}_\infty \leq \abs{\alpha} \nrm{f}_\infty$.

Therefore, $\nrm{\alpha f}_p = \abs{\alpha} \nrm{f}_p$ for every $\alpha\in \Cplx$ and $p\in \clo{1, +\infty}$.\\

\label{thm:l_p_norm_triangle} \noindent \textbf{Theorem} 9-3.
Let $\brac{\Omega, \Fcal, \mu}$ be a measure space and $f,g:\brac{\Omega, \Fcal}\to\brac{\Cplx, \borel{\Cplx}}$ be measurable maps. Then for any $p\in \clo{1,+\infty}$ the $L^p$-norms satisfy the triangle law: \[ \nrm{f+g}_p\leq \nrm{f}_p + \nrm{g}_p \]

Note, that the complex modulus satisfies the triangle inequality, whence $\abs{f+g} \leq \abs{f} + \abs{g}$ everywhere on $\Omega$.

Suppose $p=+\infty$. Theorem 9-1 applied to $f$ and $g$ implies that $\abs{f} \leq \nrm{f}_\infty$ $\mu$-a.s. and $\abs{g}\leq \nrm{g}_\infty$ $\mu$-a.s. However, since at most countably many $\mu$-a.s. properties hold $\mu$-a.s. simultaneously together, $\abs{f+g} \leq \nrm{f}_\infty + \nrm{g}_\infty$ $\mu$-a.s., whence $\nrm{f+g}_\infty \leq \nrm{f}_\infty + \nrm{g}_\infty$ by definition of a lower bound. Everything is well-defined even in case of infinite $L^\infty$-norm of one of $f$ or $g$ is infinite.

Now, suppose $p\in \clop{1, +\infty}$. Since $\phi^\alpha$ is non-decreasing for $\alpha\geq 0$, theorem 43 (Minkowski inequality) and monotone properties of the usual Lebesgue integrals imply \[\brac{ \int \abs{f+g}^p d\mu }^\frac{1}{p} \leq \brac{ \int \brac{\abs{f}+\abs{g}}^p d\mu }^\frac{1}{p} \leq \brac{ \int \abs{f}^p d\mu }^\frac{1}{p} +\brac{ \int \abs{g}^p d\mu }^\frac{1}{p}\] Therefore for all $p\in \clo{1, +\infty}$ the $L^\infty$ norm satisfies the triangle inequality.\\

\label{thm:l_p_norm_muas_zero} \noindent \textbf{Theorem} 9-4.
Let $\brac{\Omega, \Fcal, \mu}$ be a measure space and $f:\brac{\Omega, \Fcal}\to\brac{\Cplx, \borel{\Cplx}}$ be measurable maps. Then for any $p\in \clo{1,+\infty}$ the $\nrm{f}_p=0$ is equivalent to $f=0$ $\mu$-a.s. At the same time $\nrm{f}_p < +\infty$ implies that $\abs{f}<+\infty$ $\mu$-a.s.

Indeed, let $p\in \clop{1,+\infty}$. Then if $f$ is such that $\nrm{f}_p = 0$, then $\int_\Omega \abs{f}^p d\mu = 0$ and so $\mu\brac{\obj{ \abs{f}^p>0 }} = 0$, since the complex modulus is $\Cplx$-$\Real^+$ continuous. This implies that $f=0$ $\mu$-a.s., as $\obj{ \abs{f}^p > 0 } = \obj{ \abs{f} > 0 }$. Conversely, if $f=0$ $\mu$-a.s, then $\abs{f}^p = 0$ $\mu$-a.s, whence $\int \abs{f}^p d\mu = 0$ and ultimately $\nrm{f}_p=0$.

For $p=+\infty$, the argument goes as follows. If $f$ is such that $\nrm{f}_\infty = 0$, then $\abs{f}\leq 0$ $\mu$-a.s., whence $f=0$ $\mu$-a.s. Conversely, if $f=0$ $\mu$-a.s., then by definition $\nrm{f}_\infty \leq 0$ and so $\nrm{f}_\infty = 0$.

If $\nrm{f}_p < +\infty$, $p\in \clop{1, +\infty}$, then $\int_\Omega \abs{f}^p d\mu < +\infty$, whence $\mu\brac{\obj{ \abs{f}^p = +\infty }} = 0$ and $\abs{f}<+\infty$ $\mu$-a.s. If $\nrm{f}_\infty < +\infty$, then by theorem 9-1 $\abs{f}\leq \nrm{f}_\infty < +\infty$ $\mu$-a.s.\\

For any $p\in \clo{1,+\infty}$, the map $\rho:L^p_\Cplx \times L^p_\Cplx \to \Real^+$ defined as $\rho\brac{f,g} \defn \nrm{f-g}_p$ cannot be a metric on $L^p_\Cplx$ because it fails the ``zero-distance'' property: $\rho\brac{f,g}=0$ does not imply that $f=g$ everywhere on $\Omega$, just that $f=g$ outside of a $\mu$-null measurable set. However, the symmetry and the triangle law are satisfied, meaning that $\rho$ is not a ``full'' metric.

\label{thm:l_p_space_nest} \noindent \textbf{Theorem} 9-5.
Let $\brac{\Omega, \Fcal, \mu}$ be a measure space. For any $p\in \clo{1, +\infty}$ \[L^p_\Real = \obj{ \induc{ f\in L^p_\Cplx } f\brac{\Omega}\subseteq \Real } \]

If $f:\brac{\Omega, \Fcal}\to\brac{\Real,\borel{\Real}}$ is measurable then $f\brac{\Omega}\subseteq \Real\subseteq \Cplx$, whence $f$ is $\Fcal$-$\borel{\Cplx}$ measurable by theorem 4-2. The converse is also implied by theorem 4-2: If $f:\brac{\Omega, \Fcal}\to\brac{\Cplx,\borel{\Cplx}}$ is measurable and $f\brac{\Omega}\subseteq \Real\subseteq \Cplx$, then $f$ is $\Fcal$--$\borel{\Real}$ measurable.

The complex modulus of a real number is nothing but its absolute value, implying that when $p<+\infty$ the values of integrals, and thus norms, are identical, and when $p=+\infty$ the sets, over which the greatest lower bound is taken, are identical. Therefore for all $p\in \clo{1, +\infty}$ \[L^p_\Real\brac{\Omega, \Fcal, \mu} = \obj{ \induc{ f\in L^p_\Cplx\brac{\Omega, \Fcal, \mu} } f\brac{\Omega}\subseteq \Real }\]

\label{thm:l_p_space_linear} \noindent \textbf{Theorem} 9-6.
Let $\brac{\Omega, \Fcal, \mu}$ be a measure space. For any $p\in \clo{1,+\infty}$ the $L^p_K\brac{\Omega, \Fcal, \mu}$-space over the field $K$ is closed under $K$-linear combinations.

For any $f, g\in L^p\Cplx$ and $\alpha\in \Cplx$ the combination $f + \alpha g$ is a well-defined $\Cplx$-valued measurable map. By theorems 9-2 (scaling) and 9-3 (triangle law) \[\nrm{f+\alpha g}_p\leq \nrm{f}_p+\nrm{\alpha g}_p = \nrm{f}_p + \abs{\alpha} \nrm{g}_p < +\infty \] because $\nrm{f}_p,\nrm{g}_p, \abs{\alpha}<+\infty$. Hence for any $p\in \clo{1,+\infty}$, $L^p_\Cplx$ is closed under $\Cplx$-linear combinations.

If $f, g\in L^p\Real$ and $\alpha\in \Real$, then $f, g\in L^p\Cplx$ by theorem 9-5 and $\alpha\in \Cplx$ because $\Real\subseteq \Cplx$. Therefore $f+\alpha g\in L^p_\Cplx$. Since the combination $f + \alpha g$ is a well-defined $\Real$-valued map, by theorem 9-5 $f + \alpha g\in L^p_\Real$, whence $L^p_\Real$ is a $\Real$-linear s[ace.\\

\noindent \textbf{Definition} 75.
Let $p\in \clo{1, +\infty}$ and $K=\Real$ or $\Cplx$. For every $f\in L^p_K\brac{\Omega, \Fcal, \mu}$ and all $\epsilon>0$ the open ball in $L^p_K$ is defined as: \[B^p_K\brac{f, \epsilon}\defn \obj{ g\in L^p_K\,\text{with}\,\nrm{f-g}_p < \epsilon }\] The usual topology in $L^p_K$ is the set $\Tcal$ defined by \[\Tcal\defn \obj{ \induc{ U\subseteq L^p_K } \forall f\in U,\, \exists \epsilon>0,\,\text{s.t.}\, B^p_K\brac{f, \epsilon}\subseteq U } \]

In complete analogy to the metric topology every open ball in $L^p_K$ is an element of $\Tcal_{L^p_K}$. Indeed, for any $f\in L^p_K$ and $\epsilon>0$ pick any $g\in B^p_K\brac{f,\epsilon}$ and set $\eta \defn \epsilon-\nrm{f-g}_p$. Then $\eta>0$ and for any $h\in B^p_K\brac{g,\eta}$, $\nrm{f-h}_p\leq \nrm{f-g}_p+\nrm{g-h}_p<\nrm{f-g}_p+\eta=\epsilon$, whence $h\in B^p_K\brac{f,\epsilon}$.

\label{thm:L_p_K_topo} \noindent \textbf{Theorem} 9-7.
For all $p\in \clo{1, +\infty}$ \[\Tcal_{L^p_\Real} = \induc{\Tcal_{L^p_\Cplx}}_{L^p_\Real}\]

If $U\in \Tcal_{L^p_\Real}$, then for any $f\in U$ there is $\epsilon>0$ with $B^p_\Real\brac{f,\epsilon}\subseteq U$. Since $L^p_\Real\subseteq L^p_\Cplx$, $B^p_\Real\brac{f,\epsilon}=B^p_\Cplx\brac{f,\epsilon}\cap L^p_\Real$. Therefore, for every $f\in U$ exists $V\in \Tcal_{L^p_\Cplx}$ with $f\in V$ and $V\cap L^p_\Real \subseteq U$. Let $\Gamma\defn \obj{\induc{V\in \Tcal_{L^p_\Cplx}} V\cap L^p_\Real \subseteq U}$. Then $\Gamma\subseteq \Tcal_{L^p_\Cplx}$ and by the noted property above for every $f\in U$ there is $V\in \Gamma$ with $f\in V$, whence $U\subseteq \bigcup_{V\in \Gamma} V$. At the same time $U\subseteq L^p_\Real$ and $V\cap L^p_\Real \subseteq U$ for any $V\in \Gamma$, implying that $L^p_\Real \cap \bigcup_{V\in \Gamma} V \subseteq U$. Hence there is $W\in \Tcal_{L^p_\Cplx}$ such that $U = W\cap L^p_\Real$. So $U\in \induc{\Tcal_{L^p_\Cplx}}_{L^p_\Real}$.

If $U\in \induc{\Tcal_{L^p_\Cplx}}_{L^p_\Real}$, then $\exists V\in \Tcal_{L^p_\Cplx}$ such that $U=W\cap L^p_\Real$. For any $f\in U$, $f\in W$ implies that there is $\epsilon>0$ such that $B^p_\Cplx\brac{f, \epsilon}\subseteq W$, whence $B^p_\Cplx\brac{f, \epsilon}\cap L^p_\Real\subseteq U$. But $B^p_\Real\brac{f,\epsilon} = B^p_\Cplx\brac{f,\epsilon}\cap L^p_\Real$. Therefore, for every $f\in U$ there is $\epsilon>0$ such that $B^p_\Real\brac{f,\epsilon}\subseteq U$, thereby implying that $U\in \Tcal_{L^p_\Real}$ and $\Tcal_{L^p_\Real} = \induc{\Tcal_{L^p_\Cplx}}_{L^p_\Real}$ regardless of $p\in \clo{1,+\infty}$.\\

Note that the definition of $\Tcal$ is almost identical to the definition of a metric topology on $L^p_K$, were it not for the fact that $\brac{f,g}\to\nrm{f-g}_p$ is not a metric, though it has almost all properties expected of one. It turns out, that the property it lacks, makes this $L^p_K$ topological space in general not Hausdorff.

Indeed, let $p\in \clo{1,+\infty}$ and suppose there is $N\in \Fcal$ with $N\neq \emptyset$ and $\mu\brac{N}=0$. For $f=1_N$ and $g=0$, it is true, that $f\neq g$ on $\Omega$. Since $1_N=0$ $\mu$-a.s. and $g=0$ on $\Omega$, $\nrm{f}_p=\nrm{g}_p=0$, whence $f,g\in L^p_\Cplx$.

In general, let $f\in L^p_K$ and $U\in \Tcal_{L^p_K}$ be such that $f\in U$. Then there is $\epsilon>0$ such that $B\brac{f,\epsilon}\subseteq U$. Now, for every $h\in B\brac{f,\epsilon}$ and any $g\in L^p_K$ with $h=g$ $\mu$-a.s. by a prior result $\nrm{h-g}_p=0$, whence $\nrm{f-g}_p\leq \nrm{f-h}_p + \nrm{h-g}_p < \epsilon + 0$. Therefore, $g\in B\brac{f,\epsilon}\subseteq U$ as well, and, in particular, $g\in U$ for any $g\in L^p_K$ with $f=g$ $\mu$-a.s.

Therefore, if there is $g\in L^p_K$ with $f=g$ $\mu$-a.s. and $f\neq g$, then for any open neighbourhood $U$ of $f$ and $V$ of $g$, $U\cap V\neq \emptyset$, whence $L^p_K$ cannot in general be Hausdorff.

\noindent \textbf{Definition} 76.
Let $\brac{\Omega,\Tcal}$ be a topological space. A sequence $\brac{x_n}_{n\geq 1}\in \Omega$ converges to $x\in \Omega$ in $\Tcal$, $x_n\overset{\Tcal}{\to} x$, if and only if for all $U\in \Tcal$ with $x\in U$, there exists $N\geq 1$ such that $x_n\in U$ for all $n\geq N$.

\label{thm:prod_conv} \noindent \textbf{Theorem} 9-8.
\textbf{As a brief foreword}, recall, that $\omega\in \prod_{i\in I} \Omega_i$ is a map $I\to \bigcup_{i\in I} \Omega_i$ with $\omega\brac{i}\in \Omega_i$ for all $i\in I$. Let's introduce the following notation $\omega^i \defn \omega\brac{i}$ for any $i\in I$ and $\omega\in \prod_{i\in I} \Omega_i$.

Let $\brac{\Omega_i, \Tcal_i}_{i\in I}$ be a family of topological spaces, and $\brac{\Omega, \Tcal}$ be the product space, where $\Omega\defn \prod_{i\in I} \Omega_i$ and $\Tcal \defn \bigotimes_{i\in I} \Tcal_i$. If $\brac{x_n}_{n\geq 1}\in \Omega$ and $x\in \Omega$, then $x_n \to x$ in $\Tcal$ is equivalent to $x_n^j\overset{\Tcal_j}{\to}x^j$ for all $j\in I$.

Suppose $x_n \overset{\Tcal}{\to} x$. Let $j\in I$ and $U\in \Tcal_j$ be such that $x^j\in U$. Since the rectangle $V\defn U \times \prod_{i\neq j} \Omega_i$ is open in $\brac{\Omega, \Tcal}$ and $x\in V$, convergence of $\brac{x_n}_{n\geq 1}$ to $x$ in $\Tcal$ implies that there is $N\geq 1$ such that $x_n\in V$ for all $n\geq N$, whence $x_n^j\in U$ for all $n\geq N$ for such $N\geq 1$. Therefore $x_n^j\overset{\Tcal_j}{\to}x^j$ for all $j\in I$.

Conversely, suppose $x_n^j\to x^j$ in $\Tcal_j$ for all $j\in I$. Let $U\in\Tcal$ be such that $x\in U$. Since the open rectangles constitute the basis of the product topology (tutorial 6), there is $V\defn \prod_{i\in I} V_i \in \coprod_{i\in I} \Tcal_i$ such that $x\in V\subseteq U$. The set $J_V\defn \obj{\induc{i\in I} V_i \neq \Omega_i}$ is unique and finite by definition of a rectangle. Since $x\brac{i}\in V_i\in \Tcal_i$ for all $i\in I$, there is $N_i \geq 1$ such that $x_n\brac{i}\in V_i$ for all $n\geq N_i$ for each $i\in J_V$. For $N\defn \max\obj{\induc{N_i} i\in J_V}$, $N$, $N\geq 1$ and, being the greatest among a \textbf{finite} set of numbers, is finite itself. Then $x_n\brac{i}\in V_i$ for all $n\geq N \geq N_i$ for each $i\in J_V$, and, since $x_n\brac{i}\in \Omega_i$ for all $n\geq 1$ and $i\in I$, it is automatically true that $x_n\brac{i}\in V_i$ for $n\geq N$ for all $i\notin J_V$. Thus $x_n\in V$ for all $n\geq N$, whence $x_n\in U$ for all $n\geq N$. It follows that $x_n$ converges to $x$ in $\Tcal$.\\

\label{thm:subsp_conv} \noindent \textbf{Theorem} 9-9.
Let $\brac{\Omega, \Tcal}$ be a topological space, $\Omega'\subseteq \Omega$ and $\Tcal\defn \induc{\Tcal}_{\Omega'}$ be the induced topology on $\Omega'$. If $\brac{x_n}_{n\geq 1}\in \Omega'$ and $x\in \Omega'$, then $x_n\overset{\Tcal'}{\to} x$ is equivalent to $x_n\overset{\Tcal}{\to} x$.

Indeed, suppose $x_n\overset{\Tcal'}{\to} x$. If $U\in \Tcal$ is such that $x\in U$, then $U'\defn U\cap \Omega'\in \Tcal'$ is such that $x\in U'$ as $x\in \Omega'$, whence $\exists N\geq 1$ with $x_n\in U'\subseteq U$ for all $n\geq N$. Therefore $x_n\to x$ in $\Tcal$. Conversely, if $x_n\overset{\Tcal}{\to} x$ and $U'\in \Tcal'$ with $x\in U'$, then there is $U\in \Tcal$ such that $U'=U\cap \Omega'$. Since $U'\in U$, $x\in U$ implying existence of $N\geq 1$ such that $x_n\in U$ for all $n\geq N$. However, $x_n\in \Omega'$ and so $x_n\in U'$ for all such $n$. Consequently, $x_n\to x$ in $\Tcal'$.\\

\label{thm:hausdorff_conv} \noindent \textbf{Theorem} 9-10.
Let $\brac{\Omega, \Tcal}$ be a Hausdorff topological space. If $\brac{x_n}_{n\geq 1}\in \Omega$ and $x,y\in \Omega'$ are such that $x_n\overset{\Tcal}{\to} x$ and $x_n\overset{\Tcal}{\to} y$, then $x=y$.

Indeed, for any $U,V\in \Tcal$ with $x\in U$ and $y\in V$, there are $N,M\geq 1$ such that $x_n\in U$ for all $n\geq N$ and $x_n\in V$ for every $n\geq M$. Then $x_n\in U\cap V$ for all $n\geq \max\obj{M,N}$, implying that $U\cap V\neq \emptyset$. Since $\brac{\Omega, \Tcal}$ is Hausdorff this can only imply that $x=y$.\\

A corollary to theorem 9-10 is, that in non-Hausdorff spaces a convergent sequence may have many distinct points of convergence. For instance, $L^p$-spaces in general are one example. The reason for this non-uniqueness has been outlined in the discussion of the reason of $L^p_K$ for not being Hausdorff in general, unless, of course, the only $\mu$-null set is the empty set.

Let $f,g,\brac{f_n}_{n\geq 1} \in L^p_K\brac{\Omega, \Fcal, \mu}$, where $p\in \clo{1,+\infty}$. Without mentioning specific topologies, the notation $f_n\to f$ means that $f_n\brac{\omega}\to f\brac{\omega}$ in $K$ with respect to its usual topology $\Tcal_K$ for every $\omega \in \Omega$. The latter, in turn, is equivalent to $\abs{f_n\brac{\omega}-f\brac{\omega}}\to 0$ in $\Real^+$ or $\Cplx$.

\label{thm:lpto_nrm_infinitesimal} \noindent \textbf{Theorem} 9-11.
Let $\brac{\Omega, \Fcal, \mu}$ be a measure space, $p\in \clo{1,+\infty}$, $\brac{f_n}_{n\geq 1}, f\in L^p_K\brac{\Omega, \Fcal, \mu}$, $K=\Cplx$ or $\Real$. Then $f_n\lpto f$ is equivalent to $\nrm{f_n-f}_p \to 0$.

Indeed, if $f_n\lpto f$, then for every $U$ open in $L^p$ with $f\in U$ exists $N\geq 1$ such that $f_n\in U$ for all $n\geq N$. In particular, for all $\epsilon>0$ there is $N\geq 1$ such that $f_n\in B^p\brac{f, \epsilon}$ for all $n\geq N$, or equivalently, $\nrm{f_n-f}_p< \epsilon$ for all $n\geq N$. Thus, $\nrm{f_n-f}_p\to 0$.

Conversely, if $\nrm{f_n-f}_p\to 0$, then for all $\epsilon>0$ there is $N\geq 1$ with $f_n\in B^p\brac{f,\epsilon}$ for every $n\geq N$. If $U\in \Tcal_{L^p}$ with $f\in U$ then by definition of the $L^p$-space topology there is $\epsilon>0$ such that $B^p\brac{f,\epsilon}\subseteq U$, whence there exists $N\geq 1$ such that $f_n\in B^p\brac{f,\epsilon}\subseteq U$ for all $n\geq N$. Therefore, $f_n\to f$ in $L^p$.\\

\label{thm:seq_subseq_conv} \noindent \textbf{Theorem} 9-12.
Let $\brac{\Omega, \Fcal, \mu}$ be a measure space, $p\in \clo{1,+\infty}$, and $\brac{f_n}_{n\geq 1}, f\in L^p_K\brac{\Omega, \Fcal, \mu}$, $K=\Cplx$ or $\Real$. If $f_n\lpto f$ and $\brac{f_{n_k}}_{k\geq 1}$ is any sub-sequence then $f_{n_k}\lpto f$.

Indeed, since $\brac{f_{n_k}}_{k\geq 1}$ is a sub-sequence, then $1\leq n_k<n_{k+1}$ fo all $k\geq 1$. Thus $\brac{n_k}_{k\geq 1}\in \mathbb{N}$ is such that $n_k\uparrow +\infty$, whence for every $N\geq 1$ there is $K\geq 1$ with $n_k\geq N$ for all $k\geq K$. Now, for every $\epsilon>0$ there is $N\geq 1$ such that $\nrm{f_n-f}_p<\epsilon$ for all $n\geq N$, whence $\exists K\geq 1$ with $n_k\geq N$ for all $K\geq 1$. Thus for every $\epsilon>0$ there is $K\geq 1$ such that $\nrm{f_{n_k}-f}_p<\epsilon$ for all $k\geq K$. Therefore $f_{n_k}\lpto f$ and any sub-sequence of $\brac{f_n}_{n\geq 1}$ converges to the ``same'' limiting element of $L^p$.\\

\label{thm:lpto_limit_muas} \noindent \textbf{Theorem} 9-13.
Let $\brac{\Omega, \Fcal, \mu}$ be a measure space, $p\in \clo{1,+\infty}$, and $\brac{f_n}_{n\geq 1},f, g\in L^p_K\brac{\Omega, \Fcal, \mu}$, $K=\Cplx$ or $\Real$. If $f_n\lpto f$, then $f=g$ $\mu$-a.s if and only if $f_n\lpto g$. 

Indeed, if $f_n\lpto f$ and $f_n\lpto g$, then for any $\epsilon>0$ there is a common $N\geq 1$ such that $\nrm{f_n-f}_p<\frac{\epsilon}{2}$ and $\nrm{f_n-g}_p<\frac{\epsilon}{2}$ for all $n\geq N$, whence, by the triangle law for $L^p$ norms, $\nrm{f-g}_p\leq \nrm{f_n-f}_p+\nrm{f_n-g}_p<\epsilon$. Therefore $\nrm{f-g}_p=0$ and so $f=g$ $\mu$-a.s.

Conversely, if $f=g$ $\mu$-a.s. and $f_n\lpto f$, then $\nrm{f_n-g}_p \to 0$, because $\nrm{f_n-g}_p\leq \nrm{f-g}_p + \nrm{f_n-f}_p$ and $\nrm{f_n-f}_p \to 0$. Therefore, $f_n\to g$ in $\Tcal_{L^p}$.\\

Suppose, there exists non-empty $N\in\Fcal$ with $\mu\brac{N}=0$ and let $f\in L^p\brac{\Omega, \Fcal, \mu}$. Then $f_n\defn \frac{1}{2^n} f 1_{N^c} \to 0$ in $L^p$, since $\nrm{f_n}_p \leq \frac{1}{2^n} \nrm{f}_p$ and $\nrm{f}_p<+\infty$. However $f 1_N = 0$ $\mu$-a.s., whence $f_n\lpto f 1_N$.

\noindent \textbf{Definition} 77.
A sequence $\brac{f_n}_{n\geq 1}\in L^p_\Cplx\brac{\Omega, \Fcal, \mu}$, $p\in \clo{1, +\infty}$, is Cauchy if for all $\epsilon>0$ there is $N\geq 1$ such that $\nrm{f_n-f_m}_p<\epsilon$ for all $n,m\geq N$.

Suppose $\brac{f_n}_{n\geq 1}, f\in L^p_\Cplx$, $p\in \clo{1,+\infty}$ are such that $f_n\lpto f$. Let $\epsilon>0$. Then there is exists $N\geq 1$ such that $\nrm{f_n-f}<\frac{\epsilon}{2}$ for all $n\geq N$. Therefore for all $n,m \geq N$ \[\nrm{f_n-f_m}\leq \nrm{f_n-f}+\nrm{f_m-f}<\epsilon\] whence $\brac{f_n}_{n\geq 1}$ is a Cauchy sequence.

\label{thm:cauchy_subseq_conv} \noindent \textbf{Theorem} 9-14.
Let $\brac{\Omega, \Fcal, \mu}$ be a measure space and $p\in \clo{1,+\infty}$. Suppose $\brac{f_n}_{n\geq 1}\in L^p_K\brac{\Omega, \Fcal, \mu}$ is Cauchy, $K=\Cplx$ or $\Real$. If there is $\brac{f_{n_k}}_{k\geq 1}$ and $f\in L^p_K$ such that $f_{n_k}\lpto f$, then $f_n\lpto f$.

Indeed, let $\epsilon>0$. Then $\exists N\geq 1$ such that $\nrm{f_n-f_m}_p<\frac{\epsilon}{2}$ for all $m, n\geq N$. Also there is $K_1\geq 1$ with $\nrm{f_{n_k}-f}_p<\frac{\epsilon}{2}$ for all $k\geq K_1$. In addition for $N$ there is $K_2\geq 1$ with $n_k\geq N$ for all $k\geq K_2$. Therefore $\nrm{f_{n_k}-f}_p<\frac{\epsilon}{2}$ and $n_k\geq N$ for all $k\geq K_0\defn \max\obj{K_1, K_2}$. Thus for all $n\geq N$ it is true that $\nrm{f-f_n}_p \leq \nrm{f_{n_k}-f_n}_p + \nrm{f-f_{n_k}}_p < \epsilon$ for all $k\geq K_0$, whence $\nrm{f-f_n}_p \to 0$. Thus by theorem 9-11 $f_n\lpto f$.\\

\label{thm:cauchy_sum} \noindent \textbf{Theorem} 9-15.
Let $\brac{\Omega, \Fcal, \mu}$ be a measure space and $p\in \clo{1,+\infty}$. Let $\brac{f_n}_{n\geq 1} \in L^p_K\brac{\Omega, \Fcal, \mu}$ be Cauchy, $p\in \clo{1,+\infty}$ and $K=\Real$ or $\Cplx$. Then there exists a sub-sequence $\brac{f_{n_k}}_{k\geq 1}$ with \[\sum_{k=1}^{+\infty} \nrm{f_{n_{k+1}}-f_{n_k}}_p < +\infty\]

First of all, it needs to be shown that for any $M\geq 1$ and $\epsilon>0$ there is $N > M$ such that $\nrm{f_n-f_N}<\epsilon$ for all $n\geq N$. Indeed,
Cauchy sequences are such that for any $\epsilon>0$ there is $\hat{N}\geq 1$ such that $\nrm{f_n-f_m}_p<\epsilon$ for all $n,m\geq \hat{N}$. Therefore if $N\defn \max\obj{\hat{N},M}+1$, then $N > M$ and $ \nrm{f_n-f_m}_p<\epsilon$ for all $n,m\geq N$ because $N \geq \hat{N}$, and, in particular, for $m$ fixed at $N$ one gets $\nrm{f_n-f_N}_p<\epsilon$ for all $n\geq N$.

Thus for $M=0$ and $\epsilon\defn \frac{1}{2}>0$ there is $n_1\geq 1$ such that $\nrm{f_n-f_{n_1}}_p \leq \frac{1}{2}$. Furthermore, for $k\geq 1$ suppose $\brac{n_j}_{j=1}^k$ with $n_1<\ldots<n_k$ are such that $\nrm{f_n-f_{n_j}}_p\leq \frac{1}{2^j}$ for all $n\geq n_j$ and $j=1\ldots k$. Then for $M=n_k$ and for $\epsilon\defn \frac{1}{2^{k+1}}>0$ there is $n_{k+1} > n_k$ such that $\nrm{f_n-f_{n_{k+1}}}_p\leq \frac{1}{2^{k+1}}$ for all $n\geq n_{K+1}$. Thus by induction, there is $\brac{n_k}_{k\geq 1}$ with $n_k<n_{k+1}$ such that $\nrm{f_{n_{k+1}}-f_{n_k}}_p\leq \frac{1}{2^k}$ for all $k\geq 1$, whence \[\sum_{k=1}^{+\infty} \nrm{f_{n_{k+1}}-f_{n_k}}_p\leq \sum_{k\geq 1} \frac{1}{2^k} = 1<+\infty\]\\

\label{thm:lp_sum_muas} \noindent \textbf{Theorem} 9-16.
Let $\brac{\Omega, \Fcal, \mu}$ be a measure space and $p\in \clo{1,+\infty}$. For any $\brac{f_n}_{n\geq 1}\in L^p_K\brac{\Omega, \Fcal, \mu}$, $K=\Cplx$ or $\Real$, with\[\sum_{n=1}^{+\infty} \nrm{f_{n+1}-f_n}_p < +\infty\] there exists $f\in L^p_K$ such that $f_n\to f$ $\mu$-a.s in $K$.

Let $g\defn \sum_{k\geq 1} \abs{f_{k+1}-f_k}$. For each $n\geq 1$ the map $\omega\to\abs{f_{n+1}\brac{\omega}-f_n\brac{\omega}}$ is non-negative and measurable since each $f_n$ is measurable and $\abs{\cdot}$ is non-negative and $\Cplx$-$\Real^+$ continuous. Therefore $g_n\defn \sum_{k=1}^n \abs{f_{k+1}-f_k}$ for every $n\geq 1$ is non-negative and measurable. Since $g=\sup_{n\geq 1} g_n$, by tutorial 4 $g$ is non-negative and measurable.

Let $n\geq 1$ and suppose $\nrm{g_n}_p \leq \sum_{k=1}^n \nrm{f_{k+1}-f_k}_p$. Then by theorem 9-3 (the triangle law of the $L^p$ norm) \[\nrm{g_{n+1}}_p = \nrm{ g_n + \abs{f_{n+2}-f_{n+1}} }_p \leq \nrm{g_n}_p + \nrm{\abs{f_{n+2}-f_{n+1}} }_p \leq \sum_{k=1}^{n+1} \nrm{f_{k+1}-f_k}_p\] Therefore for all $n\geq 1$ the inductive argument implies \[\nrm{ \sum_{k=1}^n \abs{f_{k+1}-f_k} }_p\leq \sum_{k=1}^n \nrm{f_{k+1}-f_k}_p\leq \sum_{k=1}^\infty \nrm{f_{k+1}-f_k}_p\]

If $p\in \clop{1,+\infty}$, then for all $n\geq 1$ the definition of the $L^p$-norm implies \[ \int g_n^p d\mu \leq \brac{\sum_{k=1}^\infty \nrm{f_{k+1}-f_k}_p}^p \] By theorem 20 (Fatou) \[\int \liminf_{n\geq 1} g_n^p d\mu \leq \liminf_{n\geq 1} \int g_n^p d\mu \leq \brac{\sum_{k=1}^\infty \nrm{f_{k+1}-f_k}_p}^p\] Since $g_n\uparrow g$ everywhere on $\Omega$ in $\Rbar$ and $\phi^\alpha:\Zinf\to \Zinf$ is continuous for any $\alpha>0$, $\liminf_{n\geq 1} g_n^p = g^p$, whence \[\nrm{g}_p = \brac{\int g^p d\mu}^\frac{1}{p} \leq \sum_{k=1}^\infty \nrm{f_{k+1}-f_k}_p\] Since $\sum_{k=1}^\infty \nrm{f_{k+1}-f_k}_p < +\infty$, theorem 9-4 implies $g<+\infty$ $\mu$-a.s.

If $p=+\infty$, for any $n\geq 1$ by theorem 9-1 then $\abs{g_n}\leq \sum_{k=1}^\infty \nrm{f_{k+1}-f_k}_\infty$ $\mu$-a.s. Since countably many $\mu$-almost sure properties hold $\mu$-almost surely simultaneously together, there is $N\in \Fcal$ with $\mu\brac{N}=0$ such that $g_n\leq \sum_{k=1}^\infty \nrm{f_{k+1}-f_k}_\infty$ for all $n\geq 1$ on $N^c$, whence $g\leq \sum_{k=1}^\infty \nrm{f_{k+1}-f_k}_\infty$ $\mu$-a.s. and $g<+\infty$ $\mu$-a.s. by theorem 9-4.

Put $A\defn \obj{g < +\infty}\in \Fcal$, as $g$ is measurable, and note that $\mu\brac{A^c}=0$, since $g<+\infty$ $\mu$-a.s. Let $\omega\in A$ and for every $n\geq 1$ define $z_n\defn f_n\brac{\omega}$. Then $\sum_{k\geq 1} \abs{z_{k+1}-z_k} = g\brac{\omega} < +\infty$ and for any $\epsilon>0$ there is $N\geq 1$ with $\sum_{k\geq n} \abs{z_{k+1}-z_k}<\epsilon$ for all $n\geq N$, because $g_n\uparrow g$ and infinite non-negative series possess the partition-invariance property. Therefore for any $\epsilon>0$ there is $N\geq 1$ such that $\abs{\sum_{k\geq n+1} \abs{z_{k+1}-z_k}} < \frac{\epsilon}{2}$ for every $n\geq N$ from where it follows that for all $n > m\geq N$ \[\abs{z_n-z_m} = \abs{ \sum_{k=m}^{n-1} z_{k+1}-z_k } \leq \sum_{k=m}^{n-1} \abs{z_{k+1}-z_k} = \abs{ \sum_{k\geq m} \abs{z_{k+1}-z_k} - \sum_{k\geq n} \abs{z_{k+1}-z_k} } < \epsilon\] whence $\brac{z_n}_{n\geq 1}\in \Cplx$ is Cauchy. As the space $\brac{\Cplx, \Tcal_\Cplx}$ is complete, there is unique $z\in \Cplx$ such that $z_n\overset{\Cplx}{\to} z$. This holds for each $\omega\in A$ and so there is a map $z:A\to \Cplx$ such that $f_n\to z$ in $\Cplx$ everywhere on $A$.

Let $f\brac{\omega}\defn z\brac{\omega}$ if $\omega\in A$ and $0$ otherwise. Since $f_n\brac{\omega}\to z\brac{\omega} = f\brac{\omega}$ for all $\omega\in A$, $f_n 1_A \overset{\Cplx}{\to} f$ everywhere on $\Omega$, whence theorem 17 implies that the map $f:\brac{\Omega, \Fcal}\to\brac{\Cplx, \borel{\Cplx}}$ is measurable. Therefore $f_n\to f$ $\mu$-a.s. because $A^c$ is a $\mu$-null set.

On $\Omega$ for every $n\geq 1$ it is true that \[\abs{f_n} = \abs{f_1+\sum_{k=1}^{n-1}\brac{f_{k+1}-f_k}}\leq \abs{f_1} + \sum_{k=1}^{n-1} \abs{f_{k+1}-f_k} \leq \abs{f_1} + g\] At every point in $A$ for any $\epsilon>0$ there is $N\geq 1$ such that for all $n\geq N$ \[\abs{f} \leq \abs{f_n-f}+\abs{f_n} \leq \epsilon + \abs{f_1} + g\] whence $\abs{f}\leq \abs{f_1} + g$ $\mu$-a.s. Since $g\leq \sum_{n\geq 1} \nrm{f_{n+1}-f_n}_\infty$ $\mu$-a.s. and by theorem 9-1 $\abs{f_1}\leq \nrm{f_1}_\infty$ $\mu$-a.s., $\abs{f}\leq \nrm{f_1}_\infty + \sum_{n\geq 1} \nrm{f_{n+1}-f_n}_\infty$ $\mu$-a.s., whence $\nrm{f}_\infty < +\infty$ and $f\in L^\infty_\Cplx$.

Suppose $p\in \clop{1, +\infty}$. Since $\nrm{f_{n+1}-f_n}_p\geq 0$ and $\sum_{n\geq 1} \nrm{f_{n+1}-f_n}_p < +\infty$, for all $\epsilon>0$ there is $N\geq 1$ with \[\nrm{f_n - f_N}_p \leq \sum_{k=N}^{n-1} \nrm{f_{k+1}-f_k}_p\leq \sum_{k\geq N} \nrm{f_{n+1}-f_n}_p < \epsilon\] In particular there is $N\geq 1$ such that $\int_\Omega \abs{ f_n - f_N}^p d\mu \leq 1$ for all $n\geq N$. Now, the fact that $f_n\to f$ in $\Cplx$ on $A$ implies $\abs{f_n-f_N}^p \to \abs{f-f_N}^p$ on $A$ in $\Real^+$, whence $\liminf_{n\geq 1} 1_A\abs{f_n-f_N}^p = 1_A\abs{f_n-f_N}^p$ everywhere on $\Omega$. Thus from theorem 20 (Fatou) follows that \[\int_\Omega \liminf 1_A \abs{f_n-f_N}^p d\mu \leq \liminf \int_\Omega 1_A \abs{f_n-f_N}^p d\mu\] Furthermore, since $\mu\brac{A^c}=0$, properties of Lebesgue integral on $\mu$-null sets imply \[\int_\Omega \abs{f-f_N}^p d\mu \leq \liminf \int_\Omega \abs{f_n-f_N}^p d\mu\] Even though $\int_\Omega \abs{f_n-f_N}^p d\mu \leq 1$ only for all $n\geq N$, the definition of $\liminf$ implies that \[\liminf_{n\geq 1} \int_\Omega \abs{f_n-f_N}^p d\mu\leq 1\] Therefore $f-f_N \in L^p_\Cplx$ implying $f\in L^p_\Cplx$ by linearity of the $L^p$-space and $f_N\in L^p_\Cplx$.

Therefore for any $p\in \clo{1,+\infty}$, the sequence $\brac{f_n}_{n\geq 1}\in L^p_\Cplx$ with $\sum_{n\geq 1} \nrm{f_{n+1}-f_n}_p<+\infty$ is such that there exists $f\in L^p_\Cplx$ with $f_n\to f$ $\mu$-a.s.

Suppose $\brac{f_n}_{n\geq 1}\in L^p_\Real$ such that $\sum_{k\geq 1} \nrm{f_{k+1}-f_k}_p<+\infty$. Then by theorem 9-5 $\brac{f_n}_{n\geq 1}\in L^p_\Cplx$, whence there exists $g\in L^p_\Cplx$ such that $f_n\to g$ $\mu$-a.s. Therefore $\exists N\in \Fcal$ with $\mu\brac{N}=0$ such that $\forall \omega\notin N$ $f_n\brac{\omega}\to g\brac{\omega}$ in $\Cplx$. However, $f_n\brac{\Omega}\subseteq \Real$ and thus $g\brac{\omega}\in \Real$ for $\omega\in N^c$. If $f\defn g 1_{N^c}$, then $f=g$ $\mu$-a.s. and $f\brac{\Omega}\subseteq \Real$. Moreover, $\nrm{f-g}_p=0$ implies that $\nrm{f}_p=\nrm{g}_p$, because by the triangle law $\nrm{f}_p\leq \nrm{g}_p+\nrm{f-g}_p$ and $\nrm{g}_p\leq \nrm{f}_p+\nrm{f-g}_p$. Therefore for such sequence there exists $f\in L^p_\Real$ such that $f_n\to f$ $\mu$-a.s.\\

\label{thm:muas_lpto} \noindent \textbf{Theorem} 9-17.
Let $\brac{\Omega, \Fcal, \mu}$ be a measure space and $p\in \clo{1,+\infty}$. For any $\brac{f_n}_{n\geq 1}, f\in L^p_K\brac{\Omega, \Fcal, \mu}$ with \[\sum_{n=1}^{+\infty} \nrm{f_{n+1}-f_n}_p < +\infty\], $K=\Cplx$ or $\Real$, $f_n\to f$ $\mu$-a.s in $K$ implies $f_n\lpto f$.

Suppose $p=+\infty$ and let $m\geq n\geq 1$. Then $\abs{f_{m+1}-f_n}\leq \sum_{k=n}^m \abs{f_{k+1}-f_k}$ by the triangle inequality, whence by theorem 9-1 and features of almost surely holding properties it is $\mu$-almost surely true that for all $m\geq n$ simultaneously $\abs{f_{m+1}-f_n}\leq \sum_{k\geq n} \nrm{f_{k+1}-f_k}_\infty$. In combination with $f_n\to f$ $\mu$-a.s., for every $n\geq 1$ there is $N\in \Fcal$ with $\mu\brac{N}=0$ such that at any point outside of $N$ for every $\epsilon>0$ there exists $M\geq 1$ such that for all $m\geq \max\obj{M,n}$ \[\abs{f-f_n} \leq \abs{f-f_{m+1}} + \abs{f_{m+1}-f_n} < \epsilon + \sum_{k\geq n} \nrm{f_{k+1}-f_k}_\infty\] Therefore outside of $N$ \[\abs{f-f_n} < \epsilon + \sum_{k\geq n} \nrm{f_{k+1}-f_k}_\infty\] for every $\epsilon>0$, whence $\abs{f-f_n} < \sum_{k\geq n} \nrm{f_{k+1}-f_k}_\infty$ $\mu$-a.s. Therefore by theorem 9-1 $\nrm{f-f_n}_\infty < \sum_{k\geq n} \nrm{f_{k+1}-f_k}_\infty$ for every $n\geq 1$.

Suppose $p\in \clop{1,+\infty}$. For any $m\geq n\geq 1$ induction on theorem 9-2 (the triangle law) implies \[\nrm{f_{m+1}-f_n}_p\leq \sum_{k=n}^m \nrm{f_{k+1}-f_k}_p\leq \sum_{k\geq n} \nrm{f_{k+1}-f_k}_p\] Using the definition of the $L_p$-norm gives the following: for all $m\geq n\geq 1$ \[\int \abs{f_{m+1}-f_n}^p d\mu \leq \brac{\sum_{k\geq n} \nrm{f_{k+1}-f_k}_p}^p\] Since $f_n\to f$ $\mu$-a.s. power function is continuous, there is $N\in \Fcal$ with $\mu\brac{N}=0$ such that $f_n\brac{\omega}\to f\brac{\omega}$ for all $\omega\notin N$ and therefore $\liminf_{m\geq 1} 1_A \abs{f_{m+1}-f_n}^p = 1_A \abs{f-f_n}^p$ for $A\defn N^c$. By theorem 20 (Fatou) \[ \int \abs{f_{m+1}-f_n}^p d\mu = \int \liminf_{m\geq 1} 1_A \abs{f_{m+1}-f_n}^p d\mu \leq \liminf_{m\geq 1} \int 1_A \abs{f_{m+1}-f_n}^p d\mu\] Therefore $\nrm{f-f_n}_p \leq \brac{\sum_{k\geq n} \nrm{f_{k+1}-f_k}_p}^p$ for all $n\geq 1$ and whatever $p\in \clo{1,+\infty}$.

However $\sum_{k\geq 1} \nrm{f_{k+1}-f_k}_p<+\infty$ means that for every $\epsilon>0$ there is $N\geq 1$ such that for all $n\geq N$ \[\sum_{k\geq n} \nrm{f_{k+1}-f_k}_p < \epsilon\] whence $\nrm{f-f_n}_p<\epsilon$ for all $n\geq N$. Therefore $\nrm{f-f_n}_p\to 0$ in $\Real^+$ and by theorem 9-11 $f_n\lpto f$.\\

\label{thm:lpto_muas} \noindent \textbf{Theorem} 9-18.
Let $\brac{\Omega, \Fcal, \mu}$ be a measure space and $p\in \clo{1,+\infty}$. For any $\brac{f_n}_{n\geq 1}, f\in L^p_K\brac{\Omega, \Fcal, \mu}$ with \[\sum_{n=1}^{+\infty} \nrm{f_{n+1}-f_n}_p < +\infty\], $K=\Cplx$ or $\Real$, $f_n\lpto f$ implies $f_n\to f$ $\mu$-a.s in $K$.

Indeed, suppose $f\in L^p_K\brac{\Omega, \Fcal, \mu}$ is such that $f_n\to f$ in $L^p_K$. By theorem 9-16 there is $g\in L^p_K$ such that $f_n\to g$ $\mu$-a.s. and by theorem 9-17 $f_n\lpto g$. By theorem 9-13 $f=g$ $\mu$-a.s. whence $f_n\to f$ $\mu$-a.s. in $K$.\\

\label{thm:sum_lp_muas_conv} \noindent \textbf{Theorem} 44.
Let $\brac{\Omega, \Fcal, \mu}$ be a measure space and $p\in \clo{1,+\infty}$. Let $\brac{f_n}_{n\geq 1}\in L^p_\Cplx\brac{\Omega, \Fcal, \mu}$ be such that \[\sum_{n=1}^{+\infty} \nrm{f_{n+1}-f_n}_p < +\infty\] Then there exists $f\in L^p_\Cplx\brac{\Omega, \Fcal, \mu}$ such that $f_n\to f$ $\mu$-a.s. Moreover $f_n\lpto g$ is equivalent $f_n\to g$ $\mu$-a.s. in $\Cplx$.

In this setting first part of the theorem follows from theorems 9-16, while the second from 9-17 and 9-18.\\

\label{thm:cauchy_conv_subseq} \noindent \textbf{Theorem} 9-19.
Let $\brac{\Omega, \Fcal, \mu}$ be a measure space and $p\in \clo{1,+\infty}$. Let $\brac{f_n}_{n\geq 1}\in L^p_K\brac{\Omega, \Fcal, \mu}$ be a Cauchy in $L^p_K$, $K=\Real$ or $\Cplx$. Then there is a sub-sequence $\brac{f_{n_k}}_{k\geq 1}$ and $f\in L^p_K$ such that $f_{n_k}\lpto f$ and $f_{n_k}\to f$ $\mu$-a.s.

Indeed, since $\brac{f_n}_{n\geq 1}$ is Cauchy, by theorem 9-15 there is a sub-sequence $\brac{f_{n_k}}_{k\geq 1}$ such that \[\sum_{n\geq 1} \nrm{f_{n+1}-f_n}_p < +\infty\] By theorem 9-16 there is $f\in L^p_K$ such that $f_{n_k}\to f$ $\mu$-a.s. and by theorem 9-17 $f_{n_k}\lpto f$.\\

\label{thm:lp_conv_subseq} \noindent \textbf{Theorem} 45.
Let $\brac{\Omega, \Fcal, \mu}$ be a measure space and $p\in \clo{1,+\infty}$. Let $\brac{f_n}_{n\geq 1}, f\in L^p_K\brac{\Omega, \Fcal, \mu}$ be such that $f_n\to f$ in $L^p_K$, $K=\Real$ or $\Cplx$. Then there is a sub-sequence $\brac{f_{n_k}}_{k\geq 1}$ such that $f_{n_k}\to f$ $\mu$-a.s.

Indeed, since $f_n\lpto f$, $\brac{f_n}_{n\geq 1}$ is Cauchy. Then by theorem 9-19 there is a sub-sequence $\brac{f_{n_k}}_{k\geq 1}$ and $g\in L^p_K$ such that $f_{n_k}\lpto g$. However $f_n\lpto f$ implies $f_{n_k}\lpto f$ by theorem 9-12, whence by theorem 9-13 $g=f$ $\mu$-a.s. Therefore $f_{n_k}\to f$ $\mu$-a.s.\\

\label{thm:cauchy_lpto} \noindent \textbf{Theorem} 46.
Let $\brac{\Omega, \Fcal, \mu}$ be a measure space, $p\in \clo{1,+\infty}$ and $K=\Real$ of $\Cplx$. Let $\brac{f_n}_{n\geq 1}\in L^p_K\brac{\Omega, \Fcal, \mu}$ be a Cauchy sequence in $L^p$. Then there is $f\in L^p_K$ such that $f_n\lpto f$.

Since $\brac{f_n}_{n\geq 1}$ is Cauchy, by theorem 9-19 there is a sub-sequence $\brac{f_{n_k}}_{k\geq 1}$ and $f\in L^p_K$ such that $f_{n_k}\lpto f$. Since $f_n$ is Cauchy, by theorem 9-14 $f_n\lpto f$.\\

% section tut_9 (end)

\section{Bounded Linear Functionals in $L^2$} % (fold)
\label{sec:tut_10}
\url{http://probability.net/PRTbounded.pdf}

\noindent \textbf{Definition} 78.
A sub-sequence of a sequence $\brac{x_n}_{n\geq 1}$ is any sequence of the form $\brac{x_{\phi\brac{k}}}_{k\geq 1}$ where $\phi:\mathbb{N}\to\mathbb{N}$ is a strictly increasing map.

\label{thm:conv_equ} \noindent \textbf{Theorem} 10-1.
Let $\brac{E,d}$ be a metric space with metric topology $\Tcal$, and $\brac{x_n}_{n\geq 1}, x\in E$. Then $x_n\overset{\Tcal}{\to}x$ if and only if $d\brac{x_n, x}\to 0$.

Note that his is a restatement of theorem 9-11 in terms of convergence metric topology.

Indeed, if $x_n\overset{\Tcal}{\to} x$ then $B^d_E\brac{x, \epsilon}\in \Tcal$ for any $\epsilon > 0$, whence there is $N\geq 1$ such that $x_n\in B^d_E\brac{x, \epsilon}$ for all $n\geq N$. Therefore for any $\epsilon>0$ there is $N\geq 1$ such that $d\brac{x_n,x}<\epsilon$ for all $n\geq N$.

Conversely, suppose $d\brac{x_n,x}\to 0$. For any $U\in \Tcal$ with $x\in U$ there is $\epsilon>0$ such that $B^d_E\brac{x, \epsilon}\subseteq U$, whence there is $N\geq 1$ such that $x_n\in B^d_E\brac{x,\epsilon}$ for all $n\geq N$. Thus $x_n\in U$ for all $n\geq N$ and thus $x_n\overset{\Tcal}{\to} x$.\\

\label{thm:conv_seq_cauchy} \noindent \textbf{Theorem} 10-2.
Let $\brac{E,d}$ be a metric space with metric topology $\Tcal$, and $\brac{x_n}_{n\geq 1}\in E$. Then if there exist $x\in E$ with $x_n\overset{\Tcal}{\to} x$ then $\brac{x_n}_{n\geq 1}$ is Cauchy.

Indeed, $x_n\overset{\Tcal}{\to} x$ implies that for any $\epsilon>0$ there exists $N\geq 1$ with $d\brac{x_n, x}<\frac{\epsilon}{2}$ for all $n\geq N$. Therefore $d\brac{x_n, x_m} \leq d\brac{x_n, x} + d\brac{x_m, x}< \epsilon$ for all $n,m\geq N$, whence that $\brac{x_n}_{k\geq 1}$ is Cauchy.\\

\label{thm:compact_subseq} \noindent \textbf{Theorem} 10-3.
Let $\brac{E,\Tcal}$ be a compact topological space metrizable by $d$. For any sequence $\brac{x_n}_{n\geq 1}\in E$ there exist $x\in E$ and a sub-sequence $\brac{x_{n_k}}_{k\geq 1}$ such that $x_{n_k}\overset{\Tcal}{\to} x$.

For all $n\geq 1$ let $X_n\defn\obj{\induc{x_k} k\geq n}$ and $F_n\defn \clo{X_n}$ in $\brac{E, \Tcal}$. Then $X_{n+1}\subseteq X_n$ implies $X_{n+1}\subseteq F_n$ and by theorem sup-1 $F_{n+1}\subseteq F_n$.

Now, for any finite $F\subseteq \mathbb{N}$ there is $N\in F$ such that $n\leq N$ for all $n\in F$. Therefore $\bigcap_{n\in F} F_n = F_N$ and, since $\emptyset \neq X_N\subseteq F_N$, $\bigcap_{n\in F} F_n \neq \emptyset$ for any finite $F\subseteq \mathbb{N}$.

Since $\brac{E,\Tcal}$ is compact, $\brac{F_n^c}_{n\geq 1}$ cannot be an open cover of $E$, because by the observation above $\bigcup_{n\in F} F_n^c \neq E$ for any finite $F\subseteq \mathbb{N}$. Thus $E = \bigcup_{n\geq 1} F_n^c$ whence $\bigcap_{n\geq 1} F_n \neq \emptyset$.

Furthermore, for any $x\in \bigcap_{n\geq 1} F_n$ and any $U\in \Tcal$ with $x\in U$ it is true that $U\cap X_n \neq \emptyset$ for all $n\geq 1$ since $F_n=\clo{X_n}$ in $\brac{E, \Tcal}$. In particular, since $\Tcal$ is metrizable, $B^d_E\brac{X, \epsilon}\cap X_n\neq \emptyset$ for all $n\geq 1$ and $\epsilon > 0$.

Thus for $n=1$ and $\epsilon=1$ there is $n_1\geq 1$ such that $x_{n_1}\in B^d_E\brac{x, 1}$. Given any $n_k$ with $x_{n_k}\in B^d_E\brac{x, \frac{1}{2^k}}$ for $n=1+n_k$ and $\epsilon=\frac{1}{2^{k+1}}$ there is $n_{k+1}>n_k$ with $x_{n_{k+1}}\in B^d_E\brac{x, \frac{1}{2^{k+1}}}$.

Therefore there is a sub-sequence $\brac{x_{n_k}}_{k\geq 1}$ with $d\brac{x, x_{n_k}}\to 0$, whence $x_{n_k}\overset{\Tcal}{\to} x$.\\

\label{thm:open_ball_cover} \noindent \textbf{Theorem} 10-4.
Let $\brac{E,d}$ be a metric space such that any sequence $\brac{x_n}_{n\geq 1}$ in $E$ has a convergent sub-sequence. Then for any $r>0$ there is a finite $F\subseteq E$ such that $E=\bigcup_{x\in F} B^d_E\brac{x, r}$.

Suppose there is $r>0$ such that for any finite $F\subseteq E$ the collection $\brac{O_x}_{x\in F}$ is not a covering of $E$, where $O_x\defn B^d_E\brac{x,r}$. Let $x_1\in E$, then $O_{x_1}$ does not cover $E$ and so there must be $x_2\in E$ with $x_2\notin O_{x_1}$. Suppose $\brac{x_k}_{k=1}^n\in E$ are such that $x_{k+1}\notin \bigcup_{m=1}^k O_{x_k}$ for all $k=1\ldots {n-1}$. Since $\brac{O_{x_k}}_{k=1}^n$ cannot cover $E$ there must exist $x_{n+1}\in E$ such that $x_{n+1}\notin \bigcup_{k=1}^n O_{x_k}$.

Thus constructed sequence $\brac{x_n}_{n\geq 1}$ is such that $x_{n+1}\notin \bigcup_{m=1}^n O_{x_m}$ for all $n\geq 1$, whence $x_n\notin B^d_E\brac{x_m,r}$ or, equivalently, $d\brac{x_n, x_m}\geq r$ for all $n>m\geq 1$.

So, if $\brac{x_{n_k}}_{k\geq 1}$ is any arbitrary sub-sequence, then $d\brac{x_{n_k}, x_{n_l}}\geq r$ for any $k, l\geq 1$, whence $\brac{x_{n_k}}_{k\geq 1}$ cannot be Cauchy. Therefore by theorem 10-2 there is no $x\in E$ such that $x_{n_k}\to x$, from where it follows that $\brac{x_n}_{n\geq 1}$ is a particular sequence which contains no convergent sub-sequence.

By contraposition, the above argument implies that whenever the metric space $\brac{E,d}$ is such that every sequence has a convergent sub-sequence, then for any $r>0$ it can be covered by a finite set of open balls of common radius $r$.\\

\label{thm:conv_subseq_compact} \noindent \textbf{Theorem} 10-5.
Let $\brac{E,\Tcal}$ be a topological space metrizable by $d$ such that any sequence $\brac{x_n}_{n\geq 1}$ in $E$ has a convergent sub-sequence. Then $\brac{E, \Tcal}$ is compact.

Indeed, let $\brac{V_i}_{i\in I}$ be an open covering of $E$. For any $x\in E$ define\[r\brac{x}\defn \sup\obj{ \induc{ r > 0 } \exists i\in I\,\text{s.t.}\,B^d_E\brac{x, r}\subseteq V_i }\]

For any $x\in E$ there is $j_x\in I$ such that $x\in V_{j_x}$. Since $V_{j_x}\in \Tcal$ and $\Tcal$ is induced by metric $d$, there is $r>0$ such that $B^d_E\brac{x,r}\subseteq V_{j_x}$, whence the set over which the supremum is taken is non-empty and contains at least $r>0$. Therefore $r\brac{x} > 0$ for ever $x\in E$.

Suppose $\inf_{x\in E} r\brac{x} = 0$. Then for any $n\geq 1$ there is $x_n\in E$ such that $r\brac{x_n}<\frac{1}{n}$. Since in the space $\brac{E, d}$ every sequence contains a convergent sub-sequence there exist $\bar{x}\in E$ and $\brac{x_{n_k}}_{k\geq 1}$ such that $x_{n_k}\overset{\Tcal}{\to} \bar{x}$. Since $\bar{x}\in E$ there is $j\in I$ and $\bar{r}>0$ such that $B^d_E\brac{\bar{x}, \bar{r}}\subseteq V_j$.

Now, from convergence in $\Tcal$ there is $K_1\geq 1$ such that $x_{n_k}\in B^d_E\brac{\bar{x},\frac{\bar{r}}{2}}$ for all $k\geq K_1$. From the construction of $\brac{x_n}_{n\geq 1}$ and the fact that $n_k\uparrow +\infty$ there exists $K_2 \geq 1$ such that $r\brac{x_{n_k}} < \frac{1}{n_k} \leq \frac{\bar{r}}{4}$ for all $k\geq K_2$. Therefore for all $k\geq K\defn \max\obj{K_1, K_2}$ both $r\brac{x_{n_k}} \leq \frac{\bar{r}}{4}$ and $d\brac{\bar{x},x_{n_k}} < \frac{\bar{r}}{2}$.

The last bound implies $B^d_E\brac{x_{n_k}, \frac{\bar{r}}{2}}\subseteq B^d_E\brac{\bar{x}, \bar{r}}$ for all $k\geq K$, because for every $y\in B^d_E\brac{x_{n_k}, \frac{\bar{r}}{2}}$ \[d\brac{\bar{x},y}\leq d\brac{\bar{x},x_{n_k}} + d\brac{x_{n_k},y} < \frac{\bar{r}}{2} + \frac{\bar{r}}{2} \] Thus $B^d_E\brac{x_{n_k}, \frac{\bar{r}}{2}}\subseteq V_j$ for all $k\geq K$, whence $\frac{\bar{r}{2}}\leq r\brac{x_{n_k}}$ by definition of $r\brac{\cdot}$. However it is also true that $r\brac{x_{n_k}} \leq \frac{\bar{r}}{4}$ for all such $k\geq K$. Therefore $\frac{\bar{r}}{2}\leq \frac{\bar{r}}{4}$ which is self-contradictory for any $\bar{r}>0$, whence $\inf_{x\in E} r\brac{x} > 0$.

Therefore there exists $r_0\in \brac{0,\inf_{x\in E} r\brac{x}}$. By theorem 10-4 for $r_0>0$ there exists $F\subseteq E$ such that $E\subseteq \bigcup_{x\in F} B^d_E\brac{x, r_0}$. However, if $x\in E$, then $r\brac{x}>r_0$, whence by definition there is $r>r_0$ and $j_x\in I$ such that $B^d_E\brac{x, r}\subseteq V_{j_x}$. In particular, for each $x\in F$ pick one such $j_x$ with $B^d_E\brac{x, r_0}\subseteq V_{j_x}$ to form $J\subseteq I$, which is a finite subset of $I$ such that $E=\bigcup_{x\in F} B^d_E\brac{x, r_0} \subseteq \bigcup_{i\in J} V_i$.

Therefore when $\brac{E, \Tcal}$ is a metrizable topological space such that every sequence has a convergent sub-sequence, from every open cover of $E$ one can extract a finite sub-cover. Therefore $\brac{E, \Tcal}$ is compact topological space.\\

\label{thm:metric_compact} \noindent \textbf{Theorem} 47.
A metrizable topological space $\brac{E, \Tcal}$ is compact if and only if for every sequence $\brac{x_n}_{n\geq 1}\in E$ there exists a sub-sequence $\brac{x_{n_k}}_{k\geq 1}$ and $x\in E$ such that $x_{n_k}\overset{\Tcal}{\to} x$.

Indeed, this theorem is a result of a direct application of theorems 10-3 and 10-5.\\

Let $a<b\in \Real$ and $\brac{x_n}_{n\geq 1}\in \brac{a,b}$. Since $\brac{a,b}\subseteq \clo{a,b}$, $\clo{a,b}$ is metrizable and by theorem 34 the latter is compact, theorem 47 implies that there is $x\in \clo{a,b}$ and $\brac{x_{n_k}}_{k\geq 1}$ such that $x_{n_k}\to x$ in $\clo{a,b}$. Note that $x$ is not necessarily in $\brac{a,b}$. Indeed, the sequence $\brac{a+\frac{\Delta}{2^n}}_{n\geq 1}$ for $\Delta\defn b-a>0$ is in $\brac{a,b}$ yet it converges to $a\notin \brac{a,b}$.

\label{thm:realn_compact} \noindent \textbf{Theorem} 10-6.
A subset of $\Real^n$ is compact if and only if it is closed and bounded with respect to its usual metric.%% Wrong statement!!!!

Let $E\defn \clo{-M, M}^n\subseteq \Real^n$, $n\geq 1$ and $M\in \Real^+$. Let $\Tcal_{\Real^n}$ be the product topology on $\Real^n$ and $\Tcal_E\defn \induc{\Tcal_{\Real^n}}_E$ be the subspace topology on $E$.

Let $\brac{x_k}_{k\geq 1}, x\in E$. By theorem 9-9 $x_k\overset{\Tcal_E}{\to} x$ is equivalent to $x_k\overset{\Tcal_{\Real^n}}{\to} x$. Let $\brac{x_k}_{k\geq 1}, x\in \Real^n$. By theorem 9-8 $x_k\overset{\Tcal_{\Real^n}}{\to} x$ is equivalent to $x_k^i\overset{\Tcal_{\Real}}{\to} x^i$ for all $i=1\ldots n$.

By theorem 6-3-1 the space $\brac{\Real^n, \Tcal_{\Real^n}}$ is metrizable, at least by \[d\brac{x,y}\defn \brac{\sum_{i=1}^n d_i^2\brac{x^i,y^i}}^\frac{1}{2}\] where $d_i$ is any metric on $\Real$. Thus by theorem 12 the space $\brac{E,\Tcal_E}$ is metrizable by the metric on $E$ induced by $d$. As for the constituents of $E$, in addition to being compact by theorem 34, each $\clo{-M, M}$ as a subspace of $\Real$, is metrizable by the induced metric $d_i$ on $\Real$ according to theorem 12.

A natural metric is the Euclidean distance \[d\brac{x,y}\defn \brac{\sum_{i=1}^n d_i^2\brac{x^i,y^i}}^\frac{1}{2}\] with lower-index notation for coordinates, where $d_i=d_\Real$ and $d_\Real\brac{x,y}\defn \abs{x-y}$. So in the following, $\induc{d_\Real}_{\clo{-M,M}}$ induces the topology $\Tcal_{\clo{-M,M}}$ and $\induc{d}_E$ -- $\Tcal_E$.

Suppose $\brac{x_k}_{k\geq 1}\in E$. Then $\brac{x_k^i}_{k\geq 1}\in \clo{-M,M}$ for any $i=1\ldots n$. By theorem 47 there is $x^1\in \clo{-M,M}$ and a sub-sequence $\brac{x_{p_k}^1}_{k\geq 1}$ such that $x_{p_k}^1\to x^1$ in $\Tcal_{\clo{-M,M}}$, which is equivalent to $x_{p_k}^1\overset{\Tcal_\Real}{\to} x^1$ due to the prior observations.

%% Consider, changing the order!!!
Suppose $m=1\ldots {n-1}$ and $y_k\defn x_{q_k}$ is a sub-sequence of $\brac{x_k}_{k\geq 1}$ such that $y_k^i\overset{\Tcal_\Real}{\to} x^i$ for some $x^i\in \clo{-M,M}$ for all $i=1\ldots m$. Since $\brac{y_k^{m+1}}_{k\geq 1}\in\clo{-M,M}$, theorem 47 implies that there is $x^{m+1}\in \clo{-M,M}$ and ar a sub-sequence $\brac{y_{p_k}}_{k\geq 1}$ such that $y_{p_k}^{m+1}\overset{\Tcal_{\clo{-M,M}}}{\to} x^{m+1}$, or equivalently $y_{p_k}^{m+1}\overset{\Tcal_\Real}{\to} x^{m+1}$. Since $\brac{p_k}_{k\geq 1}$ and $\brac{q_k}_{k\geq 1}$ are strictly increasing, $l_{k_1}<l_{k_2}$ for any $k_1<k_2$, where $l_k\defn q_{p_k}$. Therefore $\brac{x_{l_k}}_{k\geq 1}$ is a sub-sequence of $\brac{x_k}_{k\geq 1}$. Since sub-sequences of a convergent sequence always converge to the same limit, for any $j=1\ldots m$ it is true that $x_{l_k}^j\overset{\Tcal_\Real}{\to} x^j$. Furthermore $x_{l_k}=x_{q_{p_k}}=y_{p_k}$ and $y_{p_k}^{m+1}\overset{\Tcal_\Real}{\to} x^{m+1}$, whence $x_{l_k}^j\overset{\Tcal_\Real}{\to} x^j$ for all $j=1\ldots {m+1}$.

Thus for any sequence $\brac{x_k}_{k\geq 1}\in E$ there is $x\in E$ and a sub-sequence $\brac{x_{p_k}}_{k\geq 1}$ such that $x_{p_k}^j\overset{\Tcal_\Real}{\to} x^j$ for all $j=1\ldots n$, which is equivalent to $x_{p_k}\overset{\Tcal_E}{\to} x$. Therefore by theorem 47 the space $\brac{E,\Tcal_E}$ is compact.

Note that $E$ is closed in $\Real^n$ by theorem 35, since $\Real^n$ is Hausdorff due to being metrizable. Also $E$ is bounded with respect to the usual metric, since $d\brac{x,y} \leq \sqrt{n}\max_{1\leq i\leq n} d_\Real\brac{x^i,y^i}\leq 2\sqrt{n} M$ for every $x,y\in E$, which implies that $\delta\brac{E}\defn \sup\obj{ \induc{ d\brac{x, y} } x, y\in E }$ is not greater than $2\sqrt{n} M < +\infty$.\\

\label{thm:bounded_metric_compact} \noindent \textbf{Theorem} 48.
A subset of $\Real^n$ is compact if and only if it is closed and bounded with respect to its usual metric.

Let $A$ be a closed subset of $\Real^n$, bounded with respect to the usual metric of $\Real^n$. With $A$ bounded, the diameter of $A$, $\delta\brac{A}$, is finite, whence there exists $B\in \Real^+$ such that $d\brac{x,y}\leq B$ for all $x,y\in A$. Let $x_0\in A$ be fixed and pick any $y\in A$. Then for all $y\in A$ \[\max_{1\leq i\leq n} \abs{x_0^i-y^i} \leq d\brac{x_0,y}\leq B\] whence $y^i\in \clo{x_0^i-B, x_0^i+B}$ for all $i=1\ldots n$ for all $y\in A$. It is here, where the ``usualness'' of the metric is employed. Indeed usual metrics, or composite metrics, expose certain properties of each ``dimension''.

If $M_i\defn \max\obj{\abs{x_0^i-B}, \abs{x_0^i+B}}$ for each $i=1\ldots n$.  then $\abs{y^i}\leq M_i$ for each $i=1\ldots n$ for every $y\in A$, whence $A\subseteq E\defn \clo{-M, M}^n$ where $M\defn \max_{i=1\ldots n} M_i$. Furthermore, since $A$ is closed in $\Real^n$, $A^c$ is open whence $E\cap A^c\in \Tcal_E$. However, since $A\subseteq E$, $E\cap A^c = E\setminus A$ whence $A$ must be closed in $\brac{E, \Tcal_E}$.

Since by theorem 10-6 $\clo{-M,M}^n$ is compact for any $M\in \Real^+$ and $A\subseteq E$, theorem 8-5 implies that $\brac{A, \induc{\Tcal_E}_A}$ is a compact space. Since $E\subseteq	\Real^n$, $\induc{\Tcal_{\Real^n}}_A = \induc{\Tcal_E}_A$. In conclusion, $\brac{A, \induc{\Tcal_{\Real^n}}_A}$ is a compact topological space.

Conversely, let $A$ be a compact subspace of $\Real^n$. Since $\Real^n$ is metrizable, by theorem 8-6 it must be Hausdorff, whence by theorem 35 $A$ is closed. Further by theorem 8-7, $A$ must be bounded with respect to any metric, which induces $\Tcal_{\Real^n}$. In particular, $A$ is bounded with respect to $d$.\\

Define a map $\phi:\Cplx^n \to \Real^{2n}$ as $\phi\brac{\brac{z_k}_{k=1}^n} \defn \brac{\brac{\re z_k, \im z_k}}_{k=1}^n$. Since the equality of real and imaginary parts of any complex numbers is equivalent to the equality of the complex numbers themselves, $\phi$ must be an injection. Furthermore, since $\Real^{2n}$ is identical to $\brac{\Real\times \Real}^n$, for any $\brac{x_k}_{k=1}^{2n} \in \Real^{2n}$ there is $z\defn\brac{z_k}_{k=1}^n \in \Cplx^n$ defined as $z_k \defn x_{2k-1} + i x_{2k}$ such that $\phi\brac{z}=\brac{x_k}_{k=1}^{2n}$. Therefore $\phi$ is a bijection between $\Cplx^n$ and $\Real^{2n}$.

The usual metric on $\Real$ is $d\brac{x,y} = \abs{x-y}$ whereas the usual metric on  $\Cplx$ is \[d_\Cplx\brac{z_1, z_2} = \sqrt{ \abs{\re z_1 - \re z_2 }^2 + \abs{ \im z_1 - \im z_2 }^2 } = \brac{ d_\Real\brac{\re z_1, \re z_2} + d_\Real\brac{\im z_1, \im z_2} }^\frac{1}{2}\] which is the usual metric on $\Real^2$.

By theorem 6-3-1, the metric $d_{\Cplx^n}\defn\sqrt{ \sum_{k=1}^n d_\Cplx\brac{z_1^k, z_2^k}^2 }$ induces the product space $\brac{\Cplx^n, \Tcal_{\Cplx^n}}$ and the metric $d_{\Real^{2n}}\defn\sqrt{ \sum_{k=1}^{2n} d_\Real\brac{x_1^k, x_2^k}^2 }$ metrizes the product space $\brac{\Real^{2n}, \Tcal_{\Real^{2n}}}$.Given the connection between $d_{\Cplx}$ and $d_{\Real^2}$ and the common structure pattern of these usual metrics it is possible to see that for any $z, z'\in \Cplx^n$ \[d_{\Cplx^n}\brac{z, z'} = d_{\Real^{2n}}\brac{ \phi\brac{z}, \phi\brac{z'} }\]

This last equality implies by the $\epsilon$-$\delta$ definition of continuity that $\phi$ is $\Tcal_{\Cplx^n}$-$\Tcal_{\Real^{2n}}$ continuous. Let $x, x'\in \Real^{2n}$, then \[d_{\Cplx^n}\brac{\phi^{-1}\brac{x}, \phi^{-1}\brac{x'}} = d_{\Real^{2n}}\brac{ x, x' }\] whence $\phi^{-1}:\Real^{2n}\to \Cplx^n$ is continuous. Thus $\phi$ defines a homeomorphism between the spaces $\Real^{2n}$ and $\Cplx^n$.

For any compact subset $K\subseteq \Cplx^n$, theorem 8-8 implies that $\induc{\phi}_K\brac{K}$ is compact, since $\induc{\phi}_K$ is continuous by theorem Sup-A-4. Therefore, $\phi\brac{K}$ is a compact subset of $\Real^{2n}$. Conversely, since $\phi^{-1}$ is continuous, if $\phi\brac{K}$ is compact, then by the same logic $K=\phi^{-1}\brac{\phi\brac{K}}$ must be compact in $\Cplx^n$.

Since $\phi$ and $\phi^{-1}$ are both continuous, by theorem Sup-A-1 $K\subseteq \Cplx^n$ is closed if and only if $\phi\brac{K}$ is closed in $\Real^{2n}$.

Since for any $x,x'\in \phi\brac{K}$ their associated pre-images $z,z'\in K$ satisfy $d_{\Cplx^n}\brac{ z, z' }=d_{\Real^{2n}}\brac{ x, x' }$, it must be true, that $d_{\Real^{2n}}\brac{ x, x' }\leq \delta\brac{K}$ for all $x,x'\in \phi\brac{K}$, whence $\delta\brac{\phi\brac{K}}\leq \delta\brac{K}$. Conversely, since $d_{\Cplx^n}\brac{z, z'} = d_{\Real^{2n}}\brac{ \phi\brac{z}, \phi\brac{z'} }$ for any $z,z'\in K$, it must be true that $d_{\Cplx^n}\brac{z, z'}\leq \delta\brac{\phi\brac{K}}$ for all $z,z'\in K$, whence $\delta\brac{\phi\brac{K}}=\delta\brac{K}$ for any $K\subseteq \Cplx^{2n}$. Therefore $K\subseteq \Cplx^n$ is bounded with respect to $d_{\Cplx^n}$ if and only if $\phi\brac{K}$ is bounded with respect to $d_{\Real^{2n}}$.

The last three statements and theorem 48 imply that $K\subseteq \Cplx^n$ is compact if and only if $K$ is closed in $\Cplx^n$ and bounded with respect to the usual metric on $\Cplx^n$.

\label{thm:real_closed_in_cplx} \noindent \textbf{Theorem} 10-7.
The set $\Real^n$ is closed in $\Cplx^n$ with respect to the usual metric on $\Cplx^n$.

Let $d_{\Cplx^n}\brac{x,y}\defn \brac{ \sum_{i=1}^n \abs{x_i-y_i}^2 }^\frac{1}{2}$ be the usual metric of $\Cplx^n$. The usual metric on $\Real^n$ is $\induc{d_{\Cplx^n}}_{\Real^n}$.

Suppose $z\notin \Real^n$ yet in $\Cplx^n$. Then there must be some $i = 1\ldots n$ such that $\im z_i \neq 0$. Let $\epsilon\defn \frac{\abs{\im z_i}}{2}>0$ and consider an open ball in $\Cplx^n$ around $z$. If $\Real^n\cap B_{\Cplx^n}\brac{z, \epsilon}\neq \emptyset$ then there is $x\in \Cplx^n$ with $d_{\Cplx^n}\brac{x,z}<\epsilon$ such that $\im x_i = 0$. However \[\abs{\im z_i-\im x_i} \leq \abs{z_i-x_i}\leq d_{\Cplx^n}\brac{z,x}\] whence $2 \epsilon = \abs{\im z_i} \leq d\brac{z,x} < \epsilon$. Thus $B_{\Cplx^n}\brac{z, \epsilon}\cap \Real^n=\emptyset$ and so $z\notin \clo{\Real^n}$. Therefore $\Real^n$ must be closed in $\Cplx^n$.\\

\label{thm:cplx_real_spaces_complete} \noindent \textbf{Theorem} 49.
The spaces $\Cplx^n$ and $\Real^n$, $n\geq 1$, are complete with respect to their usual metrics.

Let $\brac{z_k}_{k\geq 1} \in \Cplx^n$, $n\geq 1$, be a Cauchy sequence with respect to $d\brac{z,z'}\defn \nrm{z-z'}$, where \[\nrm{z}\defn \brac{ \sum_{i=1}^n \abs{z_i}^2 }^\frac{1}{2}\]

Since $\brac{z_k}_{k\geq 1}\in \Cplx^n$ is Cauchy, there exists $N\geq 1$ such that $d\brac{z_k,z_m} < 1$ for all $n,m\geq N$. In particular, $d\brac{z, z'}\leq \max\obj{1, \max_{N\geq k,m }\obj{d\brac{z_m, z_k} } }<+\infty$ for all $z,z'\in Z\defn \obj{\induc{z_k} k\geq 1}$, because there is only finitely many elements of $z_k$ with $k\leq N$. Thus there is $M\in \Real^+$ such that $d\brac{z_k, z_m}\leq M$ for all $k,m\geq 1$. Therefore $\delta\brac{Z}$ is finite and the sequence is bounded with respect to the metric $d$.

If $B\defn \obj{\induc{z\in \Cplx^n} \nrm{z}\leq M}$, then $\delta\brac{B}\leq 2M < +\infty$, since $d\brac{z,z'}\leq \nrm{z'} + \nrm{z}\leq 2M$, and whenever $z\notin B$, $\nrm{z}>M$, whence $B'\defn B_{\Cplx^n}^d\brac{z, \nrm{z}-M}$ is an open set in $\Cplx^n$ such that $B' \cap B = \emptyset$. Indeed, 
$\nrm{z'}\leq M$ and, since $\nrm{z}\leq \nrm{z-z'}+\nrm{z'}$ implies that $M < \nrm{z}-\nrm{z-z'}\leq \nrm{z'}$. Thus $B$ is closed in $\Cplx^n$.

Since $B$ is closed and bounded in $\Cplx^n$, by theorem 48 it is compact and, since $B$ as a subspace $\Cplx^n$ is metrizable by $d$, theorem 47 implies that every sequence has a convergent sub-sequence. Hence there exists $\brac{z_{p_k}}_{k\geq 1}$ and $z\in B$ such that $z_{p_k}\overset{\Tcal_B}{\to} z$, which is equivalent to $z_{p_k}\overset{\Tcal_{\Cplx^n}}{\to} z$. Thus by applying theorem 9-14 to this setting, which readily generalises to arbitrary metric spaces, the sequence $\brac{z_k}_{k\geq 1}$ converges in $\Cplx^n$ to $z$.

Since every sequence can be enclosed in a closed bounded ball in $\Cplx^n$ by the above argument there must be $z\in \Cplx^n$ with $z_k\overset{\Tcal_{\Cplx^n}}{\to} z$. Thus, the space $\Cplx^n$ is complete with respect to its usual metric. Thus theorem 9-16, which hinged upon completeness of $\Cplx$, is indeed true.

If $\brac{x_k}_{k\geq 1}\in \Real^n$ is Cauchy with respect to the usual metric, then it is Cauchy in $\Cplx^n$. Completeness of $\Cplx^n$ implies that there is $z\in \Cplx^n$ such that $x_k\overset{\Tcal_{\Cplx^n}}{\to} z$, whence the fact that by theorem 10-7 $\Real^n$ is closed in $\Cplx^n$ implies that $z\in \Real^n$. Thus the space $\Real^n$ is complete with respect to its usual metric.\\

\label{thm:metric_closed_sequence} \noindent \textbf{Theorem} 10-8.
Let $\brac{E, d}$ be a metric space with metric topology $\Tcal$. Let $F\subseteq E$. Then $F$ is closed in $\brac{E, \Tcal}$ if and only if for all $\brac{x_k}_{k\geq 1}\in F$ with $x_k\overset{\Tcal}{\to}x$ for some $x\in E$ it has to be that $x\in F$. The necessary condition holds true in general topological spaces.

Indeed, suppose $F$ is a closed subset of $E$. If $\brac{x_k}_{k\geq 1}\in F$ is such that $x_k\overset{\Tcal}{\to}x$ for some $x\in E$, then for every $U\in \Tcal$ with $x\in V$ there is $K\geq 1$ such that $x_k\in V$ for all $k\geq K$. Thus $V\cap F\neq \emptyset$ and $x\in \clo{F} = F$.

Suppose $F$ is such that for every $\brac{x_k}_{k\geq 1}\in F$ and $x\in E$ the fact that $x_k\overset{\Tcal}{\to}x$ implies $x\in F$. Then for all $x\in \clo{F}$, since $B_E^d\brac{x,\frac{1}{n}}$ is open in $\brac{E,\Tcal}$, it has to be that $B_E^d\brac{x,\frac{1}{n}}\cap F \neq \emptyset$ for any $n\geq 1$, whence there exists $\brac{x_k}_{k\geq 1}$ such that $x_k\overset{\Tcal}{\to} x$. Therefore $x\in F$ and so $F$ must be a closed subset of $E$.\\

By theorem 12 the space $\brac{F,\induc{\Tcal}_F}$ is metrizable by the induced metric $d_F\defn \induc{d}_{F\times F}$. Suppose $\brac{F, d_F}$ is complete and let $\brac{x_k}_{k\geq 1}\in F$ and $x\in E$ be such that $x_k\overset{\Tcal}{\to}x$. Then $\brac{x_k}_{k\geq 1}$ is Cauchy in $\brac{E, d}$, whence it also must be Cauchy with respect to $d_F$. Since $F$ is complete, $x\in F$ and $x_k\overset{\induc{\Tcal}_F}{\to}x$, and therefore $F$ must be closed in $E$ by theorems 9-9 and 10-8.

The usual topology on $\Rbar$, in fact the whole space $\brac{\Rbar,\Tcal_\Rbar}$ is constructed as a topological space on $\Rbar$ homeomorphic to the topological subspace $\brac{\clo{-1,1},\Tcal_{\clo{-1,1}}}$ of $\brac{\Real, \Tcal_\Real}$. Theorem 13 establishes that $\Rbar$ is metrizable. Indeed for  an arbitrary homeomorphism $h:\Rbar\to \clo{-1,1}$ the metric $d_\Rbar\brac{x,y}\defn \abs{h\brac{x}-h\brac{y}}$ induces $\Tcal_\Rbar$. Theorem 13 also implies that $\Tcal_\Real = \induc{\brac{\Tcal_\Rbar}}_{\Real}$, whence by theorem 12 the metric $d'\defn \induc{d_\Rbar}_{\Real\times \Real}$ induces the usual topology on $\Real$. However, the usual metric $d_\Real$ on $\Real$ also induces the usual topology $\Tcal_\Real$, whence $\Tcal_\Real^{d_\Real} = \Tcal_\Real^{d'}$.

Furthermore, by the same theorem 12 the metric $d_{\clo{-1,1}}\defn \induc{d_\Real}_{\clo{-1,1}\times \clo{-1,1}}$ induces the usual topology $\Tcal_{\clo{-1,1}}$ on $\clo{-1,1}$. The set $\obj{-1,1}$ is not open in $\clo{-1,1}$ since for none of its elements $x$ there is an $\epsilon>0$ with $B_{\clo{-1,1}}^{d_{\clo{-1,1}}}\brac{x,\epsilon}\subseteq \obj{-1,1}$. Thus $\obj{-\infty, +\infty} = h^{-1}\brac{\obj{-1,1}}$ is not open in $\Rbar$, whence $\Real = \Rbar\setminus\obj{-\infty, +\infty}$ cannot be closed in $\Rbar$.

Though $\Real$ is complete with respect to $d_\Real$, by the follow up to theorem 10-8 it cannot be complete with respect to the metric $\induc{d_\Rbar}_{\Real\times \Real}$ induced by a metric on $\Rbar$.

\noindent \textbf{Definition} 81.
Let $\Hcal$ be a $K$-vector space, where $K=\Real$ or $\Cplx$. The inner product on $\Hcal$ is any map $\brkt{\cdot,\cdot}:\Hcal\times\Hcal\to K$ with the following properties:\begin{itemize}
	\item $\brkt{x,y} = \bar{ \brkt{y,x} }$ for all $x,y\in \Hcal$
	\item $\brkt{x+z,y} = \brkt{x,y} + \brkt{z,y}$ for all $x,y,z\in \Hcal$
	\item $\brkt{\alpha x,y} = \alpha \brkt{x,y}$ for all $x,y\in \Hcal$ and $\alpha\in K$
	\item $\brkt{x,x}\geq 0$ for all $x\in \Hcal$
	\item $\brkt{x,x}=0$ if and only if $x=0_\Hcal$.
\end{itemize}
For all $x\in \Hcal$ the norm of $x$ is defined by $\nrm{x}\defn \sqrt{\brkt{x,x}}$.

\label{thm:cauchy_schwartz_inner_prod} \noindent \textbf{Theorem} 50.
Let $\Hcal$ be a $K$-vector space, where $K=\Real$ or $\Cplx$ and $\brkt{\cdot,\cdot}$ be the inner product on $\Hcal$. Then for all $x,y\in \Hcal$ it is true that \[\abs{\brkt{x,y}} \leq \nrm{x} \nrm{y}\]

Indeed, let $x,y\in\Hcal$ and $A\defn \nrm{x}^2$, $B\defn \abs{\brkt{x,y}}$ and $C\defn \nrm{y}^2$. Let $\alpha\in K$ be such that $\abs{\alpha}=1$ and $\alpha \bar{\brkt{x,y}} = B$. For instance $\alpha\defn \frac{z}{\abs{z}}$ is such that $\abs{\alpha} = 1$ and $\alpha \bar{z} = \abs{z}$ for any $z\in \Cplx$. All quantities $A,B$ and $C$ are non-negative by the basic properties of the inner product the complex modulus.

For any $t\in \Real$, $x-t\alpha y\in \Hcal$ and \[\nrm{x-t\alpha y} =\brkt{x-t\alpha y, x-t\alpha y} = \brkt{x,x} -t\alpha \brkt{y,x} + t\bar{\alpha}\brkt{x,y} + t^2 \alpha \bar{\alpha} \brkt{y,y}\] which is equal to $A - t B - t\bar{B} + t^2 C$. Thus if $C=0$ then $y=0_\Hcal$ and so $A = A - 2 t B$, whence $B=0$ and $B^2\leq AC$. Suppose $C>0$. By the properties of the inner product, $\nrm{x-t\alpha y}\geq 0$ for all $t$, whence the quadratic cannot have two distinct real roots implying that the discriminant $4B^2-4AC$ is at most zero. Hence in this case as well $B^2\leq AC$. Thus $\abs{\brkt{x,y}}^2 \leq \nrm{x}^2\nrm{y}^2$, from where the conclusion readily follows.\\

\label{thm:cauchy_schwartz_X} \noindent \textbf{Theorem} 10-9 (Extended Cauchy-Schwartz inequality).
Let $\brac{\Omega, \Fcal, \mu}$ be a measure space. For all $f,g\in L^2_\Cplx\brac{\Omega, \Fcal, \mu}$ define \[\brkt{f,g} \defn \int_\Omega f\bar{g} d\mu\]

Since $\abs{f\bar{g}}^2 = f\bar{g}\bar{\brac{f\bar{g}}} = f\bar{g} \bar{f} g = \abs{f}^2 \abs{g}^2$ by the first Cauchy-Schwartz inequality (theorem 42) \[\int_\Omega \abs{f\bar{g}} d\mu = \int_\Omega \abs{f}\abs{g} d\mu\leq \brac{\int_\Omega \abs{f}^2 d\mu}^\frac{1}{2} \brac{\int_\Omega \abs{g}^2 d\mu}^\frac{1}{2} \] whence $\nrm{f\bar{g}}_1\leq \nrm{f}_2 \nrm{g}_2 < +\infty$ and $f\bar{g}\in L^1_\Cplx\brac{\Omega, \Fcal, \mu}$. Therefore $\brkt{f,g}$ is a well-defined Complex number. Furthermore since $f\bar{f}=\abs{f}^2$ it is true that $\brac{\nrm{f}_2}^2 = \brkt{f,f}$.

Let $f,g\in L^2_\Cplx$. Then by theorem 24 and the Cauchy-Schwartz inequality for the usual Lebesgue integrals \[\abs{\int_\Omega f\bar{g} d\mu} \leq \int_\Omega \abs{f\bar{g}}d\mu\leq \brac{\int_\Omega \abs{f}^2 d\mu}^\frac{1}{2} \brac{\int_\Omega \abs{g}^2 d\mu}^\frac{1}{2}\] Therefore $\abs{\brkt{f,g}}\leq \nrm{f}_2 \nrm{g}_2$.

%% to theorem 5-11...
The map $\brkt{\cdot, \cdot}$ behaves mostly like an inner product. Indeed, by definition of the complex conjugate and the complex Lebesgue integral for any $h\in L^1_\Cplx$ \[\overline{\brac{\int h d\mu}} = \int \re h d\mu - i \int \im h d\mu = \int \re h d\mu + i \int \brac{-\im h} d\mu = \int \bar{h} d\mu\] since $\im \bar{h} = -\im h$ and $\re \bar{h} = \re h$. Therefore \[\overline{\brkt{g,f}} = \overline{\brac{\int g\bar{f} d\mu}} = \int \overline{g\bar{f}} d\mu = \int \bar{g}f d\mu = \brkt{f,g}\] for any $f,g\in L^2_\Cplx$.

Furthermore, for any $f,g,h\in L^2_\Cplx$ and $\alpha\in \Cplx$ the fact that $L^p_\Cplx$ is closed with respect to $\Cplx$-linear combinations implies that $\brac{f+\alpha h}\bar{g}\in L^1_\Cplx$ whence by linearity of the Complex Lebesgue integral \[\brkt{f+\alpha h,g} = \int \brac{f+\alpha h}\bar{g} d\mu = \int f\bar{g} d\mu +\alpha \int h\bar{g} d\mu = \brkt{f,g}+\alpha \brkt{h,g}\] Finally, $\brkt{f,f} = \int f\bar{f} d\mu = \int \abs{f}^2 d\mu$ is non-negative since $\abs{f}^2$ is non-negative and measurable and the usual Lebesgue integral is used.

The only property violated by this inner product is that $\brkt{x,x}=0$ does not necessarily imply $x=0_\Hcal$, just that $x=0$ $\mu$-a.s., while indeed $\brkt{x,x}=0$ when $x=0\in L^2_\Cplx$.

Let $f,g\in L^2_\Cplx$ and $t\in \Real$. The map $\brac{\abs{f}+t\abs{g}}^2$ is non-negative and measurable for every $t$, and by linearity of signed Lebesgue integral and the fact that on non-negative maps the signed integral coincides with the usual \[\int_\Omega \brac{\abs{f}+t\abs{g}}^2 d\mu = \int_\Omega \abs{f}^2 d\mu + 2 t \int_\Omega \abs{f}\abs{g} d\mu + t^2 \int_\Omega \abs{g}^2 d\mu\] Since the left hand side is non-negative for every $t\in \Real$ the right-hand side quadratic must have non-positive discriminant:\[ 4 \brac{\int_\Omega \abs{f}\abs{g} d\mu }^2 \leq 4 \int_\Omega \abs{f}^2 d\mu \int_\Omega \abs{g}^2 d\mu\]

Finally, let $f,g:\brac{\Omega, \Fcal}\to \Zinf$ be non-negative and measurable and suppose that $\int f^2 d\mu, \int g^2 d\mu<+\infty$. Then $f<+\infty$ $\mu$-a.s. and $g<+\infty$ $\mu$-a.s., whence there is $N\in \Fcal$ with $\mu\brac{n}=0$ such that $\brac{f 1_N}\brac{\Omega}$ and $\brac{g 1_N}\brac{\Omega}$ are subsets of $\Real^+$. Furthermore, $\int_\Omega \abs{f 1_N}^2 d\mu= \int_\Omega f^2 d\mu < +\infty$ and $\int_\Omega \abs{g 1_N}^2 d\mu < +\infty$. Therefore there are $\Real^+$ valued maps $f',g'\in L^2_\Cplx$ such that $f=f'$ and $g=g'$ $\mu$-a.s. Thus using the above inequality \[ \int_\Omega fg d\mu = \int_\Omega f'g' d\mu = \int_\Omega \abs{f'}\abs{g'} d\mu \leq \brac{\int_\Omega \abs{f}^2 d\mu}^\frac{1}{2} \brac{\int_\Omega \abs{g}^2 d\mu}^\frac{1}{2}\]\\

% Triangle inequality 
Let $\Hcal$ be a $K$-vector space, where $K=\Real$ or $\Cplx$, and $\brkt{\cdot,\cdot}$ be an inner product on $\Hcal$. Let pick any $x,y\in \Hcal$. By the basic properties of the inner product \[\brkt{x+y,x+y} = \nrm{x}^2 + \brkt{x,y} + \overline{\brkt{x,y}} + \nrm{y}^2\] Since $\brkt{x,y}+\overline{\brkt{x,y}} = 2 \re \brkt{x,y}$, the inner product Cauchy-Schwartz inequality (theorem 50) implies that $\nrm{x+y}^2 \leq \nrm{x}^2 + 2 \nrm{x}\nrm{y} + \nrm{y}^2$, whence $\nrm{x+y}\leq \nrm{x}+\nrm{y}$.

Define $d_{\brkt{\cdot,\cdot}}\brac{x,y}\defn \nrm{x-y}$. Firstly, by the above inequality for norms for any $x,y,z\in \Hcal$ it is true that \[d_{\brkt{\cdot,\cdot}}\brac{x,z} = \nrm{x-z}\leq \nrm{x-y}+\nrm{y-z} = d_{\brkt{\cdot,\cdot}}\brac{x,y} + d_{\brkt{\cdot,\cdot}}\brac{y,z}\] Secondly, $d_{\brkt{\cdot,\cdot}}\brac{x,y} = 0$ implies that $x=y$ since $\brkt{x-y,x-y}=0$ implies that $x-y=0_\Hcal$. And finally, $d_{\brkt{\cdot,\cdot}}\brac{x,y} = \nrm{x-y} = \nrm{y-x} = d_{\brkt{\cdot,\cdot}}\brac{y,x}$. Therefore the map $d_{\brkt{\cdot,\cdot}}:\Hcal\times\Hcal\to \Real^+$ is a metric on $\Hcal$.

\noindent \textbf{Definition} 82.
Let $\Hcal$ be a $K$-vector space, where $K=\Real$ or $\Cplx$, and $\brkt{\cdot,\cdot}$ be an inner product on $\Hcal$. The \textbf{norm} topology on $\Hcal$, $\Tcal_{\brkt{\cdot,\cdot}}$, is defined as the metric topology on $\Hcal$ induced by the metric $d_{\brkt{\cdot,\cdot}}\brac{x,y}\defn\nrm{x-y}$, i.e. $\Tcal_{\brkt{\cdot,\cdot}} \defn \Tcal_\Hcal^{d_{\brkt{\cdot,\cdot}}}$.

\noindent \textbf{Definition} 83.
A \textbf{Hilbert} space over $K$, where $K=\Real$ or $\Cplx$, is any pair $\brac{\Hcal, \brkt{\cdot,\cdot}}$ where $\brkt{\cdot,\cdot}$ is an inner product on a $K$-vector space $\Hcal$, which is complete with respect to the metric $d_{\brkt{\cdot,\cdot}}\brac{x,y}\defn\nrm{x-y}$.

\label{thm:Hilbert_subspace} \noindent \textbf{Theorem} 10-10.
Let $\brac{\Hcal, \brkt{\cdot,\cdot}}$ be a Hilbert space over $K$ and $\Mcal$ be a linear subspace of $\Hcal$. If $\Mcal$ is closed with respect to the norm topology $\Tcal_{\brkt{\cdot,\cdot}}$, then $\brac{\Mcal, \clo{\cdot,\cdot}}$ is a Hilbert (sub-)space.

For notational brevity denote $\clo{\cdot,\cdot}\defn \induc{\brkt{\cdot,\cdot}}_{\Mcal\times \Mcal}$. This $\clo{\cdot,\cdot}$ is an inner product on the $K$-vector space $\Mcal$, since $\Mcal$ is a linear subspace of $\Hcal$ and $\clo{\cdot,\cdot}$ inherits all properties of $\brkt{\cdot,\cdot}$. Furthermore, the metric $d_{\clo{\cdot,\cdot}}$ on $\Mcal$ constructed from $\clo{\cdot,\cdot}$ is in fact induced by $d_{\brkt{\cdot,\cdot}}$. Thus by theorem 12, $\Tcal_{\clo{\cdot,\cdot}} = \induc{\Tcal_{\brkt{\cdot,\cdot}}}_{\Mcal}$.

Further, if $\brac{x_n}_{n\geq 1}\in \Mcal$ is a Cauchy sequence with respect to the metric $d_{\clo{\cdot, \cdot}}$, then it is Cauchy with respect to $d_{\brkt{\cdot, \cdot}}$ in $\Hcal$, since $d_{\clo{\cdot,\cdot}}\defn \induc{d_{\brkt{\cdot, \cdot}}}_{\Mcal\times \Mcal}$. Thus by completeness of any Hilbert space there must exist $x\in \Hcal$ such that $x_n\overset{\Tcal_{\brkt{\cdot,\cdot}}}{\to}x$. Since $\Mcal$ is closed, $x\in \Mcal$ by theorem 10-8 and by theorem 9-9 $x_n\overset{\Tcal_{\clo{\cdot,\cdot}}}{\to}x$. Therefore the space $\brac{\Mcal, \clo{\cdot,\cdot}}$ is a Hilbert space over $K$, as the linear space $\Mcal$ is complete with respect to $d_{\clo{\cdot, \cdot}}$ and $\clo{\cdot, \cdot}$ is an inner product on $\Mcal$.\\

% show that it is an inner product on $\Cplx^n$.
For all $z,z'\in \Cplx^n$, $n\geq 1$, define \[\brkt{z,z'}\defn \sum_{k=1}^n z_k \bar{z}_k'\] First, due to the properties of complex conjugation, $\overline{\brkt{z',z}} = \overline{\sum_{k=1}^n z_k' \bar{z}_k} = \sum_{k=1}^n z_k \bar{z}_k' = \brkt{z,z'}$. Furthermore, since summation is linear over $\Cplx$ and multiplication in the field $\Cplx$ is distributive, $\brkt{z+\alpha z', z''} = \brkt{z,z''}+\alpha \brkt{z',z''}$. Finally, $\brkt{z,z} = \sum_{k=1}^n z_k \bar{z}_k = \sum_{k=1}^n \abs{z_k}^2 \geq 0$ and if $\brkt{z,z} = 0$, then $\abs{z_k}=0$ for all $k=1\ldots n$, whence $z = 0\in \Cplx^n$ Conversely, if $z=0$, then $\brkt{z,z} = \sum_{k=1}^n 0 = 0$. Therefore, $\brkt{z,z'}$ is an inner product on $\Cplx^n$.

\noindent \textbf{Definition} 84.
The usual inner product in $K^n$, where $K=\Real$ or $\Cplx$,  is the inner product $\brkt{\cdot,\cdot}$ defined for every $x,y\in K^n$ by \[\brkt{x,y}\defn \sum_{i=1}^n x_i \bar{y}_i\]

\label{thm:real_complex_hilbert_spaces} \noindent \textbf{Theorem} 51.
The vector spaces $\Cplx^n$ and $\Real^n$ together with their usual inner products are Hilbert spaces over $\Cplx$ and $\Real$ respectively.

The norm on the inner product space $\brac{\Cplx^n, \brkt{\cdot,\cdot}}$ is $\nrm{z}\defn \brkt{z,z} = \brac{\sum_{k=1}^n \abs{z_k}^2}^\frac{1}{2}$. Now the usual metric on $\Cplx^n$ is \[d_{\Cplx^n}\brac{z,z'} = \brac{\sum_{k=1}^n \abs{z_k-z_k'}^2}^\frac{1}{2}\] whence $\nrm{z-z'} = d_{\brkt{\cdot, \cdot}}\brac{z,z'}$. Since by theorem 49 $\Cplx^n$ is complete with respect to its usual metric $d_{\Cplx^n}$ and the norm metric $d_{\brkt{\cdot, \cdot}}\brac{z,z'}$ is equal to the usual metric of $\Cplx^n$, the space $\brac{\Cplx^n, \brkt{\cdot,\cdot}}$ is nothing but a Hilbert space over $\Cplx$.

By theorem 10-7 $\Real^n$ is closed in $\Cplx^n$ with respect to the usual metric in $\Cplx^n$, but is not a linear space over $\Cplx$ and thus theorem 10-10 cannot be applied. Nevertheless $\Real^n$ is complete due theorem 49, and $\induc{\brkt{\cdot,\cdot}}_{\Real^n\times \Real^n}$ is an inner product on a $\Real$-vector space $\Real^n$. Therefore $\brac{\Real^n,\induc{\brkt{\cdot,\cdot}}_{\Real^n\times \Real^n}}$ is indeed a Hilbert space.\\

\noindent \textbf{Definition} 85.
Let $\Hcal$ be a $K$-vector space, where $K=\Real$ or $\Cplx$. Let $\Ccal\subseteq \Hcal$. The set $\Ccal$ is a convex subset of $\Hcal$ if for all $x,y\in \Hcal$ and $t\in \clo{0,1}$ it is true that $tx + \brac{1-t}y\in \Ccal$.

\label{thm:inner_product_properties} \noindent \textbf{Theorem} 10-11.
Let $\Hcal$ be a $K$-vector space, where $K=\Real$ or $\Cplx$, and $\brkt{\cdot,\cdot}$ be an inner product on $\Hcal$.

For any $x,y\in \Hcal$ \[\nrm{x-y}^2 = \nrm{x}^2 - \brkt{x,y} - \brkt{y,x} + \nrm{y}^2\] and \[\nrm{x+y}^2 = \nrm{x}^2 + \brkt{x,y} + \brkt{y,x} + \nrm{y}^2\] Therefore for any $x,y\in \Hcal$ \[\nrm{x+y}^2+\nrm{x-y}^2 = 2\nrm{x}^2 + 2\nrm{y}^2 \] whence \[\nrm{x-y}^2 = 2\nrm{x}^2 + 2\nrm{y}^2 - 4\nrm{\frac{x+y}{2}}^2\]

Now $\nrm{x} = \nrm{\brac{x-y}+y}\leq \nrm{x-y}+\nrm{y}$ by a corollary to theorem 50. Therefore, obtaining symmetric expression for $y$ gives \[\abs{\nrm{x}-\nrm{y}}\leq \nrm{x-y}\]\\

\label{thm:convex_projection} \noindent \textbf{Theorem} 52.
Let $\brac{\Hcal, \brkt{\cdot, \cdot}}$ be a Hilbert space over $K=\Real$ or $\Cplx$. Let $\Ccal$ be a non-empty, closed and convex subset of $\Hcal$. For all $x_0\in \Hcal$ there exists a unique $x^*\in \Ccal$ such that \[\nrm{x^*-x_0} = \inf\obj{ \induc{ \nrm{x - x_0} } x\in \Ccal }\]

Indeed, let $x_0\in \Hcal$ \[\delta_{\text{min}}\defn \inf\obj{ \induc{ \nrm{x - x_0} x\in \Ccal } }\] Since $\Ccal\neq \emptyset$ and for any $x\in \Hcal$ the norm $0\leq \nrm{x}<+\infty$, $\delta_{\text{min}}\in \Real^+$. Thus for all $n\geq 1$ there is $x_n\in \Ccal$ such that $\delta_{\text{min}}\leq \nrm{x_n-x_0}<\delta_{\text{min}}+\frac{1}{2^n}$. For any $\epsilon>0$ if $n\geq 1+\floor{-\log_2{\epsilon}}$ then \[0\leq \nrm{x_n-x_0}-\delta_{\text{min}}<\frac{1}{2^n}\leq\epsilon\] Therefore the sequence $\brac{x_n}_{n\geq 1}\in \Ccal$ is such that $\nrm{x_n-x_0}\to \delta_{\text{min}}$ in $\Real$.

Since $\phi^\alpha$ is continuous for every $\alpha > 0$ and by theorem 9-9 $\nrm{x_n-x_0}\overset{\Real}{\to}\delta_{\text{min}}$ is equivalent to $\nrm{x_n-x_0}\overset{\Real^+}{\to}\delta_{\text{min}}$, it is true that $\nrm{x_n-x_0}^2\overset{\Real}{\to}\delta_{\text{min}}^2$.

By theorem 10-11 \[\nrm{x_n-x_m}^2 = 2\nrm{x_n-x_0}^2 + 2\nrm{x_m-x_0}^2 - 4\nrm{\frac{x_n+x_m}{2}-x_0}^2\] Since $\Ccal$ is a convex subset of $\Hcal$ which implies that $\frac{x_n+x_m}{2} = \frac{1}{2}x_n + \frac{1}{2}x_m \in \Ccal$ for any $n,m\geq 1$ as $\brac{x_n}_{n\geq 1}\in \Ccal$. Thus $\delta_{\text{min}}\leq \nrm{\frac{x_n+x_m}{2} - x_0}$ for all $n,m \geq 1$.
Therefore for all $n,m\geq 1$ \[\nrm{x_n-x_m}^2\leq 2\nrm{x_n-x_0}^2 + 2\nrm{x_m-x_0}^2 - 4\delta_{\text{min}}^2\] For any $\epsilon>0$ there is $N\geq 1$ such that $\nrm{x_n-x_m} < 2\frac{2 \epsilon}{4}$ for all $n,m\geq 1$. Therefore the sequence $\brac{x_n}_{n\geq 1}\in \Ccal\subseteq \Hcal$ is Cauchy with respect to the metric $d_{\brkt{\cdot,\cdot}}$.

Now, $\brac{\Hcal, \brkt{\cdot, \cdot}}$ is a Hilbert space over $K$, whence there is $x^*\in \Hcal$ such that $x_n\overset{\Tcal_{\brkt{\cdot,\cdot}}}{\to}x^*$. By theorem 10-8 $x^*\in \clo{\Ccal}$ which implies that $x^*\in \Ccal$ as $\Ccal$ is closed in $\brac{\Hcal, \Tcal_{\brkt{\cdot,\cdot}}}$. Since $\abs{\nrm{x^*-x_0} - \nrm{x_n-x_0}}\leq \nrm{x_n-x^*}$, it must be true that $\nrm{x_n-x_0}\to\nrm{x^*-x_0}$ in $\Real$. Since $\Real$ is a Hausdorff space and $\nrm{x_n-x_0}\to\delta_{\text{min}}$ it is true that there is $x^*\in \Ccal$ such that \[\nrm{x^*-x_0}=\inf\obj{\induc{\nrm{x-x_0}} x\in \Ccal}\] If $y^*\in \Ccal$ is another element with such property, then \[\nrm{x^*-y^*} = 2\nrm{x^*-x_0}^2 + 2\nrm{y^*-x_0}^2 - 4\nrm{\frac{x^*+y^*}{2}-x_0}^2 \leq 2\nrm{x^*-x_0}^2 + 2\nrm{y^*-x_0}^2 - 4\delta_{\text{min}} = 0\] Therefore $x^*=y^*$ and thus there exists a unique projection $x^*$ of any $x_0\in \Hcal$ onto a non-empty closed convex subset $\Ccal$ such that \[\nrm{x^*-x_0} = \inf\obj{ \induc{ \nrm{x - x_0} } x\in \Ccal }\]\\

\label{thm:inner_product_map} \noindent \textbf{Theorem} 10-12.
Let $\Hcal$ be a $K$-vector space, where $K=\Real$ or $\Cplx$, and $\brkt{\cdot,\cdot}$ be an inner product on $\Hcal$. For every $y\in \Hcal$ the map $x\to \brkt{x,y}$ on $\Hcal$ is linear and for every $x\in \Hcal$ the map $y\to\brkt{x,y}$ is conjugate-linear. Furthermore both are $\Tcal_{\brkt{\cdot,\cdot}}$-$\Tcal_K$ continuous.

Let $y\in \Hcal$ be some fixed vector and put $\phi_y\brac{x}\defn \brkt{x,y}$. Then for any $x\in \Hcal$ and $\alpha\in K$ by the basic properties of the inner product it must be true that $\phi_y\brac{x+z} = \brkt{x+z,y} = \brkt{x,y}+\brkt{x,y}=\phi_y\brac{x}+\phi_y\brac{z}$ and $\phi_y\brac{\alpha x} = \brkt{\alpha x,y} = \alpha \brkt{x,y} = \alpha \phi_y\brac{x}$. Therefore the map $x\to \brkt{x,y}$ on $\Hcal$ is linear for any $y\in \Hcal$.

For any $x,x'\in \Hcal$ linearity and the Cauchy-Schwartz inequality (theorem 50) imply \[\abs{\phi_y\brac{x}-\phi_y\brac{x'}} = \abs{\phi_y\brac{x-x'}} = \abs{\brkt{x-z,y}}\leq \nrm{x-z} \nrm{y}\] If $y=0_\Hcal$ then $\phi_y\brac{x}=0$ for all $x\in \Hcal$ implying that $\phi_y$ is continuous by theorem Sup-A-4. If $y\neq 0_\Hcal$ and let $x\in \Hcal$ and $\epsilon>0$. Since $\nrm{y}\neq0$, $\delta \defn \frac{\epsilon}{\nrm{y}}>0$ is such that $\abs{\phi_y\brac{x}-\phi_y\brac{z}}<\epsilon$ whenever $\nrm{x-z}<\delta$. Therefore for every $y\in \Hcal$ the map $\phi_y:\Hcal\to K$ is $\Tcal_{\brkt{\cdot,\cdot}}$-$\Tcal_K$ continuous.

Similarly, by the properties of the inner product and the complex conjugate for any $x\in \Hcal$ the map $y\to\brkt{x,y}$ is linear if $K=\Real$ and conjugate-linear if $K=\Cplx$. And is still $\Tcal_{\brkt{\cdot,\cdot}}$-$\Tcal_K$ continuous.\\

\noindent \textbf{Definition} 86.
Let $\brac{\Hcal, \brkt{\cdot,\cdot}}$ be a Hilbert space over $K$, where $K=\Real$ or $\Cplx$. Let $\mathcal{G}\subseteq \Hcal$. The orthogonal of $\mathcal{G}$ is the subset of $\Hcal$ denoted by $\mathcal{G}^\perp$ and defined by \[\mathcal{G}^\perp \defn \obj{\induc{ x\in \Hcal}\,\brkt{x,y}=0\,\forall y\in \mathcal{G}}\]

If $x,y\in \mathcal{G}^\perp$ then $x,y\in \Hcal$ are such that $\brkt{x,z}=\brkt{y,z}=0$ for all $z\in \mathcal{G}$. Therefore, $\brkt{x+y,z}=0$ for all $z\in \mathcal{G}$ by the basic properties of the inner product. If $\alpha\in K$, then $\brkt{\alpha x, z}=\alpha \brkt{x,z} = 0$ for very $z\in \mathcal{G}$. Thus $\mathcal{G}^\perp$ is a linear subspace of $\Hcal$ even if $\mathcal{G}$ is not.

For every $y\in \Hcal$ define $\phi_y\brac{x}\defn \brkt{x,y}$. By theorem 10-12 this map is continuous for all $y\in \Hcal$. Since $\obj{0}$ is a closed subset of $K$ theorem Sup-A-1 implies that $\phi_y^{-1}\brac{\obj{0}}$ is closed in $\brac{\Hcal, \Tcal_{\brkt{\cdot, \cdot}}}$ for every $y\in \Hcal$. Thus the set $\bigcap_{y\in \mathcal{G}} \phi_y^{-1}\brac{\obj{0}}$ is closed in $\Hcal$.

Now if $x\in \bigcap_{y\in \mathcal{G}} \phi_y^{-1}\brac{\obj{0}}$ then $\phi_y\brac{x} = \brkt{x,y} = 0$ for all $y\in \mathcal{G}$ and so $x\in \mathcal{G}^\perp$. On the other hand, if $x\in \mathcal{G}^\perp$, then $\brkt{x,y}=\phi_y\brac{x} = 0$ for all $y\in \mathcal{G}$, whence $x\in \phi_y^{-1}\brac{\obj{0}}$ for all $y\in \mathcal{G}$. Therefore $\mathcal{G}^\perp = \bigcap_{y\in \mathcal{G}} \phi_y^{-1}\brac{\obj{0}}$ which implies that the orthogonal of $\mathcal{G}$ is closed in $\brac{\Hcal, \Tcal_{\brkt{\cdot, \cdot}}}$ even if $\mathcal{G}$ is not.

Now, $\emptyset^\perp = \Hcal$ since for any $x\in \Hcal$ there is no $y\in \emptyset$ to contradict $\brkt{x,y}=0$. As for $\obj{0_\Hcal}^\perp$, $\brkt{x,0}=0$ for all $x\in \Hcal$, and so $\obj{0_\Hcal}^\perp = \Hcal$.

Finally, $x=0_\Hcal$ is the only element of $\Hcal$ such that $\brkt{x,y}=0$ for all $y\in \Hcal$. Indeed, if there were another analogous element $x$, then since $x\in \Hcal$ it must be true that $\brkt{x,x}=0$ whence $x=0_\Hcal$. Therefore $\Hcal^\perp = \obj{0_\Hcal}$.

Furthermore, if $x\in G\cap \mathcal{G}^\perp$, then $\brkt{x,y}=0$ for all $y\in \mathcal{G}$. In particular, $\brkt{x,x}=0$, which means that $x=0_\Hcal$.\\

\label{thm:subspace_decomposition} \noindent \textbf{Theorem} 53.
Let $\brac{\Hcal, \brkt{\cdot, \cdot}}$ be a Hilbert space over $K=\Real$ or $\Cplx$. Let $\Mcal$ be a closed linear subspace of $\Hcal$. For all $x_0\in \Hcal$ there exists a unique decomposition $x_0 = x' + x^\perp$, where $x'\in \Mcal$ and $x^\perp\in \Mcal^\perp$.

Since $\Mcal$ is a linear subspace of $\Hcal$, $0_\Hcal\in \Mcal$ implying that $\Mcal$ is non-empty. Furthermore, because $\Mcal$ is closed under $K$-linear combinations, it must be convex. Indeed, for any $x,y\in \Mcal$ and every $t\in \clo{0,1}\subseteq K$ it is true that $t x, \brac{1-t} y\in \Mcal$ whence $t x + \brac{1-t} y\in \Mcal$.

Let $x_0\in \Mcal$. By theorem 52 there exists a unique $x'\in \Mcal$ such that \[\nrm{x'-x_0} = \inf\obj{ \induc{\nrm{x - x_0}} x\in \Mcal }\] Since $\Mcal$ is a linear subspace $x'-\alpha x\in \Mcal$ for all $\alpha\in K$ and every $x\in\Mcal$. Therefore if $x^\perp \defn x_0 - x'$ then for all $y\in\Mcal$ and $\alpha\in K$ \[\nrm{x^\perp}^2=\nrm{x'-x_0}^2\leq \nrm{\brac{x' - \alpha y} - x_0}^2 = \nrm{x^\perp - \alpha y}^2\]Basic properties of the inner product imply that $\nrm{x^\perp - \alpha y}^2 = \nrm{x^\perp}^2 - \alpha \brkt{y,x^\perp} - \bar{\alpha} \brkt{x^\perp,y}+\alpha\bar{\alpha}\nrm{y}^2$. Therefore \[0\leq  - \alpha \brkt{y,x^\perp} - \overline{\alpha\brkt{x^\perp,y}}+\abs{\alpha}^2 \nrm{y}^2 = \abs{\alpha}^2 \nrm{y}^2 - 2\re\brac{\alpha \brkt{y,x^\perp}}\]

For all $y\in \Mcal$ with $y\neq 0_\Hcal$ taking $\alpha \defn \frac{\overline{\brkt{y,x^\perp}}}{\nrm{y}^2}$ means that \[\alpha \brkt{y,x^\perp} = \frac{\abs{\brkt{y,x^\perp}}^2}{\nrm{y}^2}\,\text{and}\,\abs{\alpha}^2 \nrm{y}^2 = \nrm{y}^2 \frac{\abs{\brkt{y,x^\perp}}^2}{\brac{\nrm{y}^2}^2} \] Thus this clever choice of $\alpha\in K$ yields \[\frac{\abs{\brkt{y,x^\perp}}^2}{\nrm{y}^2} \leq 0\] which in turn implies that $\brkt{y,x^\perp} = 0$ for such $y\in \Mcal$. If $y=0$, then obviously $\brkt{y,x^\perp}=0$. In conclusion, $\brkt{y,x^\perp}=0$ for all $y\in \Mcal$, whence $x^\perp\in \Mcal^\perp$ and $x_0 = x' + x^\perp$.

Suppose there is another $y^\perp\in \Mcal^\perp$ and $y'\in \Mcal$ such that $x_0 = y' + y^\perp$. Then $x' + x^\perp = x_0 = y' + y^\perp$ implies that $y'-x' = x^\perp - y^\perp$. Since $\Mcal$ and $\Mcal^\perp$ are linear spaces, this observation implies that $w \defn y'-x'$ is such that $w\in \Mcal$ and $w\in \Mcal^\perp$. Since $\Mcal\cap \Mcal^\perp = \obj{0_\Hcal}$, $w=0_\Hcal$ implying that $x' = y'$ and $x^\perp = y^\perp$.

Therefore for every $x_0\in \Hcal$ there is a unique decomposition $x_0 = x' + x^\perp$ where $x'\in \Mcal$ and $x^\perp \in \Mcal^\perp$ all thanks to theorem 52 and basic properties of the inner product.\\

\noindent \textbf{Definition} 87.
Let $\Hcal$ be a $K$-vector space, where $K=\Real$ or $\Cplx$. A linear functional is any map $\lambda:\Hcal\to K$ such that for all $x,y\in \Hcal$ and $\alpha\in K$ it is true that \[\lambda\brac{x+\alpha y} = \lambda\brac{x} + \alpha \lambda\brac{y}\]

\label{thm:lin_func_cont} \noindent \textbf{Theorem} 10-13.
Let $\lambda$ be some linear functional on a $K$-vector space $\Hcal$ with an inner product $\brkt{\cdot,\cdot}$, where $K=\Real$ or $\Cplx$. Then $\lambda$ is continuous $\Tcal_{\brkt{\cdot,\cdot}}$-$\Tcal_K$ if and only if $\lambda$ is a bounded linear functional: there exists $M\in \Real^+$ such that $\abs{\lambda\brac{x}}\leq M\nrm{x}$ for all $x\in \Hcal$.

If $\lambda$ is continuous in $\Hcal$-$K$, then by Sup-A-1 $\lambda$ is continuous at some $x_0\in \Hcal$. Since the topologies are metric, by the $\epsilon$-$\delta$ criterion there is $\delta>0$ such that $\abs{\lambda\brac{x}-\lambda\brac{x_0}}<\frac{1}{2}$ for all $d_{\brkt{\cdot,\cdot}}\brac{x,x_0}<\delta$. Since $\delta>0$, one can always pick $\eta>0$ such that $0<\eta\leq\delta$, whence $\abs{\lambda\brac{x}-\lambda\brac{x_0}}\leq 1$ for all $\nrm{x-x_0}\leq\eta$. As $\Hcal$ is a linear space over $K$ and $\lambda$ is a linear functional, it must be true that $\abs{\lambda\brac{y}}\leq 1$ for all $y\in \Hcal$ with $\nrm{y}\leq\eta$. Put $M\defn \frac{1}{\eta}\in \Real^+$.

Let $x\in \Hcal$. When $x=0_\Hcal$, $\abs{\lambda\brac{x}}=0$ by linearity of $\lambda$. Indeed, $\lambda\brac{0} = \lambda\brac{\alpha 0} = \alpha \lambda\brac{0}$ for all $\alpha\in K$ and in particular for any $\alpha\neq 1$, whence $\abs{\lambda\brac{0}} \leq 0 = M\nrm{x}$.

If $x\neq 0_\Hcal$, then $\frac{1}{\nrm{x}}x=1$ and by the above it is true that \[\abs{\lambda\brac{\frac{\eta}{\nrm{x}}} x }\leq 1\] whence by linearity of $\lambda$ it must be true that $\abs{\lambda\brac{x}}\leq \frac{\nrm{x}}{\eta}$. Thus for all non-trivial $x\in \Hcal$ it is true that $\abs{\lambda\brac{x}}\leq M\nrm{x}$. Therefore if $\lambda$ is continuous, there exists $M\in \Real^+$ such that $\abs{\lambda\brac{x}}\leq M\nrm{x}$ for all $x\in \Hcal$.

Suppose, conversely, that there exists $M\in \Real^+$ such that $\abs{\lambda\brac{x}}\leq M\nrm{x}$ for all $x\in \Hcal$. Let $x_0\in \Hcal$ and $\epsilon>0$. If $M=0$ then $\abs{\lambda\brac{y} - \lambda\brac{x_0}} = \abs{\lambda\brac{x-x_0}}\leq 0$ for all $y\in \Hcal$ whence $\lambda$ is automatically continuous at $x_0$. If $M>0$ then for $\delta\defn \frac{\epsilon}{M}>0$ it is true that for all $y\in \Hcal$ with $\nrm{y-x_0}<\delta$ \[\abs{\lambda\brac{y}-\lambda\brac{x_0}} = \abs{\lambda\brac{y-x_0}} \leq M\nrm{y-x_0}<M \frac{\epsilon}{M}\] Therefore $\lambda$ is continuous at $x_0\in \Hcal$ according to the $\epsilon$-$\delta$ criterion, whence $\lambda$ is indeed $\Tcal_{\brkt{\cdot,\cdot}}$-$\Tcal_K$.\\

\noindent \textbf{Definition} 88.
Let $\brac{\Hcal, \brkt{\cdot,\cdot}}$ be a Hilbert, where $K=\Real$ or $\Cplx$. Let $\lambda:\Hcal\to K$ be a linear functional. Then the following are equivalent (by theorem 10-13):\begin{itemize}
	\item $\lambda:\brac{\Hcal, \Tcal_{\brkt{\cdot,\cdot}}} \to \brac{K,\Tcal_K}$ is continuous
	\item $\exists M\in \Real^+$ such that $\forall x\in \Hcal$ it is true that $\abs{\lambda\brac{x}}\leq M\nrm{x}$
\end{itemize}
If $\lambda$ satisfies any of these equivalent properties, then it is called a bounded linear functional.

\label{thm:lin_func_inner_prod} \noindent \textbf{Theorem} 54.
Let $\brac{\Hcal, \brkt{\cdot, \cdot}}$ be a Hilbert space over $K=\Real$ or $\Cplx$. Let $\lambda$ be a bounded linear functional on $\Hcal$. Then there exists a unique $y\in \Hcal$ such that $\lambda\brac{x}=\brkt{x,y}$ for all $x\in \Hcal$.

Indeed, if $\lambda\brac{x}=0$ for all $x\in\Hcal$ then for $y\defn 0_\Hcal$ the basic properties of the inner product imply that $\brkt{x,y}=0$ for all $x\in\Hcal$, whence $\lambda\brac{x}=\brkt{x,y}$ for all $x\in\Hcal$. Note that no other $y\in\Hcal$ gives $\brkt{x,y}=0$ for all $x\in\Hcal$. Therefore there indeed exists a unique $y\in\Hcal$ such that $\lambda\brac{x}=\brkt{x,y}$ for all $x\in\Hcal$.

Now suppose $\lambda\brac{x_0}\neq 0$ for some $x_0\in\Hcal$. For $\Mcal\defn \lambda^{-1}\brac{\obj{0_\Hcal}}$, obviously, $x_0\notin\Mcal$.

For any $x,y\in\Mcal$ it is true that $\lambda\brac{x}=\lambda\brac{y}=0$, whence from linearity of $\lambda$ follows that $\lambda\brac{x+y}=0$. Furthermore, for any $\alpha \in K$ and $x\in\Mcal$ by linearity $\lambda\brac{\alpha x} = \alpha \lambda\brac{x} = 0$, whence $\alpha x\in\Mcal$. In conclusion, the set $\Mcal$ is a linear subspace of $\Hcal$, because $\lambda\brac{0_\Hcal}=0$ as well. Therefore by theorem 53 for $x_0$ there exist $x'\in\Mcal$ and $x^\perp\in \Mcal^\perp$ such that $x_0=x'+x^\perp$.

Since $x_0\notin\Mcal$, $\lambda\brac{x_0}\neq0$ and by linearity $\lambda\brac{x^\perp}\neq 0$. Thus $x^\perp\notin \Mcal$ and $x^\perp\notin \Mcal\cap \Mcal^\perp$, whence $x^\perp \neq 0_\Hcal$. If $z\defn \frac{x^\perp}{\nrm{x^\perp}}$, then $z\in \Mcal^\perp$, $\lambda\brac{z}\neq 0$ and $\nrm{z}=1$.

Let $x\in \Hcal$ and $\alpha\in K$ with $\alpha\neq 0$. Then $\frac{\lambda\brac{x}}{\bar{\alpha}}\in K$ and \[\frac{\lambda\brac{x}}{\bar{\alpha}} \brkt{z,\alpha z} = \frac{\lambda\brac{x}}{\abs{\alpha}^2}\brkt{\alpha z, \alpha z} = \frac{\lambda\brac{x}}{\abs{\alpha}^2} \abs{\alpha}^2\nrm{z}^2 = \lambda\brac{x}\]

Suppose there exists $\alpha\in K$ with $\alpha\neq 0$ such that $\frac{\lambda\brac{x}}{\bar{\alpha}} z - x \in \Mcal$ for all $x\in \Hcal$. Then for all $x\in\Hcal$ \[\brkt{\frac{\lambda\brac{x}}{\bar{\alpha}} z - x, \alpha z} = \bar{\alpha} \brkt{\frac{\lambda\brac{x}}{\bar{\alpha}} z - x, z} = 0\] since $z\in \Mcal^\perp$. Therefore by the basic properties of the inner product \[0 = \brkt{\frac{\lambda\brac{x}}{\bar{\alpha}} z - x, \alpha z} = \frac{\lambda\brac{x}}{\bar{\alpha}} \brkt{z, \alpha z}-\brkt{x, \alpha z} = \lambda\brac{x}-\brkt{x, \alpha z} \] whence $\lambda\brac{x} = \brkt{x, \alpha z}$ for all $x\in \Hcal$.

Now it is necessary to demonstrate that such $\alpha\in K$, $\alpha\neq 0$ as mentioned above exists. A solution to the equation \[\lambda\brac{ \frac{\lambda\brac{x}}{\bar{\alpha}} z - x } = 0\] provides a candidate for the required $\alpha\in K$. Indeed, due to linearity of $\lambda$ this equation is equivalent to \[\frac{\lambda\brac{x}}{\bar{\alpha}} \lambda\brac{z}=\lambda\brac{x}\] whence $\alpha = \overline{\lambda\brac{z}}$. However, $z\in \Mcal^\perp$ has been chosen, so that $\lambda\brac{z}\neq 0$.

Now, for $y\defn \overline{\lambda\brac{z}} z$, $y\in \Hcal$ is such that $\lambda\brac{x}=\brkt{x,y}$ for all $x\in \Hcal$. Let $y'\in \Hcal$ be another element with $\forall x\in \Hcal\, \lambda\brac{x}=\brkt{x,y'}$. Then $\brkt{x,y-y'}=0$ for all $x\in \Hcal$. Since $y-y'\in \Hcal$ this would implying that $\nrm{y-y'}=0$ whence $y=y'$ by the basic properties of an inner product. Thus such $y\in \Hcal$ is unique.\\

\noindent \textbf{Definition} 89.
%%Consider moving closer to the definition of an inner product...
Let $K=\Real$ or $\Cplx$ or any other field. A $K$-vector space is any set $\Hcal$ together with operators $\oplus$ (addition) and $\otimes$ (scalar multiplication) for which there exists an element $0_\Hcal\in \Hcal$ such that for all $x,y,z\in \Hcal$ and $\alpha, \beta\in K$ it is true \begin{itemize}
	\item Additively neutral element: $0_\Hcal \oplus x = x$
	\item Additive inverse: $\exists \brac{-x}\in \Hcal$ such that $\brac{-x} \oplus x = 0_\Hcal$
	\item Associativity of addition: $x\oplus \brac{y\oplus z} = \brac{x\oplus y}\oplus z$
	\item Commutativity of addition: $x\oplus y = y\oplus x$
	\item $1 \otimes x = x$
	\item $\alpha \otimes \brac{\beta \otimes x} = \brac{\alpha \beta} \otimes x$
	\item $\brac{\alpha + \beta} \otimes x = \alpha \otimes x \oplus \beta \otimes x$
	\item $\alpha \otimes \brac{x\oplus y} = \brac{\alpha \otimes x} \oplus \brac{\alpha \otimes y}$
\end{itemize}

Let $K=\Real$ or $\Cplx$, $p\in \clo{1,+\infty}$. For all $f\in L^P_K\brac{\Omega, \Fcal, \mu}$ define the $\mu$-a.s equivalence class of $f$ as $\clo{f}\defn \obj{\induc{g\in L^p_K}\, f=g\, \mu\text{-a.s}}$. Indeed, the binary relation on $L^p_K$ of $\mu$ almost sure equality is an equivalence relation: it is obviously reflexive and symmetric, and transitive, because if $f=g$ $\mu$-a.s and $g=h$ $\mu$-a.s then there is a $\mu$-null set outside of which $f=h$.Indeed, there are $N_1,N_2\in \Fcal$ with $\mu\brac{N_1}=\mu\brac{N_2}=0$ such that $f\brac{\omega}=g\brac{\omega}$ for all $\omega\in N_1$ and $g\brac{\omega}=h\brac{\omega}$ for all $\omega\in N_2$. Therefore for $N\defn N_1\cup N_2\in \Fcal$, $\mu\brac{N}\leq \mu\brac{N_1}+\mu\brac{N_2}=0$ and $f\brac{\omega}=h\brac{\omega}$ for all $\omega\in N$.

Being an equivalence relation, $\mu$-a.s. equality partitions $L^p_K$ into non-overlapping equivalence classes denoted $\clo{f}$ for every $f\in L^p_K$. Indeed, if $f,g\in L^p_K$ are such that $\clo{f}\cap \clo{g}\neq \emptyset$, then by transitivity of $\mu$-a.s. equality $\clo{g}=\clo{f}$. So if $f=g$ $\mu$-a.s, then $f\in \clo{g}$ implying $\clo{g}=\clo{f}$. Whenever $\clo{f}=\clo{g}$, transitivity implies that $f=g$ $\mu$-a.s.

Let $f,f',g,g'\in L^p_K$ be such that $\clo{f}=\clo{f'}$ and $\clo{g}=\clo{g'}$. Since $L^p_K$ is closed under $K$-linear combinations $f+g, f'+g'\in L^p_K$. Since $f=f'$ $\mu$-a.s. and $g=g'$ $\mu$-a.s. it must be true that $f+g=f'+g'$ $\mu$-a.s. whence $\clo{f+g}=\clo{f'+g'}$. Also, $f=f'$ $\mu$-a.s. implies $\alpha f=\alpha f'$ $\mu$-a.s. for any $\alpha\in K$, whence $\clo{\alpha f}=\clo{\alpha f'}$.

Let $\Hcal\defn \obj{ \induc{ \clo{f} } f\in L^p_K\brac{\Omega, \Fcal, \mu} }$ and define the following operations $\oplus:\Hcal\times \Hcal\to\Hcal$ and $\otimes:K\times\Hcal\to\Hcal$: \[\clo{f}\oplus\clo{g} = \oplus\brac{\clo{f},\clo{g}} \defn \clo{f+g}\,\text{and}\,\alpha\otimes\clo{f} = \otimes\brac{\alpha, \clo{f}} \defn \clo{\alpha f}\] for all $\clo{f},\clo{g}\in \Hcal$ and $\alpha\in K$.

Thus defined addition $\oplus$ is well defined, since its resulting element of $\Hcal$ is independent of the particular representatives of the equivalence classes used in addition within the scope of $L^p_K$. The scalar multiplication is well defined by the same logic.

To see that thus defined triple $\brac{\Hcal, \oplus, \otimes}$ is a $K$-vector space it is necessary to check each of the eight required properties.\begin{itemize}
	\item The equivalence class $0_\Hcal\defn\clo{0}$ of all $\mu$ almost surely negligible functions in $L^p_K$ serves as the additively neutral element of $\Hcal$. Indeed, $0_\Hcal\oplus \clo{f} = \clo{0+f} = \clo{f}$for any $\clo{f}\in\Hcal$.
	\item For every $\clo{f}\in \Hcal$ the additive inverse may be defined as $\brac{-\clo{f}}\defn\clo{-f}$, since $g=f$ $\mu$-a.s. is equivalent to $g+\brac{-f}=0$ $\mu$-a.s., whence $g+\brac{-f}\in 0_\Hcal$ for all $g\in \clo{f}$. Thus $\clo{f}+\clo{-f}=\clo{f+\brac{-f}}=\clo{g+\brac{-f}} = 0_\Hcal$.
	\item The associativity of $\oplus$ addition follows directly from the associativity of addition in $K$. Indeed, for $\clo{f}, \clo{g}, \clo{h}\in \Hcal$ direct checking yields \[\clo{f}\oplus\brac{\clo{g}\oplus\clo{h}} = \clo{f}\oplus\clo{g+h} = \clo{f+\brac{g+h}} = \clo{f+g+h}\] and \[\clo{f+g+h} = \clo{\brac{f+g}+h} = \clo{f+g}\oplus\clo{h} = \brac{\clo{f}\oplus\clo{g}}\oplus\clo{h}\]
	\item Commutativity of addition in $\Hcal$ is inherited from the addition in the field $K$. Indeed, \[\clo{f}\oplus\clo{g} = \clo{f+g} = \clo{g+f}=\clo{g}\oplus\clo{f}\]
	\item For $1\in K$ by definition of $\otimes$ it must be $1\otimes\clo{f} = \clo{1 f} = \clo{f}$.
	\item Further properties of scalar multiplication follow from the properties of addition and multiplication in $K$. Firstly, $\alpha\otimes\brac{\beta\otimes\clo{f}} = \alpha\otimes\clo{\beta f} = \clo{\alpha\brac{\beta f}}$ and so $\alpha\otimes\brac{\beta\otimes\clo{f}} = \brac{\alpha \beta}\otimes\clo{f}$. Lastly, $\brac{\alpha + \beta}\oplus\clo{f} = \clo{\alpha f + \beta f} = \clo{\alpha f}\oplus \clo{\beta f} = \brac{\alpha\otimes\clo{f}}\oplus \brac{\beta\otimes\clo{f}}$.
	\item Distributivity of $\otimes$ over $\oplus$ stems from the distributive property of multiplication over addition in the field $K$. Indeed, \[\alpha \otimes \brac{\clo{f}\oplus\clo{g}} = \alpha \otimes \clo{f+g} = \clo{\alpha\brac{f+g}} = \clo{\alpha f + \alpha g}\] and since $\clo{\alpha f + \alpha g} = \clo{\alpha f}\oplus \clo{\alpha g}$ it must be true that $\alpha \otimes \brac{\clo{f}\oplus\clo{g}} = \brac{\alpha\otimes\clo{f}}\oplus \brac{\alpha\otimes\clo{g}}$.
\end{itemize}

Suppose $p=2$ and for all $\clo{f}, \clo{g}\in \Hcal$ define\[\brkt{\clo{f},\clo{g}}\defn \int_\Omega f\bar{g} d\mu\] If $f',g'\in L^2_K$ are such that $\clo{f}=\clo{f'}$ and $\clo{g}=\clo{g'}$ then $f\bar{g} = f'\overline{g'}$ $\mu$-a.s. and $f\bar{g}, f'\overline{g'}\in L^1_K$ whence by theorem 5-11 \[\brkt{\clo{f},\clo{g}} = \int_\Omega f\bar{g} d\mu = \int_\Omega f'\overline{g'} d\mu = \brkt{\clo{f'},\clo{g'}}\] Therefore the map $\Hcal\times\Hcal\to K$ is defined unambiguously.

In theorem 10-9 it has been shown that $\brac{f,g}\to \int_\Omega f\bar{g} d\mu$ on $L^2_K$ satisfies all but the last properties of an inner product. Indeed, \[\overline{\brkt{\clo{g}, \clo{f}}} = \overline{\int_\Omega g\bar{f} d\mu} = \int_\Omega f\bar{g} d\mu = \brkt{\clo{g}, \clo{f}}\] and the linearity of the complex Lebesgue integral and the definition of $\oplus$ imply \[\brkt{\clo{f}\oplus\clo{h},\clo{g}} = \brkt{\clo{f+h}, \clo{g}} = \int_\Omega f\bar{g} d\mu + \int_\Omega h\bar{g} d\mu = \brkt{\clo{f}, \clo{g}} + \brkt{\clo{h}, \clo{g}}\] Similarly, for $\alpha\in K$ and any $\clo{f}\in \Hcal$ by definition of $\otimes$ and the linearity of the complex Lebesgue integral it is true that \[\brkt{\alpha\otimes\clo{f}, \clo{g}}=\int_\Omega \alpha f \bar{g}d\mu = \alpha \int_\Omega f \bar{g}d\mu = \alpha \brkt{\clo{f}, \clo{g}}\] Furthermore, $\brkt{\clo{f},\clo{f}} = \int_\Omega f\bar{f} d\mu = \int_\Omega \abs{f}^2 d\mu\geq 0$ for any $\clo{f}\in \Hcal$.

Finally, if $\clo{f}\in \Hcal$ is such that $\brkt{\clo{f},\clo{f}} = 0$ then $\int_\Omega \abs{f}^2 d\mu=0$ implying that $f=0$$\mu$-a.s. Therefore, $f\in 0_\Hcal$ whence $\clo{f}=0_\Hcal$. Conversely, $\int_\Omega 0 d\mu = 0$ whence $\brkt{\clo{f}, \clo{f}} = 0$ if $\clo{f}=0_\Hcal\in \Hcal$. In conclusion, $\brkt{\cdot, \cdot}$ on $\Hcal$ is indeed an inner product on $\Hcal$.

For every $\clo{f}, \clo{g}\in \Hcal$ define the metric $d_{\brac{\cdot,\cdot}}\brac{\clo{f}, \clo{g}}\defn \nrm{\clo{f}\oplus\brac{-\clo{g}}}=\nrm{\clo{f-g}}$ where $\nrm{\clo{f}}$ is defined as $\sqrt{\brkt{\clo{f}, \clo{f}}}$. Define the usual topology $\Tcal_\Hcal$ on $\Hcal$ as the metric topology generated by $_{\brac{\cdot,\cdot}}$, or in other words as the norm-topology $\Tcal_{\brkt{\cdot,\cdot}}$.

Let $\brac{\clo{f_n}}_{n\geq 1}\in \Hcal$ be a Cauchy sequence with respect to the metric $d_{\brac{\cdot,\cdot}}$. Then for any $\epsilon>0$ there is $N\geq 1$ such that $\nrm{\clo{f_n-f_m}}<\epsilon$ for all $n,m\geq N$. Since $\nrm{\clo{f}}=\nrm{f}_2$, this implies that $\brac{f_n}_{n\geq 1}\in L^2_K$ is Cauchy with respect to the $L^2$-norm, whence by theorem 46 there is $f\in L^2_K$ such that $f_n\overset{L^2}{\to}f$. Since $\nrm{f_n-f}_2\to 0$ by theorem 9-11, it must be true that $\nrm{\clo{f_n},\clo{f}}\to 0$, whence $\clo{f_n}\overset{\Hcal}{\to}\clo{f}$ in $\Tcal_\Hcal$.

Let $\brac{\clo{g_n}}\in\Hcal$ be such that $\clo{f_n}=\clo{g_n}$ for all $n\geq 1$. Then $\brac{g_n}_{n\geq 1} \in L^2_K$ is Cauchy and by theorem 46 there exists $g\in L^2_K$ such that $\nrm{g_n-g}_2\to 0$, whence $\clo{g_n}\overset{\Hcal}{\to}\clo{g}$ in $\Tcal_\Hcal$. Since $f_n=g_n$ $\mu$-a.s. for all $n\geq 1$, the $L^p$-norm $\nrm{f_n-g_n}_2$ is zero for any $n\geq 1$. Therefore by the triangle law for $L^p$-norms \[\nrm{f-g}_2\leq \nrm{f-f_n}_2+\nrm{f_n-g_n}_2+\nrm{g_n-g}_2\] whence $\nrm{f-g}_2=0$ because $f_n\overset{L^2}{\to}f$ and $g_n\overset{L^2}{\to}g$. Thus $f=g$ $\mu$-a.s. by theorem 9-4, whence $\clo{f}=\clo{g}$.

Finally, if $h\in L^2_K$ is another map, such that $f_n\overset{L^2}{\to}h$ then by theorem 9-13 it must be that $h=f$ $\mu$-a.s., implying that $\clo{f}=\clo{h}$ and the limit of $\clo{f_n}$ in $\Hcal$ is well-defined and unique.

Thus for any Cauchy sequence $\brac{\clo{f_n}}_{n\geq1}\in\Hcal$ there is a unique $\clo{f}\in \Hcal$ such that $\clo{f_n}\overset{\Hcal}{\to}\clo{f}$. Therefore the space $\Hcal$ is complete with respect to $d_{\brac{\cdot,\cdot}}$ and the pair $\brac{\Hcal, \brkt{\cdot,\cdot}}$ is a Hilbert space over $K$.

Recall, that $\clo{f,g}\defn \int_\Omega f\bar{g}d\mu$ is not an inner product on $L^2_K\brac{\Omega, \Fcal, \mu}$ since $\clo{f,g}=0$ implies that $f=g$ just $\mu$-a.s., not everywhere on $\Omega$ as required in the definition.

\label{thm:lin_func_cplx_int} \noindent \textbf{Theorem} 55.
Let $\brac{\Omega, \Fcal, \mu}$ be a measure space and $\lambda:L^2_K\brac{\Omega, \Fcal, \mu}\to K$ be a linear functional continuous with respect to $\Tcal_{L^2_K}$ and $\Tcal_K$, where $K=\Real$ or $\Cplx$. There exists $g\in L^2_K\brac{\Omega, \Fcal, \mu}$ such that for all $f\in L^2_K$ \[\lambda\brac{f} = \int_\Omega f\bar{g} d\mu\]

First, note that $\lambda$ is a bounded linear functional on $L^2_K$. Indeed since it is $\Tcal_{L^2_k}$-$\Tcal_K$ continuous, at $0\in L^2_K$ there exists $\eta>0$ such that $\abs{\lambda\brac{f}}\leq 1$ for every $f\in L^2_K$ with $\nrm{f}_2\leq\eta$ (see theorem 10-13). Put $M\defn \frac{1}{\eta}\in \Real^+$. If $f\in L^2_K$ is such that $\nrm{f}_2\neq 0$ then $h\defn\frac{\eta}{\nrm{f}_2}f\in l^2_K$ and $\nrm{h}_2\leq \eta$, whence $\abs{\lambda\brac{h}}\leq 1$. Thus for all $f\in L^2_K$ with $\nrm{f}_2>0$ it is true that $\abs{\lambda\brac{f}}\leq M\nrm{f}_2$. Therefore, there exists $M\in \Real^+$ such that $\abs{\lambda\brac{f}}\leq M\nrm{f}_2$ for all $f\in L^2_K$.

If $\clo{f}, \clo{g}\in \Hcal$ are such that $\clo{f}=\clo{g}$ then $f=g$$\mu$-a.s., whence by theorem 9-4 $\nrm{f-g}_2=0$. From the linearity of $\lambda$ it follows that \[\abs{\lambda\brac{f}-\lambda\brac{g}} = \abs{\lambda\brac{f-g}}\leq M\nrm{f-g}_2 = 0\] Therefore, $\lambda\brac{f}=\lambda\brac{g}$ for such $f$ and $g$ in $L^2_K$. Also $\lambda\brac{h}=0$ for any $h\in L^2_K$ with $h=0$ $\mu$-a.s.

Define $\Lambda:\Hcal\to K$ by $\Lambda\brac{\clo{f}}\defn \lambda\brac{f}$. The map $\Lambda$ is well defined, because $\lambda\brac{f}=\lambda\brac{g}$ for any $g\in \clo{f}$. Now, for any $\clo{f}, \clo{g}\in \Hcal$ and $\alpha\in K$ the linearity of $\lambda$ implies that\[\Lambda\brac{\clo{f}\oplus\brac{\alpha\otimes\clo{g}}} = \Lambda\brac{\clo{f+\alpha g}} = \lambda\brac{f}+\alpha\lambda\brac{g} = \Lambda\brac{\clo{f}}+\alpha \Lambda\brac{\clo{g}}\] Therefore $\Lambda$ is a linear functional in $\Hcal$. Finally, there is $M\in \Real^+$ such that for any $\clo{f}\in \Hcal$, \[\abs{\Lambda\brac{\clo{f}}}=\abs{\lambda\brac{f}}\leq M\nrm{f}_2 = M\nrm{\clo{f}}\]

Therefore $\Lambda$ is a linear bounded functional on a Hilbert space $\brac{\Hcal, \brkt{\cdot,\cdot}}$ over $K$. Thus by theorem 54 there is $\clo{g}\in \Hcal$ such that $\Lambda\brac{\clo{f}}=\brkt{\clo{f},\clo{g}}$ for all $\clo{f}\in \Hcal$. Therefore there is $g\in L^2_K$ such that for all $f\in L^2_K$ \[\lambda\brac{f} = \Lambda\brac{\clo{f}}=\int_\Omega f\bar{g} d\mu\]\\

% section tut_10 (end)

\section{Complex measures} % (fold)
\label{sec:tut_11}
\url{http://probability.net/PRTcomplex.pdf}

\noindent \textbf{Definition} 90.
A sequence $\brac{z_n}_{n\geq 1}\in\Cplx$ is said to have the permutation properties if for all bijections $\sigma:\mathbb{N}\to\mathbb{N}$ the series $\sum_{n=1}^\infty a_{\sigma\brac{n}}$ converges in $\Cplx$ (obviously excluding the $\pm\infty$ as limits).

\label{thm:absolute_summation_1} \noindent \textbf{Theorem} 11-1.
If $\brac{z_k}_{k\geq1}\in \Cplx$ has the permutation property, then $\brac{\re z_n}_{n\geq 1}$ and $\brac{\im z_n}_{n\geq 1}$ have the permutation property as well.

Let $\sigma:\mathbb{N}\to\mathbb{N}$ be any bijection. Since $\brac{z_k}_{k\geq 1}$ has the permutation property, $\sum_{k=1}^n z_{\sigma\brac{k}}\overset{\Cplx}{\to} z$ where $z\defn \sum_{k=1}^\infty z_{\sigma\brac{k}}$ is a well-defined complex number. Since the maps $\im$ and $\re$ are $\Tcal_\Cplx$-$\Tcal_\Real$ continuous, where $\Tcal_\Cplx$ and $\Tcal_\Real$ are the usual topologies on $\Cplx$ and $\Real$ respectively, it must be true that \[\re\brac{\sum_{k=1}^n z_{\sigma\brac{k}}}\overset{\Real}{\to}\re\brac{\sum_{k=1}^\infty z_{\sigma\brac{k}}}\,\text{and}\,\im\brac{\sum_{k=1}^n z_{\sigma\brac{k}}}\overset{\Real}{\to}\im\brac{\sum_{k=1}^\infty z_{\sigma\brac{k}}}\] Since $\sum_{k=1}^n \re z_{\sigma\brac{k}}=\re\brac{\sum_{k=1}^n z_{\sigma\brac{k}}}$ and $\sum_{k=1}^n \im z_{\sigma\brac{k}}=\im\brac{\sum_{k=1}^n z_{\sigma\brac{k}}}$ for all $n\geq 1$ it is therefore true that the series $\sum_{k=1}^\infty \re z_{\sigma\brac{k}}$ and $\sum_{k=1}^\infty \im z_{\sigma\brac{k}}$ converge in $\Real$ to the real and imaginary parts of $\sum_{k=1}^\infty z_{\sigma\brac{k}}\in \Cplx$ respectively. In conclusion, $\brac{\re z_n}_{n\geq 1}$ and $\brac{\im z_n}_{n\geq 1}$ have the permutation property.\\

\label{thm:absolute_summation_2} \noindent \textbf{Theorem} 11-2.
Let $\brac{x_n}_{n\geq1}\in \Real$. Then $\sum_{n\geq 1} \abs{x_n}=+\infty$ is equivalent to either $\sum_{n\geq 1} x_n^+=+\infty$, or $\sum_{n\geq 1} x_n^+=+\infty$, or both.

Indeed, suppose $\sum_{n\geq 1} x_n^+<+\infty$ and $\sum_{n\geq 1} x_n^-<+\infty$. Since infinite sums of non-negative numbers possess the generalized partition-invariance property (theorem Sup-B-3), it must therefore be true that \[\sum_{n\geq 1} \abs{x_n} = \sum_{n\geq 1} \brac{x_n^++x_n^-} = \sum_{n\geq 1} x_n^+ + \sum_{n\geq 1} x_n^-\] Therefore, $\sum_{n\geq 1} \abs{x_n}<+\infty$, or, in other words, $\sum_{n\geq 1} \abs{x_n}=+\infty$ implies $\sum_{n\geq 1} x_n^+=+\infty$ or $\sum_{n\geq 1} x_n^+=+\infty$. In fact, the converse is also true, since $x_n^-,x_n^+\leq \abs{x_n}$ for all $n\geq 1$.\\

\label{thm:absolute_summation_3} \noindent \textbf{Theorem} 11-3.
Let $\brac{x_n}_{n\geq1}\in \Real$ be such that $\sum_{n=1}^\infty x_n$ converges in $\Real$. Then either the series $\sum_{n\geq 1} x_n^+$ and $\sum_{n\geq 1} x_n^-$ both converge or diverge simultaneously.

Indeed, suppose that $\sum_{n\geq 1} x_n^-<+\infty$. Since $x_n^+ = x_n + x_n^-\geq 0$ for every $n\geq 1$ and the series $\sum_{n=1}^\infty x_n$ and $\sum_{n=1}^\infty x_n^-$ converge in $\Real$, it has to be true that $\sum_{n=1}^\infty \brac{x_n+x_n^-}$ converges in $\Real$ to some non-negative number. Therefore $\sum_{n\geq 1} x_n^+ = \sum_{n=1}^\infty \brac{x_n+x_n^-}<+\infty$. Conversely, $\sum_{n=1}^\infty x_n^+<+\infty$ implies that $\sum_{n\geq 1} x_n^-<+\infty$.\\

\label{thm:absolute_summation_4} \noindent \textbf{Theorem} 11-4.
Let $\brac{x_n}_{n\geq1}\in \Real$ be such that $\sum_{n=1}^\infty x_n$ converges in $\Real$. Then $\sum_{n\geq 1} \abs{x_n}=+\infty$ implies $\sum_{n\geq 1} x_n^+=\sum_{n\geq 1} x_n^-=+\infty$.

Indeed, by theorem 11-3 either $\sum_{n\geq 1} x_n^+=+\infty$, or $\sum_{n\geq 1} x_n^-=+\infty$, or both, and by theorem 11-2 $\sum_{n\geq 1} x_n^+<+\infty$ is equivalent to $\sum_{n\geq 1} x_n^-<+\infty$. Therefore $\sum_{n\geq 1} x_n^+=\sum_{n\geq 1} x_n^-=+\infty$.\\

\label{thm:absolute_summation_5} \noindent \textbf{Theorem} 11-5.
Let $\brac{x_n}_{n\geq1}\in \Real$ have the permutation property. Then $\sum_{n\geq 1} \abs{x_n}<+\infty$.

For such $\brac{x_n}_{n\geq1}$ it must be true that $\sum_{n=1}^\infty x_n$ converges in $\Real$. Suppose $\sum_{n\geq 1} \abs{x_n}=+\infty$. Then by theorem 11-4 $\sum_{n\geq 1} x_n^+=\sum_{n\geq 1} x_n^-=+\infty$.

Define \[N^+\defn \obj{\induc{n\geq 1}x_n\geq 0}\,\text{and}\,N^-\defn \obj{\induc{n\geq 1}x_n<0}\] Since for every $n\geq 1$ either $x_n\geq 0$ and $n\in N^+$, or $x_n<0$ and $n\in N^-$, it is true that $\mathbb{N} = N^+\uplus N^-$, as $N^+\cap N^-=\emptyset$. Furthermore by theorem Sup-B-3 \[\sum_{n\geq 1} x_n^+ = \sum_{n\in N^+} x_n^+ + \sum_{n\in N^-} x_n^+\,\text{and}\,\sum_{n\geq 1} x_n^- = \sum_{n\in N^+} x_n^- + \sum_{n\in N^-} x_n^-\] Note that $\sum_{n\in N^-} x_n^+=\sum_{n\in N^+} x_n^-=0$ by the very construction of $N^+$ and $N^-$.

If $N^+$ is finite, then there must be $k\in N^+$ such that $x_k\geq x_n$ for all $n\in N^+$. Therefore $\sum_{n\in N^+}^\text{finte}x_n^+\leq \abs{N^+} x_k<+\infty$ with the left hand side sum being a usual sum finitely many values. This, however, contradicts the fact that $\sum_{n\in N^+} x_n^+=+\infty$. Similarly, the set $N^-$ must be infinite as well. Since $N^+$ and $N^-$ are infinite subsets of a countable $\mathbb{N}$, they must themselves be countable, whence there must exist bijective maps $\phi^+:\mathbb{N}\to N^+$ and $\phi^-:\mathbb{N}\to N^-$.

Since $x_n^+=x_n$ for all $n\in N^+$, $\sum_{n\in N^+} x_n^+=+\infty$ and by theorem Sup-B-3 $\sum_{n\in N^+} x_n = \sum_{i=1}^\infty x_{\phi^+\brac{i}}$, it must be that $\sum_{i=1}^\infty x_{\phi^+\brac{i}}=+\infty$.

Fix some $A>0$. Given any $k_{p-1}\geq 0$, for $A'\defn A-\sum_{i=1}^{k_{p-1}} x_{\phi^+\brac{i}}$ there must exist $K\geq 1$ such that $\sum_{i=1}^n x_{\phi^+\brac{i}}\geq A'$ for all $n\geq K$. Hence for $k_p\defn \max\obj{K,k_{p-1}+1}$, $k_p\geq K$ and it is true that $\sum_{i=k_{p-1}+1}^{k_p} x_{\phi^+\brac{i}}\geq A$. Therefore for this $A>0$ there is a strictly increasing sequence $\brac{k_p}_{p\geq 1} \in \mathbb{N}$ such that $\sum_{i=k_p+1}^{k_{p+1}} x_{\phi^+\brac{i}}\geq A$ for all $p\geq 1$.

Based on $\brac{k_p}_{p\geq 1}$ define $k^*_p\defn k_p+p$ with $k^*_0=0$. Since $\brac{k_p}_{p\geq1}\uparrow +\infty$, it must be that $\brac{k^*_p}_{p\geq1}\uparrow +\infty$, whence for any $n\geq 1$ there is $P\geq 1$ such that $k^*_p\geq n$ for all $p\geq P$. Thus for every $n\geq1$ there is a $q\geq 1$ such that $k^*_{q-1}<n\leq k^*_q$ (the same holds for $\brac{k_p}_{p\geq 1}$). If $m\geq 1$ is another index such that $k^*_{m-1}<n\leq k^*_m$, then $k^*_{q-1}<k^*_m$ and $k^*_{m-1}<k^*_q$, whence $q\leq m$ and $m\leq q$ as $\brac{k^*_p}_{p\geq 1}$ is strictly increasing. Thus for every $n\geq1$ there is a unique $q_n\geq 1$ such that $k^*_{q_n-1}<n\leq k^*_{q_n}$.

For this $A>0$ from the constructed sequence $\brac{k^*_p}_{p\geq1}\uparrow +\infty$ derive a map $\sigma:\mathbb{N}\to\mathbb{N}$ in the following manner: $\sigma\brac{n}\defn \phi^-\brac{q_n}$ if $n=k^*_{q_n-1}+1$ and $\sigma\brac{n}\defn \phi^+\brac{n-q_n}$ if $n>k^*_{q_n-1}+1$.

Suppose $\sigma\brac{n} = \sigma\brac{m}$ for some $n,m\geq 1$. If $n=k^*_{q_n-1}+1$ and $m=k^*_{q_m-1}+1$ then $\phi^-\brac{q_n}=\phi^-\brac{q_m}$, whence $q_n=q_m$ as $\phi^-$ is an injection. Therefore $n=m$. If $n>k^*_{q_n-1}+1$, but $m=k^*_{q_m-1}+1$, then $\phi^-\brac{q_m}=\phi^+\brac{n-q_n}$. Thus $\phi^-\brac{q_m}\in N^-$ and $\phi^+\brac{n-q_n}\in N^+$, implying that $N^+\cap N^-\neq\emptyset$, which is a contradiction. Similarly, $m>k^*_{q_m-1}+1$ and $n=k^*_{q_n-1}+1$ leads to a contradiction. If $n>k^*_{q_n-1}+1$ and $m>k^*_{q_m-1}+1$, then $\phi^+\brac{n-q_n}=\phi^+\brac{m-q_m}$, which implies that $n-q_n = m-q_m$. If $q_n<q_m$, then $k^*_{q_n}<k^*_{q_m-1}+1$ and $k_{q_n}\leq k_{q_m-1}$. Therefore $k^*_p=k_p+p$, $n\leq k^*_{q_n}$ and $k^*_{q_m-1}+1<m$ imply that $n-q_n\leq k_{q_n}\leq k_{q_m-1}<m-q_m$, which contradicts $n-q_n = m-q_m$. Similarly, $q_m<q_n$ is false, which leads to a conclusion that $q_n = q_m$, whence $n=m$. Therefore $\sigma:\mathbb{N}\to\mathbb{N}$ is injective.

Let $m\geq1$. Since $m\in N^-\uplus N^+$ there are two alternatives. If $m\in N^+$ then let $N\geq1$ be such that $\phi^+\brac{N}=m$. Since $\brac{k_p}_{p\geq 1}\uparrow +\infty$, for this $N$ there exists $p\geq 1$ such that $k_{p-1}<N\leq k_p$. Therefore putting $n\defn N+p$ gives $k^*_{p-1}+1<n\leq k^*_p$. Now, for such $n$ there is $q_n\geq 1$ such that $k^*_{q_n-1}<n\leq k^*_{q_n}$. If $n=k^*_{q_n-1}+1$ then $k^*_{p-1}<k^*_{q_n-1}$ and $k^*_{q_n-1}<k^*_p$, whence $p<q_n$ and $q_n<p+1$, which contradicts $q_n\in \mathbb{N}$. If $n>k^*_{q_n-1}+1$, then $k^*_{p-1}<k^*_{q_n}$ and $k^*_{q_n-1}<k^*_p$, whence $p\leq q_n$ and $q_n\leq p$. Thus $p=q_n$ and therefore $\sigma\brac{n} = \phi^+\brac{n-q_n} = \phi^+\brac{N+p-q_n}=m$.

If $m\in N^-$, then put $n\defn k^*_{p-1}+1$, where $m=\phi^-\brac{p}$. For such $n$ there is a unique $q_n\geq 1$ with $k^*_{q_n-1}<n\leq k^*_{q_n}$. If $n>k^*_{q_n-1}+1$ then $k^*_{q_n-1}<k^*_{p-1}$ and $k^*_{p-1}+1\leq k^*_{q_n}$, whence $q_n<p$ and $p<q_n+1$. This contradicts the fact that $p$ is an integer. If $n=k^*_{q_n-1}+1$ then $k^*_{q_n-1}<k^*_{p-1}$ and $q_n=p$, whence $\sigma\brac{n}=\phi^-\brac{q_n}=\phi^-\brac{p}=m$. Therefore the map $\sigma:\mathbb{N}\to\mathbb{N}$ is a bijective map and is thus a permutation of $\mathbb{N}$.

Since $\brac{x_n}_{n\geq1}$ has the permutation property, the series $\sum_{i=1}^\infty x_{\sigma\brac{i}}$ must converge in $\Real$. Therefore $\brac{\sum_{i=1}^n x_{\sigma\brac{i}}}_{n\geq1}$ is Cauchy with respect to the usual metric on $\Real$. In particular there is $N\geq 1$ such that $\abs{\sum_{i=n+1}^{n+p} x_{\sigma\brac{i}}} < A$ for all $n\geq N$ and $p\geq 1$. For this $N\geq 1$ there must be a unique $q\geq 1$ such that $k^*_{q-1}<N\leq k^*_q$. Therefore for $n\defn k^*_q+1\geq N$ and $p\defn k_{q+1}-k_q\geq 1$ it must be true that \[\abs{\sum_{i=n+1}^{n+p} x_{\sigma\brac{i}}} = \abs{\sum_{i=\brac{k^*_q+1}+1}^{\brac{k^*_q+1}+k_{q+1}-k_q} x_{\sigma\brac{i}}} = \abs{\sum_{i=k_q+\brac{q+1}+1}^{k_{q+1}+\brac{q+1}} x_{\phi^+\brac{i-\brac{q+1}}}} = \abs{\sum_{k=k_q+1}^{k_{q+1}} x_{\phi^+\brac{k}}}\] Given the properties of the constructed sequence $\brac{k_p}_{p\geq 1}$, it must therefore be true that $A>A$. Therefore $\sum_{n\geq 1} \abs{x_n}$ must be finite, whenever $\brac{x_n}_{n\geq 1}$ has the permutation property.\\

\label{thm:absolute_summation} \noindent \textbf{Theorem} 56
If $\brac{z_k}_{k\geq1}\in \Cplx$ has the permutation property, then $\sum_{n\geq 1} \abs{z_n} < +\infty$.

Indeed, by theorem 11-1 both $\brac{\re z_n}_{n\geq1}$ and $\brac{\im z_n}_{n\geq1}$ have the permutation property. Therefore by theorem 11-5 \[\sum_{n\geq1} \abs{\re z_n}<+\infty\,\text{and}\,\sum_{n\geq1} \abs{\im z_n}<+\infty\] Since $\abs{z_n}\leq \abs{\re z_n}+\abs{\im z_n}$, by theorem Sup-B-3 $\sum_{n\geq 1} \abs{z_n} < +\infty$.\\

\noindent \textbf{Definition} 91.
Let $\brac{\Omega, \Fcal}$ be a measurable space and $E\in \Fcal$. A measurable partition of $E$ is any sequence $\brac{E_n}_{n\geq 1}\in \Fcal$ of pairwise disjoint measurable sets with $E=\uplus_{n\geq 1} E_n$.

\noindent \textbf{Definition} 92.
A complex measure on a measurable space $\brac{\Omega, \Fcal}$ is any map $\mu:\Fcal\to \Cplx$ such that for all $E\in\Fcal$ and any measurable partition $\brac{E_n}_{n\geq 1}$ of $E$ the series $\sum_{n=1}^\infty \mu\brac{E_n}$ converges to $\mu\brac{E}$ in $\Cplx$. The set of all maps which are complex measures on $\brac{\Omega, \Fcal}$ is denoted by $M^1\brac{\Omega, \Fcal}$.

\noindent \textbf{Definition} 93.
A signed measure on a measurable space $\brac{\Omega, \Fcal}$ is any complex measure on $\brac{\Omega, \Fcal}$ with values in $\Real$.

Note that a usual measure on $\brac{\Omega, \Fcal}$ may not be a complex measure, since it takes values in $\Zinf$ which is not a subset on $\Cplx$, even though for any $E\in \Fcal$ and measurable partition $\brac{E_n}_{n\geq 1}$ of $E$ \[\mu\brac{E} = \mu\brac{\uplus_{n\geq 1} E_n} = \sum_{n\geq 1} \mu\brac{E_n}\]

Let $\mu\in M^1\brac{\Omega, \Fcal}$. Since $\mu$ is a complex measure and $\brac{E_n}_{n\geq 1}\in\Fcal$ with $E_n\defn \emptyset$ is a measurable partition on $\emptyset$, $\mu\brac{\emptyset} = \sum_{n=1}^\infty \mu\brac{E_n}$ in $\Cplx$. For any permutation $\sigma:\mathbb{N}\to\mathbb{N}$ it is true that $\mu\brac{E_{\sigma\brac{n}}} = \mu\brac{E_n}=\mu\brac{\emptyset}$ for all $n\geq 1$, whence $\sum_{n=1}^\infty \mu\brac{E_{\sigma\brac{n}}} = \mu\brac{E_n} = \mu\brac{\emptyset}$. Therefore the sequence $\brac{\mu\brac{E_n}}_{n\geq1}\in \Cplx$ must have the permutation property. By theorem 56, $\sum_{n\geq 1} \mu\brac{E_n} < +\infty$, which implies that for every $\epsilon>0$ there is $N\geq 1$ such that $\sum_{n=m+1}^{m+p} \abs{\mu\brac{E_n}} < \epsilon$ for all $m\geq N$ and $p\geq 1$. As $\mu\brac{E_n}=\mu\brac{\emptyset}$ for all $n\geq 1$, it must be true that $p \abs{\mu\brac{\emptyset}}<\epsilon$ for all $\epsilon>0$ and all $p\geq1$. Therefore $\abs{\mu\brac{\emptyset}}=0$.

Suppose $\mu$ is a measure on $\brac{\Omega, \Fcal}$ with $\mu\brac{\Omega}<+\infty$. For any $E\in \Fcal$, $E\subseteq \Omega$ and $\mu\brac{E}\leq \mu\brac{\Omega}<+\infty$, implying $\mu\brac{\cdot}\in \Real^+$. Furthermore, for any $E\in\Fcal$ and measurable partition $\brac{E_n}_{n\geq 1}$ of $E$, since $\mu\brac{\cdot}$ is a (usual) measure, the series $\sum_{n\geq 1} \mu\brac{E_n}$ converges to $\mu\brac{\uplus_{n\geq 1} E_n} = \mu\brac{E}$ in $\Real^+$ and thus in $\Cplx$. Therefore $\mu\in M^1\brac{\Omega, \Fcal}$.

Conversely, if $\mu\in M^1\brac{\Omega, \Fcal}$ and $\mu\brac{\cdot}\in \Real^+$, then $\mu:\Fcal\to\Zinf$, $\mu\brac{\emptyset}=0$ and for any pairwise disjoint family $\brac{A_n}_{n\geq 1}\in \Fcal$, $\brac{A_n}_{n\geq 1}$ is a measurable partition of $\uplus_{n\geq 1} A_n$, whence $\mu\brac{\uplus_{n\geq 1} A_n}=\sum_{n\geq 1} \mu\brac{A_n}$ because $\mu$ is a complex measure. In conclusion, $\mu$ is a finite measure on $\brac{\Omega, \Fcal}$ if and only if $\mu\in M^1\brac{\Omega, \Fcal}$ and $\mu\brac{\cdot}\in \Real^+$.

Let $\mu\in M^1\brac{\Omega, \Fcal}$. Let $E\in \Fcal$ and $\brac{E_n}_{n\geq 1}$ be a measurable partition of $E$. For any permutation $\sigma:\mathbb{N}\to\mathbb{N}$, $\brac{E_{\sigma\brac{n}}}_{n\geq 1}$ is still a measurable partition of $E$, whence $\sum_{n=1}^\infty \mu\brac{E_{\sigma\brac{n}}} = \mu\brac{E}$ because $\mu$ is a complex measure. Therefore $\brac{E_n}_{n\geq 1}\in \Cplx$ has the permutation property, which implies that $\sum_{n\geq 1} \abs{\mu\brac{E_n}}<+\infty$ by theorem 56.

\label{thm:cplx_int_is_cplx_meas} \noindent \textbf{Theorem} 11-6.
If $\brac{\Omega, \Fcal, \mu}$ be a (usual) measure space and $f\in L^1_\Cplx\brac{\Omega, \Fcal, \mu}$, then $E\to \int_E f d\mu$ defined on $\Fcal$ is a complex measure on $\brac{\Omega, \Fcal}$.

Put $\nu\brac{E}\defn \int_E f d\mu$ for all $E\in \Fcal$ and observe that by definition of the partial Complex integral $\nu\brac{E}=\int_\Omega f 1_E d\mu$ for every $E\in \Fcal$. Let $E\in \Fcal$ and $\brac{E_n}_{n\geq 1}$ be a measurable partition of $E$.

Since $\sum_{k=1}^n f 1_{E_k} = 1_{\uplus_{k=1}^n E_k}$ for all $n\geq 1$, it must be true that \[\abs{f 1_E - \sum_{k=1}^n f 1_{E_k} } = \abs{f} 1_{E\setminus {\uplus_{k=1}^n E_k}}\] Then, for every $\omega\notin E$ it is true that $1_{E\setminus {\uplus_{k=1}^n E_k}}\brac{\omega}=0$ for all $n\geq 1$, and for $\omega\in E$ there must be $N\geq 1$ such that $\omega\in E_N$, whence $\omega\in \uplus_{k=1}^n E_k$ and $1_{E\setminus {\uplus_{k=1}^n E_k}}\brac{\omega}=0$ for all $n\geq N$. Hence for all $\omega\in \Omega$ there is $N\geq 1$ such that for all $n\geq 1$ \[\abs{f\brac{\omega} 1_E\brac{\omega} - \sum_{k=1}^n f \brac{\omega}1_{E_k}\brac{\omega} } = 0\]

Now, $\abs{f 1_A}\leq \abs{f}$ everywhere on $\Omega$ for any $A\in \Fcal$, whence $\sum_{k=1}^n f 1_{E_k}\in L^1_\Cplx$ for all $n\geq 1$. Since $\sum_{k=1}^n f 1_{E_k}\overset{\Cplx}{\to} f 1_E$ everywhere on $\Omega$ and $\abs{f}\in L^1_\Real$ for all $n\geq 1$, $\int_\Omega \abs{f 1_E - \sum_{k=1}^n f 1_{E_k} } d\mu \to 0$ by virtue of theorem 23 (DCT). Further theorem 24 and linearity of the complex Lebesgue integral imply \[\abs{\nu\brac{E} - \sum_{k=1}^n \nu\brac{E_k} } = \abs{ \int_\Omega f 1_E - \sum_{k=1}^n f 1_{E_k} d\mu }\leq \int_\Omega \abs{f 1_E - \sum_{k=1}^n f 1_{E_k} } d\mu \to 0\] whence $\sum_{k=1}^n \nu\brac{E_k}\overset{\Cplx}{\to} \nu\brac{E}$. Therefore the series $\sum_{k=1}^\infty \nu\brac{E_k}$ converge to $\nu\brac{E}$ in $\Cplx$.\\

\noindent \textbf{Definition} 94.
Let $\mu$ be a complex measure on a measurable space $\brac{\Omega, \Fcal}$. The total variation of $\mu$ is the map $\abs{\mu}:\Fcal\to\Zinf$ defined for every $E\in \Fcal$ by \[\abs{\mu}\brac{E}\defn \sup\obj{ \induc{\sum_{n\geq 1} \abs{\mu\brac{E_n}}\,}\, \brac{E_n}_{n\geq1}\in \Fcal,\,E=\uplus_{n=1}^\infty E_n }\]

\label{thm:tot_var_is_measure} \noindent \textbf{Theorem} 11-7.
For any $\mu\in M^1\brac{\Omega, \Fcal}$ the total variation of $\mu$, $\abs{\mu}\brac{\cdot}$ is a measure on $\brac{\Omega, \Fcal}$.

For any $E\in\Fcal$, the collection $\brac{E_n}_{n\geq1}$ defined as $E_1 \defn E$ and $E_n\defn \emptyset$ for all $n\geq 2$ constitutes a measurable partition of $E$. Therefore theorem Sup-B-3 and the definition of the total variation imply \[\abs{\mu\brac{E}} = \sum_{n\geq 1} \abs{\mu\brac{E_n}} \leq \abs{\mu}\brac{E}\]

Let $\brac{E_k}_{k\geq1}$ be any measurable partition of $\emptyset$. Then $E_n=\emptyset$ for all $n\geq 1$, whence $\sum_{n\geq 1}\abs{\mu\brac{E_n}} = 0$. Therefore $\abs{\mu}\brac{\emptyset}=0$, because $\abs{\mu}\brac{\emptyset}$ is the least upper bound and $0\leq \abs{\mu}\brac{\emptyset}$.

Let $E\in \Fcal$ and $\brac{E_n}_{n\geq1}$ be a measurable partition on $E$. Let $\brac{x_n}_{n\geq 1}\in\Real$ be such that $x_n<\abs{\mu}\brac{E_n}$ for all $n\geq 1$. Thus for every $n\geq1$ there is a measurable partition $\brac{E_n^p}_{p\geq1}$ of $E_n$ such that $x_n<\sum_{p\geq 1} \abs{\mu\brac{E_n^p}}$.

Since $\brac{E_n}_{n\geq1}$ are pairwise disjoint, $\brac{E_n^p}_{p\geq1}$ are pairwise disjoint and $E=\uplus_{n,p\geq1} E_n^p$ is a countable union, $\brac{E_n^p}_{n,p\geq1}$ is a measurable partition of $E$. By theorem Sup-B-3 $\sum_{n\geq 1}\sum_{p\geq 1} \abs{\mu\brac{E_n^p}} = \sum_{n,p\geq 1} \abs{\mu\brac{E_n^p}}$, whence $\sum_{n=1}^N x_n < \sum_{n=1}^\infty\sum_{p\geq 1} \abs{\mu\brac{E_n^p}}\leq \abs{\mu}\brac{E}$ for all $N\geq 1$.

%% Make this more rigorous...
Thus for any arbitrary $\brac{x_n}_{n\geq 1}$ with $x_n<\abs{\mu}\brac{E_n}$ for all $n\geq 1$, it is true that $\sum_{n=1}^N x_n < \abs{\mu}\brac{E}$ for all $N\geq 1$. Since $\sum_{n=1}^N x_n < \sum_{n=1}^N \abs{\mu}\brac{E_n}$, it must be true that $\sum_{n=1}^N \abs{\mu}\brac{E_n}\leq \abs{\mu}\brac{E}$ for every $N\geq 1$. Therefore $\sum_{n\geq 1} \abs{\mu}\brac{E_n}\leq \abs{\mu}\brac{E}$.

Let $\brac{A_p}_{p\geq1}$ be another arbitrary measurable partition of $E$. Since $\brac{A_p\cap E_n}_{n\geq1}$ is a measurable partition of $A_p$ for every $p\geq 1$ and $\mu$ is a complex measure on $\brac{\Omega, \Fcal}$, the series $\sum_{n=1}^\infty \mu\brac{A_p\cap E_n}$ converges to $\mu\brac{A_p}$ in $\Cplx$. Therefore \[\abs{\mu\brac{A_p}} \leq \sum_{n\geq 1} \abs{\mu\brac{A_p\cap E_n}}\] because $\abs{\cdot}$ is $\Tcal_\Cplx$-$\Tcal_\Real^+$ continuous. However $\brac{A_p\cap E_n}_{p\geq1}$ is a measurable partition of $E_n$ for every $n\geq 1$, whence for all $n\geq1$ \[\sum_{p\geq 1} \abs{\mu\brac{A_p\cap E_n}}\leq \abs{\mu}\brac{E_n}\] Therefore by theorem Sup-B-3 \[\sum_{p\geq1 }\abs{\mu\brac{A_p}} \leq \sum_{p\geq 1}\sum_{n\geq 1} \abs{\mu\brac{A_p\cap E_n}} = \sum_{n\geq 1}\sum_{p\geq 1} \abs{\mu\brac{A_p\cap E_n}}\leq \sum_{n\geq 1}\abs{\mu}\brac{E_n}\] whence $\abs{\mu}\brac{E}\leq \sum_{n\geq 1}\abs{\mu}\brac{E_n}$ by definition of the least upper bound.

To summarize, the map $\abs{\mu}$ is such that $\abs{\mu}\brac{\emptyset}=0$ and $\abs{\mu}\brac{\uplus_{n\geq 1} E_n} = \sum_{n\geq 1} \abs{\mu}\brac{E_n}$ for any collection of pairwise disjoint measurable sets $\brac{E_n}_{n\geq1}\in \Fcal$. Therefore $\mu$ is a usual measure on $\brac{\Omega, \Fcal}$.\\

\label{thm:main_calculus_theorem} \noindent \textbf{Theorem} 11-8.
Let $a<b\in \Real$ and $F\in C^1\brac{\clo{a,b}\to \Real}$ and put \[H\brac{x}\defn \int_a^x F'\brac{t} dt\] for every $x\in \clo{a,b}$ where the signed Lebesgue integral is used. Then $H\in C^1\brac{\clo{a,b}, \Real}$, $H'=F'$ and $F\brac{b}-F\brac{a} = \int_a^b F'\brac{t} dt$.

First, let's establish the space in which the Lebesgue integral is defined. The measure space is $\brac{\clo{a,b}, \borel{\clo{a,b}}, dt}$, where $dt\defn \induc{dx}_{\clo{a,b}}$ is the induced Lebesgue measure. Now, every map continuous with respect to $\Tcal_{\clo{a,b}}$-$\Tcal_\Real$ is automatically measurable with respect to $\borel{\clo{a,b}}$-$\borel{\Real}$. Furthermore, since $F'$ is $\clo{a,b}$-$\Real$ continuous and $\clo{a,b}$ is compact in $\Real$ by theorem 34, theorem 37 implies that there must exist $x_m,x_M\in \clo{a,b}$ such that $F'\brac{x_m}\leq F'\brac{x}\leq F'\brac{x_M}$ for all $x\in \clo{a,b}$. Thus there must exist $M\in \Real^+$ such that $\abs{F'}\leq M$ on $\clo{a,b}$, whence $F'\in L^1_\Real\brac{\clo{a,b}, \borel{\clo{a,b}}, dt}$. Therefore for every $x\in \clo{a,b}$ the map $H$ is actually a well-defined partial signed Lebesgue integral over $\brac{\clo{a,b}, \borel{\clo{a,b}}, dt}$: \[H\brac{x}=\int_{\clo{a,b}} 1_{\clo{a,x}}\brac{t} F'\brac{t} dt\]

Let's define an auxiliary map $I$ as follows: for all $t\in \clo{a,b}$ if $x\geq x_0$ put $I\brac{t,x,x_0}\defn 1_{\ploc{x_0,x}}\brac{t}$ and $I\brac{t,x,x_0}\defn -1_{\ploc{x,x_0}}\brac{t}$ when $x<x_0$. Therefore for all $x_0, x\in \clo{a,b}$ it is true that \[H\brac{x} - H\brac{x_0} = \int_{\clo{a,b}} \brac{ 1_{\clo{a,x}} - 1_{\clo{a,x_0}} } F'\brac{t} dt = \int_{\clo{a,b}} I\brac{t,x,x_0} F'\brac{t} dt\] Also note that $\int_{\clo{a,b}} I\brac{t,x,x_0} dt = x-x_0$ and $\int_{\clo{a,b}} \abs{I\brac{t,x,x_0}} dt = \abs{x-x_0}$.

For any $x_0\in \clo{a,b}$ theorem 24 implies that \[\abs{H\brac{x}-H\brac{x_0}}\leq \int_{\clo{a,b}} \abs{I\brac{t,x,x_0}} \abs{F'\brac{t}} dt\leq M \abs{x-x_0}\] whence for every $\epsilon>0$ for $\delta\defn\frac{\epsilon}{M+1}>0$ it must be true that $\abs{H\brac{x}-H\brac{x_0}}\leq M \delta < \epsilon$ for all $x\in \clo{a,b}$ with $\abs{x-x_0}<\delta$. In conclusion, by the $\epsilon$-$\delta$ criterion of continuity in metric spaces, the map $H:\brac{\clo{a,b}, \Tcal_{\clo{a,b}}}\to\brac{\Real, \Tcal_\Real}$ is continuous.

Since $F'$ is $\clo{a,b}$-$\Real$ continuous, for every $x_0 \in \clo{a,b}$ and any $\epsilon>0$ there is $\delta>0$ such that $\abs{F'\brac{x}-F'\brac{x_0}}<\epsilon$ for all $x\in \clo{a,b}$ with $\abs{x-x_0}<\delta$. Note that for every $x\in \clo{a,b}$ with $x\neq x_0$ it is true that \[ F'\brac{x_0} = \frac{1}{x-x_0}\int_{\clo{a,b}} I\brac{t,x,x_0} F'\brac{x_0} dt\] Therefore for all $x\in \clo{a,b}$ with $x\neq x_0$ by theorem 24 \[\abs{ \frac{H\brac{x}-H\brac{x_0}}{x-x_0} - F'\brac{x_0} } \leq \frac{1}{\abs{x-x_0}} \int_{\clo{a,b}} \abs{\brac{F'\brac{t} - F'\brac{x_0}} I\brac{t,x,x_0} } dt\] %= \abs{ \frac{1}{x-x_0}\int_{\clo{a,b}} I\brac{t,x,x_0} \brac{F'\brac{t} - F'\brac{x_0}} dt }
Since $\abs{F'\brac{t} - F'\brac{x_0}}<\epsilon$ for all $x\in \clo{a,b}$ with $0<\abs{x-x_0}<\delta$ for all $t\in \clo{a,b}$ \[\abs{I\brac{t,x,x_0}} \abs{F'\brac{t} - F'\brac{x_0} }<\epsilon \abs{I\brac{t,x,x_0}}\] from where it follows that \[\abs{ \frac{H\brac{x}-H\brac{x_0}}{x-x_0} - F'\brac{x_0} } \leq \frac{\epsilon}{\abs{x-x_0}} \int_{\clo{a,b}} \abs{I\brac{t,x,x_0}} dt = \frac{\epsilon}{\abs{x-x_0}}\abs{x-x_0}\] Therefore at every $x_0\in \clo{a,b}$ the function $H$ is differentiable and its derivative is equal to $F'\brac{x_0}$, where one-sided derivatives are used at the edges. In conclusion, $H\in C^1\brac{\clo{a,b}\to \Real}$, since $F'$ is continuous and $H'=F'$.

Note that \[\abs{H\brac{a}} = \abs{ \int_{\clo{a,b}} 1_{\obj{a}}\brac{t} F'\brac{t} dt }\leq M dx\brac{\obj{a}} = 0\]

Since $H,F\in C^1\brac{\clo{a,b}\to\Real}$, the map $G\defn H-F$ is continuous on $\brac{a,b}$ with a well-defined derivative on $\brac{a,b}$. Therefore $G$ satisfies the conditions of theorem 39 (Taylor-Lagrange), which gives an $x_0\in \brac{a,b}$ such that $G\brac{b}-G\brac{a} = \brac{b-a} G'\brac{x_0}$. However, $H\brac{a}=0$ and $G'\brac{x} = H'\brac{x} - F'\brac{x} = 0$ for all $x\in \brac{a,b}$, whence after some mild rearrangement \[F\brac{b}-F\brac{a} = H\brac{b} = \int_{\clo{a,b}} 1_{\clo{a,b}} F'\brac{t} dt = \int_a^b F'\brac{t} dt\]\\

\label{thm:dx_n_translation_invariance} \noindent \textbf{Theorem} 11-9.
Let $\tau_u:\Real^n\to \Real^n$ be the translation map $\tau_u\brac{x}\defn x+u$, where $u\in \Real^n$. Then the Lebesgue measure $dx$ on $\brac{\Real^n, \borel{\Real^n}}$ is invariant by translation $\tau_u$.

Recall that $\borel{\Real^n} = \sigma\brac{ \Ccal^n }$, where $\Ccal\defn\obj{ \induc{\clo{a,b}}\, a,b\in \Real,\, a\leq b }$, and by theorem 7-2 the Lebesgue measure on $\Real^n$, $n\geq 1$, is the unique $\sigma$-finite product measure $dx$ on $\brac{\Real^n, \borel{\Real^n}}$ such that for all \[dx\brac{\clo{a_1, b_1}\times \ldots \times \clo{a_n, b_n}} = \prod_{i=1}^n \brac{b_i-a_i}\] for all $\clo{a_1, b_1}\times \ldots \times \clo{a_n, b_n}\in \Ccal^n$.

Let $u\in \Real^n$ and $\tau_u:\Real^n\to \Real^n$ be the translation map $\tau_u\brac{x}\defn x+u$. For any $K=\clo{a_1, b_1}\times \ldots \times \clo{a_n, b_n}\in \Ccal^n$ \[\tau_u^{-1}\brac{\clo{a_1, b_1}\times \ldots \times \clo{a_n, b_n}} = \clo{a_1-u_1, b_1-u_1}\times \ldots \times \clo{a_n-u_n, b_n-u_n}\] whence \[dx\brac{\tau_u^{-1}\brac{K}} = \prod_{i=1}^n \brac{b_i-u_i-\brac{a_i-u_i}} = dx\brac{K} \] Therefore by theorem 1 (the Dynkin system theorem) or by theorem 7-2 $dx\brac{B} = dx\brac{\tau_u^{-1}\brac{B}}$ for all $B\in \borel{\Real^n}$.\\


\label{thm:dx_n_integral_translation} \noindent \textbf{Theorem} 11-10.
For every $f\in L^1_\Cplx\brac{\Real^n, \borel{\Real^n}, dx}$ and for all $u\in \Real^n$ \[\int_{\Real^n} f\brac{x+u} dx = \int_{\Real^n} f\brac{x} dx\]

Indeed, apply the Standard Machine Argument which goes like thus. For every $A\in \borel{\Real^n}$ by theorem 11-9 it is true that \[\int_{\Real^n} 1_A\brac{x+u} dx = \int_{\Real^n} 1_{\tau_u^{-1}\brac{A}}\brac{x} dx  = dx\brac{\tau_u^{-1}\brac{A}} = dx\brac{A} = \int_{\Real^n} 1_A\brac{x} dx\] By linearity of the usual Lebesgue integral for any simple $s:\brac{\Real^n, \borel{\Real^n}}\to\Real^+$ with partition $s=\sum_{i=1}^m \alpha_i 1_{A_i}$ it is true that \[\int_{\Real^n} s\brac{x+u} dx = \sum_{i=1}^m \alpha_i \int_{\Real^n} 1_{A_i}\brac{x+u} dx = \sum_{i=1}^m \alpha_i \int_{\Real^n} 1_{A_i}\brac{x} dx = \int_{\Real^n} s\brac{x} dx\] For any non-negative and measurable $f:\brac{\Real^n, \borel{\Real^n}}\to\Zinf$ by theorem 18 there exists $\brac{s_k}_{k\geq1}$ all simple on $\brac{\Real^n, \borel{\Real^n}}$ such that $s_k\uparrow f$, whence by the MCT \begin{align*}\int_{\Real^n} s_k\brac{x+u} dx &\uparrow \int_{\Real^n} f\brac{x+u} dx\\ \int_{\Real^n} s_k\brac{x} dx &\uparrow \int_{\Real^n} f\brac{x} dx\end{align*} Therefore for any $u\in \Real^n$ it is true that $\int_{\Real^n} f\brac{x+u} dx = \int_{\Real^n} f\brac{x} dx$.

For any $f\in L^1_\Real\brac{\Real^n, \borel{\Real^n}, dx}$ it must be $\int_{\Real^n} f\brac{x+u} dx = \int_{\Real^n} f\brac{x} dx <+\infty$, whence $\int_{\Real^n} f^+\brac{x+u} dx$ and $\int_{\Real^n} f^-\brac{x+u} dx$ are finite. Therefore the signed Lebesgue integral of $f\brac{x+u}$ is well-defined and by its linearity\begin{align*}
\int_{\Real^n} f\brac{x+u} dx &= \int_{\Real^n} f^+\brac{x+u} dx - \int_{\Real^n} f^-\brac{x+u} dx\\&= \int_{\Real^n} f^+\brac{x} dx - \int_{\Real^n} f^-\brac{x} dx = \int_{\Real^n} f\brac{x} dx
\end{align*} Finally for any $f\in L^1_\Cplx\brac{\Real^n, \borel{\Real^n}, dx}$ it is true that the complex Lebesgue integral of $f\brac{x+u}$ is well-defined and by definition \begin{align*}\int_{\Real^n} f\brac{x+u} dx &= \int_{\Real^n} \re f\brac{x+u} dx + i \int_{\Real^n} \im f\brac{x+u} dx\\ &= \int_{\Real^n} \re f\brac{x} dx + i \int_{\Real^n} \im f\brac{x} dx = \int_{\Real^n} f\brac{x} dx\end{align*}\\


Consider a map $\sin \theta \in C^\infty\brac{\clo{-\frac{\pi}{2}, \frac{\pi}{2}}\to\Real}$. Then $F \defn \sin \theta$ is such that $F\in C^1\brac{\clo{-\frac{\pi}{2}, \frac{\pi}{2}}\to\Real}$ and $F' = \cos\theta$. By theorem 11-8 \[\frac{1}{2\pi}\int_{-\frac{\pi}{2}}^\frac{\pi}{2} \cos \theta d\theta = \frac{1}{2\pi}\brac{\sin\frac{\pi}{2} - \sin -\frac{\pi}{2}} = \frac{1}{\pi}\]

Next, observe that $\cos^+\brac{\theta-2\pi k} = \cos^+\theta$ due to periodicity and $\cos^+\brac{\alpha - \theta} = \cos^+\brac{\theta-\alpha}$ for every $\alpha\in \Real$ due to the fact that $\cos$ is an even function.Therefore \[\int_{-\pi}^{+\pi} \cos^+\brac{\alpha - \theta} d\theta = \int_\Real 1_{\clo{-\pi, \pi}}\brac{\theta} \cos^+\brac{\theta-\alpha} d\theta\] Then by theorem 11-10 on $\brac{\Real, \borel{\Real}, dx}$ \[\int_{-\pi}^{+\pi} \cos^+\brac{\alpha - \theta} d\theta = \int_\Real 1_{\clo{-\pi, \pi}}\brac{\theta+\alpha} \cos^+\brac{\theta} d\theta = \int_\Real 1_{\clo{-\pi-\alpha, \pi-\alpha}}\brac{\theta} \cos^+\brac{\theta} d\theta\] Ultimately for all $\alpha\in \Real$ it is true that \[\int_{-\pi}^{+\pi} \cos^+\brac{\alpha - \theta} d\theta = \int_{-\pi-\alpha}^{\pi-\alpha} \cos^+\brac{\theta} d\theta \]

Now, for any $\alpha\in \Real$ let $k\defn \floor{\sfrac{\alpha}{2\pi}}$. Then $k\in \mathbb{Z}$ and is such that $2\pi k\leq \alpha < 2\pi k + 2\pi\leq \alpha + 2\pi$, whence $-\alpha - 2\pi \leq -2\pi k - 2\pi< -\alpha \leq -2\pi k$ and $-\alpha - \pi \leq -2\pi k - \pi< \pi-\alpha \leq -2\pi k+\pi$. Therefore by theorem 11-10 \[\int_{\pi-\alpha}^{-2\pi k+\pi} \cos^+\brac{\theta} d\theta = \int_{-\pi-\alpha}^{-2\pi k-\pi} \cos^+\brac{\theta+2\pi} d\theta\]

Due to theorem 24, the basic properties of the usual Lebesgue integral and specific features of the Lebesgue measure on $\Real$, the or any $a,b\in \Real$ and any $f\in L^1_\Cplx\brac{\Real, \borel{\Real}, dx}$ \[\int_\Real 1_{\clo{a,b}} f dx = \int_\Real 1_{\brac{a,b}} f dx\] 

Therefore by linearity of the signed integral\begin{align*} \int_{-\pi-\alpha}^{\pi-\alpha} \cos^+\brac{\theta} d\theta &= \int_{-\pi-\alpha}^{-2\pi k-\pi} \cos^+\brac{\theta} d\theta + \int_{-2\pi k-\pi}^{\pi-\alpha} \cos^+\brac{\theta} d\theta \\&= \int_{\pi-\alpha}^{-2\pi k+\pi} \cos^+\brac{\theta} d\theta + \int_{-2\pi k-\pi}^{\pi-\alpha} \cos^+\brac{\theta} d\theta = \int_{-2\pi k-\pi}^{-2\pi k+\pi} \cos^+\brac{\theta} d\theta\end{align*}  whence by theorem 11-10 for any $\alpha\in \Real$ it must be true that \[\int_{-\pi-\alpha}^{\pi-\alpha} \cos^+\brac{\theta} d\theta = \int_{-\pi}^{+\pi} \cos^+\brac{\theta-2\pi k} d\theta = \int_{-\pi}^{+\pi} \cos^+\brac{\theta} d\theta\]

Finally, since $1_{\clo{-\pi,-\frac{\pi}{2}}}\cos^+$ and $1_{\clo{+\frac{\pi}{2}},+\pi}\cos^+$ are identically zero everywhere on $\Real$, (forgive a slight abuse of notation here) \[\int_{-\pi}^{+\pi} \cos^+\brac{\theta} d\theta = \brac{\int_{-\pi}^{-\frac{\pi}{2}} + \int_{-\frac{\pi}{2}}^{+\frac{\pi}{2}} + \int_{+\frac{\pi}{2}}^{+\pi}  } \cos^+\brac{\theta} d\theta = \int_{-\frac{\pi}{2}}^{+\frac{\pi}{2}} \cos^+\brac{\theta} d\theta\] Therefore for all $\alpha\in \Real$ it is true that \[\frac{1}{2\pi}\int_{-\pi}^{+\pi} \cos^+\brac{\alpha - \theta} d\theta = \frac{1}{2\pi} \int_{-\pi}^{+\pi} \cos^+\brac{\theta} d\theta = \frac{1}{2\pi}\int_{-\frac{\pi}{2}}^{+\frac{\pi}{2}} \cos\brac{\theta} d\theta = \frac{1}{\pi}\]

\label{thm:reverse_triangle_law} \noindent \textbf{Theorem} 11-11.
For any $\brac{z_k}_{k=1}^N\in \Cplx$ there is $S\subseteq \obj{1,\ldots,N}$ such that \[\frac{1}{\pi} \sum_{k=1}^N \abs{z_k} \leq \abs{ \sum_{k\in S} z_k }\]

Since $\brac{z_k}_{k=1}^N\in \Cplx$, there exist $\alpha_k\in \Real$ such that $z_k = \abs{z_k}e^{i \alpha_k}$ for all $k=1,\ldots,N$. For all $\theta\in \clo{-\pi, \pi}$ define $S\brac{\theta}\defn \obj{\induc{ k=1,\ldots,N} \cos\brac{\alpha_k-\theta}>0 }$.

For any $\theta\in \clo{-\pi,\pi}$ it is true that $\abs{e^{-i\theta}}=1$ whence \[\abs{\sum_{k\in S\brac{\theta}} z_k e^{-i\theta} } = \abs{e^{-i\theta}} \abs{ \sum_{k\in S\brac{\theta}} z_k }= \abs{\sum_{k\in S\brac{\theta}} z_k }\] Since $\abs{\re z} \leq \abs{z}$ for any $z\in \Cplx$, $z_k=\abs{z_k} e^{i\alpha_k}$, and $\cos\brac{\alpha_k-\theta}>0$ for all $k\in S\brac{\theta}$, it must be true that \[\abs{\sum_{k\in S\brac{\theta}} z_k e^{-i\theta} } = \abs{\sum_{k\in S\brac{\theta}} \abs{z_k} e^{i\brac{\alpha_k-\theta} } } \geq \abs{ \re\sum_{k\in S\brac{\theta}} \abs{z_k} e^{i\brac{\alpha_k-\theta}} } = \sum_{k\in S\brac{\theta}} \abs{z_k} \cos\brac{\alpha_k-\theta}\]

Define $\phi\clo{-\pi, +\pi} \to \Real$ by $\phi\brac{\theta}\defn \sum_{k=1}^N \abs{z_k} \cos^+\brac{\alpha_k-\theta}$. The function $\phi$ is non-negative as it is a sum of non-negative functions. Since the map $\brac{t_k}_{k=1}^N\to \sum_{k=1}^N t_k$ is $\Real^N$-$\Real$ continuous, $\cos^+$ is $\Real$-$\Real$ continuous and $\theta\to\brac{\theta}_{k=1}^N$ is $\Real$-$\Real^N$ continuous, the map $\phi\brac{\theta}$ is $\clo{-\pi, +\pi}$-$\Real$ continuous and therefore is a non-negative measurable map on $\brac{\clo{-\pi, +\pi}, \borel{\clo{-\pi, +\pi}}}$. Therefore from prior calculations, due to linearity of the usual Lebesgue integral and the fact that the signed Lebesgue integral is an extension of the usual Lebesgue integral on $\brac{\clo{-\pi, +\pi}, \borel{\clo{-\pi, +\pi}},dx}$ \[\frac{1}{2\pi}\int_{-\pi}^{+\pi} \phi\brac{\theta} d\theta = \sum_{k=1}^N \frac{1}{2\pi} \int_{-\pi}^{+\pi} \abs{z_k}\cos^+\brac{\alpha_k - \theta} d\theta = \frac{1}{\pi} \sum_{k=1}^N \abs{z_k}\]

Since $\clo{-\pi, +\pi}$ is compact by theorem 34, by theorem 35 there must be such $\theta_0\in \clo{-\pi,+\pi}$ that $\phi\brac{\theta}\leq \phi\brac{\theta_0}$, which implies \[\frac{1}{\pi} \sum_{k=1}^N \abs{z_k} \leq \frac{1}{2\pi}\int_{-\pi}^{+\pi} \phi\brac{\theta_0} d\theta = \phi\brac{\theta_0}\] However $\phi\brac{\theta_0} = \brac{ \sum_{k\notin S\brac{\theta_0}} + \sum_{k\in S\brac{\theta_0}} } \abs{z_k} \cos^+\brac{\alpha_k-\theta_0}$ and $\cos\brac{\alpha_k-\theta_0}\leq 0$ for all $k\notin S\brac{\theta_0}$, whence it is true that \[\frac{1}{\pi} \sum_{k=1}^N \abs{z_k} \leq \sum_{k\in S\brac{\theta}} \abs{z_k} \cos\brac{\alpha_k-\theta} = \abs{\sum_{k\in S\brac{\theta}} z_k } \] Therefore for any $\brac{z_k}_{k=1}^N\in \Cplx$ there is $S\subseteq \obj{1,\ldots,N}$ such that \[\frac{1}{\pi} \sum_{k=1}^N \abs{z_k} \leq \abs{ \sum_{k\in S} z_k }\]\\

\label{thm:inf_tot_var_lemma} \noindent \textbf{Theorem} 11-12.
Let $\mu \in M^1\brac{\Omega, \Fcal}$. Suppose $\abs{\mu}\brac{E} = +\infty$ for some $E\in \Fcal$. Then exist $A,B\in \Fcal$ such that $E=A\uplus B$, $\abs{\mu\brac{A}}, \abs{\mu\brac{B}}>1$ and either $\abs{\mu}\brac{A} = +\infty$ or $\abs{\mu}\brac{B}=+\infty$.

Since $\mu\brac{E}\in \Cplx$, the value $t\defn \pi \brac{ 1 + \abs{\mu\brac{E}}}\in \Real^+$. Since $t<\abs{\mu}\brac{E}$, there is a measurable partition $\brac{E_n}_{n\geq1}$ of $E$ such that $t < \sum_{n\geq 1} \abs{\mu\brac{E_n}}$. Furthermore by definition of an infinite sum of non-negative values for this $t$ there exists $N\geq 1$ such that $t<\sum_{k=1}^N \abs{\mu\brac{E_n}}$.

By theorem 11-11 there is $S\subseteq \obj{1,\ldots, N}$ such that \[\sum_{k=1}^N \abs{\mu\brac{E_n}} \leq \pi \abs{ \sum_{n\in S} \mu\brac{E_n} }\] Now $\brac{E_n}_{n\in S}$ is a measurable partition of $A\defn \uplus_{n\in S} E_n$, whence $\mu\brac{A}=\sum_{n\in S} \mu\brac{E_n}$, because $\mu$ is a complex measure. Therefore $\frac{t}{\pi}<\abs{\mu\brac{A}}$.

Since $A$ and $B\defn E\setminus A\in \Fcal$ constitute another measurable partition of  $E=A\uplus B$, $\mu\brac{E}=\mu\brac{A}+\mu\brac{B}$ which implies that \[\abs{\mu\brac{E}}\leq \abs{\mu\brac{A}}+\abs{\mu\brac{E}-\mu\brac{A}} = \abs{\mu\brac{A}}+\abs{\mu\brac{B}}\] since $\mu\brac{B}\in \Cplx$. Thus $\abs{\mu\brac{B}}\geq \abs{\mu\brac{A}}-\abs{\mu\brac{E}}$.

Thus $A,B\in\Fcal$ are such that $E=A\uplus B$ and \[\abs{\mu\brac{A}}>\frac{1}{\pi} t > \abs{\mu\brac{E}} + 1\,\text{and}\,\abs{\mu\brac{B}} > \abs{\mu\brac{E}} + 1 - \abs{\mu\brac{E}}\] These sets are such that $\abs{\mu}\brac{E}=\abs{\mu}\brac{A}+\abs{\mu}\brac{B}$, because $\abs{\mu}$ is a usual measure on $\brac{\Omega, \Fcal}$, whence either $\abs{\mu}\brac{A}=+\infty$ or $\abs{\mu}\brac{B}=+\infty$.\\

\label{thm:tot_var_finte_meas} \noindent \textbf{Theorem} 57.
Let $\mu$ be a complex measure on a measurable space $\brac{\Omega, \Fcal}$. Then its total variation $\abs{\mu}$ is a finite measure on $\brac{\Omega, \Fcal}$.

Indeed, suppose that $\abs{\mu}\brac{\Omega}=+\infty$. By theorem 11-12 there exist $A_1, B_1\in \Fcal$ with $\Omega = A_1\uplus B_1$, $\abs{\mu\brac{A_1}}>1$ and $\abs{\mu}\brac{B_1}=+\infty$.

Suppose $A_k,B_k\in \Fcal$ are such that $A_k\cap B_k = \emptyset$, $\abs{\mu\brac{A_k}}>1$, and $\abs{\mu}\brac{B_k}=+\infty$. By theorem 11-12 there exist $A_{k+1}, B_{k+1}\in \Fcal$ with $B_k = A_{k+1}\uplus B_{k+1}$, $\abs{\mu\brac{A_{k+1}}} > 1$ and $\abs{\mu}\brac{B_{k+1}}=+\infty$. Since $A_{k+1}\subseteq B_k$, it must be true that $A_k\cap A_{k+1} = \emptyset$.

Therefore there exists a family of pairwise disjoint sets $\brac{A_k}_{k\geq1}\in \Fcal$ with $\abs{\mu\brac{A_k}}>1$ for all $k\geq 1$. By theorem 56 $\brac{\mu\brac{A_k}}_{k\geq1}\in \Cplx$ does not have the permutation property, since by construction $\sum_{k\geq 1} \abs{\mu\brac{A_k}} = +\infty$. This implies that there is a bijection such $\sigma:\mathbb{N}\to\mathbb{N}$ that the series $\sum_{k=1}^\infty \mu\brac{A_{\sigma\brac{k}}}$ fails to converge in $\Cplx$. Since $\uplus_{k\geq 1} A_k = \uplus_{k\geq 1} A_{\sigma\brac{k}}$, $\brac{A_{\sigma\brac{k}}}_{k\geq 1}$ is a measurable partition of $A\defn \uplus_{k\geq 1} A_k$ for which the series $\sum_{k=1}^\infty \mu\brac{A_k}$ does not converge in $\Cplx$ to anything, not to mention to $\mu\brac{A}$, which contradicts the fact that $\mu$ is a complex measure. Therefore $\abs{\mu}\brac{\Omega}<+\infty$ and $\abs{\mu}$ is a finite measure on $\brac{\Omega, \Fcal}$.\\

\label{thm:cplx_meas_linear_space} \noindent \textbf{Theorem} 11-13.
Let $\mu, \lambda\in M^1\brac{\Omega, \Fcal}$ and $\alpha\in \Cplx$. For every $E\in \Fcal$ define\begin{align*}\brac{ \mu+\lambda }\brac{E} &\defn \mu\brac{E}+\lambda\brac{E}\\\brac{ \alpha \lambda }\brac{E} &\defn \alpha \cdot \lambda\brac{E}\end{align*} Under such operations of addition and scalar multiplication the set of all complex measure on $\brac{\Omega, \Fcal}$ is a $\Cplx$-linear space.

Indeed, let $E\in \Fcal$ and $\brac{E_n}_{n\geq 1}$ be a measurable partition of $E$. Since the series $\sum_{n=1}^\infty \mu\brac{E_n}$ and $\sum_{n=1}^\infty \mu\brac{E_n}$ converge in $\Cplx$ to $\mu\brac{E}$ and $\lambda\brac{E}$ respectively. Since the function $\brac{x,y}\to x+y$ is $\Cplx^2$-$\Cplx$ continuous it must be true that the series $\sum_{n\geq 1} \brac{ \mu+\lambda }\brac{E_n}$ converges in $\Cplx$ to $\mu\brac{E} + \lambda\brac{E} = \brac{ \mu+\lambda }\brac{E}$. As for every $\alpha$ the map $x\to \alpha \cdot z$ is $\Cplx$-$\Cplx$ continuous, the series $\sum_{n\geq 1} \brac{ \alpha\lambda }\brac{E_n}$ must converge in $\Cplx$ to $\alpha \lambda\brac{E} = \brac{ \alpha \lambda }\brac{E}$. Therefore $\brac{\mu+\lambda}$ and $\alpha \lambda$ are indeed complex measures on $\brac{\Omega, \Fcal}$, whence $M^1\brac{\Omega, \Fcal}$ is closed under $\Cplx$-linear combinations.

Define the zero of $M^1$ as $0_M\brac{E}\defn0$ for all $E\in \Fcal$ and for $\mu \in M^1$ put $\brac{-\mu}\brac{E} \defn \brac{-1}\mu\brac{E}$ according to the rules of scalar multiplication exposed above. Indeed, $0_M$ is a complex measure, since it is zero everywhere on $\Fcal$. As for $-\mu$, it is a complex measure because flipping the sign of all elements of the absolutely convergent series does not affect the convergence. For any $\mu\in M^1$, obviously, $0_M + \mu = \mu$ and $\mu + \brac{-\mu}=0_M$. Commutativity and associativity of addition, as well as all properties of scalar multiplication follow from continuity of finite addition and scaling.\\

\noindent \textbf{Definition} 95.
Let $\Hcal$ be a $K$-vector space, where $K=\Real$ or $\Cplx$. A norm on $\Hcal$ is any map $N:\Hcal\to \Real^+$ with the following properties: \begin{itemize}
	\item For any $x\in \Hcal$, $N\brac{x} = 0$ if and only if $x=0_\Hcal$
	\item For any $x\in \Hcal$ and $\alpha \in K$, $N\brac{\alpha x} = \abs{\alpha} N\brac{x}$
	\item For all $x,y\in\Hcal$, $N\brac{x+y}\leq N\brac{x}+N\brac{y}$
\end{itemize}

Recall that the $L^p$-norm $\nrm{\cdot}_p$ is not a norm in the just introduced sense as it fails to have the first property. Again, by theorem 9-4 $\nrm{f}_p=0$ implies that $f=0$ $\mu$-a.s. for every $f\in L^p_K\brac{\Omega, \Fcal, \mu}$.

If $\brkt{\cdot, \cdot}$ is an inner product on some $K$-vector space $\Hcal$, then the map $\nrm{\cdot} \defn \sqrt{\brkt{\cdot, \cdot}}$ is a norm in the above sense. Indeed, by theorem 50 this norm satisfies the triangle inequality, while the rest of the properties follow from the basic properties of the inner product.

Consider the space of complex measures $M^1\brac{\Omega, \Fcal}$ and define a norm on $M^1$ as $\nrm{\mu}\defn \abs{\mu}\brac{\Omega}$. Since by theorem 57 $\abs{\mu}\brac{\Omega}<+\infty$ for every $\mu\in M^1$, $\nrm{\cdot}:M^1\brac{\Omega, \Fcal}\to\Real^+$. Furthermore by theorem 11-7 $\abs{\mu}$ is a usual measure on $\brac{\Omega, \Fcal}$, $\abs{\mu}\brac{\Omega}\geq \abs{\mu}\brac{E}$ for all $E\in \Fcal$, whence $\abs{\mu\brac{E}}\leq \nrm{\mu}$.

If $\nrm{\mu}=0$, then $\mu\brac{E} = 0$ for all $E\in \Fcal$ since $\abs{\mu\brac{E}}\leq \nrm{\mu}$. Conversely, for any any measurable partition $\brac{E_n}_{n\geq 1}$ of $\Omega$, $\abs{0_M\brac{E_n}}=0$, whence $\sum_{n\geq 1} \abs{0_M\brac{E_n}} = 0$. Therefore $\abs{\mu}\brac{\Omega}\leq 0$, whence $\nrm{0_M}=0$.

Let $\brac{E_n}_{n\geq 1}$ be a measurable partition of $\Omega$. Then since $\brac{\alpha \mu} \brac{E} = \alpha\cdot \mu\brac{E}$, $\sum_{n\geq 1} \abs{\mu\brac{E}}\leq \nrm{\mu}$ implies that $\sum_{n\geq 1} \abs{\alpha \mu\brac{E}}\leq \abs{\alpha} \nrm{\mu}$ since $\abs{\alpha}\abs{\mu\brac{E}} = \abs{\alpha \mu\brac{E}}$. By definition of the total variation of $\alpha \mu$ it is therefore true that $\nrm{\alpha \mu}=\abs{\alpha \mu}\brac{\Omega}\leq \alpha \nrm{\mu}$. If $\alpha\neq 0$, then $\sum_{n\geq 1} \abs{\brac{\alpha\mu}\brac{E}}\leq \nrm{\alpha\mu}$ implies that $\sum_{n\geq 1} \abs{\mu\brac{E}}\leq \frac{1}{\alpha}\nrm{\alpha\mu}$, whence $\alpha\abs{\mu}\brac{\Omega}\leq \nrm{\alpha \mu}$. In conclusion, $\nrm{\alpha \mu} = \abs{\alpha} \nrm{\mu}$.

Finally let $\brac{E_n}_{n\geq 1}$ be a measurable partition of $\Omega$. By the triangle inequality for the complex modulus and theorem Sup-B-3 \[\sum_{n\geq1}\abs{\mu\brac{E_n}+\lambda\brac{E_n}}\leq \sum_{n\geq 1}\abs{\mu\brac{E_n}} + \sum_{n\geq 1}\abs{\lambda\brac{E_n}} \leq \abs{\mu}\brac{\Omega} + \abs{\lambda}\brac{\Omega}\] Since the total variation of $\mu+\lambda$ is the least upper bound on such infinite sums, $\abs{\mu+\lambda}\brac{\Omega}\leq \abs{\mu}\brac{\Omega} + \abs{\lambda}\brac{\Omega}$. Therefore, the definition of the addition operation on complex measures implies that $\nrm{\mu+\lambda}\leq \nrm{\lambda} +\nrm{\lambda}$.

\label{thm:signed_meas_decomp} \noindent \textbf{Theorem} 11-14.
Let $\mu\in M^1\brac{\Omega, \Fcal}$ be a signed measure. Then there exist $\mu^+, \mu^-$ finite usual measures on $\brac{\Omega, \Fcal}$ such that $\mu=\mu^+-\mu^-$ and $\abs{\mu}=\mu^++\mu^-$.

Define $\mu^+\defn \frac{1}{2}\brac{\abs{\mu} +\mu}$ and $\mu^-\defn \frac{1}{2}\brac{\abs{\mu} -\mu}$.

First of all, since $\mu\brac{\emptyset} = \abs{\mu}\brac{\emptyset}=0$, $\mu^+\brac{\emptyset}=\mu^-\brac{\emptyset}=0$. Next, since $\abs{\mu\brac{E}}\leq \abs{\mu}\brac{E}$ for all $E\in \Fcal$, $\mu^\pm\brac{E}\geq \frac{1}{2}\brac{ \abs{\mu\brac{E}} \pm \mu\brac{E}}\geq 0$.

Now let $\brac{E_n}_{n\geq1}$ be a measurable partition of $E$ in $\brac{\Omega, \Fcal}$. Since $\mu$ is a signed measure, the series $\sum_{n\geq 1}\mu\brac{E_n}$ converges in $\Real$ to $\mu\brac{E}$, whereas $\abs{\mu}\brac{E} = \sum_{n\geq1} \abs{\mu}\brac{E_n}$ as $\abs{\mu}$ is a usual measure on $\brac{\Omega, \Fcal}$. Therefore due to continuity of multiplication and addition in $\Cplx$, it must be that $\mu^+\brac{E} = \sum_{n\geq 1} \mu^+\brac{E_n}$ and $\mu^-\brac{E} = \sum_{n\geq 1} \mu^-\brac{E_n}$, which implies that $\mu^+, \mu^-$ are usual measures on $\brac{\Omega, \Fcal}$.

Finally, since $\mu\brac{\Omega}\in \Real$ and $\abs{\mu}\brac{\Omega}<+\infty$ it must be true that $\mu^+\brac{\Omega},\mu^-\brac{\Omega}<+\infty$. Therefore $\mu^+$ and $\mu^-$ are finite measures on $\brac{\Omega \Fcal}$.

Since arithmetic operations within $\Real$ are well-defined and continuous \[\mu^+\brac{E} - \mu^-\brac{E} = \frac{1}{2}\brac{ \abs{\mu}\brac{E} + \mu\brac{E} - \abs{\mu}\brac{E} + \mu\brac{E} } = \frac{1}{2} 2 \cdot \mu\brac{E}\] Analogously, simple well-defined arithmetic manipulations yield \[ \mu^+\brac{E} + \mu^-\brac{E} = \frac{1}{2}\brac{ \abs{\mu}\brac{E} + \mu\brac{E} + \abs{\mu}\brac{E} - \mu\brac{E}} = \abs{\mu}\brac{E}\] Therefore it is indeed true that $\mu = \mu^+-\mu^-$ and $\abs{\mu} = \mu^++\mu^+$.

Define $\lambda \defn \mu^+ + \mu^-$ and let $\brac{E_n}_{n\geq 1}$ be a measurable partition of $E\in \Fcal$. Then $\sum_{n\geq1} \abs{\lambda\brac{E}} = \sum_{n\geq1} \abs{\mu}\brac{E}$.\\

\label{thm:complex_meas_decomp} \noindent \textbf{Theorem} 11-15.
Let $\mu\in M^1\brac{\Omega, \Fcal}$ be a complex measure. Then there exist $\brac{\mu_k}_{k=1}^4$ all finite usual measures on $\brac{\Omega, \Fcal}$ such that $\mu=\mu_1 - \mu_2 + i \brac{\mu_3 - \mu_4}$. In other words there are signed measures $\lambda, \nu$ such that $\mu=\lambda + i \nu$.

Let $\brac{E_n}_{n\geq1}$ be a measurable partition of $E\in \Fcal$ in $\brac{\Omega, \Fcal}$. Since the maps $\re$ and $\im$ are $\Cplx$-$\Real$ continuous and $\mu$ is a complex measure, the series $\sum_{n=1}^\infty \re \mu\brac{E_n}$ and $\sum_{n=1}^\infty \im \mu\brac{E_n}$ converge in $\Real$ to $\re \mu\brac{E_n}$ and $\im \mu\brac{E_n}$ respectively. Therefore the maps $u\defn \re \mu$ and $v\defn \im \mu$ are complex measures on $\brac{\Omega, \Fcal}$ with values in $\Real$, whence $u$ and $v$ are signed measures. Also note that $\mu\brac{E} = u\brac{E} + i v\brac{E}$ by definition of the real and imaginary parts of a complex number. By theorem 11-14 there are $u^+, u^-, v^+$ and $v^-$ all finite measures on $\brac{\Omega, \Fcal}$ such that $u=u^+-u^-$ and $v=v^+-v^-$. Therefore $\mu = \brac{u^+ - u^-} + i\brac{v^+ - v^-}$ whence $\mu = u + i v$.\\

% section tut_11 (end)

\section{Radon-Nikodym Theorem} % (fold)
\label{sec:tut_12}
\url{http://probability.net/PRTnikodym.pdf}

\noindent \textbf{Definition} 96.
Let $\mu$ and $\nu$ be two usual or complex measure on a measurable space $\brac{\Omega, \Fcal}$. The measure $\nu$ is absolutely continuous with respect to $\mu$, $\nu<<\mu$, if every $E\in \Fcal$ with $\mu\brac{E} = 0$ is such that $\mu\brac{E}=0$.

\label{thm:meas_abs_cont_equiv} \noindent \textbf{Theorem} 58.
If $\mu$ be a usual measure on $\brac{\Omega, \Fcal}$ and $\nu\im M^1\brac{\Omega, \Fcal}$, then $\nu<<\mu$ is equivalent to $\abs{\nu}<<\mu$ and equivalent to \[\forall \epsilon>0\,\exists \delta>0\,\text{s.t.}\,\abs{\nu\brac{E}}<\epsilon\,\text{for all}\, E\in \Fcal\,\text{with}\,\mu\brac{E}=0\]

$\Leftarrow$ Indeed, suppose $\nu<<\mu$ and let $E\in\Fcal$ be such that $\mu\brac{E}=0$. For any measurable partition $\brac{E_n}_{n\geq 1}$ of $E$ the fact that $\mu$ is a usual measure implies that $\mu\brac{E_n}\leq \mu\brac{E}$, whence $E_n$ is $\mu$-null for any $n\geq 1$. Thus $\nu\brac{E_n}=0$, which implies that $\sum_{n\geq 1} \abs{\nu\brac{E_n}} = 0$ for any measurable partition $\brac{E_n}_{n\geq 1}$ of this $E$. Therefore $\abs{\nu}\brac{E}=0$ for any $E\in \Fcal$ with $\mu\brac{E}=0$.

$\Rightarrow$ Conversely, if $\abs{\nu}<<\mu$, then for any $E\in \Fcal$ with $\mu\brac{E}=0$ from $\nu < \abs{\nu}$ follows that $\nu\brac{E}=0$, whence $\nu<<\mu$.

$\Leftarrow$ Let $E\in \Fcal$ be any $\mu$-null set. Since $\mu\brac{E}\leq \delta$ for any $\delta>0$ whence for every $\epsilon>0$ the property in the statement implies that $\abs{\nu\brac{E}}<\epsilon$ for any $E\in \Fcal$ with $\mu\brac{E}=0$. Therefore $\abs{\nu}<<\mu$.

$\Rightarrow$ Suppose there is $\epsilon>0$ such that for all $\delta>0$ there is $E\in \Fcal$ with $\mu\brac{E}\leq \delta$ but $\abs{\nu\brac{E}}\geq \epsilon$. Therefore for such $\epsilon$, there is $\brac{E_n}_{n\geq1}\in \Fcal$ such that $\mu\brac{E_n}\leq \frac{1}{2^n}$ and $\abs{\nu\brac{E_n}}\geq \epsilon$ for all $n\geq 1$. Define $E\defn \limsup_{n\geq 1} E_n \defn \bigcap_{n\geq 1} \bigcup_{k\geq n} E_k$.

Since $\mu$ is a measure, for all $n\geq 1$ by $\sigma$ sub-additivity \[\mu\brac{\bigcup_{k\geq n} E_k}\leq \sum_{k\geq n}\mu\brac{E_k} \leq \frac{1}{2^{n-1}}\] whence $\mu\brac{\bigcup_{n\geq 1}E_n}\leq 1$. As $\bigcup_{k\geq n} E_k \downarrow E$, by theorem 8 \[\mu\brac{E}=\inf_{n\geq 1}\mu\brac{\bigcup_{k\geq n} E_k } = 0\] At the same time, theorems 11-7 and 57 imply that $\abs{\nu}$ is a finite measure on $\brac{\Omega, \Fcal}$, whence $\abs{\nu}\brac{\bigcup_{k\geq n}E_n}<+\infty$ and $\abs{\nu}\brac{\bigcup_{n\geq 1}E_n}\geq \epsilon$ for all $n\geq1$, because $E_n\subseteq \bigcup_{n\geq 1}E_n$. Thus because $\abs{\nu}$ is finite, theorem 8 yields \[\abs{\nu}\brac{E} = \inf_{n\geq1} \abs{\nu}\brac{\bigcup_{n\geq 1}E_n} \geq \epsilon\] Therefore there is $E\in \Fcal$ with $\mu\brac{E}=0$ and such that $\abs{\nu}\brac{E}\geq \epsilon>0$, whence $\abs{\nu}$ is not absolutely continuous with respect to $\mu$.\\

\label{thm:muas_confinement} \noindent \textbf{Theorem} 99-0-1.
Let $\mu$ be a measure on $\brac{\Omega, \Fcal}$ and $\nu\in M^1\brac{\Omega, \Fcal}$ be such that $\nu<<\mu$. If $\nu$ is a complex measure, then $\re \nu, \im \nu << \mu$. If $\nu$ is a signed measure, then $\nu^+, \nu^- << \mu$.

Indeed, let $\nu_1\defn \re \nu$ and $\nu_2\defn \im \nu$. By theorem 11-15 $\nu_1$ and $\nu_2$ are signed measures on $\brac{\Omega, \Fcal}$. Since $\abs{\re \cdot}, \abs{\im \cdot}\leq \abs{\cdot}$, $\abs{\nu_1\brac{E}},\abs{\nu_2\brac{E}}\leq \abs{\nu\brac{E}}$ for all $E\in \Fcal$, whence by virtue of theorem 58 the $\epsilon$-$\delta$ criterion of absolute continuity of measures implies that $\nu_1<<\mu$ and $\nu_2<<\mu$.

If $\nu$ is a signed measure, then by theorem 11-14 there exist finite usual measures $\nu^+$ and $\nu^-$ such that $\nu=\nu^+-\nu^-$ and $\abs{\nu}=\nu^++\nu^-$. The latter expression implies that $\nu^+\brac{E}, \nu^-\brac{E}\leq \abs{\nu}\brac{E}$ for all $E\in \Fcal$. Therefore by the equivalence established in theorem 58 $\nu^+, \nu^-$ are absolutely continuous with respect to $\mu$.\\

\label{thm:muas_confinement2} \noindent \textbf{Theorem} 59.
Let $\mu$ be a finite measure on $\brac{\Omega, \Fcal}$ and $f\in L^1_\Cplx\brac{\Omega, \Fcal, \mu}$. Let $S$ be a closed subset of $\Cplx$ such that for all $E\in \Fcal$ with $\mu\brac{E}>0$ \[\frac{1}{\mu\brac{E}} \int_E f d\mu \in S \] Then there is $N\in \Fcal$ with $\mu\brac{N}=0$ such that $f\brac{\omega}\in S$ for all $\omega\in \Omega\setminus N$ ($f\subseteq S$ $\mu$-a.s. for short).

First, recall that the usual topology on $\Real$ is induced by the usual metric on $\Real$ and has an everywhere dense subset $\mathbb{Q}$. Therefore by theorem 6-4 it has a countable base, whence by theorem 27 the space $\brac{\Real^2, \Tcal_{\Real^2}}$ has a countable base as well. Since $\Real^2$ is a product space by theorem 6-3-1 it is metrizable by the usual Euclidean metric, whence by theorem theorem 6-4 it has an everywhere dense countable subset. Since $\brac{\Cplx,\Tcal_\Cplx}$ is homeomorphic (topologically equivalent) to $\brac{\Real^2, \Tcal_{\Real^2}}$, $\Cplx$ must also have a countable everywhere dense subset, namely, the set of complex numbers with rational real and imaginary parts $\mathbb{Q}^2$. Moreover the euclidean metric on $\Real^2$ can easily be identified with the complex modulus metric on $\Cplx$.

Since $S$ is closed in $\brac{\Cplx, \Tcal_\Cplx}$, the set $S^c$ is open in $\Cplx$ and non-empty. Therefore for every $z\in S^c$ there is $\delta>0$ such that the open disc $\dot{D}\brac{z,\delta}$ is a subset of$S^c$, where \[\dot{D}\brac{z,\delta}\defn \obj{\induc{z'\in \Cplx}\,\abs{z-z'}<\delta}\] Since $\dot{D}\brac{x,\delta}$ is itself open in $\Cplx$, density of $\mathbb{Q}^2$ implies that there is $c\in \mathbb{Q}^2$ such that $c\in \dot{D}\brac{z, \delta}$. For $\epsilon\defn \delta - \abs{z-c}>0$ the open disc $\dot{D}\brac{c, \epsilon}$ resides inside of the $\dot{D}\brac{z,\delta}\subseteq S^c$ and contains $z$. Furthermore for this $\epsilon>0$ there necessarily is an $n\geq 1$ with $\epsilon>\frac{1}{2^n}$, whence $z\in D\brac{c, \frac{1}{2^n}}\subseteq S^c$, where $D\brac{z, \delta}\defn \obj{\induc{z'\in \Cplx}\,\abs{z-z'}\leq \delta}$ is a closed disc in $\Cplx$. Since $\mathbb{Q}^2$ is countable, there is bijective map $a:\mathbb{N}\to\mathbb{Q}^2$. Therefore for every $z\in S^c$ there is $n,m\geq 1$ such that $z\in D\brac{a_n, \frac{1}{2^m}}\subseteq S^c$.

Let \[\Gamma\defn \obj{\induc{\brac{n,m}\in \mathbb{N}^2}\, D\brac{a_n,\frac{1}{2^m}} \subseteq S^c}\] Since $\brac{\frac{1}{2^n}}_{n\geq 1}$ and $\mathbb{Q}^2$ are at countable, $\Gamma$ is at countable. Since every disc is uniquely identified by its radius and centre, there is a sequence $\brac{D_n}_{n\geq1}$ of closed discs in $\Cplx$ such that for all $z\in S^c$ there is $N\geq 1$  with $z\in D_N$ and $D_n\subseteq S^c$ for every $n\geq 1$, where $D_n=D\brac{\alpha_n,r_n}$ for some $\brac{\alpha_n}_{n\geq 1}\in \Cplx$ and $\brac{r_n}_{n\geq1}>0$. Therefore $S^c=\bigcup_{n\geq1} D_n$.

Put $E_n\defn \obj{f\in D_n}$ for all $n\geq 1$ and suppose $\mu\brac{E_n}>0$ for some $n\geq 1$. Since $\mu$ is a finite measure, $\int_{E_n} \alpha_n d\mu = \alpha_n \mu\brac{E_n}$ and by theorem 20 \[\abs{\frac{1}{\mu\brac{E_n}} \int_{E_n} f d\mu - \alpha_n} = \frac{1}{\mu\brac{E_n}} \abs{\int_{E_n} \brac{f - \alpha_n} d\mu} \leq \frac{1}{\mu\brac{E_n}} \int_{E_n} \abs{f-\alpha_n} d\mu \leq r_n\] as $1_{E_n}\abs{f-\alpha_n}\leq r_n$ by definition of $E_n$. This observation implies that \[\frac{1}{\mu\brac{E_n}} \int_{E_n} f d\mu \in D_n\subseteq S^c\] which contradicts the assumption in the premise. Therefore $\mu\brac{E_n}=0$ for all $n\geq 1$ and $\mu\brac{\obj{f\in S^c}}\leq \sum_{n\geq 1}\mu\brac{\obj{f\in D_n}} = 0$, whence $f\in S$ $\mu$-a.s.

If $S=\Cplx$, then $f\in S$ everywhere on $\Omega$, whence, trivially, $f\subseteq S$ $\mu$-a.s. Note that the statement of this theorem is vacuously true if $\mu\brac{E}=0$ for all $E\in \Fcal$.\\

\label{thm:int_ineq_muas_finite} \noindent \textbf{Theorem} 99-1.
Let $\mu$ be a finite measure on $\brac{\Omega, \Fcal}$ and $f,g:\brac{\Omega, \Fcal}\to \Zinf$ be measurable maps with $\int_E f d\mu \leq \int g d\mu$ and $\int_E f d\mu < +\infty$ for all $E\in \Fcal$. Then $f\leq g$ $\mu$-almost surely.

Indeed, let $E_n\defn \obj{g+\frac{1}{n}<f}\in \Fcal$. Then $E_n\uparrow \obj{g<f}$ and for all $n\geq1$ \[\int_{E_n} f d\mu \geq \int_{E_n} g d\mu + \frac{1}{n}\mu\brac{E_n}\geq \int_{E_n} f d\mu + \frac{1}{n}\mu\brac{E_n}\] Since the integrals on either side are finite, it must be true that $\mu\brac{E_n}=0$ for all $n\geq1$. By theorem 7 $\mu\brac{E_n}\uparrow \mu\brac{\obj{g<f}}$, which implies that $\mu\brac{\obj{g<f}}=0$. Therefore $f\leq g$ $\mu$-a.s.\\

\label{thm:int_ineq_muas} \noindent \textbf{Theorem} 99-2.
Let $\mu$ be a $\sigma$-finite measure on $\brac{\Omega, \Fcal}$ and $f,g:\brac{\Omega, \Fcal}\to \Zinf$ be such that $\int_E f d\mu \leq \int g d\mu$ for all $E\in \Fcal$. Then $f\leq g$ $\mu$-almost surely.

Indeed, since $\mu$ is $\sigma$-finite, there is $\brac{\Omega_n}_{n\geq1}\in \Fcal$ such that $\Omega_n\uparrow \Omega$ with $\mu\brac{\Omega_n} < +\infty$.

Let $F_n\defn \Omega_n \cap \obj{f\leq n}\in \Fcal$ and put $\mu_n\defn \mu\brac{F_n\cap \cdot}$. Then $\mu_n$ is a finite measure on $\brac{\Omega, \Fcal}$, because $\mu_n\brac{E}=\mu\brac{F_n\cap E}\leq \mu\brac{\Omega_n}$. By definition of the partial Lebesgue integral for any $E\in \Fcal$ it is true that \[\int_E f d\mu_n = \int 1_E f d\mu_n = \int_{F_n} 1_E f d\mu = \int_{E\cap F_n} f d\mu\] whence $\int_E f d\mu_n \leq \int_E g d\mu_n$ for all $E\in \Fcal$ and $n\geq 1$.

Furthermore, for all $n\geq 1$ and $E\in\Fcal$ \[\int_E f d\mu_n = \int 1_E 1_{F_n} f d\mu \leq \int 1_{E\cap F_n} n d\mu = n\mu_n\brac{E}<+\infty\] By theorem 99-1 it is true that $\mu_n\brac{\obj{g<f}}=0$ for all $n\geq1$. Since $F_n\uparrow \obj{f<+\infty}$, application of theorem 7 over $\brac{\Omega, \Fcal, \mu}$ yields $\mu\brac{F_n\cap \obj{g<f}}\uparrow \mu\brac{\obj{g<f}\cap \obj{f<+\infty}}$, which implies that $\mu\brac{\obj{g<f}\cap \obj{f<+\infty}}=0$.

Now put $F_n\defn \Omega_n \cap \obj{g\leq n}$ and let $\mu_n\defn \mu\brac{F_n\cap \cdot}$. Then $\mu_n$ is a finite measure and for any $E\in \Fcal$ by definition of the partial Lebesgue integral \[\int_E g d\mu_n\leq \int_E g d\mu_n = \int 1_E 1_{F_n} g d\mu\leq n\mu_n\brac{E}<+\infty\] Therefore $\mu_n\brac{\obj{f=+\infty}} = 0$ for all $n\geq 1$, whence by theorem 7 \[\mu_n\brac{\obj{f=+\infty}}\uparrow \mu\brac{\obj{g<+\infty}\cap \obj{f=+\infty}}=0\] because $F_n\cap \obj{f=+\infty}\uparrow \obj{g<+\infty}\cap \obj{f=+\infty}$.

Finally, since $\obj{g<f} = \obj{g<f}\cap \obj{f<+\infty} \uplus \obj{g<f}\cap \obj{f=+\infty}$, by $\sigma$-additivity of usual measures $\mu\brac{\obj{g<f}}=0$ and therefore $g\geq f$ $\mu$-a.s.\\

\label{thm:positive_partial_int} \noindent \textbf{Theorem} 99-3.
Let $\mu$ be a usual measure on $\brac{\Omega, \Fcal}$ and $f\in L^1_\Real\brac{\Omega, \Fcal, \mu}$ be such that $\int_E f d\mu \geq 0$ for all $E\in \Fcal$. Then $f\geq0$ $\mu$-almost surely.

Indeed, let $A_n\defn \obj{ f + \frac{1}{n} < 0 }$. Note that $\brac{A_n}_{n\geq 1}\in \Fcal$, because $f$ is measurable, and $A_n\uparrow \obj{f < 0}$. Therefore \[0\leq \int_{A_n} f d\mu \leq + \frac{1}{n}\mu\brac{A_n}\leq 0\] whence $\mu\brac{A_n}=0$ for all $n\geq 1$. By theorem 7 $\mu\brac{A_n}\uparrow \mu\brac{\obj{f<0}}$, which implies $\mu\brac{\obj{f<0}}=0$. Thus $f\geq 0$ $\mu$-a.s.

Alternatively, by definition $\int_E f d\mu = \int_E f^+d\mu - \int_E f^-d\mu$ for all $E\in \Fcal$. In particular, for $E\defn \obj{f\geq 0}$ it is true that $\int_E f^- d\mu = \int 1_E f^- d\mu = 0$ while $\int_E d\mu f^+ d\mu = \int f^+ d\mu$. Therefore $\int_{\obj{f\geq 0}} f d\mu = \int f^+ d\mu$, and, since $h^- = \brac{-h}^+$, it is true that \[\int_{\obj{h\leq 0}} h d\mu = \int_{\obj{-h\geq 0}} -h d\mu = \int \brac{-h}^- d\mu = \int h^- d\mu\]

Note that $0\geq -\int f^-d\mu = \int_{\obj{f\geq 0}} f d\mu \geq 0$, whence $\int f^- d\mu = 0$ and $f^-=0$ $\mu$-a.s. by the basic properties of the usual Lebesgue integral. Therefore $f=f^+$ $\mu$-a.s., which implies that $f\geq 0$ $\mu$-a.s.\\

\label{thm:zero_partial_int} \noindent \textbf{Theorem} 99-4.
Let $\mu$ be a usual measure on $\brac{\Omega, \Fcal}$ and $f\in L^1_\Cplx\brac{\Omega, \Fcal, \mu}$ be such that $\int_E f d\mu = 0$ for all $E\in \Fcal$. Then $f=0$ $\mu$-almost surely.

Indeed, if $f\in L^1_\Real\brac{\Omega, \Fcal, \mu}$ is such that $\int_E f d\mu = 0$ for all $E\in\Fcal$, then $\int_E f d\mu\geq 0$ and $\int -f d\mu\geq 0$, whence by theorem 99-3 $f\geq0$ $\mu$-a.s and $f\leq0$ $\mu$-a.s., which implies that $f=0$ $\mu$-a.s.

If $f = u + i v \in L^1_\Cplx\brac{\Omega, \Fcal, \mu}$ is such that $\int_E f d\mu = 0$ for all $E\in \Fcal$, then for all $E\in \Fcal$ \[\re \int_E f d\mu = \int_E u d\mu,\,\im \int_E f d\mu = \int_E v d\mu\] whence for all $E\in \Fcal$ the functions $u,v\in L^1_\Real\brac{\Omega, \Fcal, \mu}$ are such that $\int_E u d\mu = \int_E v d\mu = 0$. Therefore $u=0$ $\mu$-a.s. and $v=0$ $\mu$-a.s., and so $f=0$ $\mu$-almost surely.

In case of a finite usual measure $\mu$ on $\brac{\Omega, \Fcal}$ there is an alternative proof. Indeed, for such $f\in L^1_\Cplx$ it is true that for all $E\in \Fcal$ with $\mu\brac{E}>0$ \[\frac{1}{\mu\brac{E}} \int_E f d\mu \in \obj{0}\] Since the space $\Cplx$ is metrizable by the usual metric on $\Cplx$, by theorem 8-6 it must be Hausdorff which implies that every finite set is closed in $\Cplx$. Thus the one-point set $\obj{0}$ is closed in $\Cplx$, whence by theorem 59 $f=0$ $\mu$-a.s., unless $\mu$ fails to be a finite measure.\\

\label{thm:meas_comb_int} \noindent \textbf{Theorem} 99-5.
Let $\mu, \nu$ be usual measures on a measurable space $\brac{\Omega, \Fcal}$, $\beta \in \Zinf$ and put $\lambda\defn \mu+\beta \nu$. For any non-negative measurable map $f:\brac{\Omega, \Fcal}\to\Zinf$ \[\int f d\lambda = \int f d\mu + \beta \int f d\nu\]

First, $\lambda\brac{\emptyset} = \mu\brac{\emptyset}+\beta \nu\brac{\emptyset}=0$. Secondly, by theorem Sup-B-3 and distributive property of multiplication on the non-negative extended real line \[\lambda\brac{\uplus_{n\geq 1} A_n} = \sum_{n\geq1} \mu\brac{A_n}+\beta \sum_{n\geq1} \nu\brac{A_n} = \sum_{n\geq 1} \mu\brac{A_n} + \beta \nu\brac{A_n} = \sum_{n\geq1} \lambda\brac{A_n}\] Therefore $\lambda$ is a usual measure on $\brac{\Omega, \Fcal}$.

Now, note that for any $A\in \Fcal$ \[\int 1_A d\lambda = \lambda\brac{A} = \mu\brac{A} + \beta \nu\brac{A} = \int 1_A d\mu + \beta \int 1_A d\nu\] For any simple function $s=\sum_{k=1}^n \alpha_k 1_{A_k}$ on $\brac{\Omega, \Fcal}$, where $\brac{\alpha_k}_{k=1}^n\in \Real^+$ and $\brac{A_k}_{k=1}^n\in \Fcal$ with $\Omega = \uplus_{k=1}^n A_k$, the linearity of the usual Lebesgue integral and the distributive property of multiplication on $\Zinf$ imply that \[ \int s d\lambda = \sum_{k=1}^n \alpha_k \int 1_{A_k} d\lambda = \sum_{k=1}^n \alpha_k \int 1_{A_k} d\mu + \beta \sum_{k=1}^n \alpha_k \int 1_{A_k} d\nu = \int s d\mu + \beta \int s d\nu \] Finally, by theorem 18 for any $f:\brac{\Omega, \Fcal}\to \Zinf$ there is a sequence of simple functions $\brac{s_n}_{n\geq 1}$ with $s_n\uparrow f$. Triple application of theorem 19 (MCT) on $\brac{\Omega, \Fcal, \eta}$, where $\eta = \lambda$, or $\mu$ or $\nu$ yields $\int s_n d\eta \uparrow \int f d\eta$. Since addition is continuous in $\Zinf$ and the latter is Hausdorff \[\int f d\lambda = \int f d\mu + \beta \int f d\nu\]\\

\label{thm:int_meas_ineq} \noindent \textbf{Theorem} 99-6.
Let $\mu, \nu$ be usual measures on a measurable space $\brac{\Omega, \Fcal}$ with $\mu\leq \nu$. For any non-negative measurable map $f:\brac{\Omega, \Fcal}\to\Zinf$ \[\int f d\mu \leq \int f d\nu\]

Indeed, for any $E\in \Fcal$ it is true that $\mu\brac{E}\leq \nu\brac{E}$ which implies that $\int 1_E d\mu \leq \int 1_E d\nu$. For any simple function $s=\sum_{k=1}^n \alpha_k 1_{E_k}$ on $\brac{\Omega, \Fcal}$ linearity of the usual Lebesgue integral and non-negativity of $\brac{\alpha_k}_{k=1}^n$ imply $\int s d\mu = \sum_{k=1}^n \alpha_k \int 1_{E_k} d\mu \leq \sum_{k=1}^n \alpha_k \int 1_{E_k} d\nu = \int s d\nu$. Finally, for any $g:\brac{\Omega, \Fcal}\to \Zinf$ theorem 18 implies that there are $\brac{s_n}_{n\geq 1}$ simple functions on $\brac{\Omega, \Fcal}$ such that $s_n\uparrow g$ in $\Zinf$ everywhere. Therefore by theorem 19 (MCT) over $\brac{\Omega, \Fcal,\mu}$ and $\brac{\Omega, \Fcal,\nu}$ it is true that $\int s_n d\mu\uparrow\int g d\mu$ and $\int s_n d\mu\uparrow\int g d\nu$ in $\Zinf$, whence $\int g d\mu \leq \int g d\nu$.\\

\label{thm:l_real_weighting_func} \noindent \textbf{Theorem} 99-7.
If $\mu$ be a $\sigma$-finite measure, then there exists $w\in L^1_\Real\brac{\Omega, \Fcal, \mu}$ with $0<w\leq 1$.

Indeed, since $\mu$ is $\sigma$-finite, there is $\brac{\Omega_n}_{n\geq 1}\uparrow \Omega$ with $\mu\brac{\Omega_n}<+\infty$ for all $n\geq 1$. Define $w:\Omega\to \Real$ as \[w \defn \sum_{n\geq1} \frac{1}{2^n} \frac{1}{1+\mu\brac{E_n}} 1_{E_n}\] 

The map $w$, being the least upper bound on the sums of measurable maps, is itself measurable. Indeed, \[w = \sup_{n\geq1} \sum_{k=1}^n \frac{1}{2^k}\frac{1}{1+\mu\brac{E_k}} 1_{E_k}\] Furthermore $w$ is non-negative. Now, for any $\omega\in \Omega$ there is $N\geq1$ such that $\omega\in E_n$ for all $n\geq N$, which means that $w\brac{\omega} \leq \sum_{n\geq N} \frac{1}{2^n} = \frac{1}{2^{N-1}}$. Furthermore $w\brac{\omega}\geq \frac{1}{2^N}\frac{1}{1+\mu\brac{E_N}}>0$. Therefore $w\brac{\omega}\in \ploc{0, 1}$ for all $\omega\in \Omega$.

Since $w$ is non-negative and measurable, by the MCT it is true that \[\int_\omega w d\mu \leq \sum_{n\geq1} \frac{1}{2^n}\frac{1}{1+\mu\brac{E_n}} \int_\Omega 1_{E_n} d\mu \leq \sum_{n\geq1} \frac{1}{2^n} = 1\] Therefore $w\in L^1_\Real\brac{\Omega, \Fcal, \mu}$.\\

\label{thm:radon_nikodym_finite_meas} \noindent \textbf{Theorem} 99-8.
Let $\mu$ be a $\sigma$-finite measure on $\brac{\Omega, \Fcal}$ and $\nu$ be a finite measure on $\brac{\Omega, \Fcal}$ such that $\nu<<\mu$. Then there exists $h\in L^1_\Real\brac{\Omega, \Fcal, \mu}$ with $h\geq 0$ and for every $E\in \Fcal$ \[\nu\brac{E} = \int_E h d\mu\]

By theorem 99-7 there exists $w\in L^1_\Real\brac{\Omega, \Fcal, \mu}$ with $0<w\leq 1$. If $\bar{\mu}\brac{E} \defn \int_E w d\mu$ for all $E\in \Fcal$, then by theorem 21 $\bar{\mu}$ is a usual measure on $\brac{\Omega, \Fcal}$ which is finite since $\int_\Omega w d\mu<+\infty$.

Let $\phi\defn \nu + \bar{\mu}$. It is a finite measure on $\brac{\Omega, \Fcal}$, because addition is $\Zinf^N$-$\Zinf$ continuous and both $\bar{\mu}$ and $\nu$ are finite measures. Now, for any $f:\brac{\Omega, \Fcal}\to \Zinf$ theorems 99-5 and 21 imply \[\int f d\phi = \int f d\nu + \int f d\bar{\mu} = \int f d\nu + \int f w d\mu\] Since $\int \abs{f} d\nu, \int \abs{f} w d\mu \leq \int \abs{f} d\mu<+\infty$, $f\in L^1_\Cplx\brac{\Omega, \Fcal,\nu}$ and $f w\in L^1_\Cplx\brac{\Omega, \Fcal,\mu}$ for any $f\in L^1_\Cplx\brac{\Omega, \Fcal, \phi}$. As the complex Lebesgue integral is a linear extension of the usual integral it is true that $\int f d\phi = \int f d\nu + \int f w d\mu$.

For all $f \in L^2_\Cplx\brac{\Omega, \Fcal, \phi}$ theorem 42 (Cauchy-Schwartz) implies \[\int \abs{f}d\phi\leq \brac{\int\abs{f}^2 d\phi}^\frac{1}{2} \brac{\phi\brac{\Omega}}^\frac{1}{2}\] from where it follows that $f\in L^1_\Cplx\brac{\Omega, \Fcal ,\phi}$. Furthermore by theorem 99-6 $\int \abs{f} d\nu \leq \int \abs{f} d\phi$, whence $L^2_\Cplx\brac{\Omega, \Fcal, \phi}\subseteq L^1_\Cplx\brac{\Omega, \Fcal, \nu}$.

If $\lambda\brac{f}\defn \int f d\nu$ for all $f \in L^1_\Cplx\brac{\Omega, \Fcal, \nu}$, then by linearity of the complex Lebesgue integral on $\brac{\Omega, \Fcal, \nu}$ the map $\lambda$ is a linear functional. For any $f \in L^2_\Cplx\brac{\Omega, \Fcal, \phi}$ theorem 24 and the above inequalities yield $\abs{\int f d\nu} \leq \sqrt{\phi\brac{\Omega}} \nrm{f}_2$. Since $L^2_\Cplx\brac{\Omega, \Fcal, \phi}$ is a linear subspace of $L^1_\Cplx\brac{\Omega, \Fcal, \nu}$, it turns out that $\lambda:L^2_\Cplx\brac{\Omega, \Fcal, \phi}\to \Cplx$ is a bounded linear functional. Therefore $\lambda$ is continuous with respect to $\Tcal_{L^2_\Cplx}$-$\Tcal_\Cplx$, whence by theorem 55 there is $\hat{g}\in L^2_\Cplx\brac{\Omega, \Fcal, \phi}$ such that for all $f\in L^2_\Cplx\brac{\Omega, \Fcal, \phi}$ \[\lambda\brac{f}=\int f \hat{g} d\phi\]

Since $\phi$ is a finite measure and $\phi = \nu + \bar{\mu}$, $1_E\in L^2_\Cplx\brac{\Omega, \Fcal, \phi}$, it is true that $0\leq \nu\brac{E} = \int 1_E d\nu = \int 1_E \hat{g} d\phi \leq \phi\brac{E}<+\infty$. Therefore for any $E\in \Fcal$ with $\phi\brac{E}>0$ \[\frac{1}{\phi\brac{E}} \int_E \hat{g} d\phi \in \clo{0,1}\] whence $\hat{g}\subseteq \clo{0,1}$ $\phi$-a.s. by theorem 59. Since $F\defn \obj{\hat{g}\in \clo{0,1}}in \Fcal$, the map $g\defn \hat{g} 1_A$ is an element of $L^2_\Cplx\brac{\Omega, \Fcal, \phi}$ with $g\brac{\Omega}\subseteq \clo{0,1}$. Furthermore $\hat{g}=g$ $\phi$-a.s., whence $\int f d\nu = \int f g d\phi$ for all $f\in L^2_\Cplx\brac{\Omega, \Fcal, \phi}$.

For any $f\in L^2_\Cplx\brac{\Omega, \Fcal, \phi}$ by theorem 42 (Cauchy-Schwartz) $f g \in L^1_\Cplx\brac{\Omega, \Fcal, \phi}$, whence $\int f g d\phi = \int f g d\nu + \int f g w d\mu$. Thus $\int f \brac{1 - g} d\nu = \int f g w d\mu$ for all $f\in L^2_\Cplx\brac{\Omega, \Fcal, \phi}$.

Now, since $\phi$ is a finite measure on $\brac{\Omega, \Fcal}$, $g\leq 1$ implies that $\nrm{g^n}_2\leq \sqrt{\phi\brac{\Omega}}$ whence $g^n \in L^2_\Cplx\brac{\Omega, \Fcal, \phi}$ for all $n\geq 0$. Furthermore, since the $L^p_\Cplx$-space is closed under $\Cplx$-linear combinations, $f\defn \sum_{k=0}^n g^k$ is in $L^2_\Cplx\brac{\Omega, \Fcal, \phi}$ for all $n\geq 0$. Therefore for every $E\in \Fcal$ \[\int_E \brac{1-g^{n+1}} d\nu = \int 1_E f \brac{1-g} d\nu = \int 1_E f w d\mu = \int_E \brac{g \sum_{k=0}^n g^k} w d\mu\]

If $h\defn g w \sum_{k\geq 0} g^k$, then $h:\brac{\Omega, \Fcal}\to\Zinf$ is non-negative and measurable and such that $g w \sum_{k=0}^n g^k \uparrow h$. Therefore $1_E g w \sum_{k=0}^n g^k \uparrow 1_E h$ in $\borel{\Rbar}$ for every $E\in \Fcal$, whence by the MCT on $\brac{\Omega, \Fcal, \phi}$ \[\int_E \brac{g \sum_{k=0}^n g^k} w d\mu = \int 1_E \brac{g \sum_{k=0}^n g^k} w d\mu \uparrow \int 1_E h d\mu = \int_E h d\mu\]

Now, put $A\defn \obj{0\leq g<1}\in \Fcal$, and note that $1_{E\cap A} \brac{1-g^{n+1}}\uparrow 1_{E\cap A}$ for all $E\in \Fcal$. Thus by the MCT on $\brac{\Omega, \Fcal, \nu}$ \[\int_{E\cap A} 1-g^{n+1} d\nu = \int 1_{E\cap A} \brac{1-g^{n+1}} d\nu \uparrow \int 1_{E\cap A} d\nu = \nu\brac{E\cap A}\] However, $\int_E 1-g^{n+1} d\nu = \int_{E\cap A} 1-g^{n+1} d\nu + \int 1_E 1_{A^c} \brac{1-g^{n+1}} d\nu$ for all $n\geq 0$ and $E\in \Fcal$. Therefore $\int_{E\cap A} 1-g^{n+1} d\nu = \int_E \brac{g \sum_{k=0}^n g^k} w d\mu$ for all $n\geq 0$, whence $\int_{E\cap A} 1-g^{n+1} d\nu \uparrow \nu\brac{E\cap A}$ for any $E\in \Fcal$.

Now, if $\omega\notin A$, then $g\brac{\omega}=1$ and $\sum_{n\geq1}g^n\brac{\omega}=+\infty$, which implies that $h\brac{\omega}=+\infty$ because $w\brac{\omega}>0$. Conversely, if $\omega\in A$, then $\sum_{k\geq 1} g^k\brac{\omega} = \frac{1}{1-g\brac{\omega}}\in \Real$, because $g\brac{\omega}<1$. As $w\brac{\omega}\leq 1$ it must be that $h<+\infty$.

Since $\nu$ is a finite measure, $\int h d\mu<+\infty$, whence $h<+\infty$ $\mu$-a.s. Thus $\mu\brac{A^c}=\mu\brac{\obj{h=+\infty}}=0$ and $\nu\brac{A^c}=0$, because $\nu<<\mu$ and $A^c$ is $\mu$-null, which ultimately implies that $\nu\brac{E}=\nu\brac{E\cap A}$ for all $E\in \Fcal$.

Since $h<+\infty$ $\mu$-a.s., there exists $h'\in L^1_\Real\brac{\Omega, \Fcal, \mu}$ with $h=h'$ $\mu$-a.s., from where it follows that $\int 1_E h d\mu = \int 1_E h' d\mu$ for all $E\in \Fcal$. To summarize, the reasoning above implies that on a measurable space $\brac{\Omega, \Fcal}$ for a $\sigma$-finite measure $\mu$ and a finite measure $\nu$ with $\nu<<\mu$ there exists $h\in L^1_\Real\brac{\Omega, \Fcal, \mu}$ with $h\geq 0$ such that $\nu\brac{E} = \int_E h d\mu$ for every $E\in \Fcal$.\\

\label{thm:radon_nikodym_complex_meas} \noindent \textbf{Theorem} 60.
Let $\brac{\Omega, \Fcal, \mu}$ be a finite measure space and $\nu$ be a complex measure on $\brac{\Omega, \Fcal}$. Then there exists a $\mu$ almost surely unique $h\in L^1_\Cplx\brac{\Omega, \Fcal, \mu}$ such that for all $E\in \Fcal$ \[\nu\brac{E} = \int h d\mu\] Furthermore if $\nu$ is as a signed measure on $\brac{\Omega, \Fcal}$, then $h\in L^1_\Real\brac{\Omega, \Fcal, \mu}$ while $\nu$ is a finite measure $h$ is a non-negative finite map.

The last case has been demonstrated in theorem 99-8. Suppose $\nu$ is a signed measure. Then by theorem 11-14 there are finite usual measures $\nu^+$ and $\nu^-$ such that $\nu=\nu^+-\nu^-$, and which are absolutely continuous with respect to $\mu$ by theorem 58. Therefore by theorem 99-8 there are non-negative $h^+, h^-\in L^1_\Real\brac{\Omega, \Fcal, \mu}$ such that for all $E\in \Fcal$ \[\nu^+\brac{E} = \int_E h^+ d\mu\,\text{and}\,\nu^-\brac{E} = \int h^- d\mu\] For $h\defn h^+-h^-$, $h\in L^1_\Real\brac{\Omega \Fcal, \mu}$ and by linearity of the complex Lebesgue integral \[\nu\brac{E} = \nu^+\brac{E} - \nu^-\brac{E} = \int_E h^+ d\mu - \int_E h^+- d\mu = \int_E h d\mu\] 

If $\nu$ is a complex measure on $\brac{\Omega, \Fcal}$, then by theorem 11-15 there are signed measures $\nu_1,\nu_2$ on $\brac{\Omega, \Fcal}$ such that $\nu = \nu_1 + i \nu_2$, which by theorem 58 are absolutely continuous with respect to $\mu$. Thus by the intermediate result above there are $h_1, h_2\in L^1_\Real\brac{\Omega, \Fcal, \mu}$ such that $\nu_k\brac{E} = \int_E h_k d\mu$ for all $E\in \Fcal$ with $k=1,2$. Since the map $h\defn h_1 + i h_2$ is a $\Cplx$-linear combination, $h\in L^1_\Cplx\brac{\Omega, \Fcal, \mu}$ and by the linearity of the complex Lebesgue integral it is such that \[\nu\brac{E} = \nu_1\brac{E}+i \nu_2\brac{E} = \int_E h_1 d\mu + i \int_E h_2 d\mu = \int_E h d\mu\]

If there is another $h'\in L^1_K\brac{\Omega, \Fcal, \mu}$ with $\nu\brac{E}=\int_E h' d\mu$ for all $E\in \Fcal$, then the map $h-h'\in L^1_K\brac{\Omega, \Fcal, \mu}$ is such that $\int_E h-h' d\mu=0$ for all $E\in \Fcal$. By theorem 99-4 $h=h'$ $\mu$-a.s.\\

\label{thm:radon_nikodym_sigma_finite_meas} \noindent \textbf{Theorem} 61.
Let $\mu$ and $\nu$ be two $\sigma$-finite measures on $\brac{\Omega, \Fcal}$ such that $\nu<<\mu$. There exists a measurable map $h:\brac{\Omega, \Fcal}\to\brac{\Real^+, \borel{\Real^+}}$ such that for all $E\in \Fcal$ \[\nu\brac{E} = \int_E h d\mu\]

First, note that there is a sequence $\brac{\Omega_n}_{n\geq 1}\in\Fcal$ such that $\Omega_n\uparrow \Omega$ and $\nu\brac{\Omega_n}<+\infty$ for all $n\geq 1$. Let $\nu_n\defn \nu^{\Omega_n}\defn \nu\brac{\Omega_n\cap \cdot}$ for all $n\geq 1$.

By construction $\nu_n$ is a finite measure on $\brac{\Omega, \Fcal}$ which is absolutely continuous with respect to $\mu$, because $\nu_k\brac{\cdot}\leq \nu\brac{\cdot}$. Therefore by theorem 60 there exist $\brac{h_n}_{n\geq1}\in L^1_\Real\brac{\Omega, \Fcal, \mu}$ with $h_n\geq 0$ and $\nu_n\brac{E} = \int_E h_n d\mu$ for every $E\in \Fcal$ and $n\geq 1$. Furthermore, since $E\cap \Omega_n\uparrow E$ for all $E\in \Fcal$, by theorem 7 on $\brac{\Omega, \Fcal, \nu}$ it must be true that $\nu_n\brac{E}\uparrow \nu\brac{E}$.

Since $\Omega_n\subseteq \Omega_{n+1}$, for any $E\in \Fcal$ the sequence $\brac{\nu_n\brac{E}}_{n\geq1}$ is non-decreasing, whence for all $E\in \Fcal$ \[\int_E h_n d\mu = \nu_n\brac{E}\leq \nu_{n+1}\brac{E} = \int_E h_{n+1} d\mu<+\infty\] with the latter following from the fact that $\nu_n$ is a finite measure.

Thus $\int_E h_{n+1} - h_n d\mu\geq 0$ for all $E\in\Fcal$, whence by theorem 99-3 $h_{n+1}\geq h_n$ $\mu$-a.s. Since there are countably many almost sure properties, $\brac{h_n}_{n\geq 1}$ is a $\mu$-almost surely non-decreasing sequence in $L^1_\Real$ with $h_n\geq0$.

Appealing to the $\mu$-a.s. uniqueness in theorem 60 and of integrals in general, there must be $\brac{h_n}_{n\geq 1}\in L^1_\Real\brac{\Omega, \Fcal, \mu}$ with $0\leq h_n\leq h_{n+1}$ for all $n\geq 1$ and such that $\nu_n\brac{E} = \int_E h_n d\mu$ for every $E\in \Fcal$ and $n\geq 1$.

Therefore $h_n 1_E\uparrow \hat{h} 1_E$ for all $E\in \Fcal$, where $\hat{h}\defn \sup_{n\geq 1} h_n$. Hence, theorem 18 (MCT) on $\brac{\Omega, \Fcal, \mu}$ implies that for every $E\in \Fcal$ \[\nu_n\brac{E} = \int_E h_n d\mu \uparrow \int_E \hat{h} d\mu\] whence by an earlier observation $\int_E \hat{h} d\mu = \nu\brac{E}$ for all $E\in \Fcal$.

Furthermore $\int_{\Omega_n} \hat{h} d\mu = \nu\brac{\Omega_n} < +\infty$ for all $n\geq 1$. Thus $N\defn \obj{\hat{h}=+\infty}\in \Fcal$ is such that $\mu\brac{\Omega_n\cap \obj N}=0$ for all $n\geq 1$, whence $\mu\brac{N}=0$ because $\mu\brac{N}\leq \sum_{n\geq 1}\mu\brac{\Omega_n\cap \obj N}$. Since $\hat{h}<+\infty$ $\mu$-a.s., there is a measurable non-negative map $h:\brac{\Omega, \Fcal}\to \Real^+$ with $h=\hat{h}$ $\mu$-a.s. such that $\nu\brac{E} = \int_E h d\mu$ for all $E\in \Fcal$, because integrals of almost surely equal functions are identical.

Suppose there is another map $g:\brac{\Omega, \Fcal}\to\Zinf$ with $\int_E \hat{h} d\mu = \nu\brac{E}$ for all $E\in \Fcal$. Then $g\geq \hat{h}$ $\mu$-a.s. and $\hat{h}\geq g$ $\mu$-a.s by theorem 99-2, whence $h=\hat{h}$ $\mu$-a.s.

In conclusion, for any $\sigma$-finite measure $\mu$ and $\nu$ on $\brac{\Omega, \Fcal}$ there is $h\in L^1_\Real\brac{\Omega, \Fcal, \mu}$ with $h\geq 0$ such that for all $E\in \Fcal$ \[\nu\brac{E} = \int_E h d\mu\] and this ``derivative'' (the Radon-Nikodym derivative) of $\nu$ with respect to $\mu$ is unique up to $\mu$-almost sure equality.\\

% Furthermore by definition of the partial Lebesgue integral on $\brac{\Omega, \Fcal, \mu}$ for all $n\geq 1$ \[\int_{\Omega_n} h d\mu = \int h d\mu\brac{\Omega_n\cap \cdot} = \int h d\mu^\Omega_n\] whence $\int_{\Omega_n} h d\mu<+\infty$ implies that $h\in L^1_\Real\brac{\Omega, \Fcal, \mu^{\Omega_n}}$ for all $n\geq 1$.

\label{thm:cplx_meas_tot_var_int} \noindent \textbf{Theorem} 62.
For all complex measures $\mu$ on $M^1\brac{\Omega, \Fcal}$ there is $h\in L^1_\Cplx\brac{\Omega, \Fcal, \abs{\mu}}$ such that $\abs{h}=1$ and for every $E\in \Fcal$ \[\mu\brac{E} = \int_E h d\abs{\mu}\] If $\mu$ is a signed measure, then $h\in L^1_\Real\brac{\Omega, \Fcal, \abs{\mu}}$.

Indeed, since $\abs{\mu\brac{\cdot}}\leq \abs{\mu}\brac{\cdot}$ by theorem 11-7, $\abs{\mu\brac{E}}=0$ for any $E\in \Fcal$ with $\abs{\mu}\brac{E}=0$. Thus $\mu<<\abs{\mu}$. Since $\brac{\Omega, \Fcal, \abs{\mu}}$ is a finite measure space by theorem 57, theorem 60 implies that there is $h\in L^1_\Cplx\brac{\Omega, \Fcal, \abs{\mu}}$ such that $\mu\brac{E} = \int_E h d\abs{\mu}$ for all $E\in \Fcal$. Note that, if $\mu$ is a signed measure, then $h\in L^1_\Real\brac{\Omega, \Fcal, \abs{\mu}}$ by theorem 60.

%% This is clever!!!
Let $r>0$ and put $A_r\defn \obj{\abs{h}<r}$. If $\brac{E_n}_{n\geq 1}$ is an arbitrary measurable partition of $A_r$, then $\sum_{n\geq 1}\abs{\mu}\brac{E_n}=\abs{\mu}\brac{A_r}$ because $\abs{\mu}$ is a measure on $\brac{\Omega, \Fcal}$. By theorem 24 for all $n\geq 1$ \[\abs{\mu\brac{E_n}} = \abs{\int_{E_n} h d\abs{\mu}}\leq \int_{E_n} \abs{h} d\abs{\mu} \leq r \abs{\mu}\brac{E_n}\] whence $\sum_{n\geq 1}\abs{\mu\brac{E_n}}\leq r \sum_{n\geq 1} \abs{\mu}\brac{E_n} = r \abs{\mu}\brac{A_r}$. Therefore $\abs{\mu}\brac{A_r}\leq r \abs{\mu}\brac{A_r}$ for all $r>0$, whence $\abs{\mu}\brac{A_r}=0$ for all $r\in\brac{0,1}$.

If $r_n\defn 1-\frac{1}{2^n}$, then $r_n\uparrow 1$ and $A_{r_n}\uparrow \obj{\abs{h}<1}$. Therefore $\abs{\mu}\brac{A_{r_n}}\uparrow\abs{\mu}\brac{\obj{\abs{h}<1}}$ by theorem 7, whence $\abs{\mu}\brac{\obj{\abs{h}<1}}=0$. Thus $\abs{h}\geq 1$ $\abs{\mu}$-a.s. Now, $\abs{\int_E h d\abs{\mu}} = \abs{\mu\brac{E}} \leq \abs{\mu}\brac{E}$ for all $E\in \Fcal$, whence, in particular, for all $E\in \Fcal$ with $\abs{\mu}\brac{E}>0$ \[\abs{ \frac{1}{\abs{\mu}\brac{E}} \int_E h d\abs{\mu} } \leq 1\] Since the set $\obj{\induc{z\in \Cplx} \abs{z}\leq1}$ is closed in $\Cplx$ and $\abs{\mu}$ is a finite measure, theorem 59 implies that $\abs{h}\leq 1$ $\abs{\mu}$-a.s.

Therefore $\abs{h}=1$ $\abs{\mu}$-a.s. and for $h'\defn h 1_{N^c} + 1_N$ with $N\defn \obj{\abs{h}\neq1}\in \Fcal$ it is true that $h'\in L^1_\Cplx\brac{\Omega, \Fcal, \abs{\mu}}$ and $\abs{h'}=1$ everywhere on $\Omega$. Furthermore, $h'=h$ $\mu$-a.s., which implies that $\int 1_E h d\abs{\mu}=\int 1_E h' d\abs{\mu}$, whence $\mu\brac{E} = \int_E h' d\abs{\mu}$ for all $E\in \Fcal$.\\

% If $A_n\uparrow A$, then $1_{A_n}\uparrow 1_A$, while if $A_n\downarrow A$, then $1_{A_n}\downarrow 1_A$. Indeed, for any $x\in A$ there is $N\geq 1$ such that $x\in A_n$ for all $n\geq N$, whence $1_{A_n}\brac{x}=1=1_A\brac{x}$ for all $n\geq N$. Otherwise, if $x\notin A$, then $x\notin A_n$ for all $n\geq 1$, whence $1_{A_n}\brac{x}=0=1_A\brac{x}$ for all $n\geq N$.

\label{thm:cplx_meas_cont} \noindent \textbf{Theorem} 99-9.
Let $\mu$ be a complex measure on $\brac{\Omega, \Fcal}$. Whenever $\brac{A_n}_{n\geq 1}, A\in \Fcal$ are such that $1_{A_n}\to 1_A$, then $\mu\brac{A} = \lim_{n\to\infty} \mu\brac{A_n}$.

Indeed, by theorem 62 there is $h\in L^1_\Cplx\brac{\Omega, \Fcal, \abs{\mu}}$ with $\abs{h}=1$ and $\mu\brac{E} = \int_E h d\abs{\mu}$ for every $E\in \Fcal$. Thus $\brac{h 1_{A_n}}_{n\geq1}\in L^1_\Cplx\brac{\Omega, \Fcal, \abs{\mu}}$ and $h 1_{A_n}\to h 1_A$ everywhere on $\Omega$. In addition $\abs{h 1_{A_n}} \leq \abs{h}$ for and $n\geq 1$ and $\abs{h}\in \in L^1_\Real\brac{\Omega, \Fcal, \abs{\mu}}$. Therefore by theorem 23 (DCT) on $\brac{\Omega, \Fcal, \abs{\mu}}$ and theorem 24 \[\abs{\int_{A_n} h d\abs{\mu} - \int_A h d\abs{\mu}}\leq \int \abs{h 1_{A_n} - h 1_A} d\abs{\mu} \to 0\] whence $\mu\brac{A} = \lim_{n\to\infty} \mu\brac{A_n}$.\\

\label{thm:tot_var_modulus_int} \noindent \textbf{Theorem} 63.
Let $\mu$ be a usual measure on $\brac{\Omega, \Fcal}$ and $f\in L^1_\Cplx\brac{\Omega, \Fcal, \mu}$. Then the map $\nu\brac{E}\defn \int_E f d\mu$ for all $E\in \Fcal$ is a complex measure with total variation for all $E\in \Fcal$ \[\abs{\nu}\brac{E} = \int_E \abs{f} d\mu\]

Indeed, by theorem 11-6 the map $\nu$ is a complex measure. By theorem 62 there exists $h\in L^1_\Cplx\brac{\Omega, \Fcal, \abs{\nu}}$ with $\abs{h}=1$ and such that $\nu = \int h d\abs{\nu}$. By definition of the partial Lebesgue integral with respect to a usual measure for all $E, F\in \Fcal$ \[\int_E f 1_F d\mu = \nu\brac{E\cap F} = \int_E h 1_F d\abs{\nu}\]

Let $g:\brac{\Omega, \Fcal}\to \brac{\Cplx,\borel{\Cplx}}$ be any bounded and measurable map. Then there exists $M\in \Real^+$ such that $\abs{g}\leq M$ on $\Omega$. Since $\abs{f g}\leq M\abs{f}$ and $\abs{h g}\leq M$, $\int \abs{f g} d\mu \leq M\int \abs{f} d\mu < +\infty$ and $\int d\abs{\nu} \leq M \nu\brac{\Omega}<+\infty$, whence $fg \in L^1_\Cplx\brac{\Omega, \Fcal, \mu}$ and $hg \in L^1_\Cplx\brac{\Omega, \Fcal, \abs{\nu}}$.

Let $s$ be a simple function on $\brac{\Omega, \Fcal}$ with partition $\sum_{k=1}^n \alpha_k 1_{F_k}$, where $\brac{\alpha_k}_{k=1}^n\in \Real^+\subseteq \Cplx$ and $\brac{F_k}_{k=1}^n\in \Fcal$. Functions of this kind are bounded, since $s\leq M\defn \max_{k=1\ldots n} \alpha_k\in \Real^+$ on $\Omega$. Therefore for all $E\in \Fcal$ the left and the right-end complex Lebesgue integrals are well defined and by linearity \[\int_E f s d\mu = \sum_{k=1}^n \alpha_k \int_E f 1_{F_k} d\mu = \nu\brac{E\cap F} = \sum_{k=1}^n \alpha_k \int_E h 1_{F_k} d\abs{\nu} = \int_E h s d\abs{\nu}\]

Suppose $g:\brac{\Omega, \Fcal}\to \brac{\Real^+,\borel{\Real^+}}$ is a non-negative, bounded and measurable function. Then by theorem 18 there exists a sequence $\brac{s_n}_{n\geq 1}$ of simple functions on $\brac{\Omega, \Fcal}$ such that $s_n\uparrow g$. Boundedness of $g$ implies that $f g\in L^1_\Cplx\brac{\Omega, \Fcal, \mu}$ and $h g \in L^1_\Cplx\brac{\Omega, \Fcal, \abs{\nu}}$, whereas $s_n\leq g$ yields $\abs{f s_n}\leq \abs{f g}\in L^1_\Real\brac{\Omega, \Fcal, \mu}$ and $\abs{h s_n}\leq \abs{h g}\in L^1_\Real\brac{\Omega, \Fcal, \abs{\nu}}$. For all $E\in \Fcal$ the sequence $1_E f s_n\overset{\Cplx}{\to} 1_E f g$ and $1_E h s_n\overset{\Cplx}{\to} 1_E h g$, whence by applying of theorem 23 (DCT) on $\brac{\Omega, \Fcal, \mu}$ and $\brac{\Omega, \Fcal, \abs{\nu}}$ one gets, respectively, \[\int 1_E\abs{f s_n - f g} d\mu\to 0\,\text{and}\,\int 1_E\abs{h s_n - h g} d\abs{\nu}\to 0\] Therefore by theorem 24 $\int 1_E f s_n d\mu \overset{\Cplx}{\to} \int 1_E f g d\mu$ and $\int 1_E h s_n d\abs{\nu} \overset{\Cplx}{\to} \int 1_E h g d\abs{\nu}$. Finally, since $\int 1_E f s_n d\mu = \int 1_E h s_n d\abs{\nu}$ for all $E\in \Fcal$ and $n\geq 1$, the fact that $\Cplx$ is a Hausdorff topological space implies $\int_E f g d\mu = \int_E h g d\abs{\nu}$ for all $E\in \Fcal$ by definition of the partial complex Lebesgue integral.

%% Revise if needed!!
The function $g:\brac{\Omega, \Fcal}\to \brac{\Real,\borel{\Real}}$ is bounded and measurable if and only if the maps $g^+,g^-:\brac{\Omega, \Fcal}\to \brac{\Real^+,\borel{\Real^+}}$ are bounded and measurable, because $\abs{g} = g^++g^-$ and $g=g^+-g^-$. Since the complex Lebesgue integral is linear and $g^+,g^-$ are bounded, it is true that $\int_E f g d\mu = \int_E h g d\abs{\nu}$ for all $E\in \Fcal$.

If $g:\brac{\Omega, \Fcal}\to \brac{\Cplx,\borel{\Cplx}}$ is bounded and measurable, then $\re g, \im g:\brac{\Omega, \Fcal}\to \brac{\Real,\borel{\Real}}$ measurable and bounded, because $\abs{\re \cdot}, \abs{\im \cdot}\leq \abs{\cdot}$. Thus linearity with respect to $\Cplx$ combinations and definition of the complex Lebesgue integral imply that for all $E\in \Fcal$ \[\int_E f g d\mu = \int_E f \re g d\mu + i \int_E f \im g d\mu = \int_E h \re g d\abs{\nu} + i \int_E h \im g d\abs{\nu} = \int_E h g d\abs{\nu}\]

Since $\abs{h}=1$, $h\in L^1_\Cplx\brac{\Omega, \Fcal, \abs{\nu}}$ is obviously a bounded measurable map, whence $\int f\bar{h} d\mu = \int h \bar{h} d\abs{\nu} = \abs{\nu}\brac{E}$ for all $E\in \Fcal$. Furthermore, since $\abs{\nu}\brac{E}\in \Real^+$, by definition of the complex Lebesgue integral $\im \abs{\nu}\brac{E} = \int \im f\bar{h} d\mu = 0$ and $\abs{\nu}\brac{E} = \re \abs{\nu}\brac{E} = \int \re f\bar{h} d\mu \geq 0$ for all $E\in \Fcal$. By theorem 99-4 $\im f\bar{h} = 0$ $\mu$-a.s., while by theorem 99-3 $\re f\brac{h} \geq 0$ $\mu$-a.s. Thus $f\bar{h}\in \Real^+$ $\mu$-a.s. and $f\bar{h}=\abs{f\bar{h}}$ $\mu$-a.s., whence $f\bar{h}=\abs{f}$ $\mu$-a.s. because $\abs{f\bar{h}} = \abs{f} \abs{h}$. Therefore $\abs{\nu}\brac{E} = \int_E \abs{f} d\mu$ for all $E\in \Fcal$.\\

\label{thm:hahn_decompsition} \noindent \textbf{Theorem} 64.
Let $\mu$ be a signed measure on $\brac{\Omega, \Fcal}$. There exist disjoint $A_+,A_-\in \Fcal$ with $\Omega = A_+ \uplus A_-$ such that $\mu^+\brac{E} = \mu\brac{A_+\cap E}$ and $\mu^-\brac{E} = -\mu\brac{A_-\cap E}$ for all $E\in \Fcal$.

Indeed, by theorem 62 there is $h\in L^1_\Real\brac{\Omega, \Fcal, \abs{\mu}}$ such that $\mu\brac{E} = \int_E h d\abs{\mu}$ for all $E\in \Fcal$. Let $A_+\def \obj{h=+1}$ and $A_-\defn \obj{h=-1}$. Note that $A_+\cap A_-=\emptyset$ and $\Omega = A_+ \cup A_-$.

By theorem 11-14, $\mu^+\defn \frac{1}{2}\brac{\abs{\mu} +\mu}$ and $\mu^-\defn \frac{1}{2}\brac{\abs{\mu} -\mu}$ are finite usual measures on $\brac{\Omega, \Fcal}$ such that $\mu=\mu^+-\mu^-$ and $\abs{\mu}=\mu^++\mu^-$. Now since $\abs{\mu}\brac{E} = \int_E d\abs{\mu}$ for all $E\in \Fcal$, \begin{align*}\mu^+\brac{E} &= \frac{1}{2}\brac{\int_E d\abs{\mu} + \int_E h d\abs{\mu}} = \int_E \frac{1}{2} \brac{1+h} d\abs{\mu}\\\mu^-\brac{E} &= \frac{1}{2}\brac{\int_E d\abs{\mu} - \int_E h d\abs{\mu}} = \int_E \frac{1}{2} \brac{1-h} d\abs{\mu}\end{align*}

Note that for all $E\in \Fcal$ \begin{align*}\mu^+\brac{E\cap A_-} &= \int_{E\cap \obj{h=-1}} \frac{1}{2} \brac{1+h} d\abs{\mu} = 0\\ \mu^-\brac{E\cap A_+} &= \int_{E\cap \obj{h=+1}} \frac{1}{2} \brac{1-h} d\abs{\mu} = 0\end{align*} Therefore $\mu\brac{E\cap A_+} = \mu^+\brac{E\cap A_+} - \mu^-\brac{E\cap A_+} = \mu^+\brac{E}$ and $\mu\brac{E\cap A_-} = \mu^+\brac{E\cap A_-} - \mu^-\brac{E\cap A_-} = -\mu^-\brac{E}$.\\

%% SO FAR SO GOOD!!!

%%%% FROM HERE REVISION SHOULD BE DONE!

%% Integral with respect to a complex measure.
\noindent \textbf{Definition} 97.
Let $\mu$ be a complex measure on $\brac{\Omega, \Fcal}$. Define $L^1_\Cplx\brac{\Omega, \Fcal, \mu}\defn L^1_\Cplx\brac{\Omega, \Fcal, \abs{\mu}}$. For all $f\in L^1_\Cplx\brac{\Omega, \Fcal, \mu}$ define the Lebesgue integral of $f$ with respect to $\mu$ as \[\int f d\mu \defn \int f h d\abs{\mu}\] where $h\in L^1_\Cplx\brac{\Omega, \Fcal, \abs{\mu}}$ is such that $\abs{h}=1$ and $\mu\brac{E}=\int_E h d\abs{\mu}$ for all $E\in \Fcal$.

Since $\abs{f h} = \abs{f}$, $f\in L^1_\Cplx\brac{\Omega, \Fcal, \abs{\mu}}$ implies that $\int \abs{f h} d\abs{\mu} = \int \abs{f} d\abs{\mu} < +\infty$, whence $f h\in L^1_\Cplx\brac{\Omega, \Fcal, \abs{\mu}}$. Therefore $\int f h d\abs{\mu}$ is a well-defined complex Lebesgue integral. Conversely, if $f:\brac{\Omega, \Fcal}\to \brac{\Cplx, \borel{\Cplx}}$ and $\int \abs{f} d\abs{\mu}<+\infty$, then $f\in L^1_\Cplx\brac{\Omega, \Fcal, \abs{\mu}}$.

Since the total variation of $\mu$ is a finite measure, $1_E\in L^1_\Cplx\brac{\Omega, \Fcal, \mu}$ for all $E\in \Fcal$, because $\int \abs{1_E} d\abs{\mu} = \abs{\mu}\brac{E}<+\infty$. Properties of $h$ imply that $\int 1_E d\mu = \int 1_E h d\abs{\mu} = \int_E h d\abs{\mu} = \mu\brac{E}$.

If $\mu$ is a finite measure, then $\mu$ can be treated as a complex measure with values in $\Real^+$. Thus $\int h d\abs{\mu}\geq 0$ for all $E\in \Fcal$, whence $h\geq 0$ $\abs{\mu}$-a.s. by theorem 99-3. Therefore $h=1$ $\abs{\mu}$-a.s. and $\mu\brac{E} = \abs{\mu}\brac{E}$ for all $E\in \Fcal$.

The source of potential ambiguity it this definition is the map $h\in L^1_\Cplx\brac{\Omega, \Fcal, \abs{\mu}}$ with $\abs{h}=1$ and $\mu=\int h d\abs{\mu}$ existence of which is guaranteed by theorem 62. Indeed, if $h'$ is another map of this kind, then $h',h\in L^1_\Cplx\brac{\Omega, \Fcal, \abs{\mu}}$ are such that $\int_E h-h' d\abs{\mu}=0$ for all $E\in \Fcal$, then by theorem 99-4 $h=h'$ $\abs{\mu}$-almost surely. Therefore $\int f h d\abs{\mu} = \int f h' d\abs{\mu}$ for all $f\in L^1_\Cplx\brac{\Omega, \Fcal, \mu}$, implying that this definition is solid.

Another ambiguity in this definition might arise when the measure $\mu$ is a finite measure on $\brac{\Omega, \Fcal}$, since in this case $\mu$ is a complex measure with values in $\Real^+$. In this case $\int f d\mu$ is either a well-studied complex Lebesgue integral, or the just introduced Lebesgue integral with respect to a complex measure. Luckily, these integrals coincide in this specific case. Indeed, since $\abs{\mu}=\mu$, $\int f d\mu = \int d\abs{\mu}$ for any $f:\brac{\Omega, \Fcal}\to \Zinf$. Therefore, $L^1_\Cplx\brac{\Omega, \Fcal, \abs{\mu}} = L^1_\Cplx\brac{\Omega, \Fcal, \mu}$ and the complex Lebesgue integrals coincide: $\int f d\mu = \int f d\abs{\mu}$.

Since $L^1_\Cplx\brac{\Omega, \Fcal, \abs{\mu}}$ is a $\Cplx$-vector space, $L^1_\Cplx\brac{\Omega, \Fcal, \mu}$, being its copy, is also a $\Cplx$-vector space. Let $f,g\in L^1_\Cplx\brac{\Omega, \Fcal, \mu}$ and $\alpha\in \Cplx$. Since $\brac{f + \alpha g} h = f h + \alpha g h$, linearity of the complex Lebesgue integral on $\brac{\Omega, \Fcal, \abs{\mu}}$ implies linearity of the freshly defined extension: \[\int \brac{f + \alpha g} d\mu = \int \brac{f + \alpha g} h d\abs{\mu} = \int f h d\abs{\mu} + \alpha \int g h d\abs{\mu} = \int f d\mu + \alpha \int g d\mu\] Finally, for all $f\in L^1_\Cplx\brac{\Omega, \Fcal, \mu}$ theorem 24 and $\abs{h}=1$ imply \[\abs{\int f d\mu} = \abs{\int f h d\abs{\mu}} \leq \int \abs{f h} d\abs{\mu} = \int \abs{f} d\abs{\mu}\]

\label{thm:cplx_meas_sum_cplx_int} \noindent \textbf{Theorem} 99-10.
Let $\mu, \nu\in M^1\brac{\Omega, \Fcal}$ and $\alpha\in \Cplx$. Then \[L^1_\Cplx\brac{\Omega, \Fcal, \mu}\cap L^1_\Cplx\brac{\Omega, \Fcal, \nu} \subseteq L^1_\Cplx\brac{\Omega, \Fcal, \mu+\alpha \nu}\] and for all $f\in L^1_\Cplx\brac{\Omega, \Fcal, \mu}\cap L^1_\Cplx\brac{\Omega, \Fcal, \nu}$ it is true that the newly defined complex integral is linear with respect to the measure\[\int f d\brac{\mu+\alpha\nu} = \int f d\mu + \alpha \int f d\nu\]

First of all $\mu+\nu, \alpha \nu \in M^1\brac{\Omega, \Fcal}$ by theorem 11-13, whence $\mu+\alpha \nu$ is a complex measure on $\brac{\Omega, \Fcal}$.

Let $E\in \Fcal$ and $\brac{E_n}_{n\geq1}$ be a measurable partition of $E$. Then $\sum_{k\geq1} \abs{\alpha \nu\brac{E_k}} = \abs{\alpha} \sum_{k\geq 1} \abs{\nu\brac{E_k}}$, whence $\abs{\alpha \nu}\brac{E}\leq \abs{\alpha}\abs{\nu}\brac{E}$. Conversely, if $\alpha=0$ then trivially $\abs{\alpha}\abs{\nu}\brac{E} = 0 \leq \abs{\alpha \nu}\brac{E}$, while if $\alpha\neq 0$, then $\sum_{k\geq 1}\abs{\nu\brac{E_k}}\leq \frac{1}{\abs{\alpha}} \sum_{k\geq1} \abs{\alpha \nu\brac{E_k}}$, whence $\abs{\alpha}\brac{\abs{\nu}\brac{E}} \leq \abs{\alpha \nu}\brac{E}$. Therefore $\abs{\alpha \nu}=\abs{\alpha} \abs{\nu}$ for all $\alpha\in \Cplx$. Furthermore, by theorem Sup-B-3, the definition of measure addition and by the triangle inequality \[\sum_{n\geq1} \abs{\brac{\mu+\nu}\brac{E_n}}\leq \sum_{n\geq1} \abs{\mu\brac{E_n}} + \sum_{n\geq1} \abs{\nu\brac{E_n}}\leq \abs{\mu}\brac{E} + \abs{\nu}\brac{E}\] whence the definition of the total variation of $\mu$ implies that $\abs{\mu+\nu}\leq \abs{\mu}+\abs{\nu}$.

By the above $\abs{\mu + \alpha \nu}\leq \abs{\mu}+\abs{\alpha}\abs{\nu}$, whence if $f\in L^1_\Cplx\brac{\Omega, \Fcal, \mu}\cap L^1_\Cplx\brac{\Omega, \Fcal, \nu}$, then by theorem 99-6 $\int \abs{f} d\abs{\mu+\alpha \nu}\leq \int \abs{f} d\abs{\mu} + \abs{\alpha} \int \abs{f} d\abs{\nu} < +\infty$, implying that $f\in L^1_\Cplx\brac{\Omega, \Fcal, \mu+\alpha\nu}$ by definition of the latter space.

Since $\lambda\defn \mu + \alpha \nu \in M^1\brac{\Omega, \Fcal}$, by theorem 62 there exists $u\in L^1_\Cplx\brac{\Omega, \Fcal, \abs{\lambda}}$ with $\abs{u}=1$ and such that $\lambda\brac{E} = \int_E h d\abs{\lambda}$ for all $E\in \Fcal$.

By definition of the partition complex Lebesgue integral, for the newly introduced extension it is true that $\int 1_E d\eta = \int 1_E u d\abs{\eta} = \int_E u d\abs{\eta} = \eta\brac{E}$ for all $E\in \Fcal$ and any complex measure $\eta$, whence \[\int 1_E d\lambda = \lambda\brac{E} = \mu\brac{E} + \alpha \nu\brac{E} = \int 1_E d\mu + \alpha \int 1_E d\nu\]

Next, if $s=\sum_{k=1}^n \beta_k 1_{B_k}$ is a simple map on $\brac{\Omega, \Fcal}$ then it is bounded, which implies that there is $M\in \Real^+$ with $s\leq M$. Therefore, \[\abs{\int s d\eta} = \abs{\int s u d\abs{\eta}}\leq\int \abs{s} d\abs{\eta}\leq \int M d\abs{\eta}<+\infty\] for any complex measure $\eta$, whence all involved Lebesgue integrals with respect to complex measures are well defined. Therefore by linearity of the complex Lebesgue integral with respect to a complex measure it must be true that \[\int s d\lambda = \sum_{k=1}^n \beta_k \int 1_{B_k} d\lambda = \sum_{k=1}^n \beta_k \int 1_{B_k} d\mu + \sum_{k=1}^n \beta_k \alpha \int 1_{B_k} d\nu = \int s d\mu + \alpha \int s d\nu\]

Finally, if $f:\brac{\Omega, \Fcal}\to \Zinf$ is such that $f\in L^1_\Cplx\brac{\Omega, \Fcal, \mu}\cap L^1_\Cplx\brac{\Omega, \Fcal, \nu}$, by theorem 18 there exist simple functions $\brac{s_n}_{n\geq1}$ with $s_n\uparrow f$ everywhere. Thus $s_n u\overset{\Cplx}{\to} f u$, $s_n u\in L^1_\Cplx\brac{\Omega, \Fcal, \abs{\lambda}}$ and $\abs{s_n u}\leq \abs{s_n} \leq \abs{f}\in L^1_\Real\brac{\Omega, \Fcal, \abs{\lambda}}$. By theorems 23 (DCT) and 24 on respective finite measure spaces and the continuity of addition in $\Cplx$ \[\int s_n d\lambda \overset{\Cplx}{\to} \int f d\mu + \alpha \int f d\nu\]

Finally, since $f\in L^1_\Cplx\brac{\Omega, \Fcal, \mu}\cap L^1_\Cplx\brac{\Omega, \Fcal, \nu}$, $f$ can be decomposed into bounded measurable non-negative functions. Therefore by linearity of the complex Lebesgue integral $\int f d\lambda = \int f d\mu + \alpha \int f d\nu$.\\

%%% REVISION IS NEEDED
%%% OK up to here!

% for all $E\in \Fcal$ it is true that \[\int 1_E d\abs{\mu + \alpha \nu} \leq \abs{\mu}\brac{E}+\abs{\alpha}\abs{\nu}\brac{E} = \int 1_E d\abs{\mu} + \abs{\nu} \int 1_E d\abs{\nu}\] For any simple $s=\sum_{k=1}^n \alpha_k 1_{E_k}$ linearity of the usual Lebesgue integral implies that $\int s d\abs{\mu + \alpha \nu} \leq \int s d\abs{\mu} + \abs{\alpha} \int s d\abs{\nu}$. Finally if $f:\brac{\Omega, \Fcal}\to\Zinf$ then by theorem 18 there exists $\brac{s_n}_{n\geq1}$ simple functions such that $s_n\uparrow g$ everywhere. Therefore by the MCT on $\brac{\Omega, \Fcal, \abs{\mu + \alpha \nu}}$, $\brac{\Omega, \Fcal, \abs{\mu}}$, and $\brac{\Omega, \Fcal, \abs{\nu}}$ it is true that \begin{align*}\int s_n d\abs{\mu + \alpha \nu}&\uparrow \int g d\abs{\mu + \alpha \nu}\\\int s_n d\abs{\mu}&\uparrow \int g d\abs{\mu}\\\int s_n d\abs{\nu}&\uparrow \int g d\abs{\nu}\\\end{align*} Therefore by the order and topological properties of $\Rbar$ it is true that 

\label{thm:cplx_meas_cplx_int_decompo} \noindent \textbf{Theorem} 99-11.
Let $\mu$ be a complex measure on $\brac{\Omega, \Fcal}$, and $\mu_1\defn \re \mu$ and $\mu_2\defn\im \mu$. Then for all $f\in L^1_\Cplx\brac{\Omega, \Fcal, \mu}$ the newly defined complex integral with respect to $\mu$ has the following decomposition \[\int f d\mu = \int f d\mu_1^+ - \int f d\mu_1^- + i \brac{ \int f d\mu_2^+ - \int f d\mu_2^-}\]

Let $\mu\in M^1\brac{\Omega, \Fcal}$, $\mu_1\defn \re \mu$ and $\mu_2\defn\im \mu$. Let $E\in \Fcal$ and $\brac{E_n}_{n\geq1}\in \Fcal$ be a measurable partition of $E$. Since $\abs{\re \cdot},\abs{\im \cdot}\leq \abs{\cdot}$, it is true that $\sum_{n\geq1} \abs{\mu_1\brac{E_n}}, \sum_{n\geq1} \abs{\mu_2\brac{E_n}}\leq \sum_{n\geq1} \abs{\mu\brac{E_n}}\leq \abs{\mu}\brac{E}$, whence $\abs{\mu_1},\abs{\mu_2}\leq \abs{\mu}$. Furthermore, since $\abs{\cdot}\leq \abs{\re \cdot}+\abs{\im \cdot}$, by theorem Sup-B-3 $\sum_{n\geq1} \abs{\mu\brac{E_n}}\leq \sum_{n\geq1} \abs{\mu_1\brac{E_n}} + \sum_{n\geq1} \abs{\mu_2\brac{E_n}}\leq \abs{\mu_1}\brac{E}+\abs{\mu_2}\brac{E}$, whence $\abs{\mu}\leq \abs{\mu_1}+\abs{\mu_2}$. In addition, $\abs{\mu_1}+\abs{\mu_2}$ is a finite measure on $\brac{\Omega, \Fcal}$ (see theorem 99-5).

Now, if $f\in L^1_\Cplx\brac{\Omega, \Fcal, \mu}$, then $\int \abs{f} d\abs{\mu}<+\infty$, whence by theorem 99-6 \[\int \abs{f} d\abs{\mu_1},\int \abs{f} d\abs{\mu_2}\leq \int \abs{f} d\abs{\mu}<+\infty\] and $f\in L^1_\Cplx\brac{\Omega, \Fcal, \mu_1}$ and $f\in L^1_\Cplx\brac{\Omega, \Fcal, \mu_2}$. Conversely, if $f\in L^1_\Cplx\brac{\Omega, \Fcal, \mu_1} \cap L^1_\Cplx\brac{\Omega, \Fcal, \mu_2}$, then by theorem 99-6 and 99-5 \[\int \abs{f} d\abs{\mu}\leq \int \abs{f} d\abs{\mu_1}+\int \abs{f} d\abs{\mu_2}<+\infty\] and therefore $f\in L^1_\Cplx\brac{\Omega, \Fcal, \mu}$.

Let $\mu\in M^1\brac{\Omega, \Fcal}$ be a signed measure. Then by theorem 11-14 there are finite usual measures $\mu^+, \mu^-$ such that $\abs{\mu} = \mu^+ + \mu^-$. Then for any $f\in L^1_\Cplx\brac{\Omega, \Fcal, \mu}$ by theorem 99-6 \[\int \abs{f} d\mu^+, \int \abs{f} d\mu^- \leq \int \abs{f} d\abs{\mu} <+\infty\] whence $f\in L^1_\Cplx\brac{\Omega, \Fcal, \mu^+}\cap L^1_\Cplx\brac{\Omega, \Fcal, \mu^-}$. Conversely, if $f\in L^1_\Cplx\brac{\Omega, \Fcal, \mu^+}\cap L^1_\Cplx\brac{\Omega, \Fcal, \mu^-}$, then by theorem 99-5 $\int \abs{f} d\abs{\mu} = \int \abs{f} d\mu^+ + \int \abs{f} d\mu^- < +\infty$, whence $f\in L^1_\Cplx\brac{\Omega, \Fcal, \mu}$.

Therefore, if $\mu\in M^1\brac{\Omega, \Fcal}$, then $\mu = \mu_1^+ - \mu_1^- + i \brac{ \mu_2^+ - \mu_2^- }$ by theorem 11-15 and for all $f\in L^1_\Cplx\brac{\Omega, \Fcal, \mu}$ by the above result and theorem 99-10 \[\int f d\mu = \int f d\mu_1 + i \int f d\mu_2 = \int f d\mu_1^+ - \int f d\mu_1^- + i \brac{\int f d\mu_2^+ - \int f d\mu_2^- }\]

%%%%%%%%%%%%%%%%%%%%%%%%%%%%%%%%%%%%%%%%%%
Let $\mu \in M^1\brac{\Omega, \Fcal}$. Let $A\in \Fcal$ and $h\in L^1_\Cplx\brac{\Omega, \Fcal, \abs{\mu}}$ with $\abs{h}=1$ be such that $\mu=\int h d\abs{\mu}$ (exists by theorem 62). Put $\mu^A\defn \mu\brac{A\cap\cdot}$, $\induc{\mu}_A\defn \induc{\mu}_{\induc{\Fcal}_A}$, where $\induc{\Fcal}_A$ is the trace $\sigma$-algebra on $A$.

First, for any $E\in \induc{\Fcal}_A$, there is $F\in \Fcal$ such that $E=F\cap A$. Since $A\in \Fcal$, $F\cap A\in \Fcal$ and $F\cap A\subseteq A$, whence $E\in \obj{\induc{E\in \Fcal}\, E\subseteq A}$. Conversely, if $E\in \Fcal$ is such that $E\subseteq A$, then $E=E\cap A\in \induc{\Fcal}_A$. Therefore \[\induc{\Fcal}_A = \obj{ \induc{E\in \Fcal}\, E\subseteq A }\]

Second, if $E\in\Fcal$ and$\brac{E_n}_{n\geq1}$ is a measurable partition of $E$, then $E_n\cap A$ is a measurable partition of $E\cap A$, whence the series $\sum_{n=1}^\infty \mu^A\brac{E_n}$ converges to $\mu^A\brac{E}$ in $\Cplx$, since $\sum_{n=1}^\infty \mu\brac{A\cap E_n}$ converges to $\mu\brac{E_\cap A}$ in $\Cplx$. Thus $\mu^A\in M^1\brac{\Omega, \Fcal}$.

Third, recall that $\brac{A, \induc{\Fcal}_A}$ is a measurable space. So, let $F\in \induc{\Fcal}_A$ and $\brac{F_n}_{n\geq1}$ be a measurable partition of $F$ in $\induc{\Fcal}_A$. Then $F, \brac{F_n}_{n\geq1}\in \Fcal$ and $F, \brac{F_n}_{n\geq1}\subseteq A$, whence $F=\biguplus_{n\geq1} F_n$ and $\brac{F_n}_{n\geq1}$ is a measurable partition of $F$ in $\Fcal$. Therefore the series $\sum_{n=1}^\infty \mu\brac{F_n}$ converges to $\mu\brac{F}$ in $\Cplx$, whence $\sum_{n=1}^\infty \induc{\mu}_A\brac{F_n}$ converges to $\induc{\mu}_A\brac{F}$ in $\Cplx$. Therefore $\induc{\mu}_A$ is indeed a complex measure on $\brac{A, \induc{\Fcal}_A}$.

Now, let $E\in \Fcal$ and $\brac{E_n}_{n\geq1}$ be a measurable partition of $E$. Then $\brac{E_n\cap A}_{n\geq1}$ is a measurable partition of $E\cap A$, whence \[\sum_{n\geq1} \abs{\mu^A\brac{E_n}} = \sum_{n\geq1} \abs{\mu\brac{A\cap E_n}} \leq \abs{\mu}\brac{A\cap E} = \abs{\mu}^A\brac{E}\] Thus $\abs{\mu^A}\leq \abs{\mu}^A$.

Now, if $E\in \Fcal$ and $\brac{E_n}_{n\geq1}$ is a measurable partition of $A\cap E$ then $\sum_{n\geq1} \abs{\mu\brac{E_n}} = \sum_{n\geq1}\abs{\mu^A\brac{E_n}}\leq \abs{\mu^A}\brac{E\cap A}$, since $\mu\brac{E_n}=\mu\brac{E_n\cap A}=\mu^A\brac{E_n}$. Thus $\abs{\mu}^A\brac{E} = \abs{\mu}\brac{E\cap A}\leq \abs{\mu^A}\brac{E\cap A}$. If $\brac{F_n}_{n\geq1}$ is a measurable partition of $A^c$, then $\sum_{n\geq1}\abs{\mu^A\brac{F_n}} = \sum_{n\geq1}\abs{\mu\brac{A\cap F_n}} = 0$, since $F_n \cap A \subseteq A^c \cap A = \emptyset$ and $\mu\brac{\emptyset}=0$. Thus $\abs{\mu^A}\brac{A^c}\leq 0$, which implies that $\abs{\mu^A}\brac{E} = \abs{\mu^A}\brac{E\cap A}$. Therefore $\abs{\mu}^A = \abs{\mu^A}$.

Let $E\in \induc{\Fcal}_A$ and $\brac{E_n}_{n\geq1}$ be a $\induc{\Fcal}_A$-measurable partition of $E$. Since $\induc{\mu}_A\brac{E_n} = \mu\brac{E_n}$, it is true that \[\sum_{n\geq1}\abs{\induc{\mu}_A\brac{E_n}} = \sum_{n\geq1}\abs{\mu\brac{E_n}}\leq \abs{\mu}\brac{E} = \induc{\abs{\mu}}_A\brac{E}\] whence $\abs{\induc{\mu}_A}\brac{E}\leq \induc{\abs{\mu}}_A\brac{E}$.

Let $E\in \induc{\Fcal}_A\subseteq \Fcal$ and $\brac{E_n}_{n\geq1}$ be an $\Fcal$-measurable partition of $E$. Then $E_n\subseteq E\subseteq A$ for all $n\geq1$, which implies that $\brac{E_n}_{n\geq1}$ is also an $\induc{\Fcal}_A$-measurable partition of $E$. Thus \[\sum_{n\geq1} \abs{\mu\brac{E_n}} = \sum_{n\geq1} \abs{\induc{\mu}_A\brac{E_n}} \leq \abs{\induc{\mu}_A}\brac{E} \] whence $\induc{\abs{\mu}}_A\brac{E} = \abs{\mu}\brac{E}\leq \abs{\induc{\mu}_A}\brac{E}$. Therefore $\induc{\abs{\mu}}_A = \abs{\induc{\mu}_A}$.

Since $\mu\brac{E} = \int_E h d\abs{\mu}$ and $\abs{\mu}^A = \abs{\mu^A}$, for all $E\in \Fcal$ definition of the complex Lebesgue integral on $\brac{\Omega, \Fcal, \abs{\mu}}$ implies \[\mu^A\brac{E} = \int_{E\cap A} h d\abs{\mu} = \int_A 1_E h d\abs{\mu} = \int 1_E d\abs{\mu}^A = \int_E h d\abs{\mu^A}\]

First, $\induc{h}_A$ is $\induc{\Fcal}_A$-$\borel{\Cplx}$ measurable and by definition of the partition complex Lebesgue integral on $\brac{\Omega, \Fcal, \abs{\mu}}$ \[\int \abs{\induc{h}_A} d\abs{\induc{\mu}_A} = \int \induc{\abs{h}}_A d\induc{\abs{\mu}}_A = \int_A \abs{h} d\abs{\mu} = \int 1_A \abs{h} d\abs{\mu}\] since $\abs{\induc{\mu}_A} = \induc{\abs{\mu}}_A$. Therefore $\induc{h}_A\in L^1_\Cplx\brac{A, \induc{\Fcal}_A, \induc{\mu}_A}$.

If $E\in \induc{\Fcal}_A$, then $E\in \Fcal$ as in the current case $\induc{\Fcal}_A\subseteq \Fcal$. Hence by definition of the complex Lebesgue integral on $\brac{\Omega, \Fcal, \abs{\mu}}$ and the fact that $\abs{\induc{\mu}_A} = \induc{\abs{\mu}}_A$ \[\induc{\mu}_A\brac{E} = \mu\brac{E} = \int_E h d\abs{\mu} = \int \induc{h}_A d\induc{\abs{\mu}}_A = \int \induc{h}_A d\abs{\induc{\mu}_A}\]

If $f\in L^1_\Cplx\brac{\Omega, \Fcal, \mu}$, then $f\in L^1_\Cplx\brac{\Omega, \Fcal, \abs{\mu}}$, implying that $f 1_A \in L^1_\Cplx\brac{\Omega, \Fcal, \abs{\mu}}$. At the same time $\int \abs{f} d\abs{\mu^A} = \int \abs{f} d\abs{\mu}^A = \int 1_A \abs{f} d\abs{\mu}<+\infty$, whence $f\in L^1_\Cplx\brac{\Omega, \Fcal, \mu^A}$. Finally, $\induc{f}_A$ is $\brac{A, \induc{\Fcal}_A}$ measurable and by definition of the complex Lebesgue integral on $\brac{\Omega, \Fcal, \abs{\mu}}$ \[\int \abs{\induc{f}_A} d\abs{\induc{\mu}_A} = \int \induc{\abs{f}}_A d\induc{\abs{\mu}}_A = \int \induc{\abs{f}}_A d\induc{\abs{\mu}}_A = \int 1_A \abs{f} d\abs{\mu} < +\infty\] whence indeed $\induc{f}_A \in L^1_\Cplx\brac{A, \induc{\Fcal}_A, \induc{\mu}_A}$.

Now, by definition of the extension to complex measures and by the basic properties of the partial complex Lebesgue integral \[\int f 1_A d\mu = \int f 1_A h d\abs{\mu} = \int f h d\abs{\mu}^A  = \int f h d\abs{\mu^A} = \int f d\mu^A \] Similarly, prior observations imply that \[\int f 1_A d\mu = \int f 1_A h d\abs{\mu} = \int \induc{f h}_A d\induc{\abs{\mu}}_A = \int \induc{f}_A \induc{h}_A d\abs{\induc{\mu}_A} = \int \induc{f}_A d\induc{\mu}_A\]

\noindent \textbf{Definition} 98.
Let $f\in L^1_\Cplx\brac{\Omega, \Fcal, \mu}$ where $\mu$ is a complex measure on $\brac{\Omega ,\Fcal}$, and $A\in \Fcal$. The partial Lebesgue integral of $f$ with respect to $\mu$ over $A$ is the integral defined as:\[\int_A d\mu \defn \int \brac{f 1_A} d\mu = \int f d\mu^A = \int \induc{f}_A d\induc{\mu}_A\] where $\mu^A\defn \mu\brac{A\cap \cdot}$ is the complex measure on $\brac{\Omega, \Fcal}$, $\induc{f}_A$ is the restriction of $f$ to $A$ and $\induc{\mu}_A$ is the restriction of$\mu$ to $\induc{\Fcal}_A$, the trace of $\Fcal$ on $A$.

\label{thm:cplx_partial_int_meas} \noindent \textbf{Theorem} 65.
Let $f\in L^1_\Cplx\brac{\Omega, \Fcal, \mu}$, where $\mu$ is a complex measure on $\brac{\Omega, \Fcal}$. Then $\nu=\int f d\mu$, defined for all $E\in \Fcal$ as \[\nu\brac{E} \defn \int_E f d\mu\] is a complex measure on $\brac{\Omega, \Fcal}$ with total variation equal for all $E\in \Fcal$ to \[\abs{\nu}\brac{E}\defn \int_E \abs{f} d\abs{\mu}\] Moreover, for any measure map $g:\brac{\Omega, \Fcal}\to\brac{\Cplx, \borel{\Cplx}}$ it is true that \[g \in L^1_\Cplx\brac{\Omega, \Fcal, \nu} \Leftrightarrow gf \in L^1_\Cplx\brac{\Omega, \Fcal, \mu} \] and when such condition is satisfied \[\int g d\nu = \int gf d\mu\]

By definition of the complex integral with respect to a complex measure (97) and by theorem 62, there is $p\in L^1_\Cplx\brac{\Omega, \Fcal, \abs{\mu}}$ with $\abs{p}=1$ and $\mu\brac{E}=\int_E p d\abs{\mu}$, whence for any $f\in L^1_\Cplx\brac{\Omega, \Fcal, \mu}$ it is true that $f p\in L^1_\Cplx\brac{\Omega, \Fcal, \mu}$ and $\int fd\mu \defn \int f p d\mu$. Thus \[\nu\brac{E}=\int_E f d\mu = \int f 1_E d\mu = \int f p 1_E d\abs{\mu} = \int_E f p d\abs{\mu}\] Since $L^1_\Cplx\brac{\Omega, \Fcal, \mu}=L^1_\Cplx\brac{\Omega, \Fcal, \abs{\mu}}$, theorem 63 on $\brac{\Omega, \Fcal, \abs{\mu}}$ implies that $\nu = \int f d\mu \defn \int f p d\abs{\mu}$ is a complex measure on $\brac{\Omega, \Fcal}$ with total variation $\abs{\nu} = \int \abs{f p} d\abs{\mu}$. Furthermore $\abs{p}=1$ implies that $\abs{\nu} = \int \abs{f} d\abs{\mu}$.

If $g$ is $\Fcal$-$\borel{\Cplx}$ measurable, then by theorems 4-7, 4-9 and 4-11 $g f$ is $\Fcal$-$\borel{\Cplx}$ measurable. Note that by theorem 21 $\int g d\abs{\nu} = \int \abs{f} g d\abs{\mu}$ for every non-negative and measurable map $g:\brac{\Omega, \Fcal}\to\Zinf$. Thus, $g\in L^1_\Cplx\brac{\Omega, \Fcal, \nu}$ if and only if $g f\in L^1_\Cplx\brac{\Omega, \Fcal, \mu}$, because \[\int \abs{g f} d\abs{\mu} = \int \abs{g} \abs{f} d\abs{\mu} = \int \abs{g} d\abs{\nu}\]

Therefore whenever $g\in L^1_\Cplx\brac{\Omega,\Fcal,\nu}$ it is true that both integrals $\int g f d\mu$ and $\int f d\nu$ are well-defined according to definition 97. By theorem 62 there is $q\in L^1_\Cplx\brac{\Omega,\Fcal,\abs{\nu}}$ with $\abs{q}=1$ and such that $\nu = \int q d\abs{\nu}$.

Now, if $g \defn 1_E$ for any $E\in \Fcal$, then \[\int g d\nu = \int 1_E d\nu = \nu\brac{E} = \int 1_E f d\mu = \int g f d\mu\] by the basic properties of the complex Lebesgue integral with respect to a complex measure.

Let $s$ be a simple function on $\brac{\Omega, \Fcal}$. First, it must be bounded on $\Omega$ since by theorem 5-2 $s\brac{\Omega}$ is a finite subset of $\Real^+$. Thus $s\in L^1_\Cplx\brac{\Omega, \Fcal, \nu}$ since the total variation of $\nu$ is a finite usual measure. Now, since $s=\sum_{k=1}^n \alpha_k 1_{E_k}$, linearity of the complex Lebesgue integral with respect to a complex measure (theorem 12-??) implies that \[\int s d\nu = \sum_{k=1}^n \alpha_k \int 1_{E_k} d\nu = \sum_{k=1}^n \alpha_k \int 1_{E_k} f d\mu = \int s f d\mu\]

Suppose $g\in L^1_\Cplx\brac{\Omega,\Fcal,\nu}$ is such that $g\brac{\Omega}\subseteq \Real^+$. Then by theorem 18 there exist simple functions $\brac{s_n}_{n\geq1}$ on $\brac{\Omega,\Fcal}$ such that $s_n\uparrow g$. Hence $\brac{s_n q}_{n\geq1}\in L^1_\Cplx\brac{\Omega, \Fcal, \abs{\nu}}$, $s_n q\to g q$ in $\Cplx$ on $\Omega$ and $\abs{s_n q} \leq \abs{g}$. Therefore by theorem 23 (DCT) on $\brac{\Omega,\Fcal,\abs{\nu}}$ and theorem 24 it is true that \[\int s_n d\nu = \int s_n q d\abs{\nu} \overset{\Cplx}{\to} \int g q d\abs{\nu} = \int g d\nu\] Furthermore, $\brac{s_n f p}_{n\geq1}\in L^1_\Cplx\brac{\Omega,\Fcal,\abs{\mu}}$, $s_n f p \to g f p$ and $\abs{s_n f p}\leq \abs{g f}$ with $\abs{g f}\in L^1_\Real\brac{\Omega,\Fcal,\abs{\mu}}$. Similarly theorems 23 and 24 imply \[\int s_n f d\mu = \int s_n f p d\abs{\mu}\overset{\Cplx}{\to}\int g f p d\abs{\mu} = \int g f d\mu\] Finally, since $\Cplx$ is Hausdorff and $\int s_n d\nu = \int s_n f d\mu$ for all $n\geq1$ it must be true that $\int g d\nu = \int g f d\mu$ for such $g$.

Now for every $g\in L^1_\Cplx\brac{\Omega,\Fcal,\nu}$ there are non-negative $\brac{g_k}_{k=1}^4\:\brac{\Omega,\Fcal}\to\brac{\Cplx,\borel{\Cplx}}$ such that $g = g_1-g_2 + i\brac{g_3-g_4}$. By $\Cplx$-linearity of the complex Lebesgue integral with respect to a complex measure (theorem 12-??) it must be true that \begin{align*}\int g d\nu &= \int g_1 d\nu - \int g_2 d\nu  + i\brac{\int g_3 d\nu  - \int g_4 d\nu }\\ &= \int g_1 f d\mu - \int g_2 f d\mu  + i\brac{\int g_3 f d\mu  - \int g_4 f d\mu } \\&= \int g f d\mu\end{align*}

Therefore, indeed, for any $f\in L^1_\Cplx\brac{\Omega, \Fcal, \mu}$ the map $\nu\defn \int f d\mu$ is a complex measure on $\brac{\Omega, \Fcal}$ with total variation $\abs{\nu}=\int \abs{f}d\abs{\mu}$ and such that $g f\in L^1_\Cplx\brac{\Omega, \Fcal, \mu}$ for all $g\in L^1_\Cplx\brac{\Omega, \Fcal, \nu}$ and \[\int g d\nu = \int g f d\mu\]

\label{thm:cplx_prod_meas} \noindent \textbf{Theorem} 66.
Let $\brac{\Omega_k, \Fcal_k}_{k=1}^n$ be measurable spaces, where $n\geq2$, and $\mu_k \in M^1\brac{\Omega_k, \Fcal_k}$ for all $k=1\ldots n$. Let $\Omega\defn\prod_{k=1}^n \Omega_k$ and $\Fcal\defn \bigotimes_{k=1}^n \Fcal_k$. There exists a unique complex measure $\mu\defn\bigotimes_{k=1}^n \mu_k$ on $\brac{\Omega, \Fcal}$ such that for any measurable rectangle $\prod_{k=1}^n A_k \in \coprod_{k=1}^n \Fcal_k$ it is true that \[\mu\brac{\prod_{k=1}^n A_k} = \prod_{k=1}^n \mu_k\brac{A_k}\]

Let $h_k\in L^1_\Cplx\brac{\Omega_k,\Fcal_k,\abs{\mu_k}}$ be such that $\abs{h_k}=1$ and $\mu_k = \int h_k d\abs{\mu_k}$, which are guaranteed to exists by virtue of theorem 62. For all $E\in \bigotimes_{k=1}^n \Fcal_k$ define \[\mu\brac{E}\defn \int_E \prod_{k=1}^n h_k d\abs{\mu_1}\otimes\ldots\otimes\abs{\mu_n}\]

%%%%%%%%%% IMPORTANT!!!!!!!!
The map $h\defn \prod_{k=1}^n h_k$ is $\Fcal$-$\borel{\Cplx}$ measurable because $\dots$

Since each $\abs{\mu_k}$ is finite, $\abs{\mu_1}\otimes\ldots\otimes\abs{\mu_n}$ is a finite product measure on $\brac{\prod_{k=1}^n \Omega_k, \bigotimes_{k=1}^n \Fcal_k}$, whence by theorem 32 \[\int \abs{\prod_{k=1}^n h_k} d\abs{\mu_1}\otimes\ldots\otimes\abs{\mu_n} = \abs{\mu_1}\otimes\ldots\otimes\abs{\mu_n}\brac{\prod_{k=1}^n \Omega_k} = \prod_{k=1}^n \abs{\mu_k}\brac{\Omega_k}<+\infty\] Therefore \[\prod_{k=1}^n h_k\in L^1_\Cplx\brac{\prod_{k=1}^n \Omega_k, \bigotimes_{k=1}^n \Fcal_k, \bigotimes_{k=1}^n \abs{\mu_k} }\] By theorem 63 it is therefore true that $\mu$ is a complex measure on the measurable product space.

For any $E=\prod_{k=1}^n E_k\in \coprod_{k=1}^n \Fcal_k$ it is true that $1_E = \prod_{k=1}^n 1_{E_k}$, whence by definition 97 it must be true that \[\mu\brac{E}=\int 1_E \prod_{k=1}^n h_k d\bigotimes_{k=1}^n \abs{\mu_k} = \int \prod_{k=1}^n 1_{E_k} h_k d\bigotimes_{k=1}^n \abs{\mu_k}\] Hence by the extension of theorem 33 (Fubini) it is true $\mu\brac{E} = \prod_{k=1}^n \int_{E_k} h_k \abs{\mu_k} = \prod_{k=1}^n \mu_k\brac{E_k}$, since the integrated complex  function is multiplicatively separable.

Let $\Omega\defn\prod_{k=1}^n \Omega_k$ and $\Fcal\defn \bigotimes_{k=1}^n \Fcal_k$ and $\nu$ be another complex measure on $\brac{\Omega, \Fcal}$ such that $\nu\brac{E} = \prod_{k=1}^n \mu_k\brac{E_k}$ for every $E=\prod_{k=1}^n E_k \in \coprod_{k=1}^n \Fcal_k$. Consider the following auxiliary structure:\[\Dcal\defn \obj{ \induc{ E\in \bigotimes_{k=1}^n \Fcal_k }\, \nu\brac{E} = \mu\brac{E} }\]

Since $\Omega$ is a measure rectangle, $\nu\brac{\Omega}=\mu\brac{\Omega}$, whence $\Omega\in \Dcal$. Next if $A,B\in \Dcal$ with $A\subseteq B$ then $A,B\in \Fcal$ and $B\setminus A\in \Fcal$. Thus since $\nu$ and $\mu$ are complex measures \begin{align*}\nu\brac{B\setminus A} &= \nu\brac{B\setminus A} + \nu\brac{A} - \nu\brac{A}\\&= \nu\brac{B} - \nu\brac{A} = \mu\brac{B} - \mu\brac{A}\\& \mu\brac{B\setminus A} + \mu\brac{A} - \mu\brac{A}\\ &= \mu\brac{B\setminus A} + \mu\brac{A}\end{align*} whence $B\setminus A\in \Dcal$. Finally, if $\brac{A_k}_{k\geq1}\in\Dcal$ is such that $A_k\subseteq A_{k+1}$ for all $k\geq1$, then $A_k\uparrow A\defn \bigcup_{k\geq1}A_k$ and $1_{A_k}\to 1_A$. Thus by theorem 99-9 for $\nu$ and $\mu$ it is true that $\mu\brac{A_k}\overset{\Cplx}{\to}\mu\brac{A}$ and $\nu\brac{A_k}\overset{\Cplx}{\to}\nu\brac{A}$, whence $\nu\brac{A}=\mu\brac{A}$, implying that $A\in\Dcal$. Therefore $\Dcal$ is a Dynkin system on $\Omega$ with $\coprod_{k=1}^n \Fcal_k \subseteq \Dcal$. Since rectangle sets form a $\pi$-system, theorem 1 (Dynkin) implies that $\Fcal\subseteq \Dcal$. Thus $\nu = \mu$.\\

Since $\mu=\int \prod_{k=1}^n h_k d\otimes_{k=1}^n\abs{\mu_k}$ and $\abs{\prod_{k=1}^n h_k}=1$, theorem 63 and uniqueness of a product measure (theorem 7-1) imply $\abs{\mu} = \bigotimes_{k=1}^n\abs{\mu_k}$ since $\abs{h} = \prod_{k=1}^n \abs{h_k} = 1$. Therefore $\nrm{\mu} = \abs{\mu}\brac{\Omega} = \prod_{k=1}^n \abs{\mu_k}\brac{\Omega_k} = \prod_{k=1}^n \nrm{\mu_k}$. Finally, for any $E\in \Fcal$ it is true that \[\int_E h d\abs{\mu} = \int_E \prod_{k=1}^n h_k d\otimes_{k=1}^n \abs{\mu_k} = \mu\brac{E}\] Therefore by definition 97 for any $f\in L^1_\Cplx\brac{\Omega, \Fcal, \mu}$ \[\int f d\mu = \int f h d\abs{\mu} = \int f \prod_{k=1}^n h_k d\otimes_{k=1}^n \abs{\mu_k}\]

By theorem 33 \begin{align*}\int f d\mu_1\otimes\mu_2 &= \int f h_1 h_2 d\abs{\mu_1}\otimes\abs{\mu_2} = \int_{\Omega_1} \brac{\int_{\Omega_2} f h_1 h_2 d\abs{\mu_2}} d\abs{\mu_1}\\ &= \int_{\Omega_1} \brac{\int_{\Omega_2} f h_2 d\abs{\mu_2}} h_1 d\abs{\mu_1} = \int_{\Omega_1} \brac{\int_{\Omega_2} f d\mu_2} h_1 d\abs{\mu_1}\\ &= \int_{\Omega_1} \brac{\int_{\Omega_2} f d\mu_2} d\mu_1\end{align*} and $\brac{\int_{\Omega_2} f d\mu_2}$ is $\abs{\mu_1}$-almost sure equal to an element of $L^1_\Cplx\brac{\Omega_1,\Fcal_1,\mu_1}$.

Let $\sigma$ be any permutation of $\obj{1\ldots n}$. Then by extension to theorem 33, uniqueness of product measure, induction on linearity of the complex Lebesgue integral with respect to usual measures \begin{align*}\int f d\mu &= \int f h d\otimes_{k=1}^n \abs{\mu_k}\\&= \int_{\Omega_{\sigma\brac{n}}}\ldots \int_{\Omega_{\sigma\brac{1}}} f h d\abs{\mu_{\sigma\brac{1}}} \ldots d\abs{\mu_{\sigma\brac{n}}}\\&= \int_{\Omega_{\sigma\brac{n}}}\ldots \int_{\Omega_{\sigma\brac{1}}} f h_{\sigma\brac{n}} \cdot \ldots \cdot h_{\sigma\brac{1}} d\abs{\mu_{\sigma\brac{1}}} \ldots d\abs{\mu_{\sigma\brac{n}}}\\&=\int_{\Omega_{\sigma\brac{n}}}\ldots \int_{\Omega_{\sigma\brac{1}}} f d\mu_{\sigma\brac{1}} \ldots d\mu_{\sigma\brac{n}}\end{align*} 

% section tut_12 (end)

\section{Regular measure} % (fold)
\label{sec:tut_13}
\url{http://probability.net/PRTregularity.pdf}

In the following let $K$ be either $\Real$ or $\Cplx$.

\noindent \textbf{Definition} 99
Let $\brac{\Omega, \Fcal}$ be a measurable space. A map $s:\Omega\to \Cplx$ is a complex simple function on $\brac{\Omega,\Fcal}$ if and only if it is of the form $s = \sum_{i=1}^n \alpha_i 1_{A_i}$, where $n\geq1$, $\brac{\alpha_i}_{i=1}^n\in \Cplx$ and $\brac{A_i}_{i=1}^n\in \Fcal$. The set of all complex simple functions $\brac{\Omega, \Fcal}$ is denoted $S_\Cplx\brac{\Omega, \Fcal}$ and the set of all $\Real$-valued complex simple functions $\brac{\Omega, \Fcal}$ is denoted $S_\Real\brac{\Omega, \Fcal}$. The (usual) simple function of definition 40 is just an non-negative element of $S_\Real\brac{\Omega,\Fcal}$.

\label{thm:cplx_simple_func_partition} \noindent \textbf{Theorem} 13-1.
Let $\brac{\Omega, \Fcal}$ be a measurable space and $s$ be a complex simple function on $\brac{\Omega, \Fcal}$. Then there exists $n\geq 1$, non-zero $\brac{\alpha_k}_{k=1}^n\in\Cplx$ and pairwise disjoint $\brac{A_k}_{k=1}^n\in \Fcal$ such that $s=\sum_{k=1}^n \alpha_k 1_{A_k}$. Such representation of is called a partition.

Indeed, let $\phi:\Omega\to\obj{0,1}^n$ be defined as $\phi\brac{\omega}\defn \brac{1_{A_i}\brac{\omega}}_{i=1}^n$ and for any $z\in s\brac{\Omega}$ pick some fixed $\omega_z\in \Omega$ such that $s\brac{\omega_z}=z$. Put $\psi\brac{z}\defn \phi\brac{\omega_z}$.

Suppose $z_1, z_2\in s\brac{\Omega}$ are such that $\phi\brac{\omega_{z_1}}=\phi\brac{\omega_{z_2}}$. Thus $1_{A_i}\brac{\omega_{z_1}}=1_{A_i}\brac{\omega_{z_2}}$ for all $i=1\ldots n$, whence \[z_1 = s\brac{\omega_{z_1}} = \sum_{i=1}^n \alpha_i 1_{A_i}\brac{\omega_{z_1}} = \sum_{i=1}^n \alpha_i 1_{A_i}\brac{\omega_{z_2}} = s\brac{\omega_{z_2}} = z_2\] Thus $\psi:s\brac{\Omega}\to \obj{0,1}^n$ is an injective map, whence $s\brac{\Omega}$ is finite.

Now $\obj{z}\in \borel{\Cplx}$ for any $z\in \Cplx$, because $\Cplx$ is a metric space and thus Hausdorff and every finite set in a Hausdorff topological space is closed. Therefore $\obj{s=z}\in\Fcal$ for every $z\in s\brac{\Omega}$. Furthermore, $\obj{s=z_1}\cap \obj{s=z_2} \neq \emptyset$ implies that $z_1=z_2$ for any $z_1,z_2\in s\brac{\Omega}$.

Thus enumerating $S\defn s\brac{\Omega}\setminus\obj{0}$ as $\obj{\induc{z_i}\,i=1\ldots N}$, where $N\defn \abs{s\brac{\Omega}}$, and setting $F_i\defn \obj{s=z_i}$ yields $s=\sum_{i=1}^N z_i 1_{F_i}$. Indeed, pick any $\omega\in\Omega$. If $s\brac{\omega}=0$, then $\omega\notin \uplus_{i=1}^n F_i$, whence $z_i 1_{F_i}\brac{\omega}=0$ for all $i=1\ldots N$. If $s\brac{\omega}\neq 0$, then there is always only one $k=1\ldots N$ with $\omega\in F_k$ and $z_k = s\brac{\omega}$.

Therefore, for any $s\in S_\Cplx\brac{\Omega,\Fcal}$ there is $n\geq1$, $\brac{\alpha_i}_{i=1}^N\in \Cplx\setminus\obj{0}$ and pairwise disjoint $\brac{A_i}_{i=1}^N\in\Fcal$ such that $s=\sum_{i=1}^n \alpha_i 1_{A_i}$.\\

\label{thm:cplx_simple_lp_spaces} \noindent \textbf{Theorem} 13-2.
Let $\brac{\Omega, \Fcal, \mu}$ be a measure space and $p\in\clop{1, +\infty}$. Then $s\in L^p_\Cplx\brac{\Omega, \Fcal, \mu}\cap S_\Cplx\brac{\Omega, \Fcal}$ if and only if $s$ is a complex simple function of $\brac{\Omega, \Fcal}$ with $\mu\brac{A_i}<+\infty$, where $\brac{A_i}_{i=1}^n\in \Fcal$ are from definition 99. Furthermore \[L^\infty_\Cplx\brac{\Omega, \Fcal, \mu}\cap S_\Cplx\brac{\Omega, \Fcal} = S_\Cplx\brac{\Omega, \Fcal}\]

If $s:\Omega\to\Cplx$ is such that $s=\sum_{i=1}^n \alpha_i 1_{A_i}$, where $n\geq1$, $\brac{\alpha_i}_{i=1}^n\in \Cplx$, $\brac{A_i}_{i=1}^n\in \Fcal$ and $\mu\brac{A_i}<+\infty$ for all $i=1\ldots n$. Then by definition 99 $s\in S_\Cplx\brac{\Omega, \Fcal}$. Furthermore being a sum of $\Fcal$-$\borel{\Cplx}$ measurable maps, $s$ is itself $\Fcal$-$\borel{\Cplx}$ measurable.

Since \[\abs{s}=\abs{\sum_{i=1}^n \alpha_i 1_{A_i}}\leq \sum_{i=1}^n \abs{\alpha_i} 1_{A_i}\] and $\abs{s}$ is non-negative ans $\Fcal$-$\borel{\Rbar}$ measurable, theorem 5-5 implies that\[\int \abs{s}^p d\mu \leq \int \brac{\sum_{i=1}^n \abs{\alpha_i} 1_{A_i}}^p d\mu \] Since each $\abs{\alpha_i} 1_{A_i}$ is non-negative and measurable with respect to $\Fcal$ and $\borel{\Rbar}$, by a straightforward generalization of Minkowski inequality (theorem 43) and by linearity of the usual Lebesgue integral (theorem 5-7) it is true that \[\int \abs{s}d\mu \leq \sum_{i=1}^n \brac{\abs{\alpha_i}^p \int 1_{A_i} d\mu}^\frac{1}{p} = \sum_{i=1}^n \abs{\alpha_i} \mu\brac{A_i}^\frac{1}{p}<+\infty\] Therefore $s\in L^p_\Cplx\brac{\Omega, \Fcal, \mu}\cap S_\Cplx\brac{\Omega, \Fcal}$.

Conversely, suppose $s\in L^p_\Cplx\brac{\Omega, \Fcal, \mu}\cap S_\Cplx\brac{\Omega, \Fcal}$. Since $s\in S_\Cplx\brac{\Omega, \Fcal}$, theorem 13-1 imples that there exist $n\geq1$, $\brac{\alpha_i}_{i=1}^n\in \Cplx\setminus\obj{0}$ and pairwise disjoint $\brac{A_i}_{i=1}^n \in \Fcal$ such that $s=\sum_{i=1}^n \alpha_i 1_{A_i}$. Therefore $\abs{\alpha_k}^p 1_{A_k} \leq \sum_{i=1}^n \abs{\alpha_i}^p 1_{A_i} = \abs{s}^p$ for all $k=1\ldots n$ since $\brac{A_i}_{i=1}^n$ are pairwise disjoint. Therefore by theorem 5-5 and 5-7 and since $s\in L^p_\Cplx$ for each $k=1\ldots n$ \[\abs{\alpha_k}\mu\brac{A_k}^\frac{1}{p} = \brac{\int \abs{\alpha_k}^p 1_{A_k} d\mu}^\frac{1}{p} \leq \brac{\int \abs{s}^p d\mu}^\frac{1}{p} < +\infty\] whence $\mu\brac{A_k}<+\infty$ because $\abs{\alpha_k}\neq 0$. Thus $s$ is of the form in definition 99 with $\mu\brac{A_i}<+\infty$ for all $i=1\ldots n$.

Now if $s\in S_\Cplx\brac{\Omega, \Fcal}$, then by the above $\abs{s}\leq \sum_{i=1}^n \abs{\alpha_i} 1_{A_i}$, whence for $M\defn \sum_{i=1}^n \abs{\alpha_i}\in \Real^+$ it is true that $\abs{s}\leq M$ everywhere on $\Omega$. Thus $\nrm{s}_\infty\leq M<+\infty$, which implies that $s\in L^\infty_\Cplx$. Therefore $L^\infty_\Cplx \cap S_\Cplx = S_\Cplx$.\\

\label{thm:lp_DCT} \noindent \textbf{Theorem} 13-3.
Let $p\in \clop{1,+\infty}$, $\brac{\Omega, \Fcal, \mu}$ be a measure space and $\brac{f_n}_{n\geq 1}:\brac{\Omega,\Fcal}\to\brac{\Cplx,\borel{\Cplx}}$ be such that $f_n\overset{\Cplx}{\to}f$ everywhere on $\Omega$. If there exists $g\in L^p_\Real\brac{\Omega,\Fcal,\mu}$, such that $\abs{f_n}\leq g$, then $f, \brac{f_n}_{n\geq1}\in L^p_\Cplx\brac{\Omega,\Fcal,\mu}$ and \[\int \abs{f_n-f}^p d\mu \to 0\]

This follows from the DCT. Indeed, if $f\in L^p_\Cplx\brac{\Omega,\Fcal,\mu}$, then $f$ is $\Fcal$-$\borel{\Cplx}$ measurable, whence $g\defn \abs{f}^p$ is $\Fcal$-$\borel{\Rbar}$ measurable, since the map $x\to \abs{x}^p$ is $\Tcal_\Cplx$-$\Tcal_\Rbar$ measurable. Furthermore, $\int \abs{g} d\mu = \int \abs{f}^p d\mu <+\infty$, whence $g\in L^1_\Real$.

Now, the functional sequence $\brac{\abs{f_n-f}^p}_{n\geq1}$ consists of elements from $L^1_\Cplx$, $\abs{f_n-f}^p\overset{\Cplx}{\to}0$ everywhere on $\Omega$ and its absolute value is dominated by $2\abs{g}^p$ which is an element of $L^1_\Real$. Therefore by theorem 23 (DCT) it is true that $\int \abs{f_n-f}^p d\mu \to 0$.

The following replicates the proof of theorem 23. Indeed, continuity of $\abs{\cdot}$ implies $\abs{f_n}\to \abs{f}$, whence $\abs{f_n}^p,\abs{f}^p\leq \abs{g}^p$ for all$n\geq1$. By theorem 17 $f$ is $\Fcal$-$\borel{\Cplx}$ measurable and by theorem 5-5\[\int \abs{f_n}^p d\mu, \int \abs{f}^p d\mu \leq \int \abs{g}^p d\mu < +\infty\] Thus $f,\brac{f_n}_{n\geq1}\in L^p_\Cplx$.

Define $h_n\defn 2\abs{g}^p-\abs{f_n-f}$ for every $n\geq1$. Since $f_n\overset{\Cplx}{\to}f$ it must be true that $\abs{f_n-f}\to 0$, whence $\lim_{n\to\infty}\abs{f_n-f}^p=0$. Then, by basic properties of lower and upper limits it is true that \[\liminf_{n\to \infty} h_n = 2 \abs{g}^p - \limsup_{n\to \infty} \abs{f_n-f}^p = 2\abs{g}^p\] whence $\int \liminf h_n d\mu = 2\int\abs{g}^p d\mu < +\infty$. By linearity of the usual Lebesgue integral (theorem 5-7) it is true that $\int h_n d\mu = 2\int \abs{g}^p d\mu - \int \abs{f_n-f}^p d\mu$ whence \[\liminf_{n\to \infty} \int h_n d\mu = 2\int \abs{g}^p d\mu - \limsup_{n\to \infty} \int \abs{f_n-f}^p d\mu\] Since $g\in L^p_\Real$, $\int \abs{g}^p d\mu$ is finite, whence by theorem 20 (Fatou) \[\limsup_{n\to \infty} \int \abs{f_n-f}^p d\mu \leq 0\] Basic properties of the upper limit further imply that $\lim_{n\to\infty} \int \abs{f_n-f}^p d\mu = 0$ by observations in theorem 23. Therefore $\nrm{f_n-f}_p\to 0$, which also means that $f_n\lpto f$ by theorem 9-11.\\

\label{thm:cplx_simple_dense_lp} \noindent \textbf{Theorem} 67.
Let $\brac{\Omega,\Fcal,\mu}$ be a measure space and $p\in \clo{1, +\infty}$. Then the set $S_K\brac{\Omega, \Fcal}\cap L^p_K\brac{\Omega,\Fcal,\mu}$ is dense in $L^p_K\brac{\Omega,\Fcal,\mu}$.

Indeed, consider $p\in\clop{1,+\infty}$. Let $f$ be a non-negative element of $L^p_\Real\brac{\Omega,\Fcal,\mu}$. Since $f$ is $\Fcal$-$\borel{\Rbar}$ measurable, by theorem 18 there exists a sequence $\brac{s_n}_{n\geq1}$ of simple functions on $\brac{\Omega, \Fcal}$ such that $s_n \uparrow f$ everywhere on $\Omega$. By definition 99 the functions $\brac{s_n}_{n\geq1}\in S_\Real\brac{\Omega, \Fcal}$.

By theorem 13-3 $s_n\overset{\Real}{\to}f$ point-wise and $s_n\leq f\in L^p_\Real$ imply that $\brac{s_n}_{n\geq1}\in L^p_\Real$ and $\int \abs{s_n-f}^p d\mu \to 0$. Hence $\nrm{s_n-f}_p\to 0$. Furthermore $\brac{s_n}_{n\geq1}\in S_\Real\cap L^p_\Real$ by theorem 13-2. 

Since $\nrm{s_n-f}_p\to 0$, for every $\epsilon>0$ there is $N\geq1$ such that $\nrm{s_n-f}_p<\epsilon$ for all $n\geq N$. This means that for any $\epsilon>0$ there is $s\in L^p_\Real\cap S_\Real$ such that $\nrm{s-f}_p<\epsilon$.

Now, let $f$ be a non-negative element of $L^\infty_\Real\brac{\Omega,\Fcal,\mu}$. For all $n\geq1$ let $F^n_k\defn \obj{\frac{k}{2^n}\leq f < \frac{k+1}{2^n}}$ and define \[s_n\defn \sum_{k=0}^{n 2^n-1} \frac{k}{2^n} 1_{F^n_k} + n 1_{\obj{n\geq f}}\] The sequence $\brac{s_n}_{n\geq1}$ is a sequence of simple functions on $\brac{\Omega, \Fcal}$ since each element of it has the form of a simple function as in definition 40. Thus by definition 99 $\brac{s_n}_{n\geq1}\in S_\Real\brac{\Omega, \Fcal}$.

Since $\nrm{f}_\infty < +\infty$ there is $N_0\geq1$ such that $\abs{f} < N_0$ $\mu$-almost surely, whence there exists $N\in \Fcal$ with $\mu\brac{N}=0$ and such that $f\brac{\omega}\in\clop{0,N_0}$ for all $\omega\in N^c$.

Let $n\geq N_0$ and pick any $\omega\in N^c$. Then $f\brac{\omega} < n$ and for $k\defn \floor{ 2^n f\brac{\omega} }$ it is true that $0\leq k \leq {n2^n-1}$ and $\omega\in \obj{\frac{f}{2^n}\leq f < \frac{k+1}{2^n}}$, implying $0\leq f\brac{\omega} - s_n\brac{\omega}<\frac{1}{2^n}$. Since $\mu\brac{N}=0$, $\abs{f-s_n}<\frac{1}{2^n}$ $\mu$-a.s. and so $\nrm{f-s_n}_\infty \frac{1}{2^n}$ for all $n\geq N_0$. If $\epsilon>0$, then there is $N_1\geq 1$ such that $\frac{1}{2^n}\leq \epsilon$ for all $n\geq N_1$. Hence for all $n\geq N$ with $N\defn \max\obj{N_0,N_1}$ it is true that $\nrm{f-s_n}_\infty<\epsilon$, whence $\nrm{f-s_n}_\infty \to 0$.

Therefore if $f\in L^\infty_\Real$ is a non-negative element, for every $\epsilon>0$ there is a non-negative $s\in L^\infty_\Real\cap S_\Real$ such that $\nrm{f-s}_\infty < \epsilon$, since by theorem 13-2 $L^\infty_\Real\cap S_\Real=S_\Real$.

Now, let $p\in\clo{0,+\infty}$. If $f\in L^p_\Real\brac{\Omega,\Fcal,\mu}$, then there are non-negative $f^+, f^-\in L^p_\Real$ such that $f=f^+-f^-$. For every $\epsilon>0$ by the above there are non-negative $s^+, s^-\in L^p_\Real\cap S_\Real$ such that $\nrm{s^+ - f^+}_p, \nrm{s^- - f^-}_p < \frac{\epsilon}{2}$. Put $s\defn s^+-s^-$ and note that $s\in l^p_\Real$ by linearity of the space and that $s\in S_\Real$ by definition 99. Hence by theorems 9-2 and 9-3 \[\nrm{s-f}_p\leq \nrm{s^+-f^+}_p + \nrm{s^--f^-}_p < \epsilon\] Therefore for any $\epsilon>0$ there is $s\in L^p_\Real\cap S_\Real$ such that $\nrm{s-f}_p<\epsilon$.

If $f\in L^p_\Cplx\brac{\Omega,\Fcal,\mu}$, then there are $f_1,f_2\in L^p_\Real$ such that $f = u + i v$. For every $\epsilon>0$ by the above there are $s_1, s_2\in L^p_\Real\cap S_\Real$ such that $\nrm{s_i-f_i}_p < \frac{\epsilon}{2}$ for all $i=1,2$. Put $s\defn s_1 + i s_2$ and note that $s\in l^p_\Cplx$ by linearity of the space and that $s\in S_\Cplx$ by definition 99. Hence by theorems 9-2 and 9-3 \[\nrm{s-f}_p\leq \nrm{s_1-f_1}_p + \nrm{s_2-f_2}_p < \epsilon\] Therefore for any $\epsilon>0$ there is $s\in L^p_\Cplx\cap S_\Cplx$ such that $\nrm{s-f}_p<\epsilon$.

Let $f\in L^p_K\brac{\Omega,\Fcal,\mu}$ and $U\in \Tcal_{L^p}$ with $f\in U$. By definition 75 of the usual topology on $L^p_K$ there is $\epsilon>0$ such that $B^p_K\brac{f,\epsilon}\subseteq U$. Since by the above there is $s\in L^p_K\cap S_K$ such that $s\in B^p_K\brac{f,\epsilon}$, which means that $U\cap \brac{L^p_K\cap S_K}\neq \emptyset$. Therefore the closure of the set $L^p_K\cap S_K$ in $\brac{L^p_K, \Tcal_{L^p}}$ is the entire space $L^p_K$, which implies that $L^p_K\cap S_K$ is dense in this space for all $p\in\clo{1, +\infty}$.\\

\label{thm:topo_meas_space} \noindent \textbf{Theorem} 68.
Let $\brac{\Omega, \Tcal}$ be a metrizable topological space and $\mu$ a finite measure on $\brac{\Omega, \borel{\Omega}}$. Then for all $B\in \borel{\Omega}$ and every $\epsilon>0$ there exist $F$ closed and $G$ open in $\Omega$ with $F\subseteq B \subseteq G$ such that $\mu\brac{G\setminus F}< \epsilon$.

Define $\Sigma$ to be the collection of all $B\in \borel{\Omega}$ such that for all $\epsilon>0$ there exist $F$ closed and $G$ open in $\Omega$ with $F\subseteq B \subseteq G$ such that $\mu\brac{G\setminus F}< \epsilon$. Let $d$ be the metric which induces the topological space $\brac{\Omega, \Tcal}$.

Let $F$ be a closed subset of $\Omega$. First $F^c\in \Tcal$, whence $F\in \borel{\Omega}$. Now, for all $n\geq1$ define \[G_n\defn \obj{ \induc{x\in \Omega}\,\Phi_F\brac{x}<\frac{1}{n} }\] where $\Phi_F\defn d\brac{\cdot,F}$ is the distance from $x$ to $F$ as defined in theorem 4-8. That theorem also shows that the map $x\to d\brac{x,A}$ is $\Tcal$-$\Tcal_\Rbar$ continuous for all $A\subseteq \Omega$ and that if $A$ is closed in $\Omega$, then $d\brac{x,A}=0$ is equivalent to $x\in A$.

Now, since $G_n \defn \Phi_F^{-1}\brac{B_n}$ with $B_n\defn B_\Rbar\brac{0,\frac{1}{n}}$ and $\brac{B_n}_{n\geq1}\in \Tcal_\Rbar$, continuity of $\Phi_F\brac{\cdot}$ implies that $\brac{G_n}_{n\geq1}$ is open in $\brac{\Omega, \Tcal}$. Furthermore, $G_{n+1}\subseteq G_n$, since $B_{n+1}\subseteq B_n$ for all $n\geq1$, and $\Phi_F\brac{x}=0$ whenever $x\in F$, which implies that $F\subseteq G_n$ for all $n\geq1$. Therefore $G_n\downarrow F$.

Since $\mu$ is finite on $\brac{\Omega, \borel{\Omega}}$, for all $n\geq1$ it is true that $\mu\brac{G_n\setminus F} = \mu\brac{G_n}-\mu\brac{F}$ while theorem 8 implies that $\mu\brac{G_n}\downarrow \mu\brac{F}$. Thus $\mu\brac{G_n\setminus F}\to 0$, which in turn implies that $F\in\borel{\Omega}$ is such that for all $\epsilon>0$ there is $N\geq1$ such that $\mu\brac{G_n\setminus F}<\epsilon$ for all $n\geq N$. Therefore $F\in \Sigma$.

Since, $\Omega$ is both closed and open in $\brac{\Omega, \Tcal}$ and $\mu\brac{\Omega\setminus\Omega}=\mu\brac{\emptyset}=0$, it must be true that $\Omega\in \Sigma$. Now if $B\in \Sigma$, then for every $\epsilon>0$ there exist $F$ closed and $G$ open in $\Omega$ with $F\subseteq B\subseteq G$ such that $\mu\brac{G\setminus F}<\epsilon$. Now, $G^c$ is closed and $F^c$ is open in $\Omega$ and are such that $G^c \subseteq B^c \subseteq F^c$ with $F^c\setminus G^c = F^c\cap G = G\setminus F$. Thus $\mu\brac{F^c\setminus G^c}<\epsilon$, whence $B^c\in \Sigma$.

Let $\brac{B_n}_{n\geq1}\in \Sigma$, $B\defn \bigcup_{n\geq1} B_n$ and pick any $\epsilon>0$. Then for all $n\geq1$ there exist $F_n$ closed and $G_n$ open in $\Omega$ with $F_n\subseteq B_n\subseteq G_n$ and $\mu\brac{G_n\setminus F_n}<\frac{\epsilon}{2^n}$. Since $\mu$ is finite and $\brac{\cup_{k\geq1} F_k}\setminus\brac{\cup_{k=1}^n F_k} \downarrow \emptyset$, theorem 8 implies that \[\mu\brac{\brac{\cup_{k\geq1} F_k}\setminus\brac{\cup_{k=1}^n F_k}}\downarrow 0\] whence for $\epsilon>0$ there is $N\geq1$ with $\mu\brac{\brac{\cup_{k\geq1} F_k}\setminus\brac{\cup_{k=1}^N F_k}}<\epsilon$.

Let $G\defn \bigcup_{n\geq1} G_n$ and $F\defn \bigcup_{k=1}^N F_k$. By basic properties of open and closed sets, $F$ is closed and $G$ is open in $\Omega$. Additionally, $F_n\subseteq B_n$ implies that $F\subseteq B$ and $B_n\subseteq G_n$ yields $B\subseteq G$. Now by basic properties of set operations $F\subseteq \cup_{k\geq1} F_k$ and \begin{align*}G\setminus F &= G\cap F^c \cap \brac{\cup_{k\geq1} F_k}^c \uplus \brac{\cup_{k\geq1} F_k} \cap G\cap F^c\\&= G\setminus\brac{\cup_{k\geq1} F_k} \uplus G\cap \brac{\cup_{k\geq1} F_k}\setminus F\\&\subseteq G\setminus\brac{\cup_{k\geq1} F_k} \uplus \brac{\cup_{k\geq1} F_k}\setminus F\end{align*} Furthermore, if $\omega\in G\setminus\brac{\cup_{k\geq1} F_k}$, by definition of $G$ there is $m\geq1$ such that $\omega\in G_m$ while $\omega\notin F_k$ for all $k\geq1$ and, in particular, for $k=m$. Therefore $\omega\subseteq \bigcup_{k\geq1} G_k\setminus F_k$.

Theorems 2-3 and 2-4 imply that \[\mu\brac{G\setminus F}\leq \mu\brac{\bigcup_{k\geq1} G_k\setminus F_k}+\mu\brac{\brac{\cup_{k\geq1} F_k}\setminus F}< \sum_{n\geq1} \frac{\epsilon}{2^n} + \epsilon\] Therefore for any $\epsilon>0$ there are $F$ closed and $G$ open in $\Omega$ with $F\subseteq B\subseteq G$ and such that $\mu\brac{G\setminus F}<\epsilon$. Hence $\bigcup_{n\geq1} B_n\in \Sigma$.

Above reasoning shows that $\Sigma$ is a actually a $\sigma$-algebra on $\Omega$ with $\Sigma\subseteq \borel{\Omega}$. Furthermore, $\Tcal\subseteq \Sigma$ since every open set in $\Omega$ is a complement of a closed set in $\Omega$, whence $\borel{\Omega}\subseteq \Sigma$.\\

\noindent \textbf{Definition} 100.
Let $\brac{\Omega, \Tcal}$ be a topological space. Denote $C^b_K\brac{\Omega}$ the $K$-vector space of all continuous, bounded maps $\phi:\Omega\to K$, where $K=\Real$ or $\Cplx$.

\label{thm:bound_cont_sub_lp} \noindent \textbf{Theorem} 13-4.
Let $\brac{\Omega, \Tcal}$ be a topological space and $\mu$ a finite measure on $\brac{\Omega, \borel{\Omega}}$. Then for all $p\in \clo{1,+\infty}$ \[C^b_K\brac{\Omega,\Tcal}\subseteq L^p_K\brac{\Omega,\borel{\Omega},\mu}\]

Indeed, every $\Tcal$-$\Tcal_K$ continuous map is $\borel{\Omega}$-$\borel{K}$ measurable. Furthermore if $\phi\in C^b_k$, then there is $M\in \Real^+$ such that $\abs{\phi}\leq M$ on $\Omega$. Therefore $\int \abs{\phi}^p d\mu\leq M^p \mu\brac{\Omega}<+\infty$ for any $p\in\clop{1,+\infty}$ and $\abs{\phi}\leq M$ $\mu$ almost surely imply that $\nrm{\phi}_p<+\infty$ for all $p\in \clo{1,\infty}$. Therefore $\phi\in L^p_K$ the conclusion follows.\\

\label{thm:bound_cont_indic_seq} \noindent \textbf{Theorem} 13-?-4.
Let $\brac{\Omega,\Tcal}$ be a metrizable topological space. For any $F$ closed subset of $\brac{\Omega, \Tcal}$ there exists $\brac{\phi_n}_{n\geq1}\in C^b_\Real$ with $\abs{\phi}\leq 1$ such that $\phi_n\to 1_F$ point-wise on $\Omega$.

Indeed, let $d$ be the metric, which induces $\Tcal$. For all $n\geq1$ define $\phi_n:\Omega\to\Real$ by \[\phi_n\brac{x}\defn 1 - 1\wedge\brac{n\cdot d\brac{x,F}}\] Being a composition of continuous maps, by theorem Sup-A-3 maps $\phi_n$ are continuous, since by theorem 4-8 $x\to d\brac{x,F}$ is continuous for every $F\subseteq \Omega$. Next, $\phi_n\leq1$ for all $n\geq1$, which implies that $\brac{\phi_n}_{n\geq1}\in C^b_\Real$.

Because $F$ is closed, it it true that if $x\notin F$, then $d\brac{x,F}\neq 0$, which implies that there is $N\geq1$ with $n\cdot d\brac{x,F}>1$ for all $n\geq N$. Thus $\phi_n\brac{x}=0=1_F\brac{x}$ for all $n\geq N$. If $x\in F$ then $d\brac{x,F}=0$, which implies that $\phi_n\brac{x}=1=1_F\brac{x}$ for all $n\geq1$. Therefore for all $x\in \Omega$ and any $\epsilon>0$ there is $N\geq1$ such that $\abs{\phi_n\brac{x}-1_F\brac{x}}<\epsilon$ for all $n\geq N$. Hence $\phi_n\to 1_F$ everywhere on $\Omega$.\\

\label{thm:bound_cont_indic_lp} \noindent \textbf{Theorem} 13-?-5.
Let $\brac{\Omega, \Tcal}$ be a metrizable topological space, $\mu$ a finite measure on $\brac{\Omega, \borel{\Omega}}$ and $p\in \clop{1,+\infty}$. Then for any closed subset $F$ of $\Omega$ for all $\epsilon>0$ there exists $\phi\in C^b_K\subseteq L^p_K$ with $\abs{\phi}\leq 1$ such that $\nrm{\phi-1_F}_p<\epsilon$.

By theorem 13-?-4 there exists $\brac{\phi_n}_{n\geq1}\in C^b_\Real\subseteq C^b_K$ with $\phi_n\to 1_F$ and by theorem 13-4 the finiteness of $\mu$ implies $C^b_K\subseteq L^p_K$, whence $\brac{\phi_n}_{n\geq1}\in L^p_K$.

Since $\mu$ is finite it is true that $\mu\brac{\Omega}<+\infty$, whence for $g\defn 1$ it is true that $\int \abs{g}^p d\mu <+\infty$. Since $\phi_n,1_F\leq 1$ everywhere on $\Omega$ and $\phi_n\to 1_F$, theorem 13-3 implies that $\lim_{n\to\infty}\int \abs{\phi_n-1_F}^p d\mu = 0$, whence $\nrm{\phi_n-1_F}_p\to 0$. Thus for every $\epsilon>0$ there is $N\geq1$ with $\nrm{\phi_n-1_F}<\epsilon$ for all $n\geq N$. Therefore, in particular, for all $\epsilon>0$ there is $\phi\in C^b_\Real$ with $\abs{\phi}\leq 1$ such that $\nrm{\phi-1_F}_p<\epsilon$.\\

\label{thm:bound_cont_dense_lp} \noindent \textbf{Theorem} 70.
Let $\brac{\Omega,\Tcal}$ be a metrizable topological space and $\mu$ be a finite measure on $\brac{\Omega,\borel{\Omega}}$. Then $C^b_K\brac{\Omega}$ is dense in $L^p_K\brac{\Omega,\borel{\Omega},\mu}$ for every $p\in \clop{1, +\infty}$.

Indeed, let $s\in S_\Cplx\brac{\Omega,\borel{\Omega}}$ be a complex simple function with representation $s=\sum_{i=1}^n \alpha_i 1_{A_i}$, where $n\geq1$, $\brac{\alpha_i}_{i=1}^n\in \Cplx$ and $\brac{A_i}_{i=1}^n\in\borel{\Omega}$. Let $p\in \clop{1,+\infty}$.

By theorem 68 for every $\delta>0$ there exists $\brac{F_i}_{i=1}^n$ closed and $\brac{G_n}_{i=1}^n$ open in $\Omega$ with $F_i\subseteq A_i \subseteq G_i$ such that $\mu\brac{A_i\setminus F_i}\leq \mu\brac{G_i\setminus F_i}<\delta$. Note that \[\nrm{1_{A_i}-1_{F_i}}_p = \brac{\int \abs{1_{A_i\setminus F_i}}^p d\mu}^\frac{1}{p} = \brac{\mu\brac{A_i\setminus F_i}}^\frac{1}{p} < \delta^\frac{1}{p}\] Let $s'\defn \sum_{i=1}^n \alpha_i 1_{F_i}$. By theorems 9-3 and 9-2 \[\nrm{s-s'}_p\leq \sum_{i=1}^n \abs{\alpha_i}\nrm{1_{A_i}-1_{F_i}}_p < \delta^\frac{1}{p}\] where $M\defn \sum_{i=1}^n \abs{\alpha_i}$ with $M\in \Real^+$.

Now in theorem 13-?-5 it has been shown that for all $i=1\ldots n$ the fact that $F_i$ is closed in a metrizable topological space $\brac{\Omega,\Tcal}$ implies existence of $\brac{\phi^i_n}_{n\geq1}\in C^b_\Real$ with $\phi^i_n\overset{\Cplx}{\to}1_{F_i}$ point-wise and $\abs{\phi^i_n}\leq 1$. Since finite sums of $\Tcal$-$\Tcal_\Cplx$ continuous maps are continuous, the map $\phi_n\defn \sum_{i=1}^n \alpha_i \phi^i_n$ is itself $\Tcal$-$\Tcal_\Cplx$ continuous. Furthermore, $\phi_n\overset{\Cplx}{\to}s'$ everywhere on $\Omega$ and $\abs{\phi_n}\leq M$, which implies that $\brac{\phi_n}_{n\geq1}\in C^b_\Cplx$.

Since $\mu$ is finite, $\int M^p d\mu = M^p \mu\brac{\Omega}<+\infty$ and by theorem 13-4 $\brac{\phi_n}_{n\geq1}\in L^p_\Cplx$. Therefore theorem 13-3 implies that $\int \abs{\phi_n-s'}^p d\mu \to 0$, whence $\nrm{\phi_n-s'}_p\to 0$. Hence, for every $\delta>0$ there is $\phi\in C^b_\Cplx$ with $\nrm{\phi-s'}_p<\delta$.

Consider $\nrm{\phi-s}_p\leq \nrm{\phi-s'}_p+\nrm{s'-s}_p$ and let $\epsilon>0$. If $M=0$, then $\alpha_i=0$ for all $i=1\ldots n$, whence $s=0$ and $\phi\defn 0$ is such continuous and bounded map that $\nrm{\phi-s}_p<\epsilon$. If $M>0$ then for $\delta\defn \brac{\frac{\epsilon}{2 M}}^p$ there is $s'\in S_\Cplx$ with $\nrm{s'-s}_p< M \delta^p$ and further there is $\phi\in C^b_\Cplx$ such that $\nrm{\phi-s'}_p<\frac{\epsilon}{2}$. To summarize, there is $\phi\in C^b_\Cplx$ such that $\nrm{\phi-s}_p<\epsilon$, and if $s\in S_\Real$, then $\phi\in C^b_\Real$ as well.

Now if $f\in L^p_K\brac{\Omega,\borel{\Omega},\mu}$ and $U\in \Tcal_{L^p}$ with $f\in U$, then there is $\delta>0$ with $B_{L^p}\brac{f,\delta}\subseteq U$ and $s\in S_K\cap L^p_K$ with $s\in B_{L^p}\brac{f,\frac{\delta}{2}}$. By the above there is $\phi\in C^b_K$ with $\phi\in B_{L^p}\brac{s,\frac{\delta}{2}}$, whence $\phi\in B_{L^p}\brac{f,\delta}$. Thus $U\cap C^b_K \neq \emptyset$ for any $U\in \Tcal_{L^p}$ with $f\in U$. Therefore $C^b_K\brac{\Omega}$ is dense in $L^p_K\brac{\Omega,\borel{\Omega},\mu}$.\\

\label{thm:cplx_meas_cb_prrox}\noindent\textbf{Theorem} 13-?-6.
Let $\brac{\Omega,\Tcal}$ be a metrizable topological space and $\nu$ be a complex measure on $\brac{\Omega, \borel{\Omega}}$. Then $C^b_\Cplx\brac{\Omega}\subseteq L^1_\Cplx\brac{\Omega, \borel{\Omega}, \nu}$ and for any $F$ closed in $\Omega$ there exists $\brac{\phi_n}_{n\geq1}\in C^b_\Cplx\brac{\Omega}$ with $\phi_n\overset{\Cplx}{\to} 1_F$ and such that \[\nu\brac{F} = \lim_{n\to \infty}\int \phi_n d\nu\]

Since the total variation $\abs{\nu}$ of the complex measure $\nu$ is a finite measure on $\brac{\Omega, \borel{\Omega}}$, by theorem 13-4 it is true that $C^b_\Cplx\brac{\Omega}\subseteq L^1_\Cplx\brac{\Omega, \borel{\Omega}, \abs{\nu}}$. However the space $L^1_\Cplx\brac{\Omega, \borel{\Omega}, \nu}$ is defined in 97 as $L^1_\Cplx\brac{\Omega, \borel{\Omega}, \abs{\nu}}$, which means that $C^b_\Cplx\subseteq L^1_\Cplx\brac{\Omega,\borel{\Omega},\nu}$.

Let $F$ be a closed subset of $\Omega$. By theorem 13-?-4 there exists a sequence $\brac{\phi_n}_{n\geq1}\in C^b_\Real\subseteq L^1_\Cplx\brac{\Omega,\Fcal,\nu}$ with $\phi_n\to 1_F$ and $\abs{\phi_n}\leq1$. By theorem 63 there exits $h\in L^1_\Cplx\brac{\Omega,\Fcal,\abs{\nu}}$ with $\abs{h}=1$ and such that $\nu\brac{E}=\int_E h d\abs{\nu}$ and by definition 97 $\int \phi_n d\nu = \int \phi_n h d\abs{\nu}$. Since $\phi_n \to 1_F$ and $\abs{\phi_n}\leq 1$, $\phi_n h \overset{\Cplx}{\to} 1_F h$ and $\abs{\phi_n h}\leq \abs{h}\in L^1_\Real$, whence by theorem 23 (DCT) on $\brac{\Omega,\borel{\Omega},\abs{\nu}}$ it is true that $\int \abs{\phi_n h - 1_F h} d\abs{\nu} \to 0$. Therefore by theorem 24 \[\abs{\int \phi_n d\nu - \nu\brac{F} } = \abs{\int \phi_n h d\abs{\nu} - \int 1_F h d\abs{\nu} } \leq \int \abs{\phi_n - 1_F} \abs{h} d\abs{\nu} \to 0\] whence $\nu\brac{F} = \lim_{n\to\infty}\int\phi_n d\nu$.\\

\label{thm:meas_equality} \noindent \textbf{Theorem} 69.
Let $\brac{\Omega,\Tcal}$ be a metrizable topological space. If $\mu,\nu$ are two complex measures on $\brac{\Omega, \borel{\Omega}}$ such that for all $\phi\in C^b_\Cplx$ \[\int \phi d\mu = \int \phi d\nu\] then $\nu = \mu$.

Now, if $\nu,\mu$ are two complex measures on $\brac{\Omega,\borel{\Omega}}$ with $\int \phi d\nu = \int \phi d\mu$ for all $\phi \in C^b_\Real$, then in particular by theorem 13-?-6 for every $F$ closed in $\Omega$ there exists $\brac{\phi_n}_{n\geq1}\in C^b_\Real$ such that for any $\lambda\in M^1\brac{\Omega,\borel{\Omega}}$ it is true that $\lambda\brac{F} = \lim_{n\to\infty}\int \phi_n d\lambda$. Since $\Cplx$ is a Hausdorff space and $\int \phi_n d\nu = \int \phi_n d\mu$ for all $n\geq1$, it must be true that $\nu\brac{F}=\mu\brac{F}$. Therefore $\nu\brac{F}=\mu\brac{F}$ for all $F$ closed in $\Omega$.

Let $\Dcal\defn \obj{\induc{B\in\borel{\Omega}} \nu\brac{B}=\mu\brac{B}}$. First of all, $F\in \Dcal$ for every $F$ closed in $\Omega$. If $A,B\in \Dcal$ with $A\subseteq B$ then, since $\nu,\mu:\borel{\Omega}\to\Cplx$, \begin{align*}\nu\brac{B\setminus A}&=\nu\brac{B\setminus A} + \nu\brac{A} - \nu\brac{A}\\ &= \nu\brac{B}-\nu\brac{A} = \mu\brac{B}-\mu\brac{A}\\&= \mu\brac{B\setminus A}+\mu\brac{A}-\mu\brac{A}\\&= \mu\brac{B\setminus A}\end{align*} Thus $B\setminus A\in \Dcal$. And finally if $\brac{A_n}_{n\geq1}\in \Dcal$ and $A\defn \bigcup_{n\geq1} A_n$, then $A_n\uparrow A$ and $1_{A_n}\to 1_A$, whence by theorem 99-9 $\lambda\brac{A} = \lim_{n\to\infty} \lambda\brac{A_n}$ for bot $\lambda=\mu$ and $\lambda=\nu$. Therefore $\nu\brac{A}=\mu\brac{A}$ and $A\in \Dcal$. Since closed sets in $\Omega$ are closed under finite intersection, theorem 1 (Dynkin) implies that $\borel{\Omega}\subseteq \Dcal$, since the $\sigma$-algebra of open sets in $\brac{\Omega, \Tcal}$ is also generated by sets closed in $\Omega$. Hence $\mu=\nu$.\\

\noindent \textbf{Definition} 101.
A topological space $\brac{\Omega,\Tcal}$ is said to be $\sigma$-compact if and only if there are $\brac{K_n}_{n\geq1}$ compact subsets of $\Omega$ such that $K_n\uparrow \Omega$.

\label{thm:sigma_compact_subspace} \noindent \textbf{Theorem} 71.
Let $\brac{\Omega, \Tcal}$ be a metrizable $\sigma$-compact space. Then for every $\Omega'$ open subset of $\Omega$, the induced topological space $\brac{\Omega', \induc{\Tcal}_E}$ is itself metrizable and $\sigma$-compact.

Let $\brac{\Omega, \Tcal}$ be a metrizable $\sigma$-compact space with metric $d$. Let $E$ be open in $\Omega$. For all $n\geq1$ define \[F_n\defn \obj{ \induc{x\in \Omega}\, d\brac{x,E^c}\geq \frac{1}{n} }\] Let $\brac{K_n}_{n\geq1}$ be a sequence of compact subsets of $\Omega$ with $K_n\uparrow \Omega$.

Since $B_\Rbar^c\brac{x,\frac{1}{n}}$ is closed in $\Rbar$ for all $n\geq1$ and $d\brac{\cdot,E^c}$ is $\Tcal_E^d$-$\Tcal_\Rbar$ continuous by theorem 4-8, theorem Sup-A-1 implies that each $F_n$ is closed in $\Omega$. Furthermore, if $x\in F_n$ then $d\brac{x,E^c}\geq \frac{1}{n}\geq\frac{1}{n+1}$, whence $x\in F_{n+1}$.

Now if $x\in F_n$ for some $n\geq1$, then $d\brac{x,E^c}>0$ which implies that $x\notin \clo{E^c}$. But $E$ is open in $\Omega$, and thus $E^c$ is closed in $\Omega$, whence $x\notin E^c$ and thus $x\in E$. Conversely, if $x\in E$, then $x\notin E^c$ and $x\notin \clo{E^c}$, which implies that $d\brac{x,E^c}>0$ by theorem 4-8. Hence there is $N\geq1$ with $N\cdot d\brac{x,E^c}\geq 1$, whence $x\in F_N$. Therefore $E = \bigcup_{n\geq1} F_n$ and $F_n\uparrow E$.

By theorem 35 the set $K_n$ is closed in $\Omega$ for every $n\geq1$, since it is compact and any metrizable topological space is Hausdorff by theorem 8-6. Therefore $H_n\defn F_n\cap K_n$ is closed in $\Omega$ for all $n\geq1$, whence $H_n^c$ is open in $\Omega$ and thus $H_n^c\cap K_n$ is open in $K_n$. But \[K_n\cap \brac{H_n^c\cap K_n}^c = K_n\cap \brac{H_n\cup K_n^c} = F_n\cap K_n\] whence $H_n$ is closed in $K_n$. Since $K_n\uparrow \Omega$ and $F_n\uparrow E$, $H_n\uparrow E$. By theorem 8-5 $H_n$ is a compact subset of $K_n$, whence by basic properties of compactness it also is a compact subset of $\Omega$. However, $H_n\subseteq E$ for all $n\geq1$, which implies that $H_n$ is compact in $\brac{E,\induc{\Tcal}_E}$ for all $n\geq1$.

Thus there is a sequence $\brac{H_n}_{n\geq1}$ of compact subsets of $E$ such that $H_n\uparrow E$. By theorem 12 the topological subspace $\brac{E,\induc{\Tcal}_E}$ is metrizable by the induced metric $d_E\defn \induc{d}_{E\times E}$.\\

\noindent \textbf{Definition} 102.
Let $\brac{\Omega,\Tcal}$ be a topological space. A measure $\mu$ on $\brac{\Omega,\borel{\Omega}}$ is locally finite if and only if for every $x\in \Omega$ there is $U\in \Tcal$ with $x\in U$ such that $\mu\brac{U}<+\infty$.

\noindent \textbf{Definition} 103.
If $\mu$ is a measure on a Hausdorff topological space $\brac{\Omega,\Tcal}$. The measure $\mu$ is inner-regular if for all $B\in\borel{\Omega}$ it is true that \[\mu\brac{B}=\sup\obj{\induc{\mu\brac{K}}\,K\subseteq B,\,K\text{-- compact}}\] The measure $\mu$ is outer-regular if for all $B\in\borel{\Omega}$ it is true that \[\mu\brac{B}=\inf\obj{\induc{\mu\brac{K}}\,B\subseteq G,\,G\text{-- open}}\] The measure $\mu$ is regular if it is both inner- and outer- regular.


\label{thm:compact_meas_finite} \noindent \textbf{Theorem} 13-5.
Let $\brac{\Omega, \Tcal}$ be a Hausdorff topological space, $\mu$ a locally finite measure on $\brac{\Omega,\borel{\Omega}}$. Then $\nu\brac{K}<+\infty$ for every $K$ compact subset of $\Omega$. 

The Hausdorff property is very important, since by theorem 35 every compact subset of a Hausdorff space is closed, which implies that it is $\borel{\Omega}$ measurable.

Local finiteness of $\mu$ by its definition implies that for every $x\in K$ there is $U_x\in\Tcal$ with $x\in U_x$ such that $\mu\brac{U_x}<+\infty$. Now, $\brac{U_x}_{x\in K}$ constitutes an open cover of $K$ for which compactness implies that there is $F\subseteq K$ finite such that $K\subseteq \bigcup_{x\in K} U_x$. By theorem 3-6 since $F$ is finite \[\mu\brac{K}\leq \sum_{x\in F} \mu\brac{U_x}<+\infty\]

\label{thm:metr_sig_comp_meas_finite} \noindent \textbf{Theorem} 13-6.
In a metrizable and $\sigma$-compact topological space $\brac{\Omega,\Tcal}$ every finite measure $\mu$ on $\brac{\Omega,\borel{\Omega}}$ is inner-regular.

Indeed, pick any $B\in \borel{\Omega}$ and define \[\alpha\defn \sup\obj{\induc{\mu\brac{K}}\,K\subseteq B,\,K\text{-- compact}}\] Now, $\sigma$-compactness implies that there exist $\brac{K_n}_{n\geq1}$ compact subsets of $\Omega$ such that $K_n\uparrow \Omega$.

Pick any $\epsilon>0$. Since$\mu$ is finite, by theorem 68 there exists $F$ closed in $\Omega$ with $F\subseteq B$ such that $\mu\brac{B\setminus F}<\frac{\epsilon}{2}$.

By theorem 8-6 $\brac{\Omega, \Tcal}$ is Hausdorff, whence by theorem 35 each compact subset $K_n$ of $\Omega$ is closed in $\Omega$. Therefore $H_n\defn F\cap K_n$ is closed in $\Omega$ and thus in $K_n$, since $H_n\subseteq K_n$. Hence theorem 8-5 implies that $H_n$ is a compact subset of $\Omega$. Also $H_n\subseteq F$ for all $n\geq1$.

Now $K_n\uparrow \Omega$ implies $H_n\uparrow F$ and $F\setminus H_n\downarrow \emptyset$. Since $\mu$ is finite, $\mu\brac{F\setminus H_n}\downarrow 0$ by theorem 8, whence for $\epsilon>0$ there is $N\geq1$ such that $\mu\brac{F\setminus H_n}<\frac{\epsilon}{2}$ for all $n\geq N$. Thus for a given $\epsilon>0$ there is a compact subset $K\defn H_N$ of $\Omega$ such that $K\subseteq F$ and $\mu\brac{F\setminus K}<\frac{\epsilon}{2}$. Hence \begin{align*}\mu\brac{B} &= \mu\brac{B\setminus F} + \mu\brac{F} \\&= \mu\brac{B\setminus F} + \mu\brac{F\setminus K} + \mu\brac{K}\\ &<\mu\brac{K} + \epsilon\end{align*}

Therefore for any $\epsilon>0$ there is a compact subset $K$ of $\Omega$ such that $\mu\brac{B} < \mu\brac{K} + \epsilon \leq \alpha + \epsilon$, whence $\mu\brac{B}\leq \alpha$. If $K$ is any compact subset of $\Omega$ with $K\subseteq B$, then by theorem 2-4 $\mu\brac{K}\leq \mu\brac{B}$, whence $\alpha \leq \mu\brac{B}$. Therefore, by definition 103 $\mu$ is an inner-regular measure.\\

\label{thm:metr_sig_comp_local_finite} \noindent \textbf{Theorem} 13-7.
In a metrizable and $\sigma$-compact topological space $\brac{\Omega,\Tcal}$ every locally finite measure $\mu$ on $\brac{\Omega,\borel{\Omega}}$ is inner-regular.

Indeed, since $\brac{\Omega,\Tcal}$ is $\sigma$-compact, there exist $\brac{K_n}_{n\geq1}$ compact subsets of $\Omega$ such that $K_n\uparrow \Omega$. Let $B\in \borel{\Omega}$ and consider any $\alpha\in\Real$ such that $\alpha < \mu\brac{B}$.

Since $K_n\uparrow \Omega$, $K_n\cap B\uparrow B$ implies $\mu\brac{K_n\cap B}\uparrow \mu\brac{B}$ by theorem 7. Thus $\alpha<\mu\brac{B}$ implies that there is $N\geq 1$ with $\alpha < \mu\brac{K_n\cap B}$ for all $n\geq N$, whence for any such $\alpha$ there exists a compact subset $K\defn K_N$ is such that $\alpha<\mu\brac{K\cap B}$.

If $\mu^K\defn \mu\brac{\cdot \cap K}$, then $\mu^K$ is a measure on $\brac{\Omega,\borel{\Omega}}$, which is finite because $\mu\brac{K}<+\infty$ for any compact subset $K$ of $\Omega$ by theorem 13-5 and $\mu^K\brac{E}\leq \mu\brac{K}$ for all $E\in\borel{\Omega}$. Hence theorem 13-6 implies that $\mu^K$ is an inner-regular measure, since $\brac{\Omega,\Tcal}$ is metrizable and $\sigma$-compact. Therefore $\mu^K\brac{B}=\sup\obj{\induc{\mu^K\brac{G}}\,G\subseteq B,\,G\text{-- compact}}$, whence there is a compact subset $G$ of $\Omega$ with $G\subseteq B$ such that $\alpha < \mu^K\brac{G} = \mu\brac{K\cap G}$.

Now, if $H\defn K\cap G$, then $H\subseteq B$. Since $\brac{\Omega,\Tcal}$ is Hausdorff by theorem 8-6, theorem 35 implies that $G$ is closed in $\brac{\Omega,\Tcal}$, whence $H$ is closed in $\brac{K,\induc{\Tcal}_K}$. By theorem 8-5 $H$ is compact in $\brac{K,\induc{\Tcal}_K}$ and thus in the entire $\brac{\Omega,\Tcal}$.

Therefore for any $\alpha<\mu\brac{B}$ there exists a compact subset $H$ of $\Omega$ with $H\subseteq B$ such that $\alpha<\mu\brac{H}$, whence \[\mu\brac{B}\leq \sup\obj{\induc{\mu\brac{G}}\,G\subseteq B,\,G\text{-- compact}}\] Now, since $\mu\brac{H}\leq \mu\brac{B}$ for every compact subset $H$ of $\Omega$ with $H\subseteq B$ by theorem 2-4, it must be true that $\mu$ is an inner-regular measure on $\brac{\Omega,\borel{\Omega}}$.\\

\label{thm:metr_compact_separable} \noindent \textbf{Theorem} 13-8.
Every metrizable compact topological space $\brac{\Omega, \Tcal}$ is separable.

Indeed, let $d$ be the metric, which induces the topology $\Tcal$. Since $\brac{\Omega,\Tcal}$ is a compact topological space, for every $n\geq1$ for the collection $\brac{B^d_\Omega\brac{x,\frac{1}{n}}}_{x\in \Omega}$, which is an open covering of $\Omega$, there exists a finite $F_n\subseteq \Omega$ such that $\Omega = \bigcup_{x\in F_n} B^d_\Omega\brac{x,\frac{1}{n}}$.

Consider the set \[\Hcal\defn \obj{\induc{B^d_\Omega\brac{x,\frac{1}{n}}}\, n\geq1,\,x\in F_n}\] Since each $F_n$ is a finite subset of $\Omega$, the set $F\defn \bigcup_{n\geq1} F_n$ is at most countable, whence $\Hcal$ is a collection of open sets in $\brac{\Omega,\Tcal}$ with at most countably many members.

The collection $\Hcal$ is a topological base of $\Tcal$. Indeed, $B^d_\Omega\brac{x,\delta}$ is open in $\Omega$ for any $x\in \Omega$ and $\delta>0$ and for every $x\in \Omega$ by construction there is $\omega\in F_1$ such that $x\in B^d_\Omega\brac{\omega,1}$.

Next, if $U,V\in \Hcal$ are such that $U\cap V\neq \emptyset$, then there is $W\in \Hcal$ with $W\subseteq U\cap V$. Indeed, $U=B^d_\Omega\brac{x,\frac{1}{n}}$ and $V=B^d_\Omega\brac{y,\frac{1}{m}}$ for some $n,m\geq1$, $x\in F_n$ and $y\in F_m$. For any $z\in U\cap V$ for $k\geq1+\floor{\frac{1}{\delta}}$, where \[\delta\defn \frac{1}{2}\min\obj{ \frac{1}{n}-d\brac{x,z}, \frac{1}{m}-d\brac{y,z} }>0\] there is $\omega\in F_k$ with $z\in B^d_\Omega\brac{\omega, \frac{1}{k}}$. By the triangle inequality for every $p\in B^d_\Omega\brac{\omega, \frac{1}{k}}$ \[d\brac{x,p}\leq d\brac{x,z}+d\brac{z,\omega}+d\brac{\omega,p} < d\brac{x,z} + \frac{2}{k} < d\brac{x,z} + 2\delta < \frac{1}{n}\] whence $B^d_\Omega\brac{\omega, \frac{1}{k}}\subseteq U$. Similarly $B^d_\Omega\brac{\omega, \frac{1}{k}}\subseteq V$. Therefore $W\defn B^d_\Omega\brac{\omega, \frac{1}{k}}$ is such that $W\in \Hcal$ and $W\subseteq U\cap V$.

Finally, let $U\in \Tcal$ and pick any $x\in U$. Since $\Tcal$ is metrizable, there is $\delta>0$ with $B^d_\Omega\brac{x,\delta}\subseteq U$. However, by construction for $n\defn1+\floor{\frac{1}{\delta}}$ there is $\omega\in F_n$, such that $x\in B^d_\Omega\brac{\omega,\frac{1}{n}}$. By the triangle inequality \[x\in B^d_\Omega\brac{\omega,\frac{1}{n}}\subseteq B^d_\Omega\brac{\omega,\frac{\delta}{2}} \subseteq B^d_\Omega\brac{x,\delta}\] Therefore for every $x\in U$ there is $V\in \Hcal$ with $x\in V$ and $V\subseteq U$.

Therefore a compact metrizable topological space $\brac{\Omega,\Tcal}$ has a countable base, whence it is separable by theorem 6-4.\\

\label{thm:metr_sigma_compact_base} \noindent \textbf{Theorem} 72.
Every metrizable $\sigma$-compact topological space $\brac{\Omega, \Tcal}$ is separable and has a countable base.

Let $d$ be the metric on $\Omega$ which metrizes the space $\brac{\Omega, \Tcal}$. Since $\brac{\Omega, \Tcal}$ is a $\sigma$-compact topological space, there exists a countable collection $\brac{K_n}_{n\geq1}$ of compact subsets of $\Omega$, such that $K_n\uparrow \Omega$. By theorem 12 the subspace topology $\Tcal_n\defn \induc{\Tcal}_{K_n}$ is induced by the induced metric $d_n\defn \induc{d}_{K_n\times K_n}$:\[\Tcal_n = \induc{\Tcal}_{K_n} = \induc{\Tcal^d_\Omega}_{K_n} = \Tcal^{d_n}_{K_n}\]

Since for each $n\geq1$ the topological subspace $\brac{K_n,\Tcal_n}$ is metrizable and compact, by theorem 13-8 it must be separable. Therefore for every $n\geq1$ there is a dense subset $X_n\subseteq K_n$ with at most countably many elements. Hence the set $X\defn \bigcup_{n\geq1} X_n$ being a countable union of at most countable sets it itself at most countable.

Let $x\in \Omega$ and $U\in \Tcal$ with $x\in U$. Since $K_n\uparrow \Omega$, there is $N\geq1$ such that $x\in K_N$. Further, as $V\defn U\cap K_n$ is open in $\brac{K_n,\Tcal_n}$ and $x\in V$, the fact that $X_n$ is dense in $\brac{K_n,\Tcal_n}$ implies that $V\cap X_n \neq \emptyset$. Thus $U\cap X\neq \emptyset$, because $X_n\subseteq X$ and $V\subseteq U$.

Therefore for every $x\in \Omega$ and $U\in \Tcal$ with $x\in U$ it is true that $U\cap X\neq \emptyset$, whence $\Omega = \clo{X}$. Hence the set $X$ is dense in $\brac{\Omega,\Tcal}$. Hence $\brac{\Omega, \Tcal}$ is separable and by theorem 6-4 has a countable base.\\

\label{thm:metr_sigma_compact_count_cover} \noindent \textbf{Theorem} 13-9.
Let $\brac{\Omega, \Tcal}$ be a metrizable $\sigma$-compact topological space and $\mu$ be a locally finite measure on $\brac{\Omega,\borel{\Omega}}$. Then there exists $\brac{V_n}_{n\geq1}\in \Tcal$ with $\mu\brac{V_n}<+\infty$ and such that $\Omega = \bigcup_{n\geq1} V_n$.

Indeed, by theorem 72 there exists a countable base $\Hcal$ of $\brac{\Omega,\Tcal}$. Put \[\Hcal'\defn \obj{\induc{V\in \Hcal}\,\mu\brac{V}<+\infty}\]

Let $U\in \Tcal$ and $x\in U$. Since $\mu$ is locally-finite there is $W_x$ open in $\Omega$ such that $x\in W_x$ and $\mu\brac{W_x}<+\infty$. Therefore $U_x\defn U\cap W_x$ is an open set in $\Omega$ with $x\in U_x\subseteq U$ and $\mu\brac{U_x}\leq \mu\brac{W_x}<+\infty$ by theorem 2-4. Since $\Hcal$ is a base of $\brac{\Omega,\Tcal}$, every open set in $\Omega$ is an arbitrary union of sets from the base, whence there is $V_x\in \Hcal$ such that $x\in V_x\subseteq U_x$. From $V_x\subseteq U_x$ theorem 2-4 implies that $\mu\brac{V_x}<+\infty$, whence $V_x\in \Hcal'$.

Therefore for any $U\in \Tcal$ and any $x\in U$ there is $V_x\in\Hcal'$ with $x\in V_x\subseteq U$. Hence for \[\Gamma\defn\obj{\induc{W\in \Hcal'}\,W\subseteq U}\] it is true that $\Gamma$ is at most countable and $U=\bigcup_{V\in \Gamma} V$. Thus $\Hcal'$ is a countable base of $\brac{\Omega, \Tcal}$.

Finally, since $\Hcal'$ is a countable base, there exists $\brac{V_n}_{n\geq1}\in \Hcal'$ such that $\brac{V_n}_{n\geq1}$ open in $\Omega$, $\Omega=\bigcup_{n\geq1} V_n$ and $\mu\brac{V_n}<+\infty$.\\

\label{thm:loc_fin_metr_compact_meas} \noindent \textbf{Theorem} 73.
Let $\brac{\Omega, \Tcal}$ be a metrizable $\sigma$-compact topological space. Then every locally finite measure $\mu$ on $\brac{\Omega,\borel{\Omega}}$ is a regular measure.

Indeed, by theorem 13-9 there exists $\brac{V_n}_{n\geq1}$ open in $\Omega$ with $\Omega = \bigcup_{n\geq1} V_n$ and $\mu\brac{V_n}<+\infty$ for all $n\geq1$. For every $n\geq1$ define $\mu_n=\mu^{V_n}\defn \mu\brac{V_n\cap \cdot}$. Note that for each $n\geq1$ the measure $\mu_n$ is a finite measure on $\brac{\Omega,\borel{\Omega}}$ by theorem 2-4, since $\mu\brac{V_n}<+\infty$.

Let $B\in\borel{\Omega}$ and $\alpha \defn \inf\obj{\induc{\mu\brac{K}}\,B\subseteq G,\,G\text{-- open}}$.

By theorem 68 for any $\epsilon>0$ there exists $\brac{G_n}_{n\geq1}$ open in $\Omega$ with $B\subseteq G_n$ and such that $\mu_n\brac{G_n\setminus B}<\frac{\epsilon}{2^n}$ for all $n\geq1$. Since for each $n\geq1$ the set $V_n\cap G_n$ is open in $\Omega$, the set $G\defn \bigcup_{n\geq1} \brac{V_n\cap G_n}$ is itself open in $\Omega$. Furthermore $B\subseteq G$ since for every $x\in B$ there is $N\geq1$ such that $x\in V_N$, whence $x\in V_N\cap G_N$.

Trivial properties of set operations imply that $G\setminus B=\bigcup_{n\geq 1} V_n \cap \brac{G_n\setminus B}$. Hence by theorem 3-6\begin{align*}\mu\brac{G\setminus B}&\leq \sum_{n\geq1}\mu\brac{V_n \cap \brac{G_n\setminus B}} \\&= \sum_{n\geq1}\mu_n\brac{G_n\setminus B}\\&\leq \sum_{n\geq1} \frac{\epsilon}{2^n} = \epsilon\end{align*}

Thus for every $\epsilon>0$ there exists an open set $G\subseteq \Omega$ such that $B\subseteq G$ and \[\mu\brac{G}=\mu\brac{B}+\mu\brac{G\setminus B}\leq \mu\brac{B}+\epsilon\] which implies that $\mu\brac{G}\leq \mu\brac{B}$. Therefore $\alpha \leq \mu\brac{B}$ by definition of the greatest lower bound. Obviously, by theorem 2-6 $\mu\brac{B}\leq \mu\brac{G}$ for every $G$ open in $\Omega$ with $B\subseteq G$. Hence, according to definition 101 the measure $\mu$ is outer-regular, and thus by theorem 13-7 $\mu$ is a regular measure.\\

\label{thm:rn_loc_fin_regular_meas} \noindent \textbf{Theorem} 74.
Let $\Omega$ be an open subset of $\Real^n$, where $n\geq1$. Then any locally finite measure on $\brac{\Omega,\brac{\Omega}}$ is regular.

The following closed nested subsets of $\Real^n$ \[K_m\defn \clo{-m,m}\times \ldots \clo{-m,m}\] are such that each $K_m$ is compact in $\Real^n$ by theorem 10-6 and $K_m\uparrow \Real^n$. Therefore the space $\Real^n$ is $\sigma$-compact and by theorem 6-3-1 is metrizable by a natural Euclidean metric on $\Real^n$.

By theorem 71 the topological subspace $\brac{\Omega, \Tcal}$ is metrizable and $\sigma$-compact. Thus by theorem 73 any locally finite measure on $\brac{\Omega, \borel{\Omega}}$ is regular.\\

\noindent\textbf{Definition} 104.
A topological space $\brac{\Omega, \Tcal}$ is strongly $\sigma$-compact if there exist $\brac{V_n}_{n\geq1}$ open in $\Omega$ with $V_n\uparrow \Omega$ which have compact closure in $\brac{\Omega, \Tcal}$.

\noindent\textbf{Definition} 105.
A topological space $\brac{\Omega, \Tcal}$ is locally compact if for every point of $\Omega$ there exists an open neighbourhood with compact closure. In other words for any $x\in \Omega$ there is $U\in \Tcal$ with $x\in U$ and such that $\clo{U}$ is compact in $\brac{\Omega, \Tcal}$.

\label{thm:finite_union_closed}\noindent\textbf{Theorem} 13-10.
Let $\brac{\Omega,\Tcal}$ be a topological space and $\brac{W_k}_{k=1}^n$ be subsets of $\Omega$, $n\geq1$. Then $\clo{\bigcup_{k=1}^n W_k} = \bigcup_{k=1}^n \clo{W_k}$.

Indeed, if $x\in \bigcup_{k=1}^n \clo{W_k}$, then there exists $k=1\ldots n$ with $x\in \clo{W_k}$. For any $U\in \Tcal$ with $x\in U$ it is true that $W_k\cap U\neq \emptyset$, whence $u\cap \bigcup_{k=1}^n W_k \neq \emptyset$ as well. Thus $x\in \clo{\bigcup_{k=1}^n W_k}$.

Conversely, if $x\notin \bigcup_{k=1}^n \clo{W_k}$ then for $U\defn \bigcap_{k=1}^n \clo{W_k}^c$ it is true that $x\in U$ and $U$ is open in $\brac{\Omega,\Tcal}$ since $\clo{W_k}^c$ is open in $\Omega$ and $U$ is an intersection of finitely many open sets. Now, $W_k\cap U\subseteq w_k\cap \clo{W_k}^c = \emptyset$ for all $k=1\ldots n$, whence $\brac{\bigcup_{k=1}^n W_k}\cap U = \emptyset$. Therefore $x\notin \clo{\bigcup_{k=1}^n W_k}$.\\

\label{thm:finite_union_cmpact}\noindent\textbf{Theorem} 13-11.
Let $\brac{\Omega,\Tcal}$ be a topological space and $\brac{K_m}_{m=1}^n$ be compact subsets of $\Omega$, $n\geq1$. Then $\bigcup_{m=1}^n K_m$ is compact in $\brac{\Omega,\Tcal}$ as well.

Indeed, let $\brac{U_i}_{i\in I}$ be an open in $\brac{\Omega,\Tcal}$ cover of $K\defn \bigcup_{m=1}^n K_m$. Then for every $m=1\ldots n$ the collection $\brac{U_i}_{i\in I}$ is an open covering of $K_m$, whence there exists a finite $F_n\subseteq I$ with $K_m\subseteq \bigcup_{i\in F_n} U_i$. If $F\defn \bigcup_{m=1}^n F_m$, then $F$ as a finite union of finite sets is a finite subset of $I$ such that \[\bigcup_{m=1}^n K_m \subseteq \bigcup_{m=1}^n \bigcup_{i\in F_n} U_i = \bigcup_{i\in F} U_i\] Thus, there exists a finite set $F\subseteq I$ such that $\brac{U_i}_{i\in F}$ is a covering of $\bigcup_{m=1}^n K_m$.\\

\label{thm:nested_closure}\noindent \textbf{Theorem} 13-12.
Let $\brac{\omega, \Tcal}$ be a topological space. Then $\clo{A}\subseteq \clo{B}$ whenever $A\subseteq B$.

Indeed, if $x\in \clo{A}$ and $U$ is open in $\Omega$ with $x\in U$, then $\emptyset\neq U\cap A\subseteq U\cap B$, whence $\clo{A}\subseteq \clo{B}$.\\

\label{thm:strong_compactness} \noindent \textbf{Theorem} 75.
A topological space is strongly $\sigma$-compact if and only if it is both $\sigma$-compact and locally compact.

Suppose $\brac{\Omega,\Tcal}$ is a $\sigma$-compact and a locally compact topological space. Then for every $x\in \Omega$ there exists $U_x\in \Tcal$ with $x\in U_x$ and $\clo{U_x}$ compact in $\Omega$. Also there exists $\brac{K_n}_{n\geq1}$ compact such that $K_n\uparrow \Omega$.

For every $n\geq1$ the collection $\brac{U_x}_{x\in K_n}$ is an open covering of $K_n$, whence there is $F_n\subseteq K_n$ finite such that $K_n\subseteq \bigcup_{x\in F_n} U_x$. Let $W_n\defn \bigcup_{x\in F_n} U_x$.

First, since $W_n$ is a finite union, theorem 13-10 implies that $\clo{W_n}=\bigcup_{x\in F_n} \clo{U_x}$, whence $\clo{W_n}$ is compact in $\brac{\Omega,\Tcal}$ by theorem 13-11. Similarly $V_n\defn \bigcup_{k=1}^n W_n$ is such that $\clo{V_n} = \bigcup_{k=1}^n \clo{W_n}$ by theorem 13-10 and $\clo{V_n}$ is compact in $\brac{\Omega,\Tcal}$ by theorem 13-11.

Now, since $K_n\subseteq W_n$ for all $n\geq1$, it is true that \[\Omega = \bigcup_{n\geq1} K_n \subseteq \bigcup_{n\geq1}W_n = \bigcup_{n\geq1}V_n\] whence $V_n\uparrow \Omega$. Therefore there exists $\brac{V_n}_{n\geq1}$ open in $\brac{\Omega,\Tcal}$ and such that $\clo{V_n}$ is compact. Thus by definition 104 the space $\brac{\Omega, \Tcal}$ is strongly $\sigma$-compact.

Conversely, suppose the space $\brac{\Omega, \Tcal}$ is strongly $\sigma$-compact. Then there exist $\brac{V_n}_{n\geq1}$ open in $\Omega$ with $V_n\uparrow \Omega$ which have compact closure in $\brac{\Omega, \Tcal}$.

For any $x\in \Omega$ there is $n\geq1$ with $x\in V_n$ and $\clo{V_n}$ compact in $\Omega$. Hence, $\brac{\Omega,\Tcal}$ is locally compact.

Now, for $K_n\defn \clo{V_n}$ it is true that $K_n\subseteq K_{n+1}$ by theorem 13-12 and \[\Omega = \bigcup_{n\geq1} V_n \subseteq \bigcup_{n\geq1} K_n\] whence $K_n\uparrow \Omega$. Since $\clo{V_n}$ is compact in $\Omega$, the space $\brac{\Omega,\Tcal}$ is $\sigma$-compact.\\

\label{thm:subspace_closure} \noindent \textbf{Theorem} 13-13.
Let $\brac{\Omega, \Tcal}$ be a topological space and $\Omega'$ be a subset of $\Omega$. Then $\clo{A}_{\Omega'} = \clo{A}\cap \Omega'$ for any $A\subseteq \Omega'$, where $\clo{A}_{\Omega'}$ the closure of $A$ with respect to the subspace $\brac{\Omega',\Tcal'}$ with $\Tcal'\defn \induc{\Tcal}_{\Omega'}$.

Let $A\subseteq \Omega'$ and denote by $\clo{A}_{\Omega'}$ the closure of $A$ with respect to the subspace $\brac{\Omega',\Tcal'}$, where $\Tcal'\defn \induc{\Tcal}_{\Omega'}$.

Let $\brac{\Omega, \Tcal}$ be a topological space and $\Omega'$ be a subset of $\Omega$. Let $A\subseteq \Omega'$ and denote by $\clo{A}_{\Omega'}$ the closure of $A$ with respect to the subspace $\brac{\Omega',\Tcal'}$, where $\Tcal'\defn \induc{\Tcal}_{\Omega'}$.

First $A\subseteq \clo{A}$, whence $A\subseteq \Omega'\cap\clo{A}$. Since $\clo{A}$ is closed in $\Omega$, $U\defn \brac{\clo{A}}^c$ is open. Thus $\Omega'\cap U$ is open in $\Omega'$ and since \[\Omega'\cap \clo{A} = \Omega'\cap U^c = \Omega'\setminus \brac{\Omega'\cap U}\] the set $\Omega'\cap \clo{A}$ must be closed in $\Omega'$.

Suppose $x\notin \Omega'\cap \clo{A}$. Then $U\defn \brac{\Omega'\cap \clo{A}}$ is an open set in $\Omega'$ with $x\in U$. Since $A\subseteq \clo{A}$ and $A\subseteq \Omega'$, $A\subseteq \Omega'\cap \clo{A}$, whence $A\cap U = \emptyset$. Thus $x\notin \clo{A}_{\Omega'}$.

Let $x\in \Omega'\cap \clo{A}$ and pick any $U\in \Tcal$ with $x\in U'$. For this $U'$ there exists $U\in \Tcal$ such that $U'=U\cap \Omega'$. Since $x\in \clo{A}$ and $x\in U$, it is true that $U\cap A \neq \emptyset$. As $A\subseteq \Omega'$, $U'\cap A = U\cap A$ and $U'\cap A \neq \emptyset$, whence $x\in \clo{A}_{\Omega'}$. Therefore $\clo{A}_{\Omega'} = \Omega' \cap \clo{A}$.\\

\label{thm:subspace_loc_compact} \noindent \textbf{Theorem} 13-14.
Let $\brac{\Omega,\Tcal}$ be a metrizable locally compact space and $\Omega'$ be an open subset of $\Omega$. Then the induced topological space $\Omega'$ is itself metrizable and locally compact.

By theorem 12 any topological subspace of a metrizable space is metrizable by the induced metric. Let $d$ be the metric, which metrizes $\brac{\Omega, \Tcal}$.

Now, for any $x\in \Omega'$ by local compactness there exists $U$ open in $\Omega$ with $x\in U$ such that $\clo{U}$ is compact in $\brac{\Omega,\Tcal}$. Since $U\cap \Omega'\in \Tcal$ and the space is metrizable by $d$, there is $\epsilon>0$ with $B\brac{x,\epsilon}\subseteq U\cap \Omega'$. Furthermore $\clo{B\brac{x,\frac{\epsilon}{2}}}\subseteq \clo{U}$ and is closed in the topological subspace $\clo{U}$. Since $\clo{U}$ is compact, $\clo{B\brac{x,\frac{\epsilon}{2}}}$ is compact in $\clo{U}$ by theorem 8-5 and hence is a compact subset of $\Omega$.

If $y\in \clo{B\brac{x,\frac{\epsilon}{2}}}$ then, in particular, there exists $z\in B\brac{y,\frac{\epsilon}{2}} \cap B\brac{x,\frac{\epsilon}{2}}$, whence by the triangle inequality $d\brac{x,y}\leq d\brac{x,z} + d\brac{z,y} < \epsilon$. Therefore $y\in B\brac{x,\epsilon}$ and \[\clo{B\brac{x,\frac{\epsilon}{2}}}\subseteq \Omega'\] whence $\clo{B\brac{x,\frac{\epsilon}{2}}}$ is compact in $\Omega'$.

So, if $W\defn B\brac{x,\frac{\epsilon}{2}}\cap \Omega'$, then $x\in W\in \induc{\Tcal}_{\Omega'}$ and \[\clo{W}_{\Omega'} = \Omega'\cap \clo{B\brac{x,\frac{\epsilon}{2}}}=\clo{B\brac{x,\frac{\epsilon}{2}}}\] by theorem 13-13, whence $\clo{W}_{\Omega'}$ is compact in $\Omega'$. Therefore the induced topological space $\brac{\Omega',\induc{\Tcal}_{\Omega'}}$ is metrizable and locally compact.\\

\label{thm:subspace_strong_compact} \noindent \textbf{Theorem} 76.
Let $\brac{\Omega,\Tcal}$ be a metrizable strongly $\sigma$-compact space. Then for all $\Omega'$ open subsets of $\Omega$, the topological subspace $\brac{\Omega',\Tcal'}$, where $\Tcal'\defn \induc{\Tcal}_{\Omega'}$, is itself metrizable and strongly $\sigma$-compact.

Indeed, by theorem 75 the space $\brac{\Omega, \Tcal}$ is $\sigma$-compact and locally compact. Thus by theorem 13-14 the induced topological space $\brac{\Omega',\Tcal'}$ is locally compact. Furthermore theorem 71 implies that $\brac{\Omega',\Tcal'}$ is a $\sigma$-compact topological space. Therefore, by theorem 75 the space $\brac{\Omega',\Tcal'}$ is strongly $\sigma$-compact and metrizable by theorem 12.\\

\noindent\textbf{Definition} 106.
Let $\brac{\Omega,\Tcal}$ be a topological space and $\phi:\Omega\to \Cplx$ be a map. The support of $\phi$ is the closure of the set $\obj{\phi\neq 0}$:\[\Supp{\phi}\defn \clo{\obj{ \induc{\omega\in\Omega}\,\phi\brac{\omega}\neq 0 }}\]

\noindent\textbf{Definition} 107.
Let $\brac{\Omega, \Tcal}$ be a topological space. Denote by $C^c_K\brac{\Omega,\Tcal}$ the set of all continuous maps $\phi:\Omega\to K$ with $\Supp{\phi}$ compact in $\brac{\Omega,\Tcal}$.

\label{thm:compact_support_functions}\noindent \textbf{Theorem} 13-15.
Let $\brac{\Omega,\Tcal}$ be a topological space and $K=\Cplx$ or $\Real$. The set $C^c_K\brac{\Omega,\Tcal}$ is a $K$-vector space, with $C^c_K\brac{\Omega,\Tcal}\subseteq C^b_K\brac{\Omega,\Tcal}$.

The map $\phi\brac{\omega}\defn 0$ for all $\omega\in \Omega$ is $\Tcal$-$\Tcal_K$ continuous by theorem Sup-A-4. The support of $\phi$ is $\emptyset$ since $\phi\brac{\omega}\neq 0$ for no $\omega\in\Omega$. Since $\emptyset$ is trivially a compact subset of $\Omega$, it must be true that $0\in C^c_K\brac{\Omega,\Tcal}$.

Suppose $\phi\in C^c_K\brac{\Omega,\Tcal}$ and $\alpha\in K$. Let the map $\alpha \phi$ be defined as $\brac{\alpha \phi}\brac{\omega}\defn \alpha \phi\brac{\omega}$. If $\alpha=0$ then $\alpha \phi = 0$ and $0\in C^c_K\brac{\Omega,\Tcal}$, so suppose $\alpha\neq 0$. For every $x\in \Omega$ via theorem Sup-A-1 continuity of $\phi$ implies that for any $\epsilon>0$ there exists $V\in \Tcal$ with $x\in V$ such that $\abs{\phi\brac{x}-\phi\brac{y}}<\frac{\epsilon}{\abs{\alpha}}$ for all $y\in V$. Therefore $\abs{\brac{\alpha\phi}\brac{x}-\brac{\alpha\phi}\brac{y}}<\epsilon$ for any $y\in V$, whence $\alpha\phi$ is $\Tcal$-$\Tcal_K$ continuous, because open balls constitute a topological basis of any metric topology.

Since $\alpha\phi\neq 0$ if and only if $\phi\neq 0$, $\Supp{\alpha\phi} = \Supp{\phi}$ and $\alpha\phi\in C^c_K\brac{\Omega,\Tcal}$.

Suppose $\phi_1,\phi_2\in C^c_K\brac{\Omega,\Tcal}$ and let the map $\phi_1+\phi_2$ be defined as $\brac{\phi_1+\phi_2}\brac{\omega}\defn \phi_1\brac{\omega}+\phi_2\brac{\omega}$.

Since $\phi_1$ and $\phi_2$ are $\Tcal$-$\Tcal_K$ continuous, for any $x\in \Omega$ and for every $\epsilon>0$ there exist $V_1,V_2\in \Tcal$ with $x\in V_1,V_2$ such that $\abs{\phi_i\brac{x}-\phi_i\brac{y}}<\frac{\epsilon}{2}$ for all $y\in V_i$, where $i=1,2$. Since intersection of finitely many open sets is open, $V\defn V_1\cap V_2$ is open in $\brac{\Omega, \Tcal}$, whence by the triangle inequality for all $y\in V$ it is true that \[\abs{\brac{\phi_1+\phi_2}\brac{x}-\brac{\phi_1+\phi_2}\brac{y}}\leq \abs{\phi_1\brac{x}-\phi_1\brac{y}} + \abs{\phi_2\brac{x}-\phi_2\brac{y}} < \epsilon\] Therefore $\phi_1+\phi_2$ is $\Tcal$-$\Tcal_K$ continuous.

Now, \[\obj{\phi_1+\phi_2\neq 0}\subseteq \obj{\phi_1\neq 0}\cup \obj{\phi_2\neq 0}\] whence by theorem 13-10 $\Supp{\phi_1+\phi_2}\subseteq \Supp{\phi_1}\cup\Supp{\phi_2}$. By theorem 13-11 $\Supp{\phi_1+\phi_2}$ is compact in $\brac{\Omega,\Tcal}$, which implies that $\phi_1+\phi_2\in C^c_K\brac{\Omega,\Tcal}$.

Therefore $C^c_K\brac{\Omega,\Tcal}$ is a $K$-vector space.

Let $\phi\in C^c_K\brac{\Omega,\Tcal}$. If $\Supp{\phi}=\emptyset$ then $\phi = 0$ everywhere on $\Omega$, whence $\phi\in C^b_K\brac{\Omega,\Tcal}$.

Suppose $\Supp{\phi}\neq\emptyset$. Since $\phi$ is continuous, theorem Sup-A-2 implies that $\bar{\phi}\defn \induc{\phi}_{\Supp{\phi}}$ is continuous $\induc{\Tcal}_{\Supp{\phi}}$-$\Tcal_K$ continuous. Since $\Supp{\phi}$ is compact, by the extreme value theorem (theorem 37) there exist $x^m,x^M\in \Supp{\phi}$ such that $\bar{\phi}\brac{x^m}\leq \bar{\phi}\brac{x}\leq \bar{\phi}\brac{x^M}$ for all $x\in \Supp{\phi}$. Since $\bar{\phi}=\phi$ on $\Supp{\phi}$, letting $M\defn \max\obj{\abs{\phi\brac{x^M}},\abs{\phi\brac{x^m}}}$ yields $M\in \Real^+$ such that $\abs{\phi\brac{\omega}}\leq M$ for any $\omega\in\Omega$, since $\phi = 0$ outside of $\Supp{\phi}$. Therefore $\phi\in C^b_K\brac{\Omega,\Tcal}$.\\

\label{thm:loc_compact_closure}\noindent\textbf{Theorem} 13-16.
Let $\brac{\Omega,\Tcal}$ be a metrizable locally compact topological space. For any $K\neq \emptyset$ compact and $G$ open subsets of $\Omega$ with $K\subseteq G$, there exists $V$ open in $\Omega$ with compact closure such that $K\subseteq V\subseteq G$.

Since $\brac{\Omega, \Tcal}$ is locally compact, for every $x\in \Omega$ there exists at least one $U_x\in \Tcal$ with $x\in U_x$ such that $\clo{U_x}$ is compact in $\brac{\Omega,\Tcal}$. Thus $\brac{U_x}_{x\in K}$ is an covering of $K$ open in $\Omega$. Since $K$ is compact there exist a finite $F\subseteq K$ such that $K\subseteq \bigcup_{x\in F} U_x$.

Let $V\defn G\cap \bigcup_{x\in F} U_x$. Since $G$ is open in $\Omega$, $V\in \Tcal$ and $K\subseteq V\subseteq G$, because $K\subseteq G$ and $K\subseteq  \bigcup_{x\in F} U_x$. By theorem 13-12 $\clo{V}\subseteq \clo{\bigcup_{x\in F} U_x}$ and by theorem 13-10 $\clo{V} \subseteq \bigcup_{x\in F} \clo{U_x}$ because $F$ is finite. Furthermore, since $\clo{U_x}$ is compact in $\Omega$ for each $x\in F$, by theorem 13-11 $\bigcup_{x\in F} \clo{U_x}$ is compact in $\brac{\Omega,\Tcal}$. By theorem 8-5 $\clo{V}$ must be compact in the topological subspace induced by $\bigcup_{x\in F} \clo{U_x}$, whence it must be compact in $\brac{\Omega,\Tcal}$.\\

\label{thm:compact_support_map}\noindent\textbf{Theorem} 77.
Let $\brac{\Omega,\Tcal}$ be a metrizable locally compact topological space. Let $K$ be a compact subset of $\Omega$ and $G$ be open in $\Omega$ with $K\subseteq G$. Then there exists $\phi\in C^c_\Real\brac{\Omega,\Tcal}$,  a real-valued continuous map with compact support, such that \[1_K\leq \phi \leq 1_G\]

If $K=\emptyset$, then for $\phi\defn 0$ it is true that $\phi\in C^c_\Real\brac{\Omega,\Tcal}$ and $1_K = 0 = \phi\leq 1_G$ everywhere of $\Omega$.

Suppose $K\neq \emptyset$. By theorem 13-16 there is $V\in \Tcal$ with $K\subseteq V\subseteq G$ such that $\clo{V}$ is compact in $\brac{\Omega,\Tcal}$.

Suppose $V\neq \Omega$. Define $\phi:\Omega\to\Real$ by \[\phi\brac{x}\defn \frac{d\brac{x,V^c}}{d\brac{x,V^c}+d\brac{x,K}}\] Since $V\neq \Omega$, $d\brac{x,V^c}\in \Real^+$ for every $x\in \Omega$, because for every $y\in V^c$ \[d\brac{x,V^c}\leq d\brac{x,y}<+\infty\] Similarly, $d\brac{x,K}\in \Real^+$.

Now, $d\brac{x,V^c}+d\brac{x,K}=0$ if and only if $x\in \Omega$ is such that $d\brac{x,V^c}=d\brac{x,K}=0$, which by theorem 4-8 occurs if and only if $x\in \clo{V^c}$ and $x\in \clo{K}$. However $V^c$ is closed in $\Omega$ and $K$ is closed in $\brac{\Omega, \Tcal}$ by theorem 35, since $\brac{\Omega, \Tcal}$ is metrizable and thus by theorem 8-6 Hausdorff. Therefore $d\brac{x,V^c}+d\brac{x,K}=0$ if and only if $x\in V^c$ and $x\in K$, which is impossible since $K\subseteq V$. Therefore, as a ratio of real numbers the map $\phi$ is well-defined everywhere on $\Omega$. Furthermore as a composite function of continuous map, $\phi$ must be $\Tcal$-$\Tcal_\Real$ continuous.

Now, $\phi\brac{x}\neq 0$ if and only if $d\brac{x,V^c}\neq 0$. Thus, since $V^c$ is closed in $\Omega$, $x\in V$, which is equivalent to $x\notin V^c$, implies $d\brac{x,V^c}>0$. Conversely, if $d\brac{x,V^c}>0$ implies that $x\notin \clo{V^c}=V^c$, whence $x\in V$. Therefore $\obj{\phi\neq 0} = V$. Now, since the closure of $V$ is compact in $\brac{\Omega,\Tcal}$, it must be true that $\Supp{\phi} = \clo{V}$ is compact in $\Omega$, whence $\phi\in C^c_\Real\brac{\Omega,\Tcal}$.

Now, let $x\in \Omega$. If $x\notin G$, then $x\in V^c$ and $x\notin K$, whence by theorem 4-8 $d\brac{x,V^c}=0$ and $d\brac{x,K}>0$. Thus $\phi\brac{x}=0$ and $1_K\brac{x}=0 = \phi\brac{x} = 1_G\brac{x}$. If $x\in G$, but $x\notin V$, then $x\in V^c$ and $x\notin K$, whence by theorem 4-8 $d\brac{x,V^c}=0$ and $d\brac{x,K}>0$. Hence $\phi\brac{x}=0$ and $1_K\brac{x}=0=\phi\brac{x} \leq 1 = 1_G\brac{x}$.

If $x\in V$, but $x\notin K$, then by theorem 4-8 $d\brac{x,V^c}>0$ and $d\brac{x,K}>0$. Since $1_K\brac{x}=0$, $1_G\brac{x}=1$ and \[\phi\brac{x}=1-\frac{d\brac{x,K}}{d\brac{x,V^c}+d\brac{x,K}}\] it is true that $1_K\brac{x} \leq \phi\brac{x} \leq 1_G\brac{x}$.

If $x\in K$, then by theorem 4-8 $d\brac{x,V^c}>0$ and $d\brac{x,K}=0$, whence $\phi\brac{x}=1$. Since $1_K\brac{x}=1$ and $1_G\brac{x}=1$, it is true that $1_K\brac{x} \leq \phi\brac{x} \leq 1_G\brac{x}$. Thus $1_K\leq \phi 1_G$ everywhere on $\Omega$.

Now, if $V=\Omega$, then $\Omega=\clo{\Omega}$ is compact, since by construction $\clo{V}$ is compact in $\brac{\Omega,\Tcal}$. Thus setting $\phi\defn 1$ means that by theorem Sup-A-4 $\phi$ is a $\Tcal$-$\Tcal_\Real$ continuous map with compact support, since $\Supp{\phi}=\Omega$ is compact. Thus $\phi\in C^c_\Real\brac{\Omega,\Tcal}$. Now, $V\subseteq G$ implies that $1_G=1$ everywhere on $\Omega$ and that $1_K\leq \phi \leq 1_G$.

Therefore for any $K$ compact and $G$ open in $\Omega$ with $K\subseteq G$ there exists $\phi\in C^c_\Real\brac{\Omega,\Tcal}$ with $1_K\leq \phi \leq 1_G$.\\

\label{thm:compact_support_indicator} \noindent\textbf{Theorem} 13-17.
Let $\brac{\Omega,\Tcal}$ be a metrizable strongly $\sigma$-compact topological space. Let $\mu$ be a locally finite measure on $\brac{\Omega,\borel{\Omega}}$ and $p\in \clop{1,+\infty}$. Then $C^c_K\subseteq L^p_K$ and 
for any $B\in \borel{\Omega}$ with $\mu\brac{B}<+\infty$ and every $\epsilon>0$ there exists $\phi\in C^c_K$ such that $\nrm{\phi-1_B}_p<\epsilon$.

Indeed, let $p\in\clop{1,+\infty}$. By theorem 13-15 $C^c_K\brac{\Omega,\Tcal}\subseteq C^b_K\brac{\Omega,\Tcal}$, whence there exists $M\in \Real^+$ with $\abs{\phi}\leq M$. Now, since by theorem 8-6 $\brac{\Omega,\Tcal}$ is Hausdorff and $\Supp{\phi}$ is compact in $\brac{\Omega,\Tcal}$, by theorem 35 it must be closed in $\Omega$ and thus $\borel{\Omega}$ measurable. Since $\mu$ is locally finite, theorem 13-5 implies that $\mu\brac{\Supp{\phi}}<+\infty$, whence $\mu\brac{A}<+\infty$, where $A\defn \obj{\phi\neq 0}$. Since by definition $\phi = \phi 1_A$, \[\int \abs{\phi}^p d\mu = \int 1_A \abs{\phi}^p d\mu \leq M^p \mu\brac{A} < +\infty\] whence $\phi\in L^p_K\brac{\Omega,\borel{\Omega},\mu}$. Hence $C^c_K\brac{\Omega,\Tcal}\subseteq L^p_K\brac{\Omega,\borel{\Omega},\mu}$.

Since $\brac{\Omega,\Tcal}$ is a metrizable strongly $\sigma$-compact topological space, theorem 75 implies that it is $\sigma$-compact and locally compact. Therefore by theorem 73 $\mu$ is a regular measure on $\brac{\Omega,\borel{\Omega}}$.

Let $B\in \borel{\Omega}$ be such that $\mu\brac{B}<+\infty$ and pick any $\epsilon>0$. Since $\mu$ is inner-regular there exists $K$ compact in $\Omega$ with $K\subseteq B$ such that \[\mu\brac{B}-\frac{\epsilon^p}{2}< \mu\brac{K}\leq \mu\brac{B}\] Since $\mu$ is outer-regular, there must be $G$ open in $\Omega$ with $B\subseteq G$ such that \[\mu\brac{B}\leq \mu\brac{G}<\mu\brac{B}+\frac{\epsilon^p}{2}\] Therefore there exist $K$ compact and $G$ open in $\Omega$ with $K\subseteq B\subseteq G$ such that $\mu\brac{B\setminus K}<\frac{\epsilon^p}{2}$ and $\mu\brac{G\setminus B}<\frac{\epsilon^p}{2}$. Since $G\setminus K = G\setminus B \uplus B\setminus K$, by theorem 2-3, $\mu\brac{G\setminus K}<\epsilon^p$.

By theorem 77 there exists $\phi\in C^c_\Real$ with $1_K\leq\phi\leq1_G$. Since $K\subseteq B\subseteq G$, it is true that $1_K\leq1_B\leq1_G$, whence $1_K-1_G\leq \phi-1_B\leq 1_G-1_K$. Thus $\abs{\phi-1_B}\leq \abs{1_G-1_K}$ and \[\int\abs{\phi-1_B}^p d\mu \leq \int \abs{1_G-1_K}^p d\mu =  \int 1_{G\setminus K} d\mu = \mu\brac{G\setminus K}\] Hence $\phi$ is such that $\nrm{\phi-1_B}_p\leq \brac{\epsilon^p}^\frac{1}{p} = \epsilon$.

Thus for any $B\in \borel{\Omega}$ with $\mu\brac{B}<+\infty$ for every $\epsilon>0$ there exists $\phi\in C^c_\Real\brac{\Omega,\Tcal}$ such that $\nrm{\phi-1_B}_p<\epsilon$.\\

\label{thm:compact_support_lp_dense} \noindent\textbf{Theorem} 78.
Let $\brac{\Omega,\Tcal}$ be a metrizable strongly $\sigma$-compact topological space. Let $\mu$ be a locally finite measure on $\brac{\Omega,\borel{\Omega}}$. Then for all $p\in \clop{1,+\infty}$ the space $C^c_K\brac{\Omega,\Tcal}$ is dense in $L^p_K\brac{\Omega,\borel{\Omega},\mu}$.

Let $s\in S_\Cplx\brac{\Omega,\borel{\Omega}}\cap L^p_\Cplx\brac{\Omega,\borel{\Omega},\mu}$. Then by theorems 13-1 and 13-2 $s=\sum_{k=1}^n \alpha 1_{A_k}$ where $n\geq1$, non-zero $\brac{\alpha_k}_{k=1}^n\in \Cplx$ and $\brac{A_k}_{k=1}^n\in \borel{\Omega}$ are such that $\mu\brac{A_k}<+\infty$ for all $k=1\ldots n$. For any $\epsilon>0$ by theorem 13-17 there exist $\brac{\phi_k}_{k=1}^n\in C^c_\Real\brac{\Omega,\Tcal}$ with $\nrm{\phi_k-1_{A_k}}_p<\frac{\epsilon}{n\abs{\alpha_k}}$ for each $k=1\ldots n$.

The map $\phi\defn \sum_{k=1}^n \alpha \phi_k$, being a finite $K$-linear combination of functions from $C^c_K$, by theorem 13-15 $\phi\in C^c_K\brac{\Omega,\Tcal}$.

Alternatively, the map $\phi\defn \sum_{k=1}^n \alpha \phi_k$, being a finite weighted sum of $\Tcal$-$\Tcal_K$ continuous functions, is itself $\Tcal$-$\Tcal-K$ continuous. Furthermore $\bigcap_{k=1}^n\obj{\phi_k=0}\subseteq \obj{\phi=0}$, whence $\obj{\phi\neq0}\subseteq \bigcup_{k=1}^n \obj{\phi_k\neq 0}$. By theorems 13-10 and 13-12 \[\Supp{\phi}\subseteq \bigcup_{k=1}^n \Supp{\phi_k}\] and by theorem 13-11 $\Supp{\phi}$ is compact in $\brac{\Omega,\Tcal}$. Hence $\phi\in C^c_K$.

By theorem 9-2 $\nrm{\alpha_k \phi_k - \alpha_k 1_{A_k}}_p = \abs{\alpha_k}\nrm{\phi_k-1_{A_k}}<\frac{\epsilon}{n}$ for each $k=1\ldots n$. By the triangle inequality for $L^p$ norms (theorem 9-3) implies that \[\nrm{\phi - s}_p\leq \sum_{k=1}^n \nrm{\alpha_k \phi_k - \alpha_k 1_{A_k}}_p < \epsilon\]

Let $f\in L^p_K$. For any $U\in \Tcal_{L^p}$ with $f\in U$ there exists $\epsilon>0$ such that $g\in U$ for every $g\in L^p_K$ with $\nrm{f-g}_p<\epsilon$. By theorem 67 there is $s\in S_K$ with $\nrm{s-f}_p<\frac{\epsilon}{2}$. But by the above there also exists $\phi\in C^c_K$ with $\nrm{\phi-s}_p<\epsilon$ , whence by theorem 9-3 $\nrm{f-\phi}_p\leq \nrm{s-\phi}_p+\nrm{f-s}_p<\epsilon$. Since $C^c_K\subseteq L^p_K$, it must be true that $\phi\in U$, whence $U\cap C^c_K \neq \emptyset$. Therefore the closure of the space $C^c_K$ in $L^p_K$ is the whole $L^p_K$ space itself.\\

\label{thm:rn_compact_support_dense} \noindent\textbf{Theorem} 13-18.
Let $\Omega$ be an open subset of $\Real^n$, where $n\geq1$. Then for any locally finite measure $\mu$ on $\brac{\Omega,\borel{\Omega}}$ and $p\in \clop{1,+\infty}$ the space $C^c_K\brac{\Omega,\induc{\Tcal_{\Real^n}}_\Omega}$ is dense in $L^p_K\brac{\Omega,\borel{\Omega},\mu}$.

This follows from theorem 78 if it is shown that the topological subspace $\brac{\Omega,\induc{\Tcal_{\Real^n}}_\Omega}$ of $\Real^n$ is metrizable and strongly $\sigma$-compact. Furthermore the latter follows from theorem 76 if it is shown that $\brac{\Real^n,\Tcal_{\Real^n}}$ is metrizable and strongly $\sigma$-compact.

Repeating the argument in theorem 74, the space $\Real^n$ is metrizable by a natural Euclidean metric on $\Real^n$ and $\sigma$-compact. As for local compactness, let $x\in \Real^n$. Then the set $V\defn B\brac{x, 1}$ is open with the closure that is bounded in $\Real^n$ with respect to its usual metric. Thus by theorem 48 $\clo{V}$ is compact, whence the space $\Real^n$ is locally compact. Thus by theorem 75 $\Real^n$ is strongly $\sigma$-compact, whence by theorem 71 the topological subspace $\brac{\Omega, \Tcal}$ is metrizable and strongly $\sigma$-compact.\\


%%%%%%%%%% L^\infty attempt...
Now suppose $p=\infty$. By the properties of the $L^\infty$ norm 9-1 $\abs{g}\leq \nrm{g}_\infty$ $\mu$-a.s. Hence $\abs{f_n}\leq \abs{g}$ implies that $\abs{f_n}\leq \nrm{g}_\infty$ $\mu$-a.s., whence $\nrm{f_n}_\infty\leq \nrm{g}_\infty<+\infty$ for all $n\geq1$. Continuity of $\abs{\cdot}$ implies $\abs{f_n}\to \abs{f}$, whence $\abs{f}\leq \abs{g}$. Thus $\nrm{f}_\infty\leq \nrm{g}_\infty$. In conclusion $f,\brac{f_n}_{n\geq1}\in L^p_\Cplx$ for any $p\in \clo{1,+\infty}$.

% section tut_13 (end)

\section{Maps of Finite Variation} % (fold)
\label{sec:tut_14}
\url{http://probability.net/PRTfinitevar.pdf}

\noindent\textbf{Definition} 108.
A total variation of a map $b:\Real^+\to \Cplx$ is the map $\abs{b}:\Real^+\to \Zinf$ defined as: \[\abs{b}\brac{t}\defn \abs{b\brac{0}}+\sup \sum_{i=1}^n \abs{b\brac{t_i} - b\brac{t_{i-1}}}\] where the least upper bound is taken over all finite $t_0\leq t_1\leq\ldots\leq t_n$ in $\clo{0,t}$, $n\geq1$.

The map $b$ is of finite variation if and only if $\abs{b}\brac{t}<+\infty$ for all $t\in \Real^+$. The map $b$ is of bounded variation if and only if $\sup_{t\in \Real^+}\abs{b}\brac{t}<+\infty$.

Note, that the notation $\abs{b}$ can be misleading, since it can refer to the modulus map $t\to\abs{b\brac{t}}$ or to the total variation $t\to \abs{b}\brac{t}$.

\label{thm:non_decr_bounded}\noindent \textbf{Theorem} 14-1.
Let $a:\Real^+\to\Real^+$ be a non-decreasing map with $a\brac{0}\geq0$. Then $\abs{a}=a$ and $a$ is of finite variation. Also $a$ is of bounded variation if and only if $a\brac{\infty}<+\infty$.

Indeed, pick any $t\in \Real^+$ and let $\brac{t_i}_{i=0}^n\in \clo{0,t}$ with $t_{i-1}\leq t_i$ for all $i=1\ldots n$. Since $a$ is non-decreasing, $\abs{a\brac{t_i}-a\brac{t_{i-1}}} = a\brac{t_i}-a\brac{t_{i-1}}$, whence \[\sum_{i=1}^n \abs{a\brac{t_i}-a\brac{t_{i-1}}} = \sum_{i=1}^n a\brac{t_i}-a\brac{t_{i-1}} = a\brac{t_n}-a\brac{t_0}\leq a\brac{t}-a\brac{0}\] for all finite $t_0\leq\ldots\leq t_n$ in $\clo{0,t}$, $n\geq1$. However $t_0\defn 0$ and $t_1\defn t$ are such that $t_0\leq t_1$ and $t_0,t_1\in \clo{0,t}$, whence \[a\brac{t}-a\brac{0} = \abs{a\brac{t_1}-a\brac{t_0}}\leq \sup\sum_{i=1}^n \abs{a\brac{t_i}-a\brac{t_{i-1}}}\] where the least upper bound is taken over all finite $t_0\leq t_1\leq\ldots\leq t_n$ in $\clo{0,t}$, $n\geq1$. Therefore since $a\brac{0}\geq0$ \[\abs{a}\brac{t}=\abs{a\brac{0}}+ \sup\sum_{i=1}^n \abs{a\brac{t_i}-a\brac{t_{i-1}}}=a\brac{0}+a\brac{t}-a\brac{0}=a\brac{t}\], whence $\abs{a} = a$.

Since $\abs{a}=a$ and $a\brac{t}<+\infty$ for any $t\in \Real^+$, the map $a$ must be of finite variation.

Let $\alpha\defn \sup_{t\in \Real^+} a\brac{t}\in \Rbar$. Then for any $L\in \Real$ with $L<\alpha$ there is $T\in \Real^+$ such that $L<a\brac{T}\leq \alpha$. Since $a$ is non-decreasing for any $t\in \Real^+$ with $t\geq T$ it is true that $a\brac{T}\leq a\brac{t}\leq \alpha$. Therefore for any $L\in \Real$ with $L<\alpha$ there exists $T\in \Real^+$ such that $L<a\brac{t}\leq \alpha$ for all $t\geq T$. Therefore the limit $a\brac{\infty}\defn\lim_{t\to\infty} a\brac{t}$ exists in $\Rbar$ and is equal to $\alpha$.

For a non-decreasing map $a$ by definition \[a\brac{\infty}=\sup_{t\in \Real^+}a\brac{t}=\sup_{t\in \Real^+}\abs{a}\brac{t}\] Therefore $a$ is of bounded variation if and only if $a\brac{\infty}<+\infty$.\\

\label{thm:cplx_total_var_ineq}\noindent\textbf{Theorem} 14-2.
Let $b=b_1 + i b_2:\Real^+\to\Cplx$ be a map, $b_1$,$b_2$ real-valued. Then the total variations obey the following set of inequalities \[\abs{b_1},\abs{b_2}\leq \abs{b}\leq \abs{b_1}+\abs{b_2}\]

Since $\abs{\re\cdot},\abs{\im\cdot}\leq \abs{\cdot}$, for any $t\in \Real^+$ and for all $\brac{t_i}_{i=0}^n\in\clo{0,t}$ with $t_{i-1}\leq t_i$ it is true that \[\sum_{i=1}^n \abs{b_k\brac{t_i}-b_k\brac{t_{i-1}}}\leq \sum_{i=1}^n \abs{b\brac{t_i}-b\brac{t_{i-1}}}\] for $k=1,2$, because $\re b = b_1$ and $\im b = b_2$. Therefore \[\abs{b_k}\brac{t}=\abs{b_k\brac{0}}+\sup\sum_{i=1}^n \abs{b_k\brac{t_i}-b_k\brac{t_{i-1}}}\leq \abs{b\brac{0}}+\sup\sum_{i=1}^n \abs{b\brac{t_i}-b\brac{t_{i-1}}}=\abs{b}\brac{t}\] by definition of the least upper bound, whence $\abs{b_k}\leq \abs{b}$ for $k=1,2$.

Now, $\abs{\cdot}\leq \abs{\re\cdot}+\abs{\im\cdot}$ implies that for any $t\in \Real^+$ and all $\brac{t_i}_{i=0}^n\in\clo{0,t}$ with $t_{i-1}\leq t_i$ it is true that \begin{align*}\sum_{i=1}^n \abs{b\brac{t_i}-b\brac{t_{i-1}}}&\leq \sum_{i=1}^n \abs{b_1\brac{t_i}-b_1\brac{t_{i-1}}} + \sum_{i=1}^n \abs{b_2\brac{t_i}-b_2\brac{t_{i-1}}} \\&\leq \sup\sum_{i=1}^n \abs{b_1\brac{t_i}-b_1\brac{t_{i-1}}} + \sup\sum_{i=1}^n \abs{b_2\brac{t_i}-b_2\brac{t_{i-1}}} \end{align*} Therefore by definition of the least upper bound $\abs{b}\leq \abs{b_1}+\abs{b_2}$.\\

\label{thm:cplx_total_var}\noindent\textbf{Theorem} 14-3.
Let $b=b_1 + i b_2:\Real^+\to\Cplx$ be a map, $b_1$,$b_2$ real-valued. Then $b$ is of finite (bounded) variation if and only if $b_1$ and $b_2$ are of finite (bounded) variation. Furthermore $\abs{b}\brac{0}=\abs{b\brac{0}}$.

Indeed, if $\abs{b}\brac{t}<+\infty$ for any $t\in \Real^+$, then by theorem 14-2 $\abs{b_1}\brac{t},\abs{b_2}\brac{t}\leq \abs{b}\brac{t}<+\infty$. If $\abs{b_1}\brac{t},\abs{b_2}\brac{t}<+\infty$, then by the same theorem 14-2 $\abs{b}\brac{t}\leq \abs{b_1}\brac{t}+\abs{b_2}\brac{t}<+\infty$. Thus $b$ is of finite variation if and only if $b_1$ and $b_2$ are of finite variation.

Inequalities of theorem 14-2 imply that \[\sup_{t\in\Real^+}\abs{b_1}\brac{t},\sup_{t\in\Real^+}\abs{b_2}\brac{t}\leq\sup_{t\in\Real^+}\abs{b}\brac{t}\leq \sup_{t\in\Real^+}\abs{b_1}\brac{t}+\sup_{t\in\Real^+}\abs{b_2}\brac{t}\], whence $b$ is of bounded variation if and only if $b_1$ and $b_2$ are of bounded variation.

Now, let $t=0\in \Real^+$. Since for any $\brac{t_i}_{i=0}^n\in\clo{0,t}$ with $t_{i-1}\leq t_i$ it is true that $t_i=0$, it must thus be true that $\sum_{i=1}^n\abs{b\brac{t_i}-b\brac{t_{i-1}}} = 0$, whence $\abs{b}\brac{0} = \abs{b\brac{0}}+0$.\\

\label{thm:tot_var_map_lin_comb}\noindent\textbf{Theorem} 14-4.
If $b_1,b_2:\Real^+\to\Cplx$ are maps of finite (bounded) variation and $\alpha\in\Cplx$, then $b_1+\alpha b_2$ is also a map of finite (bounded) variation.

Put $b\defn b_1+\alpha b_2$. Let $t\in \Real^+$ and pick any $\brac{t_i}_{i=0}^n\in\clo{0,t}$ with $t_{i-1}\leq t_i$. Then by the triangle inequality\begin{align*} \sum_{i=1}^n\abs{b\brac{t_i}-b\brac{t_{i-1}}} &= \sum_{i=1}^n\abs{\brac{b_1\brac{t_i}-b_1\brac{t_{i-1}}} + \alpha \brac{ b_2\brac{t_i}-b_2\brac{t_{i-1}}}}\\ &\leq \sum_{i=1}^n \abs{b_1\brac{t_i}-b_1\brac{t_{i-1}}} + \abs{\alpha} \abs{ b_2\brac{t_i}-b_2\brac{t_{i-1}}}\\ &\leq \sup\sum_{i=1}^n\abs{b_1\brac{t_i}-b_1\brac{t_{i-1}}} + \abs{\alpha} \sup\sum_{i=1}^n\abs{ b_2\brac{t_i}-b_2\brac{t_{i-1}}}\end{align*} where the least upper bounds are taken over all $\brac{t_i}_{i=0}^n\in\clo{0,t}$ with $t_{i-1}\leq t_i$. Therefore \begin{align*}\abs{b}\brac{t} &= \abs{b\brac{0}} + \sup\sum_{i=1}^n\abs{ b\brac{t_i}-b\brac{t_{i-1}}}\\&\leq \abs{b_1\brac{0}} + \sup\sum_{i=1}^n\abs{b_1\brac{t_i}-b_1\brac{t_{i-1}}} + \abs{\alpha} \brac{\abs{b_2\brac{0}}+\sup\sum_{i=1}^n\abs{ b_2\brac{t_i}-b_2\brac{t_{i-1}}}}\\& = \abs{b_1}\brac{t} + \abs{\alpha}\abs{b_2}\brac{t}\end{align*} Furthermore, $\sup_{t\in\Real^+} \abs{b}\brac{t}\leq \sup_{t\in\Real^+} \abs{b_1}\brac{t} + \abs{\alpha} \sup_{t\in\Real^+} \abs{b_2}\brac{t}$.

Therefore $b_1+\alpha b_2$ is of finite (bounded) variation if $b_1$ and $b_2$ are of finite (bounded) variation.\\

\label{thm:real_diff_finite_var}\noindent\textbf{Theorem} 14-5.
Let $b:\Real^+\to\Real$ be a differentiable map, such that its derivative $b'$ is bounded on every compact interval of $\Real^+$. Then $b$ is a map of finite variation.

Let $t\in \Real^+$. Then by theorem 35 the interval $\clo{0,t}$ is compact in $\Real$. Therefore there exists $M\in \Real^+$ such that $\abs{b'\brac{x}}\leq M$ for all $x\in \clo{0,t}$.

If $x_1<x_2\in \clo{0,t}$, then the facts $b\in C^0\brac{\clo{0,t}}$ and $b'$ exists on $\brac{0,t}$ imply that $b\in C^0\brac{\clo{x_1,x_2}}$ and $b'$ exists on $\brac{x_1,x_2}$. Thus by Taylor-Lagrange theorem (39) there is $c\in \brac{x_1,x_2}$ such that $b\brac{x_2}-b\brac{x_1} = \brac{x_2-x_1} b'\brac{c}$, whence $\abs{b\brac{x_2}-b\brac{x_1}}\leq M \brac{x_2-x_1}$.

For all $\brac{t_i}_{i=0}^n\in\clo{0,t}$ with $t_{i-1}\leq t_i$ it is true that \[\sum_{i=1}^n\abs{b\brac{t_i}-b\brac{t_{i-1}}}\leq \sum_{i=1}^n M\brac{t_i - t_{i-1}} = M \brac{t_n - t_0}\leq M t\] whence $\abs{b}\brac{t}\leq \abs{a\brac{0}} + M t < +\infty$. Therefore the map $b$ is of finite variation.\\

\label{thm:cplx_diff}\noindent\textbf{Theorem} 14-6.
The map $f:\Real\to\Cplx$ is differentiable if and only if $\re f,\im f$ are differentiable.

Indeed, suppose $f$ is differentiable and pick any $x\in \Real$. For any $\epsilon>0$ there exists $\delta>0$ such that for all $y\in\Real$ with $0<\abs{x-y}<\delta$ \[\abs{\frac{f\brac{x}-f\brac{y}}{x-y}-f'\brac{x}}<\epsilon\] Since $\abs{\re\cdot},\abs{\im\cdot}\leq \abs{\cdot}$ and $\re$ is linear, it must therefore be true that for all $y\in\Real$ with $0<\abs{x-y}<\delta$ \begin{align*}\abs{\frac{\re f\brac{x} - \re f\brac{y}}{x-y} - \re f'\brac{x}} &= \abs{\re\frac{f\brac{x}-f\brac{y}}{x-y} - \re f'\brac{x}}\\&= \abs{\re\brac{\frac{f\brac{x}-f\brac{y}}{x-y} - f'\brac{x}}}\\&\leq \abs{\frac{f\brac{x}-f\brac{y}}{x-y}-f\brac{x}}\\&<\epsilon\end{align*} Therefore $\re f$, and similarly $\im f$, is differentiable at $x$.

Conversely, if the real-valued maps $u,v:\Real\to\Real$ are differentiable, then $\abs{\cdot}\leq \abs{\re\cdot}+\abs{\im\cdot}$ implies that the map $f:\Real\to\Cplx$ defined as $f\defn u+iv$ is differentiable.\\

\label{thm:cplx_cont_diff_fin_var}\noindent\textbf{Theorem} 14-7.
Let $b:\Real^+\to\Cplx$ be a map of class $C^1\brac{\Real^+}$, i.e. continuous and differentiable with continuous derivative. Then $b$ is a map of finite variation.

By theorem 14-6 the maps $\re f$ and $\im f$ are differentiable with derivatives equal to $\re f'$ and $\im f'$ respectively. Thus $\re f$ and $\im f$ have continuous derivatives as well.

For any compact interval of $\Real^+$ by theorem 37 the maps $\re f'$ and $\im f'$ attain their maximal and minimal values, whence are bounded. Thus by theorem 14-5 the maps $\re f$ and $\im f$ are of finite variation, whence by theorem 14-3 the map $f=\re f + i \im f$ is of finite variation.\\

\label{thm:leb_integr_total_var}\noindent\textbf{Theorem} 14-8.
Let $f:\brac{\Real^+,\borel{\Real^+}}\to\brac{\Cplx,\borel{\Cplx}}$ be a measurable map with $\int_0^t \abs{f\brac{s}}ds <+\infty$ for all $t\in \Real^+$. Let $b:\Real^+\to\Cplx$ be defined as\[b\brac{t}\defn \int_{\Real^+} f 1_{\clo{0,t}} ds\] Then the map $b$ is of finite variation. If $f\in L^1_\Cplx\brac{\Real^+,\borel{\Real^+},ds}$ then $b$ is of bounded variation.

The Complex Lebesgue integral is defined over the measure space $\brac{\Real^+,\borel{\Real^+},ds}$, where $ds$ is the Lebesgue measure as in definition 20.

Let $t\in \Real^+$. By theorem 24, linearity of the usual Lebesgue integral (theorem 5-7) and theorem 5-5 for any $\brac{t_i}_{i=0}^n\in\clo{0,t}$ with $t_{i-1}\leq t_i$ it is true that \begin{align*}\sum_{i=1}^n \abs{b\brac{t_i}-b\brac{t_{i-1}}} &\leq \sum_{i=1}^n \int_{\Real^+} \abs{f\brac{s}} \brac{1_{\clo{0,t_i}}\brac{s} - 1_{\clo{0,t_{i-1}}}} ds \\ &= \int_{\Real^+} \abs{f\brac{s}} \brac{1_{\clo{0,t_n}} - 1_{\clo{0,t_0}}} ds \\ &= \int_{\Real^+} \abs{f\brac{s}} 1_{\ploc{t_0,t_n}} ds \\ &\leq \int_{\Real^+} \abs{f\brac{s}} 1_{\clo{0,t}} ds \\ &= \int_0^t \abs{f\brac{s}} ds\end{align*} Now, since $ds\brac{\obj{0}}=0$ it must be true that $\abs{b\brac{0}}\leq \int_{\Real^+} \abs{f\brac{s}} 1_{\obj{0}} ds = 0$, whence $\abs{b}\brac{t}\leq \int_0^t \abs{f\brac{s}} ds<+\infty$. Therefore the map $b$ is of finite variation.

If $f\in L^1_\Cplx\brac{\Real^+,\borel{\Real^+},ds}$ then $\int_{\Real^+} \abs{f\brac{s}}ds <+\infty$, which implies that $\sup_{t\in \Real^+}\abs{b}\brac{t}<+\infty$ since for any $t\in \Real^+$\[\int_0^t\abs{f\brac{s}}ds = \int_{\Real^+}\abs{f\brac{s}} 1_{\clo{0,t}}ds \leq \int_{\Real^+}\abs{f\brac{s}}ds\] Therefore in this case $b$ is of bounded variation.\\

\label{thm:tot_var_sum_prop}\noindent\textbf{Theorem} 14-9.
Let $b:\Real^+\to\Cplx$ be a map. Let $\brac{t_i}_{i=0}^n\in\clo{0,t}$ with $t_{i-1}\leq t_i$ and $\brac{s_i}_{i=0}^p\in\clo{0,t}$ with $s_{i-1}\leq s_i$ be such that $\obj{t_0,\ldots, t_n}\subseteq \obj{s_0,\ldots, s_n}$. Then \[\sum_{i=1}^n\abs{b\brac{t_i}-b\brac{t_{i-1}}}\leq \sum_{j=1}^p\abs{b\brac{s_j}-b\brac{s_{j-1}}}\]

For all $t\in \Real^+$ let $\Scal\brac{t}$ be the sets of all finite subsets of $\clo{0,t}$ with at least two elements. For all $A\in \Scal\brac{t}$ define \[S\brac{A}\defn \sum_{i=1}^n \abs{b\brac{t_i} - b\brac{t_{i-1}}}\] where $\brac{t_i}_{i=1}^n\in\clo{0,t}$ are such that $t_{i-1}< t_i$ for $i=1\ldots n$ and $A\defn \obj{\induc{t_i}\,i=0\ldots n}\subseteq \clo{0,t}$.

Suppose $t\in \Real^+$ and pick any $\brac{s_i}_{i=0}^p\in\clo{0,t}$ with $s_{i-1}\leq s_i$. Let $S\defn \sum_{i=1}^p \abs{b\brac{s_i}-b\brac{s_{i-1}}}$. If $s_{i-1}<s_i$ for all $i=1\ldots p$, then $A\defn \obj{s_0,\ldots,s_p}$ is such that $A\in \Scal\brac{t}$ since $A\subseteq \clo{0,t}$. Thus $S=S\brac{A}$.
If there exists at least one $i=1\ldots p$ such that $s_{i-1}=s_i$ and at least one $j=1\ldots p$ with $s_{j-1}\neq s_j$ then removing all $I\defn \obj{\induc{i=1\ldots p}\,s_{i-1}=s_i}$ does not change the value of $S$, but results in a finite subset of $\clo{0,t}$. Thus there exists $A\in \Scal\brac{t}$ such that $S=S\brac{A}$. If $s_{i-1}=s_i$ for all $i=1\ldots p$, then $S=0$.

Therefore \[\sup \sum_{i=1}^n \abs{b\brac{t_i}-b\brac{t_{i-1}}}\leq \sup\obj{\induc{S\brac{A}}\,A\in \Scal\brac{t}}\] where the first least upper bound is taken over all $\brac{t_i}_{i=0}^p\in\clo{0,t}$ with $t_{i-1}\leq t_i$. Conversely, since every $A\in \Scal\brac{t}$ is such that $A=\obj{\induc{t_i}\,i=0\ldots n}$ with $t_{i-1}\leq t_i$ it must be true that \[\sup\obj{\induc{S\brac{A}}\,A\in \Scal\brac{t}}\leq \sup \sum_{i=1}^n \abs{b\brac{t_i}-b\brac{t_{i-1}}}\] where the second least upper bound is taken over all $\brac{t_i}_{i=0}^p\in\clo{0,t}$ with $t_{i-1}\leq t_i$. Thus for all $t\in\Real^+$ it is true that $\abs{b}\brac{t} = \abs{b\brac{0}}+\sup\sup\obj{\induc{S\brac{A}}\,A\in \Scal\brac{t}}$.

Suppose $A\in\Scal\brac{t}$ and $s\in\clo{0,t}$. By definition $A=\obj{\induc{t_i}\,i=0\ldots n}$ with $t_{i-1}<t_i$ for some $n\geq1$. If $s\in S$ then $A\cup \obj{s} = A$, whence $S\brac{A} = S\brac{A\cup \obj{s}}$. However, if $s\notin A$ then there exists $j=1\ldots n$ with $t_{j-1}<s<t_j$, whence $A\cup \obj{s}\in \Scal\brac{t}$ and \begin{align*}S\brac{A} &= \sum_{i\neq j} \abs{b\brac{t_i}-b\brac{t_{i-1}}} + \abs{b\brac{t_j} - s + s - b\brac{t_{j-1}}}\\&\leq \sum_{i\neq j} \abs{b\brac{t_i}-b\brac{t_{i-1}}} + \abs{b\brac{t_j} - s} + \abs{s - b\brac{t_{j-1}}}\\&=S\brac{A\cup \obj{S}}\end{align*}

Now let $A,B\in \Scal\brac{t}$ with $A\subseteq B$ and let $C\defn B\setminus A$. If $C=\emptyset$ then $S\brac{A}=S\brac{B}$. Otherwise let $C=\obj{\induc{s_i}\,i=1\ldots p}$ and put $A_i\defn A_{i-1}\cup\obj{s_i}$ where $A_0\defn A$. By the above $S\brac{A_{i-1}}\leq S\brac{A_i}$ for all $i=1\dots p$, whence $S\brac{A}=S\brac{A_0}\leq S\brac{A_p}=S\brac{B}$.

Let $\brac{t_i}_{i=0}^n\in\clo{0,t}$ with $t_{i-1}\leq t_i$ and $\brac{s_i}_{i=0}^p\in\clo{0,t}$ with $s_{i-1}\leq s_i$ be such that $\obj{t_0,\ldots, t_n}\subseteq \obj{s_0,\ldots, s_n}$. Then for every $i=0\ldots n$ there exists $j=0\ldots p$ with $t_i=s_j$. By the above there exist $A,B\in \Scal\brac{t}$ with $A\subseteq \obj{t_0,\ldots, t_n}$ and $B\subseteq \obj{s_0,\ldots, s_p}$. If $A=\emptyset$ then \[0=S\brac{A}\leq S\brac{b} = \sum_{j=1}^p\abs{b\brac{s_j}-b\brac{s_{j-1}}}\] If $x\in A$, then $x=t_i$ for some $i=0\ldots n$, whence there exists $j=0\ldots p$ with $x=s_j$. Now, $B$ contains only distinct $s_j$, which means that $x$ must have gotten into $B$ during the removal of duplicates. Thus $x\in B$ and $S\brac{A}\leq S\brac{B}$. Therefore \[\sum_{i=1}^n\abs{b\brac{t_i}-b\brac{t_{i-1}}}\leq \sum_{j=1}^p\abs{b\brac{s_j}-b\brac{s_{j-1}}}\]

\label{thm:finite_var_diff}\noindent\textbf{Theorem} 80.
Let $b:\Real^+\to\Cplx$ be of finite variation. Then for all $s,t\in \Real^+$ with $s\leq t$ it is true that \[\abs{b}\brac{t}-\abs{b}\brac{s} = \sup \sum_{i=1}^n \abs{b\brac{t_i}-b\brac{t_{i-1}}}\] where the least upper bound is taken over $\brac{t_i}_{i=0}^n\in\clo{s,t}$ with $t_{i-1}\leq t_i$, $n\geq1$.

Indeed, let $s,t\in \Real^+$ with $s\leq t$ and put \[\delta \defn \sup \sum_{i=1}^n \abs{b\brac{t_i}-b\brac{t_{i-1}}}\] where the least upper bound is taken over $\brac{t_i}_{i=0}^n\in\clo{s,t}$ with $t_{i-1}\leq t_i$, $n\geq1$.

Let $\brac{s_i}_{i=0}^n\in\clo{0,s}$ with $s_{i-1}\leq s_i$ and $\brac{t_i}_{i=0}^n\in\clo{s,t}$ with $t_{i-1}\leq t_i$ where $n,p\geq1$. Then $x_i\defn s_i$ for $i=0\ldots p$ and $x_i\defn t_{i-n}$ for $i={n+1}\ldots {n+p}$ it is true that $\brac{x_i}_{i=0}^{n+p}\in\clo{0,t}$ with $x_{i-1}\leq x_i$, whence \begin{align*}\sum_{j=1}^p\abs{b\brac{s_j}-b\brac{s_{j-1}}} + \sum_{i=1}^p\abs{b\brac{t_i}-b\brac{t_{i-1}}} &= \sum_{k=1}^{n+p}\abs{b\brac{x_k}-b\brac{x_{k-1}}}\\&\leq \sup\sum_{k=1}^m\abs{b\brac{y_k}-b\brac{y_{k-1}}}\\&=\abs{b}\brac{t} - \abs{b\brac{0}}\end{align*} with the least upper bound taken over $\brac{y_i}_{i=0}^m\in\clo{s,t}$ with $y_{i-1}\leq y_i$, $n\geq1$. Therefore for a fixed $\brac{t_i}_{i=0}^n\in\clo{s,t}$ it must be true that \[\abs{b}\brac{s}+\sum_{i=1}^p\abs{b\brac{t_i}-b\brac{t_{i-1}}}\leq \abs{b}\brac{t}\] Since $b$ is of finite variation, $\abs{b}\brac{s}<+\infty$, whence by definition of the least upper bound \[\delta \leq \abs{b}\brac{t}-\abs{b}\brac{s}\]

Let $\brac{t_i}_{i=0}^n\in\clo{0,t}$ with $t_{i-1}\leq t_i$ where $n\geq1$ and suppose that for some $j=1\ldots{n-1}$ it is true that $s=t_j$. Then \begin{align*}\sum_{i=1}^n\abs{b\brac{t_i}-b\brac{t_{i-1}}} &= \sum_{i=1}^j \abs{b\brac{t_i}-b\brac{t_{i-1}}}+\sum_{i=j+1}^n \abs{b\brac{t_i}-b\brac{t_{i-1}}}\\&\leq \abs{b}\brac{s} - \abs{b\brac{0}} + \sup \sum_{k=1}^m \abs{b\brac{x_k}-b\brac{x_{k-1}}}\\&= \abs{b}\brac{s} - \abs{b\brac{0}} + \delta\end{align*} where the least upper bound is taken over $\brac{x_k}_{k=0}^m\in\clo{s,t}$ with $x_{k-1}\leq x_k$, $m\geq1$.

Suppose $\brac{t_i}_{i=0}^n\in\clo{0,t}$ with $t_{i-1}\leq t_i$ where $n\geq1$ is such that $s\neq t_j$ for all $j=0 \ldots n$. Then place $s$ in the correct order within the finite sequence $\brac{t_i}_{i=0}^n$ (either between, in front or at the back) to get $\brac{s_j}_{j=0}^p$. Thus $\obj{t_0,\ldots,t_n}\subseteq \obj{s_0,\ldots,s_m}$, whence by theorem 14-9 \begin{align*}\sum_{i=1}^n\abs{b\brac{t_i}-b\brac{t_{i-1}}}&\leq \sum_{j=1}^p\abs{b\brac{s_j}-b\brac{s_{j-1}}}\\&\leq \abs{b}\brac{s} - \abs{b\brac{0}} + \delta\end{align*} Therefore by definition of the least upper bound it is true that $\abs{b}\brac{t}\leq \abs{b}\brac{s}+\delta$, whence $\delta = \abs{b}\brac{t}-\abs{b}\brac{s}$.\\

\label{thm:finte_var_prop}\noindent\textbf{Theorem} 14-10.
Let $b:\Real^+\to\Cplx$ be a map of finite variation. Then $\abs{b}\brac{\cdot}$ is non-decreasing with $\abs{b}\brac{0}\geq 0$ and $\abs{\abs{b}} = \abs{b}$.

Indeed, by theorem 80 the total variation of $b$ must be a non-decreasing map and by theorem 14-3 $\abs{b}\brac{0} = \abs{b\brac{0}}\leq 0$. Therefore by theorem 14-1 $\abs{\abs{b}} = \abs{b}$ and $\abs{b}$ is itself of finite variation.\\

\noindent\textbf{Definition} 109.
Let $b:\Real^+\to\Real$ be a map of finite variation. Let $\abs{b}^+\defn \frac{1}{2}\brac{\abs{b}+b}$ and $\abs{b}^-\defn \frac{1}{2}\brac{\abs{b}-b}$. Then $\abs{b}^+$ and $\abs{b}^-$ are respectively the positive and negative variation of $b$.

\label{thm:pos_neg_var_decomp}\noindent\textbf{Theorem} 14-11.
Let $b:\Real^+\to\Real$ be a map of finite variation. Then the positive and negative variations of $b$ are non-decreasing non-negative maps.

Since $\abs{b}\brac{t}\in \Real^+$ and $b\brac{t}\in \Real$ for all $t\in\Real^+$, simple arithmetic yields $\abs{b}\brac{t} = \abs{b}^+\brac{t}+\abs{b}^-\brac{t}$ and $b\brac{t}=\abs{b}^+\brac{t}-\abs{b}^-\brac{t}$.

By theorem 14-10 $\abs{b}\brac{0}=\abs{b\brac{0}}$, which means that $\abs{b}^+\brac{0} = \frac{1}{2}\brac{\abs{b\brac{0}}+b\brac{0}} = b^+\brac{0}$ and $\abs{b}^-\brac{0} = \frac{1}{2}\brac{\abs{b\brac{0}}-b\brac{0}} = b^-\brac{0}$. Thus $\abs{b}^+\brac{0},\abs{b}^-\brac{0}\geq 0$.

Now for any $s,t\in \Real^+$ with $s\leq t$ theorem 80 implies that $\abs{b\brac{t}-b\brac{s}}\leq \abs{b}\brac{t}-\abs{b}\brac{s}$.

Let $s,t\in \Real^+$ be such that $s\leq t$. Then \begin{align*}\abs{b}^+\brac{t}-\abs{b}^+\brac{s} &= \frac{1}{2}\brac{\abs{b}\brac{t}-\abs{b}\brac{s}+b\brac{t}-b\brac{s}}\\&\geq \frac{1}{2}\brac{\abs{b\brac{t}-b\brac{s}}+\brac{b\brac{t}-b\brac{s}}}\\&=\brac{b\brac{t}-b\brac{s}}^+\end{align*} which implies that $\abs{b}^+$ is non-decreasing. Similarly the negative variation is non-decreasing. Note that by theorem 14-1 each $b_i$ is of finite variation.\\

\label{thm:cplx_tot_var_decomp}\noindent\textbf{Theorem} 14-12.
Let $b:\Real^+\to\Cplx$ be a map. Then $b$ is of finite variation if and only if there exist non-decreasing maps $b_1,b_2,b_3,b_4:\Real^+\to \Real^+$ with $b_i\brac{0}\geq 0$ such that $b=b_1-b_2+i\brac{b_3-b_4}$.

Indeed, let $u\defn \re b$ and $v\defn \im b$. Then by theorem 14-3 the maps $u,v:\Real^+\to\Real$ are of finite variation, whence by theorem 14-11 there exist non-decreasing maps $b_1,b_2,b_3,b_4:\Real^+\to \Real^+$ with $b_i\brac{0}\geq 0$ such that $u=b_1-b_2$ and $v=b_3-b_4$. This implies that $b=b_1-b_2+i\brac{b_3-b_4}$.

Suppose there exist non-decreasing maps $b_1,b_2,b_3,b_4:\Real^+\to \Real^+$ with $b_i\brac{0}\geq 0$ such that $b=b_1-b_2+i\brac{b_3-b_4}$. First, by theorem 14-1 each $b_i$ is of finite variation. By theorem 14-4 the maps $b_1-b_2$ and $b_3-b_4$ are of finite variation, whence by theorem 14-3 the map $b$ must be of finite variation.\\

\label{thm:tot_var_right_cont}\noindent\textbf{Theorem} 14-13.
Let $b:\Real^+\to\Cplx$ be a map of finite variation. If $b$ is right-continuous then $\abs{b}\brac{\cdot}$ is right-continuous as well.

Let $x_0\in \Real^+$. The fact that $b$ is right-continuous at $x_0$ means that for every $\epsilon>0$ there exits $\delta>0$ such that $\abs{f\brac{x}-f\brac{x_0}}<\epsilon$ for all $x\in\brac{x_0,x_0+\delta}$.

Let $\alpha\defn \inf_{t>x_0} \abs{b}\brac{t}$. First of all $\alpha\in \Real$ because $0\leq \abs{b}\brac{t}<+\infty$ for all $t\in \Real^+$. Thus for any $\epsilon>0$ the number $\alpha+\epsilon$ cannot be the greatest lower bound of $\obj{\induc{\abs{b}\brac{t}}\,t>x_0}$, whence there exists $T>x_0$ such that $\alpha\leq \abs{b}\brac{T}<\alpha+\epsilon$. Since by theorem 14-10 the total variation of a finite variation map is non-decreasing, for all $t\in\brac{x_0,T}$ it must be true that \[\alpha\leq \abs{b}\brac{t}\leq \abs{b}\brac{T}<\alpha+\epsilon\] Therefore the limit $\abs{b}\brac{x_0+}\defn \lim_{t\downarrow x_0} \abs{b}\brac{t}$ exists in $\Real$ and is equal to $\inf_{t>x_0} \abs{b}\brac{t}$.

Now, $\abs{b}\brac{x_0}\leq \abs{b}\brac{t}$ for all $t>x_0$, which means that $\abs{b}\brac{x_0}$ is a lower bound for $\obj{\induc{\abs{b}\brac{t}}\,t>x_0}$, whence $\abs{b}\brac{x_0}\leq \abs{b}\brac{x_0+}$.

Since $\abs{b}\brac{x_0+} = \inf_{y>x_0}\abs{b}\brac{y}$ and $\abs{b}$ is non-decreasing, for any $\epsilon>0$ there exists $y_1>x_0$ such that $0\leq \abs{b}\brac{u}-\abs{b}\brac{x_0+}<\frac{\epsilon}{4}$ for all $u\in \ploc{x_0,y_0}$. Thus for all $u\in \ploc{x_0,y_1}$ \[\abs{b}\brac{y_1}-\abs{b}\brac{u} = \brac{\abs{b}\brac{y_1}-\abs{b}\brac{x_0+}}+\brac{\abs{b}\brac{x_0+}-\abs{b}\brac{u}}< \frac{\epsilon}{2}\]

By right-continuity of $b$ there exists $\delta>0$ and $y_2\in \brac{x_0,x_0+\delta}$ such that $\abs{b\brac{u}-b\brac{x_0}}\leq\frac{\epsilon}{2}$ for all $u\in\ploc{x_0,y_2}$. Therefore for $y_0\defn\min\obj{y_1,y_2}$ it must be true that for all $u\in \ploc{x_0,y_0}$ \begin{align*}\abs{b\brac{u}-b\brac{x_0}}&\leq\frac{\epsilon}{2}\\\abs{b}\brac{y_0}-\abs{b}\brac{u}&\leq\abs{b}\brac{y_1}-\abs{b}\brac{u}\\&<\frac{\epsilon}{2}\end{align*}

Now, pick any $\brac{t_i}_{i=0}^n\in\clo{0,y_0}$ with $t_{i-1}\leq t_i$ such that there exists $j=1\ldots{n-1}$ with $x_0=t_j$ and $x_0<t_{j+1}$. By theorems 80 and 14-10 it must be true that \[\sum_{i=1}^j\abs{b\brac{t_i}-b\brac{t_{i-1}}}\leq \abs{b}\brac{x_0}-\abs{b}\brac{0}=\abs{b}\brac{x_0}-\abs{b\brac{0}}\] Since $x_0=t_j<t_{j+1}\leq y_0$ and $y_0$ was specifically chosen, it must be true that $\abs{b\brac{t_{j+1}}-b\brac{t_j}}<\frac{\epsilon}{2}$. Now, since $\brac{t_i}_{i=j+1}^n\in\clo{t_{j+1},y_0}$ by theorem 80 it is true that\[\sum_{i=j+2}^n\abs{b\brac{t_i}-b\brac{t_{i-1}}}\leq \abs{b}\brac{y_0}-\abs{b}\brac{t_{j+1}}<\frac{\epsilon}{2}\] where the last inequality follows from $t_{j+1}\in\ploc{x_0, y_0}$ and the properties of $\abs{b}$ given the choice of $y_0$. Therefore for this specific anchored finite sequence $\brac{t_i}_{i=0}^n\in\clo{0,y_0}$ it is true that \[\sum_{i=1}^n\abs{b\brac{t_i}-b\brac{t_{i+1}}}\leq \abs{b}\brac{x_0}-\abs{b\brac{0}} + \epsilon\]

Now, let $\brac{t_i}_{i=0}^n\in \clo{0,y_0}$ be any arbitrary finite sequence with $t_i\leq t_{i+1}$. If there is no $j=0\ldots {n-1}$ with $x_0=t_j<t_{j+1}$ then inserting $x_0$ at the right place (either at the beginning, between two consecutive points, or after the last) by theorem 14-9 since the new finite sequence is larger it must still be true that \[\sum_{i=1}^n\abs{b\brac{t_i}-b\brac{t_{i+1}}}\leq \abs{b}\brac{x_0}-\abs{b\brac{0}} + \epsilon\] Thus for this fixed $\epsilon>0$ there exists $y_0>x_0$ such that $\abs{b}\brac{y_0}\leq \abs{b}\brac{x_0}+\epsilon$. Hence $\abs{b}\brac{x_0+}\leq \abs{b}\brac{x_0}$ since $\abs{b}\brac{x_0+}\leq \abs{b}\brac{y_0}$. Therefore the map $\abs{b}\brac{\cdot}$ is right-continuous because $\abs{b}\brac{x_0}=\abs{b}\brac{x_0+}$ for any $x_0\in \Real^+$.\\

\label{thm:tot_var_left_cont}\noindent\textbf{Theorem} 14-13.
Let $b:\Real^+\to\Cplx$ be a map of finite variation. If $b$ is left-continuous then $\abs{b}\brac{\cdot}$ is left-continuous as well.

Indeed, suppose $b$ is left-continuous map of finite variation. Let $x_o\in \Real^+\setminus\obj{0}$. Consider $\alpha\defn \sup_{t<x_0} \abs{b}\brac{t}$. First, since $\abs{b}\brac{x_0}<+\infty$ and $0\leq \abs{b}\brac{t}\leq \abs{b}\brac{x_0}$ by theorem 14-10, it must be true that $\alpha\in \Real$. Now, for any $\epsilon>0$ there must be $T\in \Real^+$ with $T\in \clop{0,x_0}$ such that $\alpha-\epsilon<\abs{b}\brac{T}\leq \alpha$. However $\abs{b}\brac{T}\leq \abs{b}\brac{t}\leq \alpha$ for all $t\in \clop{T,x_0}$. Thus the limit $\abs{b}\brac{x_0-}\defn \lim_{t\to x_0-}\abs{b}\brac{t}$ exists in $\Real$ and is equal to $\sup_{t<x_0} \abs{b}\brac{t}$.

Since $\abs{b}\brac{t}\leq \abs{b}\brac{x_0}$ for all $t<x_0$, it must be true that $\abs{b}\brac{x_0-}\leq \abs{b}\brac{x_0}$.

Let $\epsilon>0$. Then there exists $T_1\in\clop{0,x_0}$ such that $\abs{b}\brac{x_0-}-\abs{b}\brac{t}<\frac{\epsilon}{4}$ for all $t\in \clop{T_1,x_0}$. Therefore for all $u\in\clop{T_1,x_0}$ it must be true that \[\abs{b}\brac{t}-\abs{b}\brac{T_1} = \abs{b}\brac{t}-\abs{b}\brac{x_0-} - \brac{\abs{b}\brac{T_1}-\abs{b}\brac{x_0-}}< \frac{\epsilon}{2}\] By left-continuity of $b$ there must be $T_2\in \clop{0,x_0}$ such that $\abs{b\brac{x_0}-b\brac{t}}<\frac{\epsilon}{2}$ for all $t\in \clop{T_2,x_0}$. Hence for $y_0\defn \max{T_1,T_2}$ it must be true that for all $u\in\clop{y_0,x_0}$ \begin{align*}\abs{b\brac{x_0}-b\brac{u}}&<\frac{\epsilon}{2}\\\abs{b}\brac{u}-\abs{b}\brac{y_0}&<\frac{\epsilon}{2}\\\end{align*}

Now, pick any $\brac{t_i}_{i=0}^n\in\clo{0,x_0}$ with $t_{i-1}\leq t_i$ such that there exists $j=1\ldots{n-1}$ with $y_0=t_j$ and $x_0=t_n$. Let \[k\defn\max\obj{\induc{i=j\ldots n}\,t_i<x_0}\]

Obviously $k\leq {n-1}$ and $t_k\in \clop{y_0,x_0}$, because $x_0=t_n$ and $t_j=y_0<x_0$. by theorem 80 and 14-10 it must be true that \[\sum_{i=1}^j\abs{b\brac{t_i}-b\brac{t_{i-1}}}\leq \abs{b}\brac{y_0}-\abs{b}\brac{0}=\abs{b}\brac{y_0}-\abs{b\brac{0}}\] If $j<k$ then by the same theorem 80 it is true that \[\sum_{i=j+1}^k \abs{b\brac{t_i}-b\brac{t_{i-1}}}\leq \abs{b}\brac{t_k}-\abs{b}\brac{y_0}<\frac{\epsilon}{2}\] since $t_k\in \clop{y_0,x_0}$. Finally, since $t_{k+1}\not{<}x_0$ yet $t_{k+1}\leq x_0$ for all $i={k+1}\ldots n$ it is true that $t_i=x_0$, whence $\sum_{i=k+1}^n\abs{b\brac{t_i}-b\brac{t_{i-1}}}=\abs{b\brac{t_{k+1}}-b\brac{t_k}}$. Thus \[\sum_{i=k+1}^n\abs{b\brac{t_i}-b\brac{t_{i-1}}}<\frac{\epsilon}{2}\] since again $t_k\in \clop{y_0,x_0}$. Therefore \[\sum_{i=1}^n\abs{b\brac{t_i}-b\brac{t_{i-1}}}<\abs{b}\brac{y_0}-\abs{b\brac{0}} + \epsilon\]

Now for every $\brac{t_i}_{i=0}^n\in\clo{0,x_0}$ with $t_{i-1}\leq t_i$, by theorem 14-9 refining the ``sample'' cannot decrease the partial variation $\sum_{i=1}^n \abs{b\brac{t_i}-b\brac{t_{i-1}}}$. Thus the new ``sample'', obtained by adding the points $x_0$ and $y_0$ to $\brac{t_i}_{i=0}^n$, despite having not lesser partial variation, is also itself strictly less than $\abs{b}\brac{y_0}-\abs{b\brac{0}}+\epsilon$. Hence for all $\epsilon>0$ there exists  $y_0\in\clop{0,x_0}$ such that \[\abs{b}\brac{x_0}\leq \abs{b}\brac{y_0}+\epsilon\leq \abs{b}\brac{x_0-}+\epsilon\] whence $\abs{b}\brac{x_0}\leq \abs{b}\brac{x_0-}$. Therefore the total variation of a left-continuous map is itself continuous.\\

\label{thm:finite_tot_var_cont}\noindent\textbf{Theorem} 81.
Let $b:\Real^+\to\Cplx$ be a map of finite variation. If $b$ is continuous then $\abs{b}\brac{\cdot}$ is continuous as well.

Since $b$ is continuous, at every $x_0\in \Real^+$ it must be both left- and right- continuous. Thus by theorems 14-12 and 14-13 the total variation of $b$, the map $\abs{b}\brac{\cdot}$ is similarly both left- and right- continuous respectively. Thus $\abs{b}\brac{\cdot}$ is a $\Real^+$-$\Real$ continuous map.\\

\label{thm:tot_var_cont_real_decomp}\noindent\textbf{Theorem} 14-14.
Let $b:\Real^+\to\Real$ be a map of finite variation. The kind of continuity of $b$ is inherited by both $\abs{b}^+$ and $\abs{b}^-$.

Indeed, if $b$ is right-continuous, then by theorem 81 $\abs{b}\brac{\cdot}$ is right-continuous too. By definition 109 and theorem 14-11 the positive ($\abs{b}^+$) and the negative ($\abs{b}^-$) variations of $b$, being a linear combination of right-continuous maps, are themselves right-continuous. Similar result holds for left-continuity, and hence for continuity.\\

\label{thm:cplx_tot_var_cont_cplx_decomp}\noindent\textbf{Theorem} 14-15.
Let $b:\Real^+\to\Cplx$ be a right-continuous map of finite variation. Then there exist right-continuous and non-decreasing maps $b_1,b_2,b_3,b_4:\Real^+\to\Real^+$ with $b_i\brac{0}\geq 0$ such that \[b=b_1-b_2 + i\brac{b_3-b_4}\]

Indeed, by theorem 14-3 $\re b$ and $\im b$ are maps of finite variation, which in addition are right-continuous since $\abs{\re \cdot},\abs{\im \cdot}\leq \abs{\cdot}$. By theorem 14-14 there exist non-decreasing right-continuous maps $b_1,b_2,b_3,b_4:\Real^+\to \Real^+$ with $b_i\brac{0}\geq 0$ such that $\re b=b_1-b_2$ and $\im =b_3-b_4$. Thus \[b=\re b + i \im b = b_1-b_2+i\brac{b_3-b_4}\]

\label{thm:tot_var_dense_calc}\noindent\textbf{Theorem} 82.
If $b$ is right- or left-continuous, then for all $t\in\Real^+$ it is true that \[\abs{b}\brac{t}=\abs{b\brac{0}} + \lim_{p\to\infty} \sum_{k=1}^{2^p}\abs{b\brac{\frac{k t}{2^p}}-b\brac{\frac{\brac{k-1} t}{2^p}}}\]

Indeed, let $t\in \Real^+$ and for all $p\geq 1$ define \[S_p\defn \abs{b\brac{0}} + \sum_{k=1}^{2^p}\abs{b\brac{\frac{k t}{2^p}}-b\brac{\frac{\brac{k-1} t}{2^p}}}\] then $\abs{b}\brac{t}=\lim_{p\to \infty} S_p$.

For any $k=1\ldots 2^p$ the $k$-th component of the sum $S_p$ is not greater than the $m$ and $m-1$-th components of $S_{p+1}$, where $m\defn 2k$. Indeed, by the triangle law \begin{align*}\abs{b\brac{\frac{k t}{2^p}}-b\brac{\frac{\brac{k-1} t}{2^p}}}&\leq \abs{b\brac{\frac{2 k t}{2^{p+1}}}-b\brac{\frac{\brac{2k-1} t}{2^{p+1}}}}\\&+\abs{b\brac{\frac{\brac{2k-1} t}{2^{p+1}}}-b\brac{\frac{2\brac{k-1} t}{2^{p+1}}}}\end{align*} Note that $\frac{\brac{2k-1}t}{2^{p+1}}\in \brac{\frac{k t}{2^p}, \frac{\brac{k-1} t}{2^p}}$.

Let $S\defn \sup_{p\geq1} S_p$. Since for every $p\geq1$ the ``sample'' $t_k\defn \frac{k}{2^p}t$ is such that $\brac{t_k}_{k=0}^{2^p}\in \clo{0,t}$ and $t_k\geq t_{k-1}$ for all $k=1\ldots{2^n}$. Therefore by the definition of total variation 108 it must be true that $S_p\leq \abs{b}\brac{t}$.

If $t=0$ then $S_p=\abs{b\brac{0}}$ for all $p\geq1$, whence $S=\abs{b\brac{0}}$ and $\abs{b}\brac{0}=\abs{b\brac{0}}=S$. So in the following suppose $t>0$.

Suppose $b$ is right-continuous, let $\brac{t_i}_{i=0}^n\in\clo{0,t}$ be such that $t_i<t_{i+1}$ and pick any $\epsilon>0$. For every $i=0\ldots n$ there exists $T_i\in\ploc{t_i,t_{i+1}}$, where $t_{n+1}$ denotes $t$, such that $\abs{b\brac{t_i}-b\brac{u}}<\epsilon$ for all $u\in\ploc{t_i,T_i}$. Since the binary rational number are dense in $\clo{0,1}$, there must exist $p_i\geq1$ and $q_i=0\ldots {p^{q_i}}$ such that $\frac{q_i t}{2^{p_i}}\in \clop{t_i,T_i}$, whence \[0\leq t_0\leq \frac{q_0 t}{2^{p_0}}<t_i\leq \frac{q_1 t}{2^{p_1}}<\ldots<t_n\leq \frac{q_n t}{2^{p_n}}\leq t\] and for all $i=0\ldots n$ \[\abs{b\brac{t_i}-b\brac{\frac{q_i t}{2^{p_i}}}}<\epsilon\] Let $M$ denote the least common multiple of $\brac{2^{p_i}}_{i=0}^n$, and put $k_i\defn q_i \frac{M}{2^{p_i}}$. It this particular case $M=2^p$ and $k_i=q_i 2^{p-p_i}$, where $p\defn\max\obj{\induc{p_i}\,i=0\dots n}$. Then $\frac{k_i t}{2^p} = \frac{q_i t}{2^{p_i}}$. Note that it cannot be otherwise than $n+1\leq 2^p$.

Thus by the right-continuity and density for every $\epsilon>0$ there exists $p\geq1$ and distinct $\brac{k_i}_{i=0}^n\in \obj{0,\ldots 2^p}$ with $\frac{k_i}{2^p}\in \clop{t_i, t_{i+1}}$ for all $i=0\ldots n-1$ and $\frac{k_n}{2^p}\in \clo{t_n,t}$ such that for all $i=0\ldots n$ \[\abs{b\brac{t_i}-b\brac{\frac{k_i t}{2^p}}}<\epsilon\] Furthermore \[\sum_{i=1}^n\abs{b\brac{\frac{k_i t}{2^p}}-b\brac{\frac{k_{i-1} t}{2^p}}} \leq \sum_{k=1}^{2^p}\abs{b\brac{\frac{k t}{2^p}}-b\brac{\frac{\brac{k-1} t}{2^p}}} = S_p - \abs{b\brac{0}}\] By the triangle inequality \begin{align*}\sum_{i=1}^n\abs{b\brac{t_i}-b\brac{t_{i-1}}}&\leq\sum_{i=1}^n\abs{b\brac{t_i}-b\brac{\frac{k_i t}{2^p}}}\\&+\sum_{i=1}^n\abs{b\brac{\frac{k_i t}{2^p}}-b\brac{\frac{k_{i-1} t}{2^p}}}\\&+\sum_{i=1}^n\abs{b\brac{t_{i-1}}-b\brac{\frac{k_{i-1} t}{2^p}}}\\&< n \epsilon + S_p -\abs{b\brac{0}} + n \epsilon\\&\leq S-\abs{b\brac{0}}+2 n \epsilon\end{align*} Hence $\abs{b\brac{0}}+\sum_{i=1}^n\abs{b\brac{t_i}-b\brac{t_{i-1}}}\leq S$ for all $\brac{t_i}_{i=1}^n\in\clo{0,t}$ be such that $t_i<t_{i+1}$, whence by definition 108 $\abs{b}\brac{t}\leq S$.

Now assume that $b$ is a left-continuous map and that $t\neq 0$. Let $\brac{t_i}_{i=0}^n\in\ploc{0,t}$ be such that $0<t_i<t_{i+1}\leq t$ and pick any $\epsilon>0$. Left-continuity of $b$ implies that there exist $\brac{T_i}_{i=0}^n$ with $T_i\in\clop{t_{i-1}, t_i}$ and $T_0\in\clop{0,t_0}$ such that for each $i=0\ldots n$ and all $u\in \clop{T_i,t_i}$ it is true that $\abs{b\brac{t_i}-b\brac{u}}<\epsilon$. Since the binary rationals are dense in $\clo{0,1}$ for each $i=0\ldots n$ there exist $p_i\geq 1$ and $q_i=0\ldots 2^{p_i}$ with $\frac{q_i t}{2^{p_i}}\in \clop{T_i,t_i}$, whence for $p\defn \max\obj{\induc{p_i}\,i=0\ldots n}$ and $k_i\defn q_i 2^{p-p_i}$ it is true that $\brac{k_i}_{i=0}^n\in \obj{0,\ldots,2^p}$ and $\frac{k_i t}{2^p}\in \clop{T_i,t_i}$. Thus by the triangle law the  \begin{align*}\sum_{i=1}^n \abs{b\brac{t_i}-b\brac{t_{i-1}}} &\leq \sum_{i=1}^n \abs{b\brac{t_i}-b\brac{\frac{k_i t}{2^p}}} \\&+ \sum_{i=1}^n \abs{b\brac{\frac{k_i t}{2^p}}-b\brac{\frac{k_{i-1} t}{2^p}}} \\&+ \sum_{i=1}^n \abs{b\brac{t_{i-1}}-b\brac{\frac{k_{i-1} t}{2^p}}} \end{align*} is dominated by $S+2n\epsilon$, since \[\sum_{i=1}^n \abs{b\brac{\frac{k_i t}{2^p}}-b\brac{\frac{k_{i-1} t}{2^p}}}\leq S_p-\abs{b\brac{0}}\leq S-\abs{b\brac{0}}\] Hence by definition 108 $\abs{b}\brac{t}\leq S+2n\epsilon$ for all $\epsilon>0$, whence $\abs{b}\brac{t}\leq S$.\\

%% Exercise 20 ex:20
Let $b:\Real^+\to \Real^+$ be defined by $b\defn 1_{\mathbb{Q}^+}$. Since $\frac{k}{2^n}\in\mathbb{Q}$ for all $k=0\ldots 2^n$ and $n\geq1$, $b\brac{\frac{k}{2^n}}=1$. Thus for all $n\geq1$ \[S_n=\sum_{k=1}^n \abs{b\brac{\frac{k}{2^n}}-b\brac{\frac{k-1}{2^n}}}=0\] whence $\lim_{n\to \infty} S_n = 0$.

Let $\brac{t_i}_{i=0}^n\in\ploc{0,1}$ be such that $0\leq t_i<t_{i+1}\leq 1$ for $i=1\ldots {n-1}$ and $t_i\in \mathbb{Q}$ for even $i$ and $t_i\notin \mathbb{Q}$ for odd. Then \[\sum_{i=1}^n \abs{b\brac{t_i}-b\brac{t_{i-1}}} = \sum_{i=1}^n \abs{\brac{-1}^i} = n\] whence $\abs{b}\brac{1}\leq n$ for all $n\geq1$. Thus $\abs{b}\brac{1}=+\infty\neq 0$.\\

\label{thm:bound_var_cplx_limit}\noindent\textbf{Theorem} 14-16.
Let $b:\Real^+\to \Cplx$ be a map of bounded variation. Then the limit $\lim_{t\to\infty} b\brac{t}$ exists in $\Cplx$.

Since bounded variation implies finite variation, by theorem 80 the total variation of $b$ is a non-decreasing map, whence \[\abs{b}\brac{\infty}\defn \sup_{t\in \Real^+}\abs{b}\brac{t}=\lim_{t\to\infty}\abs{b}\brac{t}\] Since $b$ is of bounded variation there exists $M\in\Real^+$ such that $\abs{b}\brac{t}\leq M$ for all $t\in \Real^+$, whence $\abs{b}\brac{\infty}\leq M$. Therefore $\abs{b}\brac{\infty}\in \Real$.

By theorems 14-11 and 14-2 it is true that for any $t\in \Real^+$ \[\abs{b_1}^+\brac{t}, \abs{b_1}^-\brac{t}, \abs{b_2}^+\brac{t}, \abs{b_2}^-\brac{t}\leq \abs{b}\brac{t}\leq \abs{b}\brac{\infty}<+\infty\] By theorem 14-11 $\abs{b_1}^+$, $\abs{b_1}^-$, $\abs{b_2}^+$ and $\abs{b_2}^-$ are non-decreasing, whence $\abs{b_1}^+\brac{\infty}$, $\abs{b_1}^-\brac{\infty}$, $\abs{b_2}^+\brac{\infty}$ and $\abs{b_2}^-\brac{\infty}$, defined similarly to $\abs{b}\brac{\infty}$, are finite and exist in $\Real^+$.

By the above the value $b_i\brac{\infty}\defn \abs{b_i}^+\brac{\infty} - \abs{b_i}^-\brac{\infty}$ is a well-defined real number. Both $\abs{b_i}^+\brac{t}\to\abs{b_i}^+\brac{\infty}$ and $\abs{b_i}^-\brac{t}\to\abs{b_i}^-\brac{\infty}$ as $t\to \infty$ imply that \[b_i\brac{t} = \abs{b_i}^+\brac{t} - \abs{b_i}^-\brac{t} \overset{\Real}{\to} \abs{b_i}^+\brac{\infty} - \abs{b_i}^-\brac{\infty} = b_i\brac{\infty}\] for each $i=1,2$ as $t\to \infty$.

Let $b\brac{\infty}\defn b_1\brac{\infty} + i b_2\brac{\infty}$. Since $b\brac{\infty}$ is a well-defined complex number,  $b)i\brac{t}\overset{\Real}{\to} b_i\brac{\infty}$ and $\abs{\cdot}\leq \abs{\re \cdot}+\abs{\im \cdot}$, it must be true that \[b\brac{t} = b_1\brac{t}+i b_2\brac{t}\overset{\Cplx}{\to} b_1\brac{\infty}+i b_2\brac{\infty} = b\brac{\infty}\] as $t\to \infty$. Therefore the limit $\lim_{t\to\infty}b\brac{t}$ exists in $\Cplx$.\\

\label{thm:cplx_stieltjes_meas}\noindent\textbf{Theorem} 14-17.
For any right-continuous map $b:\Real^+\to\Cplx$ of bounded variation there exists a unique complex measure $\mu$ on $\brac{\Real^+,\borel{\Real^+}}$ with: \begin{align*}\mu\brac{\obj{0}}&=b\brac{0}\\\mu\brac{\ploc{s,t}}&=b\brac{t}-b\brac{s}\,\forall s,t\in\Real^+,\,s\leq t\end{align*}

Let $b_1\defn\re b$, $b_2\defn \im b$. Since bounded variation implies finite variation, by theorem 14-15 the maps $\abs{b_1}^+$, $\abs{b_1}^-$, $\abs{b_2}^+$ and $\abs{b_2}^-$ are real-valued non-decreasing right-continuous maps and which due to theorem 14-3 are of bounded variation .

By theorem 11 for each ``partial'' variation map there exists a unique measure on $\brac{\Real^+,\borel{\Real^+}}$ known as the Stieltjes measure associated with $\abs{b_1}^+$, $\abs{b_1}^-$, $\abs{b_2}^+$ and $\abs{b_2}^-$ respectively.

Note that for all $n\geq1$ it is true that $\clo{0,n}=\obj{0}\uplus\ploc{0,n}$. Thus $dF\brac{\clo{0,n}} = dF\brac{\ploc{0,n}}+dF\brac{\obj{0}}$, whence $dF\brac{\clo{0,n}} = F\brac{n}$, where $F$ is either $\abs{b_1}^+$, $\abs{b_1}^-$, $\abs{b_2}^+$ or $\abs{b_2}^-$. By theorems 14-2 and 14-11 $F\brac{t}\leq \abs{\re b}\brac{t},\abs{\im b}\brac{t}\leq \abs{b}\brac{t}$ for all $t\in\Real^+$, whence $dF\brac{\clo{0,n}}\leq \abs{b}\brac{n}$ for all $N\geq1$.

Now the fact that $b$ is of bounded variation implies that there exists $M\in \Real^+$ such that $\abs{b}\brac{t}\leq M$ for all $t\in \Real^+$, whence $dF\brac{\clo{0,n}}\leq M$ for all $n\geq1$. By theorem 7 $dF\brac{\clo{0,n}}\uparrow dF\brac{\Real^+}$ which implies that $dF\brac{\Real^+}\leq M<+\infty$.

The measures $d\abs{b_1}^+$, $d\abs{b_1}^-$, $d\abs{b_2}^+$ or $d\abs{b_2}^-$ are finite and therefore are complex measures on $\brac{\Real^+,\borel{\Real^+}}$ by definition 92. By theorem 11-13 their linear combination $db\defn d\abs{b_1}^+-d\abs{b_1}^-+i\brac{d\abs{b_2}^+-d\abs{b_2}^-}$ must be a complex measure on $\brac{\Real^+,\borel{\Real^+}}$ as well.

By theorems 14-3 and 14-11 \[db\brac{\obj{0}} = \abs{b_1}^+\brac{\obj0}-\abs{b_1}^-\brac{\obj0}+i\brac{\abs{b_2}^+\brac{\obj0}-\abs{b_2}^-\brac{\obj0}} = b_1\brac{0} + i b_2\brac{0} = b\brac{0}\]

By definition 109 and theorem 14-11 for any $s,t\in\Real^+$ with $s\leq t$ it is true \begin{align*}d\abs{b_i}^+\brac{\ploc{s,t}}-d\abs{b_i}^-\brac{\ploc{s,t}} &=\abs{b_i}^+\brac{t}-\abs{b_i}^+\brac{s} - \brac{\abs{b_i}^-\brac{t}-\abs{b_i}^-\brac{s}}\\&=\abs{b_i}^+\brac{t}-\abs{b_i}^-\brac{t} - \brac{\abs{b_i}^+\brac{s}-\abs{b_i}^-\brac{s}}\\&= b_i\brac{t}-b_i\brac{s}\end{align*} Thus $db\brac{\ploc{s,t}}=b\brac{t}-b\brac{s}$.

For $A_n\defn \clo{0,n}$, $A_n\uparrow \Real^+$ and by theorem 99-9 $db\brac{A_n}\overset{\Cplx}{\to}db\brac{\Real^+}$. Now $db\brac{A_n} = b\brac{n}$ for every $n\geq1$, which implies that $db\brac{A_n}\overset{\Cplx}{\to} b\brac{\infty}$, since by theorem 14-16 the limit $b\brac{\infty}=\lim_{t\to\infty}b\brac{t}$ exists in $\Cplx$. Therefore $db\brac{\Real^+}$ is nothing but $b\brac{\infty}$.

Now on to proving the uniqueness of $db$ with such properties on the semi-ring.
Let $\mu$ be another complex measure on $\brac{\Real^+,\borel{\Real^+}}$ such that $\mu\brac{\ploc{s,t}}=b\brac{t}-b\brac{t}$ for all $s\leq t$ in $\Real^+$ and put \[\Dcal\defn \brac{\induc{B\in \borel{\Real^+}}\, \mu\brac{B} = db\brac{B}}\]

Let's show that $\Dcal$ is a Dynkin system on $\Real^+$ containing the $\pi$-system of the half-open intervals, which in turn generate the Borel $\sigma$-algebra on $\Real^+$. Then by theorem 1 (the Dynkin system) $\borel{\Real^+}\subseteq \Dcal$, whence $\mu = db$ on $\borel{\Real^+}$.

Indeed, similarly $\mu\brac{\ploc{s,t}}=b\brac{t}-b\brac{s}$ for all $s,t\in \Real^+$ implies that $\mu\brac{\Real^+} = b\brac{\infty} = db\brac{\Real^+}$ whence $\Real^+\in \Dcal$. If $A\subseteq B\in \Dcal$, then for the partition $B = A \uplus B\setminus A$ it is true that \[\mu\brac{B\setminus A}=\mu\brac{B}-\mu\brac{A}=db\brac{B}-db\brac{A}=db\brac{B\setminus A}\] because $\mu$ and $db$ are complex measure on $\brac{\Real^+,\borel{\Real^+}}$. Finally if $\brac{A_n}_{n\geq1}\in \Dcal$ is such that $A_b\subseteq A_{n+1}$ with $A\defn \bigcup_{n\geq1} A_n$, then by theorem 99-9 it must be true that $\mu\brac{A_n}\uparrow \mu\brac{A}$ and $db\brac{A_n}\uparrow db\brac{A}$, whence since $\Cplx$ is Hausdorff it must be true that $\mu\brac{A}=db\brac{A}$. Thus $\bigcup_{n\geq1} A_n\in \Dcal$, implying that it is a Dynkin system on $\Real^+$.

Recall that $\Scal$ the collection of half open intervals $\ploc{s,t}$ with $s,t\in \Real$ constitutes a semi-ring on $\Real$, which generates $\borel{\Real}$. By theorem 10 $\borel{\Real^+} = \induc{\borel{\Real}}_{\Real^+}$, which means that $\borel{\Real^+}=\sigma\brac{\induc{\Scal}_{\Real^+}}$. Thus the collection of non-negative left-open right-closed intervals with an atom at zero generates $\borel{\Real^+}$ and constitute a $\pi$-system.

The assumptions on $\mu$ imply that $\ploc{s,t}\in\Dcal$ for all $s,t\in \Real^+$ and $\obj{0}\in\Dcal$, whence by theorem 1 (Dynkin system) it must be true that $\borel{\Real^+}\subseteq \Dcal$ and $\mu=db$.\\

\noindent\textbf{Definition} 101.
Let $b:\Real^+\to\Cplx$ be a right-continuous map of bounded variation. Then there exists a unique complex measure on $\brac{\Real^+,\borel{\Real^+}}$, called the complex Stieltjes measure and denoted by $db$, such that\begin{align*}db\brac{\obj{0}}&=b\brac{0}\\db\brac{\ploc{s,t}}&=b\brac{t}-b\brac{s},\,\forall s,t\in \Real^+,\,s\leq t\end{align*}

%% Exercise 22 ex:22
Let $a:\Real^+\to \Real^+$ be a right-continuous non-decreasing map with $a\brac{0}\leq 0$ and $a\brac{\infty}<+\infty$. Then by theorem 11 $da$ is a unique Lebesgue-Stieltjes measure on $\brac{\Real^+, \borel{\Real^+}}$ with $da\brac{\obj{0}} = a\brac{0}\geq 0$ and such that $da\brac{\ploc{s,t}}=a\brac{t}-a\brac{s}$ for all $s,t\in \Real^+$ with $s\leq t$. Since $da\brac{\Real^+} = a\brac{\infty}<+\infty$ by theorem 7, it must be true that $da$ is a finite measure, whence it must also be a complex measure on $\brac{\Real^+, \borel{\Real^+}}$.

Now, by theorem 14-1 $a$ is a map of bounded variation, whence by theorem 14-17 for it there exists a unique complex Stieltjes measure $\mu$ with $\mu\brac{\obj{0}} = a\brac{0}$ and such that $\mu\brac{\ploc{s,t}}=a\brac{t}-a\brac{s}$ for all $s,t\in \Real^+$ with $s\leq t$. Thus $\mu=da$ and so for non-negative non-decreasing maps with $a\brac{\infty}\in \Real^+$ the new definition 101 coincides with the already known one stated in theorem 11.\\

%% Exercise 23 ex:23
Let $b:\Real^+\to \Cplx$ be a right-continuous map of finite-variation and let $b_1\defn\re b$, $b_2\defn \im b$. Since $b$ is of finite variation, the maps $\abs{b_1}^+$, $\abs{b_1}^-$, $\abs{b_2}^+$ and $\abs{b_2}^-$ are real-valued non-decreasing by theorem 14-11 and right-continuous by theorem 14-14. By theorem 11 there exist unique Stieltjes measures $d\abs{b_1}^+$, $d\abs{b_1}^-$, $d\abs{b_2}^+$ and $d\abs{b_2}^-$ on $\brac{\Real^+, \borel{\Real^+}}$ with the usual properties.

However these measures are not necessarily finite, which means that not for all measurable sets they ascribe a value in $\Real^+\subseteq \Cplx$. In the last theorem 14-17 the fact of bounded variation of $b$ provided a finite upper bound on the values of the positive and negative variations, which also acted as an upper bound on the value of the associated measures.

Thus unless $b$ is right-continuous non-decreasing map with $b\brac{0}\geq 0$ or a $\Cplx$-valued right-continuous map of bounded variation, the notation $db$ makes no sense.\\

%% Exercise 24 ex:24
Let $b:\Real^+\to\Cplx$ be a map. For all $T\in \Real^+$ define $b^T:\Real^+\to\Cplx$ as $b^T\brac{t}\defn b\brac{t\wedge T}$ for all $t\in\Real^+$.

Let $t\in \Real^+$ and $\brac{t_i}_{i=0}^n\in \clo{0,t}$ for some $n\geq1$ with $t_{i-1}\leq t_i$. Then $s_i\defn t_i\wedge T$ are such that $\brac{s_i}_{i=0}^n\in \clo{0,t\wedge T}$ with $s_{i-1}\leq s_i$. Then \[\sum_{i=1}^n\abs{b^T\brac{t_i}-b^T\brac{t_{i-1}}} = \sum_{i=1}^n\abs{b\brac{s_i}-b\brac{s_{i-1}}}\leq \abs{b}\brac{t\wedge T}\] whence $\abs{b^T}\brac{t}\leq \abs{b}^T\brac{t}$ by definition 108 of total variation.

Conversely, if $\brac{t_i}_{i=0}^n\in \clo{0,t\wedge T}$ for some $n\geq1$ with $t_{i-1}\leq t_i$ then in particular $\brac{t_i}_{i=0}^n\in \clo{0,t}$, which implies that \begin{align*}\sum_{i=1}^n\abs{b\brac{t_i}-b\brac{t_{i-1}}} &= \sum_{i=1}^n\abs{b\brac{t_i\wedge T}-b\brac{t_{i-1}\wedge T}} \\&= \sum_{i=1}^n\abs{b^T\brac{t_i}-b^T\brac{t_{i-1}}}\\&\leq \abs{b^T}\brac{t}\end{align*} Therefore by definition 108 $\abs{b}\brac{t\wedge T}\leq \abs{b^T}\brac{t}$.

Now, if $b$ is $\Real$-value and of finite variation, then \[\abs{b^T}^+ = \frac{1}{2}\brac{\abs{b^T}-b^T} = \frac{1}{2}\brac{\abs{b}^T-b^T} = \brac{\abs{b}^+}^T\] by the first observation. Similarly, $\abs{b^T}^-=\brac{\abs{b}^-}^T$.

%% $b^T$ is of bounded variation whenever $b$ is of finite variation.
Suppose the map $b$ is of finite variation. For any $t\in \Real^+$ it is true that $t\wedge T\leq T$ whence by theorem 80 \[\abs{b\brac{T}-b\brac{t\wedge T}}\leq \abs{b}\brac{T}-\abs{b}\brac{t\wedge T} = \abs{b}\brac{T}-\abs{b}^T\brac{t}\] Hence $\abs{b}\brac{T}\geq \abs{b}^T\brac{t}$, which means that $\abs{b^T}\brac{t}\leq \abs{b}\brac{T}<+\infty$ for all $t\in \Real^+$. Therefore $b^T$ is of bounded variation and thus non-decreasing by theorem 80. So by definition of $\abs{b^T}\brac{\infty}$ as the least upper bound on $\abs{b^T}\brac{t}$ over all $t\in \Real^+$ it must be true that $\abs{b^T}\brac{\infty}\leq \abs{b}\brac{T}<+\infty$.

%% right-continuity of $b^T$
Suppose $b$ is right-continuous and let $t\in \Real^+$. If $t\geq T$ then $\abs{b^T\brac{t}-b^T\brac{s}}=0$ for all $s\in \Real^+$ with $t<s$, which means that $b^T$ is right-continuous on $\clop{T,+\infty}$. Now if $0\leq t<T$, then $b^T\brac{s}=b\brac{s}$ for all $s\in \clop{t, T}$, which implies that for all $\epsilon>0$ by right-continuity of $b$ there is $\delta>0$ such that for all $s\in \Real^+$ with $s\in\brac{t,t+\delta}$ it is true that $s<T$ and \[\abs{b^T\brac{t}-b^T\brac{s}}=\abs{b\brac{t}-b\brac{s}}<\epsilon\] Therefore $b^T$ is a right-continuous map.

Since by the above $b^T$ is of bounded variation, by theorem 14-17 there exists a unique complex measure $db^T$ on $\Real^+$ with $db^T\brac{\obj{0}}=b\brac{0\wedge T}=b\brac{0}$ and $db^T\brac{\ploc{s,t}}=b\brac{t\wedge T}-b\brac{s\wedge T}$ for all $s,t\in \Real^+$ with $s\leq t$.

Suppose $b$ is right-continuous function of bounded variation. By theorem 14-17 there is a unique complex measure $db$ on $\brac{\Real^+, \borel{\Real^+}}$ with $db\brac{\obj{0}}=b\brac{0}=b\brac{0}$ and $db\brac{\ploc{s,t}}=b\brac{t}-b\brac{s}$ for all $s,t\in \Real^+$ with $s\leq t$. Note that for all $s,t\in \Real^+$ with $s\leq t$ \[db^{\clo{0,T}}\brac{\ploc{s,t}} = db\brac{\ploc{s\wedge T, t\wedge T}} = b\brac{t\wedge T}-b\brac{s\wedge T} = db^T\brac{\ploc{s,t}}\] Since $db^T$ is unique by theorem 14-17, it must be true that $db^T = db^{\clo{0,T}}$ for all $T\in \Real^+$ on $\brac{\Real^+, \borel{\Real^+}}$.

Suppose $b$ is right-continuous and non-decreasing map with $b\brac{0}\leq 0$. The truncated map $b^T$ is also right-continuous and non-decreasing, whence by theorem 11 there exists a unique measure $db^T$ on $\brac{\Real^+, \borel{\Real^+}}$ with $db^T\brac{\ploc{s,t}}=b\brac{t}-b\brac{s}$ for all $s,t\in \Real^+$ wit h $s\leq t$ and $db^T\brac{\obj{0}}=b\brac{0}$. However, for $b$ by theorem 11 there is a unique measure $db$ on $\brac{\Real^+, \borel{\Real^+}}$ with $db\brac{\ploc{s,t}}=b\brac{t}-b\brac{s}$ for all $s,t\in \Real^+$ wit h $s\leq t$ and $db\brac{\obj{0}}=b\brac{0}$. Since $db\brac{\obj{0}}=b\brac{0}=b^T\brac{0}=db^T\brac{\obj{0}}$ and for all $s,t\in \Real^+$ wit h $s\leq t$  \[db^{\clo{0,T}}\brac{\ploc{s,t}} = b\brac{t\wedge T}-b\brac{s\wedge T} = db^T\brac{\ploc{s,t}}\] by uniqueness of $db^T$ it must be true that $db^{\clo{0,T}}=db^T$ on $\brac{\Real^+, \borel{\Real^+}}$.\\

\label{thm:finite_meas_map}\noindent\textbf{Theorem} 14-18.
If $\mu$ is a usual finite measure on $\brac{\Real^+, \borel{\Real^+}}$, then $f\brac{t}\defn \mu\brac{\clo{0,t}}$ is a non-decreasing right-continuous map with $f\brac{0}\geq 0$.

Indeed, for any $s,t\in \Real^+$ with $s\leq t$ it is true that $\clo{0,s}\subseteq \clo{0,t}$, which by theorem 2-4 implies \[f\brac{s}=\mu\brac{\clo{0,s}}\leq \mu\brac{\clo{0,t}} = f\brac{t}\] Furthermore $f\brac{s}\geq 0$ since $\mu$ is a usual measure, and $f\brac{t}<+\infty$ for all $t\in \Real^+$ as $\mu$ is finite. Thus $f:\Real^+\to \Real^+$ is a non-negative non-decreasing map.

Now pick any $t\in \Real^+$ and let $\brac{t_n}_{n\geq1}\in \Real^+$ be such that $t_n>t$ and $t_n\downarrow t$. Then $\clo{0,t_n}\downarrow \clo{0,t}$, whence via theorem 8 the finiteness of $\mu$ implies that $f\brac{t_n}\downarrow f\brac{t}$. Hence for every $\epsilon>0$ there exists $N\geq1$ with $f\brac{t_N}-f\brac{t}<\epsilon$ from which for $\delta\defn T_n-t>0$ the non-strict monotonicity of $f$ implies that for all $s\in \clop{t,t+\delta}$ \[0\leq f\brac{s}-f\brac{t} \leq f\brac{t_N}-f\brac{t}<\epsilon\] In conclusion for every $\epsilon>0$ there exists $\delta>0$ such that $\abs{f\brac{t}-f\brac{s}}<\epsilon$ for all $s\in \clop{t, t+\delta}$, whence $f$ is right-continuous.\\

%% Exercise 25 ex:25
\label{thm:stieltjes_monotonicity}\noindent\textbf{Theorem} 83.
Let $\mu,\nu$ be two finite measures on $\brac{\Real^+,\borel{\Real^+}}$ such that\begin{itemize}
	\item $\mu\brac{\obj{0}}\leq \nu\brac{\obj{0}}$
	\item $\mu\brac{\ploc{s,t}}\leq \nu\brac{\ploc{s,t}}$ for all $s,t\in \Real^+$ and $s\leq t$
\end{itemize} Then $\mu\brac{B}\leq \nu\brac{B}$ for all $B\in \borel{\Real^+}$.

Indeed, the map $a:\Real^+\to \Real^+$ defined by $a\brac{t}\defn \mu\brac{\clo{0,t}}$ is by theorem 14-18 a non-decreasing right-continuous map with $a\brac{0}\geq 0$. Note that for any $s,t\in \Real^+$ with $s\leq t$ it is true that $\mu\brac{\ploc{s,t}}=a\brac{t}-a\brac{s}$, which by uniqueness of the Stieltjes measure, shown in theorem 11, implies that $\mu = da$. The map $c:\Real^+\to\Real^+$ defined as $c\brac{t}\defn \nu\brac{\clo{0,t}}$ has similar properties.

Now, for any $t\in \Real^+$ the basic properties of measures imply that \[\mu\brac{\clo{0,t}}=\mu\brac{\obj{0}}+\mu\brac{\ploc{0,t}}\leq\nu\brac{\obj{0}}+\nu\brac{\ploc{0,t}}=\nu\brac{\clo{0,t}}\] whence $a\brac{t}\leq c\brac{t}$.

Let the map $b:\Real^+\to\Real^+$ be defined as $b\defn c-a$. Since $a\leq c$ it is true that $b\brac{t}\geq 0$ for all $t\in \Real^+$. Furthermore both maps are non-decreasing for any $s,t\in \Real^+$ with $s\leq t$ it is true that \[a\brac{t}-a\brac{s} = \mu\brac{\ploc{s,t}}\leq \nu\brac{\ploc{s,t}} = c\brac{t}-c\brac{s}\] whence $b\brac{s}\leq b\brac{t}$.

Now pick any $t\in \Real^+$ and note that for $\epsilon>0$ there exist $\delta_a,\delta_c>0$ such that $\abs{c\brac{s}-c\brac{t}}<\frac{\epsilon}{2}$ for all $s\in \clop{t,t+\delta_c}$ and $\abs{a\brac{s}-a\brac{t}}<\frac{\epsilon}{2}$ for all $s\in \clop{t,t+\delta_a}$. Hence for $\delta\defn \min\obj{\delta_a,\defn_c}>0$ it is true that $\abs{c\brac{s}-c\brac{t}}<\frac{\epsilon}{2}$ and $\abs{a\brac{s}-a\brac{t}}<\frac{\epsilon}{2}$ for all $s\in \clop{t,t+\delta}$, whence \[\abs{c\brac{s}-a\brac{s} - \brac{c\brac{t}-a\brac{t}}} \leq \abs{c\brac{s}-c\brac{t}} + \abs{a\brac{s}-a\brac{t}} < \epsilon\] Therefore $b$ is right-continuous at $t$.

By theorem 11 there is a unique Stieltjes measure $db$ on $\brac{\Real^+}$ with $db\brac{\obj{0}}=b\brac{0}$ and $db\brac{\ploc{s,t}}=b\brac{t}-b\brac{s}$ for all $s,t\in \Real^+$ with $s\leq t$. Consider $\gamma:\borel{\Real^+}\to \Real^+$ defined by $\gamma\brac{E}\defn da\brac{E}+db\brac{E}$ for any $E\in \borel{\Real^+}$. If $E=\emptyset$, then $\gamma\brac{\emptyset} = 0$ since $da$ and $db$ are measures.

Let $E\in \borel{\Real^+}$ and $\brac{E_n}_{n\geq1}$ be a measurable partition of $E$. Then $da\brac{E}=\sum_{n\geq1} da\brac{E_n}$ and $db\brac{E}=\sum_{n\geq1} db\brac{E_n}$ whence by theorem Sup-B-3 \[\gamma\brac{E} = \sum_{n\geq1} da\brac{E_n}+\sum_{n\geq1}db\brac{E_n} = \sum_{n\geq1} da\brac{E_n}+db\brac{E_n} = \sum_{n\geq1} \gamma\brac{E_n}\] Thus $\gamma$ is a measure on $\brac{\Real^+,\borel{\Real^+}}$.

Next, $\gamma\brac{\obj{0}} = a\brac{0}+b\brac{0} = c\brac{0}\geq 0$ and for any $s,t\in \Real^+$ with $s\leq t$ it is true that \[\gamma\brac{\ploc{s,t}} = da\brac{\ploc{s,t}}+db\brac{\ploc{s,t}} = a\brac{t}-a\brac{s}+b\brac{t}-b\brac{s} = c\brac{t}-c\brac{s}\] Therefore by uniqueness of the Stieltjes measure associated with $c$ (theorem 11) it is true that $dc=\gamma$, whence indeed $dc=da+db$.

For any $B\in \borel{\Real^+}$ the above equation implies that $da\brac{B}\leq da\brac{B}+db\brac{B}=dc\brac{B}$, whence $\mu\brac{B}\leq \nu\brac{B}$.\\

\label{thm:cplx_siteltjes_tot_var}\noindent\textbf{Theorem} 14-19.
If $b:\Real^+\to\Cplx$ is a right-continuous map of bounded variation, then the total variation of its associated complex Stieltjes measure is equal to the Stieltjes measure associated with its total variation, $\abs{db}=d\abs{b}$.

Let $b:\Real^+\to\Cplx$ be right-continuous map of bounded variation 
Indeed, by theorem 14-10 the total variation of $b$ is a non-decreasing map and by theorem 81 $\abs{b}$ is right-continuous. Therefore by theorem 11 there exists a unique usual Stieltjes measure $d\abs{b}$ such that $d\abs{b}\brac{\obj{0}}=\abs{b}\brac{0}=\abs{b\brac{0}}$.

Since $b$ is of bounded variation by theorem 14-17 there is a unique complex Stieltjes measure $db$ associated with it. Let $\abs{db}$ be the total variation of the measure $db$ as in definition 97. Let $\brac{E_n}_{n\geq1}$ be any measurable partition of $\obj{0}\in \borel{\Real^+}$. If $y\in \Real^+$ is such that $y\in \biguplus_{n\geq1}E_n$, then $y=0$ and further more there is $n\geq1$ such that $E_k=\emptyset$ for all $k\neq n$ whereas $E_n=\obj{0}$. Thus $\abs{db}\brac{\obj{0}} = \abs{db\brac{\obj{0}}} = \abs{b\brac{0}}$.

Pick any $s,t\in \Real^+$ with $s\leq t$ and let $\brac{t_i}_{i=0}^n\in \clo{s,t}$ be such that $t_{i-1}\leq t_i$ for $i=1\ldots n$, where $n\geq1$. If $E_i\defn \ploc{t_{i-1},t_i}$ for $i=1\ldots n$, then $\brac{E_i}_{i=1}^n$ is a measurable partition of $\ploc{s,t}$, whence by definition 97 \[\abs{db}\brac{\ploc{s,t}} \geq \sum_{i=1}^n \abs{db\brac{\ploc{t_{i-1},t_i}}} = \sum_{i=1}^n \abs{b\brac{t_i}-b\brac{t_{i-1}}}\] Since $b$ is a map of bounded variation and bounded variation implies finite variation, from theorem 80 it follows that $d\abs{b}\brac{\ploc{s,t}} = \abs{b}\brac{t} - \abs{b}\brac{s}\leq \abs{db}\brac{\ploc{s,t}}$. Since by theorem 57 (and 11-7) $\abs{db}$, the total variation of the measure $db$, is a finite measure on $\brac{\Real^+, \borel{\Real^+}}$, by theorem 83 it must be true that $d\abs{b}\leq \abs{db}$.

Let $f\in L^1_\Cplx\brac{\Real^+, \borel{\Real^+}, \abs{db}}$. Then from the above and by theorem 99-6 it is true that \[\int \abs{f} d\abs{b} \leq \int \abs{f} d\abs{db}<+\infty\] whence $f\in L^1_\Cplx\brac{\Real^+, \borel{\Real^+}, d\abs{b}}$.

Now the topological space $\brac{\Real^+, \Tcal_{\Real^+}}$ is actually a topological subspace of $\brac{\Real, \Tcal_\Real}$ which is metrizable by the usual metric on $\Real$. Thus by theorem 12 the induced space $\brac{\Real^+, \Tcal_{\Real^+}}$ is metrizable as well.

Now, for any $x\in\Real^+$ let $V\defn \brac{x-1,x+1}\cap \Real^+$. Since $\brac{\Real^+, \Tcal_{\Real^+}}$ is an induced topological space and $\brac{x-1,x+1}$ is open in $\Real$, $V$ must be open in $\Real^+$. Since $V\subseteq \clo{0, x+1}$, by theorem 13-12 $\clo{V}\subseteq \clo{0, x+1}$, and by theorem 8-5 $\clo{V}$ is compact in $\brac{\clo{0, x+1}, \induc{\Tcal_\Real^+}_{\clo{0, x+1}}}$. Thus $\clo{V}$ it compact in $\brac{\Real^+,\Tcal_{\Real^+}}$, whence the space $\Real^+$ is locally compact. Furthermore the sets $F_n\defn \clo{0,n}$ constitute a collection of nested compact subsets of $\Real^+$ such that $F_n\uparrow \Real^+$, which implies that the space $\Real^+$ is $\sigma$-compact. Therefore by theorem 75 $\Real^+$ is strongly $\sigma$-compact.

Since by theorem 57 $\abs{db}$ is a finite measure and by the above $\Real^+$ is metrizable, by theorem 70 $C^b_\Cplx\brac{\Real^+}$ is dense in $L^1_\Cplx\brac{\Real^+, \borel{\Real^+}, \abs{db}}$.

Let $x\in \Real^+$ and set $V\defn \clop{0,x+1}$. Then $V$ is open in $\Real^+$ and $\abs{db}\brac{V}<\abs{db}\brac{\Real^+}<+\infty$, since $\abs{db}$ is a finite measure. Therefore by theorem 78 the space of continuous maps with compact support $C^c_\Cplx\brac{\Real^+}$ is dense in $L^1_\Cplx\brac{\Real^+, \borel{\Real^+}, \abs{db}}$.

Hence for every $f\in L^1_\Cplx\brac{\Real^+, \borel{\Real^+}, \abs{db}}$ density of $C^b_\Cplx$ implies that for any $\epsilon>0$ there exists $\phi\in C^b_\Cplx$ such that $\nrm{\phi-f}_1\leq\epsilon$, whence $\int \abs{\phi-f}d\abs{db}<\epsilon$. By definition 97 and theorem 24 over $\brac{\Real^+, \borel{\Real^+}, \abs{db}}$ \[\abs{\int \phi db -\int f db } = \abs{\int \phi h d\abs{db} -\int f h d\abs{db} } \leq \int \abs{\phi - f} \abs{h} d\abs{db} \leq \epsilon\] where $h\in L^1_\Cplx\brac{\Real^+, \borel{\Real^+}, \abs{db}}$ is such that $db = \int h d\abs{db}$, which exists by theorem 62. Thus $\abs{\int f db}\leq \abs{\int \phi db} + \epsilon$ by the triangle law for the complex modulus.

Next $\abs{\abs{\phi}-\abs{f\brac{\cdot}}}\leq \abs{\phi-f}$ and theorems 24 and 99-6 imply that \begin{align*}\abs{\int \abs{\phi\brac{s}} d\abs{b} - \int \abs{f\brac{s}} d\abs{b}\brac{s} }&= \abs{\int \abs{\phi\brac{s}} - \abs{f\brac{s}} d\abs{b}\brac{s} }\\&\leq \int \abs{\abs{\phi\brac{s}} - \abs{f\brac{s}}} d\abs{b}\brac{s}\\&\leq \int \abs{\phi - f} d\abs{b}\\&\leq \int \abs{\phi - f} d\abs{db}\\\end{align*} Hence $\int \abs{\phi\brac{s}} d\abs{b}\brac{s}\leq \int \abs{f\brac{s}} d\abs{b}\brac{s} + \epsilon$.

%% Could be made into a separate result for $\phi$, $db$ and $d\abs{b}$
For this fixed $\phi$ it can be show that $\abs{\int \phi db} \leq \int \abs{\phi\brac{s}}d\abs{b}\brac{s}$. Indeed, for every $n\geq1$ define \[\phi_n\defn \phi\brac{0} 1_{\obj{0}} + \sum_{k=0}^{n 2^n-1} \phi\brac{\frac{k}{2^n}} 1_{\ploc{ \frac{k}{2^n}, \frac{k+1}{2^n} }}\] Since $\phi\in C^b_\Cplx$ there exists $M\in \Real^+$ such that $\abs{\phi\brac{\cdot}}\leq M$ on $\Real^+$. Since $\obj{0}$ and $\brac{\ploc{ \frac{k}{2^n}, \frac{k+1}{2^n} } }_{k=0}^{n 2^n-1}$ partition $\clo{0,n}$ for the absolute value of $\phi_n$ it must be true that \[\abs{\phi_n} = \abs{\phi\brac{0}} 1_{\obj{0}} + \sum_{k=0}^{n 2^n-1} \abs{\phi\brac{\frac{k}{2^n}}} 1_{\ploc{ \frac{k}{2^n}, \frac{k+1}{2^n} }}\] whence $\abs{\phi_n\brac{x}}\leq M 1_{\clo{0,n}}\brac{x}\leq M$ for all $x\in\Real^+$.

Let $x\in \Real^+$. If $x=0$ then $\phi_n\brac{x} = \phi\brac{0}$ for all $n\geq1$ and $\phi_n\overset{\Cplx}{\to}\phi\brac{0}$. Now, if $x>0$ then for integers $\brac{k_n}_{n\geq1}$ defined as $k_n\defn \left \lceil 2^n x\right \rceil$, where $\left \lceil r \right \rceil$ denotes the least integer not less than $r$, it is true that $0<x-\frac{k_n}{2^n}\leq \frac{1}{2^n}$ for all $n\geq 1$. Therefore $\frac{k_n}{2^n}\to x$, whence $\phi\brac{\frac{k_n}{2^n}}\overset{\Cplx}{\to}\phi\brac{x}$ since $\phi$ is $\Real^+$-$\Cplx$ continuous.

Now for any $\epsilon>0$ there exists $N\geq 1$ such that $x \leq N$ and $\abs{\phi\brac{\frac{k_n}{2^n}}-\phi\brac{x}}<\epsilon$ for all $n\geq N$. Since $x\in \ploc{ \frac{k_n}{2^n}, \frac{k_n+1}{2^n} }$ by construction, for all $n\geq N$ it is true that \[\phi_n\brac{x}=0+\phi\brac{\frac{k_n}{2^n}} 1_{\ploc{ \frac{k}{2^n}, \frac{k+1}{2^n} }}\brac{x}+0 = \phi\brac{\frac{k_n}{2^n}}\] whence $\abs{\phi_n\brac{x}-\phi\brac{x}}<\epsilon$ for all $n\geq N$. Therefore $\phi_n\brac{x}\overset{\Cplx}{\to} \phi\brac{x}$ for all $x\in \Real^+$.

Since $\abs{db}$ is finite, the constant map $M 1_{\Real^+}$ is a non-negative element of $L^1_\Cplx\brac{\Real^+, \borel{\Real^+}, \abs{db}}$. Furthermore, if $h\in L^1_\Cplx\brac{\Real^+, \borel{\Real^+}, \abs{db}}$ is as in definition 97, $\phi_n h \overset{\Cplx}{\to} \phi h$ point-wise in $\Real^+$ and $\abs{\phi_n h}\leq M$, whence by the theorem 23 (the DCT), the integral-modulus inequality, shown in theorem 24, and definition 97 it must be true that \[\abs{\int \phi_n db - \int \phi db } = \abs{\int \phi_n h d\abs{db}-\int \phi h d\abs{db}} \to 0\] Therefore it is indeed true that $\int \phi_n db \overset{\Cplx}{\to} \int \phi db$.

Furthermore since $b$ is of bounded variation, $d\abs{b}$ must be a finite measure on $\brac{\Real^+, \borel{\Real^+}}$, whence $M\in L^1_\Real\brac{\Real^+, \borel{\Real^+}, d\abs{b}}$. Thus since $\abs{\phi_n\brac{x}}\to\abs{\phi\brac{x}}$ everywhere on $\Real^+$, by the DCT on $\brac{\Real^+, \borel{\Real^+}, d\abs{b}}$ it must be true that $\int \abs{\phi_n} d\abs{b} \to \int \abs{\phi} d\abs{b}$.

Now for any $n\geq1$ the linearity of the complex Lebesgue integral over a complex measure (theorem ??) implies that \begin{align*}\int \phi_n db &= \phi\brac{0} db\brac{\obj{0}}+\sum_{k=0}^{n 2^n-1} \phi\brac{\frac{k}{2^n}} db\brac{\ploc{ \frac{k}{2^n}, \frac{k+1}{2^n} }} \\&= \phi\brac{0} b\brac{0} + \sum_{k=0}^{n 2^n-1} \phi\brac{\frac{k}{2^n}} \brac{ b\brac{\frac{k+1}{2^n}} - b\brac{\frac{k}{2^n}} } \end{align*}

Since $b$ is of finite variation, for all $s,t\in \Real^+$ with $s\leq t$ by theorem 80 $\abs{b\brac{t}-b\brac{s}} \leq \abs{b}\brac{t}-\abs{b}\brac{s}$, whence for all $n\geq1$ by linearity of the complex Lebesgue integral \begin{align*}\abs{\int \phi_n db} &\leq \abs{\phi\brac{0}} \abs{b\brac{0}} + \sum_{k=0}^{n 2^n-1} \abs{\phi\brac{\frac{k}{2^n}}} \abs{ b\brac{\frac{k+1}{2^n}} - b\brac{\frac{k}{2^n}} }\\&\leq \abs{\phi\brac{0}} \abs{b}\brac{0} + \sum_{k=0}^{n 2^n-1} \abs{\phi\brac{\frac{k}{2^n}}} \brac{ \abs{b}\brac{\frac{k+1}{2^n}} - \abs{b}\brac{\frac{k}{2^n}} }\\&=\int \abs{\phi_n} d\abs{b}\end{align*}

Since $\abs{\int \phi_n db} \to \abs{\int \phi db}$ and $\int \abs{\phi_n\brac{s}} d\abs{b}\brac{s} \to \int \abs{\phi\brac{s}} d\abs{b}\brac{s}$, the fact that $\abs{\int \phi_n db}\leq \int \abs{\phi_n\brac{s}} d\abs{b}\brac{s}$ implies $\abs{\int \phi db}\leq \int \abs{\phi\brac{s}} d\abs{b}\brac{s}$. Therefore from previous reasoning \[\abs{ \int f db } \leq \abs{ \int \phi db } + \epsilon \leq \int \abs{\phi\brac{s}} d\abs{b}\brac{s} + \epsilon \leq \int \abs{f\brac{s}} d\abs{b}\brac{s} + 2\epsilon\] whence for every $f\in L^1_\Cplx\brac{\Real^+, \borel{\Real^+}, \abs{db}}$ it is true that $\abs{ \int f db } \leq \int \abs{f\brac{s}} d\abs{b}\brac{s}$.

Let $E\in \borel{\Real^+}$ and let $h\in L^1_\Cplx\brac{\Real^+, \borel{\Real^+}, \abs{db}}$ be such that $\abs{h}=1$ and $db = \int h d\abs{db}$, where such map exists by theorem 62. Then for $f\defn \bar{h} 1_E$, by definition 97 it is true that \[\int \bar{h} 1_E db = \int \bar{h} 1_E h d\abs{db} = \int 1_E d\abs{db} = \abs{db}\brac{E}\] Therefore for any $E\in \borel{\Real^+}$ it is true that $\abs{db}\brac{E} = \abs{\int \bar{h} 1_E db}$, whence by the previous integral inequality \[\abs{db}\brac{E} \leq \int \abs{\bar{h\brac{s}}} 1_E d\abs{b}\brac{s} = \int 1_E d\abs{b} = d\abs{b}\brac{E} \] Hence $\abs{db}\leq d\abs{b}$ and $\abs{db}=d\abs{b}$.\\

\label{thm:trunc_cplx_siteltjes_tot_var}\noindent\textbf{Theorem} 84.
If $b:\Real^+\to\Cplx$ is a right-continuous map of bounded variation, then the total variation of its associated complex Stieltjes measure is equal to the Stieltjes measure associated with its total variation, $\abs{db}=d\abs{b}$. If $b:\Real^+\to\Cplx$ is a right-continuous map of finite variation, then for all $T\in \Real^+$ the truncated map $b^T\brac{t}\defn b\brac{t\wedge T}$ is right-continuous of bounded variation, and $\abs{db^T} = d\abs{b}^T = d\abs{b}\brac{\cdot \cap \clo{0,T}}$.

The first statement follows from theorem 14-19. So, let $b:\Real^+\to\Cplx$ be right-continuous of finite variation. Then for all $T\in \Real^+$ by ex:24 it is true that $b^T$ is a right-continuous map of bounded variation, whence by theorems 14-19 and ex:24 $\abs{db^T} = d\abs{b^T} = d\abs{b}^T$.

By the uniqueness of the usual Stieltjes measure associated with $\abs{b}^T$ it therefore must be true that $d\abs{b^T} = d\abs{b}^T = d\abs{b}\brac{\cdot\cap \clo{0,T}}$. Therefore whenever $b:\Real^+\to\Cplx$ is a right-continuous map of finite variation, the total variation of the complex Stieltjes measure associated with the truncated map $b^T$ is equal to the usual Stieltjes measure associated with the truncated total variation of $b$.\\

So as one can see the total variation of measures and the total variation of complex-valued maps share many striking similarities.

\noindent\textbf{Definition} 111.
Let $b:\Real^+\to E$ be a map, where $\brac{E, \Tcal_E}$ is a Hausdorff topological space. The map $b$ is c\`adl\`ag (continue \`a droite, limite \`a gauche) with respect to $E$ if $b$ is right-continuous and the limit $b\brac{t-}\defn \lim_{s\to t, s<t} b\brac{s}$ exists in $E$ for all $t\in \Real^+$, $t\neq 0$.

When $E$ is either $\Cplx$ or $\Real$ we define $b\brac{0-}=0$ and for all $t\in \Real^+$ we let $\Delta b\brac{t}\defn b\brac{t}-b\brac{t-}$.

%% Exercise 28 ex:28
\label{thm:cadlag_properties}
Let $E$ be a Hausdorff topological space and $b:\Real^+\to E$ be cadlag having values entirely in $E'\subseteq E$.

Let $t>0$ and suppose there are $x,y\in E$ such that $x=\lim_{s\uparrow t} b\brac{s}$ and $y=\lim_{s\uparrow t} b\brac{s}$. Hence if $U,V\in\Tcal_E$ are such that $x\in U$ and $y\in V$, then there exist $S_x,S_y<t$ with $b\brac{s}\in U$ for all $s\in \brac{S_x,t}$ and $b\brac{s}\in V$ for all $s\in \brac{S_y,t}$. Thus for any $s\in \brac{S_x\vee S_y, t}$ it is true that $b\brac{s}\in U\cap V$, whence $U\cap V\neq \emptyset$. Therefore $x,y\in E$ are such that $U\cap V\neq \emptyset$ for all $U,V\in\Tcal_E$ with $x\in U$ and $y\in V$, whence the fact that $\brac{E,\Tcal_E}$ is Hausdorff implies that $x=y$.

Let $x,y\in E'$ be such that $x\neq y$. Then, since $x,y\in E$ there exist $U,V\in \Tcal_E$ with $x\in U$ and $y\in V$ such that $U\cap V=\emptyset$. Now $U'\defn U\cap E'$ and $V'\defn V\cap E'$ are open in $\brac{E', \induc{\Tcal_E}_{E'}}$ and are such that both $x\in U'$ and $y\in V'$ and finally that $U'\cap V'\subseteq U\cap V = \emptyset$. Therefore the topological subspaces inherit the Hausdorff property.

Now the map $b$ may not be cadlag with respect to $E'$ in the specific case when there is some $t>0$ with $b\brac{t-}\notin E'$, since $E'$ is not required to include its limiting points. Another case is when $b$ is not right-continuous in the subspace $E'$.

Let $t\in \Real^+$ and pick any $U'\in \induc{\Tcal_E}_{\clo{E'}}$ with $b\brac{t}\in U$. Then there is $U\in \Tcal_E$ with $b\brac{t}\in U$, whence there is $S>t$ such that $b\brac{s}\in U$ for all $s\in \clop{t,S}$. Since $b$ takes values form $E'$, it must be true that $b\brac{s}\in U'$ for all $s\in \clop{t,S}$, which actually implies that $b$ is right-continuous with respect to $\Tcal_{\Real^+}$-$\induc{\Tcal_E}_{\clo{E'}}$. Now let $t>0$ and pick any $U\in \Tcal_E$ with $b\brac{t-}\in U$. Then by the definition of the limit in topological spaces there must be $S\in \Real^+$ with $S<t$ such that $b\brac{s}\in U$ for all $s\in \brac{S,t}$, whence $E'\cap U\neq \emptyset$. Therefore $b\brac{t-}\in \clo{E'}$ for all $t>0$, which implies that $b$ is cadlag with respect to the closure of $E'$ in $\brac{E,\Tcal_E}$.\\

%% This is an application of the established result to $\Real$ and $\Cplx$.
Since by theorem 4-?9 the subset $\Real$ is closed in $\Cplx$, it must be true that any cadlag map $b:\Real^+\to \Cplx$ with $b\brac{\Real^+}\subseteq \Real$ is also cadlag with respect to $\Real$.

Conversely, let $b:\Real^+\to \Real$ be a cadlag map. If $t\in \Real^+$ then for all $U\in \Tcal_\Cplx$ with $b\brac{t}\in U$, by theorem 4-?9 $U\cap \Real\in \Tcal_\Real$, whence there is $S>t$ with $b\brac{s}\in U\cap \Real$ for all $s\in \clop{t,S}$. Thus $b$ is right-continuous with respect to $\Tcal_\Cplx$.

Next for any $t>0$ the left limit $b\brac{t-}$ exists in $\Real$ and by inclusion in $\Cplx$. Further for any $U\in\Tcal_\Cplx$ with $b\brac{t-}\in U$ there is $\delta>0$ with $B_\Cplx\brac{b\brac{t-},\delta}\subseteq U$, whence for $B_\Real\brac{b\brac{t-}, \delta}$ there is $S\in \clop{0,t}$ with $b\brac{s}\in B_\Real\brac{b\brac{t-}, \delta}$. Since $B_\Real\brac{b\brac{t-}, \delta}=B_\Cplx\brac{b\brac{t-},\delta}\cap \Real$ this implies that there is $S\in \clop{0,t}$ with $b\brac{s}\in U$ for all $s\in \brac{S,t}$. In conclusion, a cadlag map $b:\Real^+\to\Real$ is also cadlag with respect to $\Cplx$.

%% Exercise 29 ex:29
Suppose $b:\Real^+\to \Cplx$ is cadlag and $\Delta b\brac{t} = 0$ for all $t\in\Real^+$. On the one hand, $\Delta b\brac{0} = 0$ implies that $b\brac{0}=b\brac{0-}=0$ while $b\brac{0+}=b\brac{0}$ by right-continuity. On the other, at every point $t>0$ the left limit $b\brac{t-}=\lim_{s\uparrow t} b\brac{s}$ is equal to $b\brac{t}=\lim_{s\downarrow t} b\brac{s}$, which means that $b$ is left-continuous at $t$ and thus simply continuous. Conversely if $b$ is continuous with $b\brac{0}=0$, then at any $t\in \Real$ $\lim_{s\uparrow t} b\brac{s}=b\brac{t}$, whence $\Delta b\brac{t} = b\brac{t}-b\brac{t-}=b\brac{t}-b\brac{t}=0$.

If $a:\Real^+\to\Real^+$ is a right-continuous, non-decreasing map with $a\brac{0}\geq 0$, then for all $t>0$ it is true that $a\brac{s}\leq a\brac{t}$ for all $s\in \clop{0,t}$, whence $\alpha\defn \sup_{s<t} a\brac{s}$ is not greater than $a\brac{t}$, implying that $\alpha$ a finite real number. Furthermore monotonic nature of $a$ implies that $\alpha=\lim_{s\uparrow t}a\brac{s}$, since for any $l<\alpha$ there is $S<t$ with $l<a\brac{S}\leq a\brac{s}\leq \alpha$ fro all $s\in\clop{S,t}$. Thus $a\brac{t-}=\alpha\leq a\brac{t}$ and the map has a finite left-limit. Since it is given that $a$ is right-continuous, by definition 111 it must be cadlag with respect to $\Real^+$ and $\Delta a\brac{t} = a\brac{t}-a\brac{t-}\geq 0$.

Let $a,b:\Real\to \Cplx$ be two cadlag maps and $\alpha\in \Cplx$. If $\alpha=0$ then $a+\alpha b = a$ and the $\Cplx$-linear combination is cadlag, so $\alpha\neq 0$. Since both $a$ and $b$ are right-continuous, for any $t\in \Real^+$ and $\epsilon>0$ there exist $S_a, S_b>t$ such that $\abs{a\brac{s}-a\brac{t}}<\frac{\epsilon}{2}$ for all $s\in \clop{t,S_a}$ and $\abs{b\brac{s}-b\brac{t}}<\frac{\epsilon}{2\abs{\alpha}}$ for all $s\in \clop{t,S_b}$. Thus for $S\defn S_a\wedge S_b>t$ is must be true that \begin{align*}\abs{a\brac{s}+\alpha b\brac{s} - a\brac{t}+\alpha b\brac{t} }&<\abs{a\brac{s} - a\brac{t}}+\abs{\alpha}\abs{b\brac{s} - b\brac{t} }\\&<\frac{\epsilon}{2} + \abs{\alpha}\frac{\epsilon}{2\abs{\alpha}}\end{align*} whence $a+\alpha b$ is right-continuous at $t\in \Real^+$.

Now since both $a$ and $b$ have left limits at any $t>0$, one can define $\gamma\defn a\brac{t-}+\alpha b\brac{t-}$. For this $\gamma\in \Cplx$ it is true that for all $\epsilon>0$ there is $S_a,S_b\in \clop{0,t}$ such that $\abs{a\brac{s}-a\brac{t-}}<\frac{\epsilon}{2}$ for all $s\in \brac{S_a,t}$ and $\abs{b\brac{s}-b\brac{t-}}<\frac{\epsilon}{2\abs{\alpha}}$ for all $s\in \brac{S_b,t}$. Hence for all $s\in \brac{S_a\vee S_b, t}$ \[\abs{a\brac{s}+\alpha b\brac{t} - \gamma} \leq \abs{a\brac{s} - a\brac{t-}}+\abs{\alpha}\abs{b\brac{s} - b\brac{t-}} < \epsilon\] whence $a+\alpha b$ has a left limit at $t>0$ equal to $a\brac{t-}+\alpha b\brac{t-}$. Therefore the map $a+\alpha b$ is cadlag with respect ot $\Cplx$.

Let $b:\Real^+\to \Cplx$ be a right-continuous map of finite variation. By theorem 14-15 there are non-decreasing right-continuous maps $a_1, a_2, a_3$ and $a_4$ with $a_i\brac{0}\geq 0$ such that $b=a_1-a_2+i\brac{a_3-a_4}$. By the above result $a_i:\Real^+\to\Real^+$ are cadlag with respect to $\Real^+$, and by extension $\Cplx$. Since $b$ is a linear combination of $\brac{a_i}_{i=1}^4$, by another result $b$ must be cadlag itself.

Let $a:\Real^+\to \Real^+$ be right-continuous non-decreasing with $a\brac{0}\geq 0$. Then by theorem 11 there exists a unique usual Stieltjes measure $da$ associated with $a$. Since $da\brac{\obj{0}}=a\brac{0}$ and by definition $a\brac{0-}=0$, it is true that $da\brac{\obj{0}}=\Delta a\brac{0}$. Now for any $t>0$ it is true that $\ploc{t-\frac{1}{n},t}\downarrow \obj{t}$ and that $da\brac{\ploc{t-\frac{1}{n}},t}=a\brac{t}-a\brac{t-\frac{1}{n}}<+\infty$, which together imply that $da\brac{\ploc{t-\frac{1}{n}}}\downarrow da\brac{\obj{t}}$ by theorem 8. Hence, since $a$ is cadlag it must be true, that $\lim_{n\to\infty} a\brac{t-\frac{1}{n}}=a\brac{t-}$, whence $da\brac{\obj{t}} = a\brac{t}-a\brac{t-} = \Delta a\brac{t}$.

Let $b:\Real^+\to \Cplx$ be a right-continuous map of bounded variation. Then first it is cadlag by the above reasoning, and second, by theorem 14-17 there exists a unique complex Stieltjes measure $db$ associated with $b$, such that $db\brac{\ploc{s,t}}=b\brac{t}-b\brac{s}$ for all $s,t\in \Real^+$ with $s\leq t$ and $db\brac{\obj{0}}=b\brac{0}$. Similarly to the case of non-negative non-decreasing map the nested sets for every $t>0$ it is true that $\ploc{t-\frac{1}{n},t}\downarrow \obj{t}$, which implies that $1_{\ploc{t-\frac{1}{n}, t}}\to 1_{\obj{t}}$. Therefore by theorem 99-9 $db\brac{\ploc{t-\frac{1}{n},t}}\to db\brac{\obj{t}}$, whence \[db\brac{\obj{t}} = b\brac{t}-\lim_{n\to\infty} b\brac{t-\frac{1}{n}} = b\brac{t}-b\brac{t-} = \Delta b\brac{t}\] Finally, $db\brac{\obj{0}}=b\brac{0}-0=\Delta b\brac{0}$. Hence $db\brac{\obj{t}}=\Delta b\brac{t}$ for all $t\in \Real^+$.

Let $b:\Real^+\to\Cplx$ be right-continuous map of finite variation and $T\in \Real^+$. If $t>T$ then within a sufficiently small neighbourhood of $t$ the map $b^T\brac{\cdot}=b\brac{T}$. So for any $\epsilon>0$ by the nature of $\Real$ permits picking $S\in \brac{T,t}$ with $\abs{b^T\brac{s}-b\brac{T}}=\abs{b\brac{s\wedge T}-b\brac{T}}<\epsilon$. Hence $b^T\brac{t-}=b\brac{T}$ for all $t>T$. Otherwise, if $t\in \clo{0,T}$ then $b^T\brac{s}=b\brac{s}$ for any $s\in \clop{0,t}$, whence $b^T\brac{t-} = b\brac{t-}$ for such $t\in \Real^+$.

Now, \begin{align*}\Delta b^T\brac{t} &= b^T\brac{t}-b^T\brac{t-} \\&= b^T\brac{t} 1_{\clo{0,T}} + b^T\brac{t} 1_{\brac{T,+\infty}} - b^T\brac{t-} 1_{\clo{0,T}} - b^T\brac{t-} 1_{\brac{T,+\infty}} \\ &= b\brac{t} 1_{\clo{0,T}} + b\brac{T} 1_{\brac{T,+\infty}} - b\brac{t-} 1_{\clo{0,T}} - b\brac{T} 1_{\brac{T,+\infty}}\\&=b\brac{t} 1_{\clo{0,T}} - b\brac{t-} 1_{\clo{0,T}}\end{align*} whence $\Delta b^T\brac{t} = \Delta b\brac{t} 1_{\clo{0,T}}$.

Since by theorem ex:24 the map $b^T$ is right-continuous of bounded variation, there is a unique complex Stieltjes measure $db^T$ associated with it and by the results above $db^T\brac{\obj{t}}=\Delta b^T\brac{t} = \Delta b\brac{t} 1_{\clo{0,T}}\brac{t}$ for all $t\in \Real^+$.

%% Exercise 30 ex:30
\label{thm:cadlag_bounded}\noindent\textbf{Theorem} 85.
Let $b:\Real^+\to\Cplx$ be a cadlag map and $T\in \Real^+$. Then $b$ and the map $t\to b\brac{t-}$ are bounded on $\clo{0,T}$, i.e. there is such $M\in \Real^+$ that for all $t\in \clo{0,T}$ \[\abs{b\brac{t}}\vee \abs{b\brac{t-}}\leq M\]

Indeed, let $b:\Real^+\to\Cplx$ be a cadlag map and $T\in \Real^+$. Since $b$ is cadlag the map $t\to b\brac{t-}$ is well-defined.

Suppose that the map $t\to b\brac{t-}$ is not bounded on $\clo{0,T}$. Then for any $M\in \Real^+$ there is $t\in\clo{0,T}$ such that $\abs{b\brac{t-}}> M$. However for such $t$ for $\epsilon\def \abs{b\brac{t-}}-M>0$ there must be $S\in \clop{0,t}$ with $\abs{b\brac{s}-b\brac{t-}}<\epsilon$ for all $s\in \brac{S,t}$. Thus by the triangle inequality for any $s\in \brac{S,t}$ \[\abs{b\brac{t-}}\leq \abs{b\brac{s}}+\abs{b\brac{s}-b\brac{t-}}< \abs{b\brac{s}}+\epsilon\] whence $M=\abs{b\brac{t-}}-\epsilon<\abs{b\brac{s}}$. Hence is a sequence $\brac{t_n}_{n\geq1}\in \clo{0,T}$ with $\abs{b\brac{t_n}}\geq n$ for all $n\geq1$, implying that $\abs{b\brac{t_n}}\to\infty$, whence the map $b$ is not bounded on $\clo{0,T}$.

Suppose that the map $b$ is not bounded on $\clo{0,T}$. Then for any $M\in \Real^+$ there exists $t\in \clo{0,T}$ with $\abs{b\brac{t}}\leq M$. Thus there exists $\brac{t_n}_{n\geq1}\in \clo{0,T}$ with $\abs{b\brac{t_n}}\leq n$ for all $n\geq1$. Since by theorem 34 $\clo{0,T}$ is compact in $\Real$ and $\clo{0,T}\subseteq	\Real^+$, it must be compact in $\Real^+$ as well. Furthermore $\Tcal_{\clo{0,T}} = \induc{\Tcal_\Real}_{\clo{0,T}}$, whence $\brac{\clo{0,T}, \Tcal_{\clo{0,T}}}$ is a topological subspace of $\brac{\Real, \Tcal_\Real}$, which is metrizable. Thus by theorem 12 the space $\clo{0,T}$ is metrizable as well. Therefore by theorem 47 every sequence in $\clo{0,T}$ must have a convergent subsequence. Hence there exists $t\in \clo{0,T}$ and $\brac{n_k}_{k\geq1}\uparrow\infty$ such that $t_{n_k}\to t$ in $\clo{0,T}$ and thus in $\Real$. In conclusion there are $t,\brac{t_n}_{n\geq1}\in \clo{0,T}$ such that $t_n\to t$ and $\abs{b\brac{t_n}}\to \infty$.

Define $R\defn \obj{\induc{n\geq 1}\,t_n\geq t}$ and $L\defn \obj{\induc{n\geq 1}\,t_n < t}$. Since for any $n\geq$ either $t_n<t$ or $t_n\geq t$, it must be true that $\mathbb{N}=L \uplus R$.  Thus it cannot possibly be so that both $L$ and $R$ are finite, since a union of finite sets is itself finite.

Recall that $\brac{t_n}_{n\geq1}\in \clo{0,T}$ and $t_n\to t$, which means that for any $\epsilon>0$ there must be $N\geq1$ such that $\abs{t_n-t}<\epsilon$ for all $n\geq N$. If $R$ is infinite, then there must be $n\in R$ with $n>N$. Indeed, if there were no $n\in R$ with $n>N$, then $R\subseteq \obj{1,\ldots,N}$, which would mean that the set $R$ is finite. Therefore there must be $n_1>N$ and $n_1\in R$ for which $t_{n_1}\in \clop{t,t+1}\cap \clo{0,T}$.

Furthermore, if $n_0=0$ and $t_0=t+1$, then for every $k\geq1$ there must be $N\geq n_{k-1}+1$ such that $\abs{t_n-t}<t_{k-1}-t$ for all $n\geq N$. Due to $R$ being infinite there must be $n_k\in R$ with $n_k\geq N$, whence $t_{n_k}\in \clop{t,t_{n_{k-1}}}\cap \clo{0,T}$. Therefore there is $\brac{n_k}_{k\geq1}\in R$ with $n_k<n_{k+1}$, such that $t_{n_k}\downarrow t$, whence by right-continuity of $\abs{b\brac{\cdot}}$ at $t$, it must be true that $\abs{t_{n_k}}\to\abs{b\brac{t}}$ in $\Real^+$. This contradicts the fact that the fundamental property of $\brac{t_n}_{n\geq1}$ is that $\abs{b\brac{t_n}}\to\infty$, which implies that $R$ must be finite.

Suppose $L$ is infinite. Then, obviously, $L\neq \emptyset$, which actually implies that $t>0$. Let $n_0=0$ and $t_0=\frac{t}{2}>0$. For any $k\geq1$ the fact that $t_n\to t$ implies that there exists $N\geq n_{k-1}+1$ such that $\abs{t-t_n}<t-t_{n_{k-1}}$ for all $n\geq N$. Since $L$ is infinite, there must be $n_k\in L$ with $n_k\geq N$, whence \[t-t_n=\abs{t-t_n}<t-t_{n_{k-1}}\,\Rightarrow\, t_{n_{k-1}}<t_{n_k}\] and $t_{n_k}\in \brac{t_{n_{k-1}}, t}\clo{0,T}$. Hence there exists $\brac{n_k}_{k\geq1}\in L$ with $n_k\leq n_{k+1}$ and such that $t_{n_k}\uparrow t$. Since $b$ is cadlag, at $t>0$ its left-limit $b\brac{t-}$ must exist in $\Cplx$, whence $\abs{b\brac{t_{n_k}}}\to\abs{b\brac{t-}}\in \Real^+$. Thus $\abs{b\brac{t_{n_k}}}\not\to\infty$, despite the fact that it does, which is a contradiction.

In conclusion, neither $R$ nor $L$ can be infinite, whence it cannot be true that $t_n\to t$ in $\clo{0,T}$, which means that there cannot be a sequence in $\brac{t_n}_{n\geq1}\in \clo{0,T}$ in the first place. Thus the stack of assumptions, which led to a contradiction, unwinds back to the culprit, which is the one that $b$ is not bounded on $\clo{0,T}$. Therefore the map $b$ must be bounded on $\clo{0,T}$, whence by the very first observation the map $t\to b\brac{t-}$ must be bounded on $\clo{0,T}$ as well.

Hence there exist $M$ and $M_-$ in $\Real^+$ such that $\abs{b\brac{t}}\leq M$ and $\abs{b\brac{t-}}\leq M_-$ for all $t\in \Real^+$.\\

% section tut_14 (end)

\section{Stieltjes Integration} % (fold)
\label{sec:tut_15}
\url{http://probability.net/PRTstieltjesint.pdf}



% section tut_15 (end)

\section{Differentiation} % (fold)
\label{sec:tut_16}
\url{http://probability.net/PRTdifferentiation.pdf}

% section tut_16 (end)

\section{Image Measure} % (fold)
\label{sec:tut_17}
\url{http://probability.net/PRTimage.pdf}

% section tut_17 (end)

\section{Jacobian} % (fold)
\label{sec:tut_18}
\url{http://probability.net/PRTjacobian.pdf}

% section tut_18 (end)

\section{Fourier Transform} % (fold)
\label{sec:tut_19}
\url{http://probability.net/PRTfourier.pdf}

% section tut_19 (end)

\section{Gaussian Measures} % (fold)
\label{sec:tut_20}
\url{http://probability.net/PRTgaussian.pdf}

% section tut_20 (end)

\section{Supplementary results - A} % (fold)
\label{sec:sup_a}

\label{thm:continuity} \noindent \textbf{Theorem} Sup-A-1.
Let $\brac{\Omega,\Tcal}$ and $\brac{S,\Tcal_S}$ be two topological spaces and $f:\Omega\to S$ be a map. The following are equivalent:
\begin{enumerate}
	\item $f:\brac{\Omega, \Tcal}\to\brac{S, \Tcal_S}$ is continuous
	\item $f^{-1}\brac{M}$ is closed in $\Omega$ for every $M$ closed in $\brac{S, \Tcal_S}$
	\item continuity at every point of $\Omega$: for any $x\in \Omega$ and all $U\in \Tcal_S$ with $f\brac{x}\in U$ there exists $V\in \Tcal$ such that $x\in V$ and $f\brac{V}\subseteq U$
	\item $f\brac{\clo{A}}\subseteq \clo{f\brac{A}}$ for all $A\subseteq \Omega$
\end{enumerate}

$\Rightarrow$ Suppose $f$ is continuous, then for any $M\subseteq S$ closed in $\brac{S,\Tcal_S}$, $M^c\in \Tcal_S$, whence $\brac{f^{-1}\brac{M}}^c = f^{-1}\brac{M^c} \in \Tcal$, because $f^{-1}\brac{Y\setminus X} = f^{-1}\brac{Y}\setminus f^{-1}\brac{X}$, and so $f^{-1}\brac{M}$ is closed in $\brac{\Omega, \Tcal}$.

$\Leftarrow$ Suppose $f$ is such that for any $M$ closed in $S$, $f^{-1}\brac{M}$ is closed in $\Omega$. For any $U\in \Tcal_S$, $U^c$ is closed in $S$, from where it follows that $f^{-1}\brac{U^c}$ is closed in $\Omega$. Therefore $f^{-1}\brac{U}\in \Tcal$ and $f$ is continuous.


$\Rightarrow$ Suppose $f$ is continuous. Let $x\in \Omega$ and $U\in \Tcal_S$ is such that $f\brac{x}\in U$. Since $f$ is continuous, $V\defn f^{-1}\brac{U}\in \Tcal$ is such that $x\in V$ and $f\brac{V}\subseteq U$.

$\Leftarrow$ Suppose $f$ is such that for every $x\in \Omega$ and $U\in \Tcal_S$ with $f\brac{x}\in U$ there is $V\in \Tcal$ such that $x\in V$ and $f\brac{V}\subseteq U$. For any $U\in \Tcal_S$ and any $x\in f^{-1}\brac{U}$ there is $V\in \Tcal$ such that $x\in V$ and $f\brac{V}\subseteq U$, whence $V\subseteq f^{-1}\brac{U}$. Then $\Gamma\defn \obj{\induc{V\in \Tcal} V\subseteq f^{-1}\brac{U}}$ is non-empty, $\Gamma\subseteq \Tcal$ and $f^{-1}\brac{U} = \bigcup_{V\in \Gamma} V$, implying that $f^{-1}\brac{U}\in \Tcal$.


$\Rightarrow$ Suppose $f$ is continuous. For $A\subseteq S$, $\clo{f\brac{A}}$ is closed in $\brac{S,\Tcal_S}$, whence continuity implies that $f^{-1}\brac{\clo{f\brac{A}}}$ is closed in $\Omega$. If $x\in A$ then $f\brac{x}\in f\brac{A}\subseteq \clo{f\brac{A}}$ whence $A\subseteq f^{-1}\brac{\clo{f\brac{A}}}$. As the closure of $A$ is the smallest closed set containing $A$ and $f\brac{f^{-1}\brac{X}}\subseteq X$ for any $x\subseteq S$, $f\brac{\clo{A}}\subseteq \clo{f\brac{A}}$. Indeed, if $A\subseteq M$ and $M$ is closed in $\brac{\Omega, \Tcal}$, then $M^c\in \Tcal$. For any $x\notin M$, $x\in M^c$ and $A\cap M^c\subseteq M\cap M^c=\emptyset$, whence $x\notin \clo{A}$, implying $\clo{A}\subseteq M$.

Alternatively, if $x\in \clo{A}$, then $U\cap A\neq \emptyset$ for every $U\in \Tcal$ with $x\in U$. If $V\in \Tcal_S$ is such that $f\brac{x}\in V$, then $f^{-1}\brac{V}\cap A\neq \emptyset$. Since $f\brac{X\cap Y}\subseteq f\brac{X}\cap f\brac{Y}$ and $f\brac{f^{-1}\brac{X}}\subseteq X$, $f\brac{f^{-1}\brac{V}\cap A\subseteq}\subseteq V\cap f\brac{A}$ whence $V\cap f\brac{A} \neq \emptyset$, and so $f\brac{x}\in \clo{f\brac{A}}$. Thus $f\brac{\clo{A}}\subseteq \clo{f\brac{A}}$.

$\Leftarrow$ Suppose $f$ is such that $f\brac{\clo{A}}\subseteq \clo{f\brac{A}}$ for every $A\subseteq \Omega$. Then for $M$ closed in $S$, then \[f\brac{\clo{f^{-1}\brac{M}}}\subseteq \clo{f\brac{f^{-1}\brac{M}}}\subseteq \clo{M}\] whence $\clo{f^{-1}\brac{M}}\subseteq f^{-1}\brac{M}$, since $\clo{M}=M$. Therefore $f^{-1}\brac{M}$ is closed in $\brac{\Omega,\Tcal}$ and $f$ is continuous.\\

\label{thm:cont_prop_2} \noindent \textbf{Theorem} Sup-A-2.
Let $\brac{\Omega,\Tcal}$ and $\brac{S,\Tcal_S}$ be topological spaces.

If $f:\brac{\Omega, \Tcal}\to\brac{S, \Tcal_S}$ is continuous and $f\brac{\Omega} \subseteq F\subseteq S$ then $g:\Omega\to F$ defined as $g\brac{x}\defn f\brac{x}$ on $x\in \Omega$ is $\brac{\Omega, \Tcal}$-$\brac{F, \Tcal_F}$ continuous, where $\Tcal_F\defn \induc{\Tcal_S}_F$. Indeed, let $U'\in \Tcal_F$ then there is $U\in \Tcal_S$ with $U'=U\cap F$. Since $g\brac{\Omega} = f\brac{\Omega}\subseteq F$, $g^{-1}\brac{U\cap F} = g^{-1}\brac{U} = f^{-1}\brac{U}$ and $h^{-1}\brac{U'}\in \Tcal$.

If $f:\brac{\Omega, \Tcal}\to\brac{S, \Tcal_S}$ is continuous and $S\subseteq F$ with $\Tcal_S = \induc{\Tcal_F}_S$ then $h:\Omega\to F$ defined as $h\brac{x}\defn f\brac{x}$ on $x\in \Omega$ is $\brac{\Omega, \Tcal}$-$\brac{F, \Tcal_F}$ continuous. Indeed, since $h\brac{\Omega} = f\brac{\Omega}\subseteq S$, $h^{-1}\brac{U} = h^{-1}\brac{U\cap S} = f^{-1}\brac{U\cap S}$ for any $U\subseteq F$. However $U\cap S\in \Tcal_S$ for any $U\in \Tcal_F$, whence $h^{-1}\brac{U}\in \Tcal$.\\

\label{thm:cont_comp} \noindent \textbf{Theorem} Sup-A-3.
Let $\brac{\Omega,\Tcal}$, $\brac{S,\Tcal_S}$ and $\brac{F,\Tcal_F}$ be topological spaces. If $f\brac{\Omega,\Tcal}\to\brac{F,\Tcal_F}$ and $g:\brac{F,\Tcal_F}\to \brac{S,\Tcal_S}$ are continuous, then $g\circ f:\brac{\Omega,\Tcal}\to\brac{S,\Tcal_S}$ is continuous.

Indeed, $V\defn g^{-1}\brac{U}\in \Tcal_F$ for any $U\in \Tcal_S$ and $f^{-1}\brac{V}\in \Tcal$ for such $V$. Therefore, $h\defn g\circ f$ is such that $h^{-1}\brac{U} \in \Tcal$.\\

\label{thm:cont_prop_3} \noindent \textbf{Theorem} Sup-A-4.
Let $\brac{\Omega,\Tcal}$ and $\brac{S,\Tcal_S}$ be topological spaces.

Indeed, let $y_0\in S$ $f:\Omega\to S$ be such that $f\brac{\omega}=y_0$ for all $\omega \in \Omega$. Then for any $U$ open in $S$, $f^{-1}\brac{U}=\Omega$ in case $y_0\in U$ and $f^{-1}\brac{U}=\emptyset$ otherwise. Therefore $f$ is a continuous map.

Let $i:\Omega\to \Omega$ be defined as $i\brac{\omega}\defn\omega$ for all $\omega\in \Omega$. Since $i^{-1}\brac{U}=U$, $i:\brac{\Omega, \Tcal}\to\brac{\Omega, \Tcal}$ is continuous. Let $A\subseteq \Omega$ be any subset. The inclusion map $j_A:A \to \Omega$, defined as the expansion of $i$'s range to $\Omega$ is $\brac{A,\induc{\Tcal}_A}$-$\brac{\Omega, \Tcal}$ continuous by theorem Sup-A-2. Let $f:\brac{\Omega,\Tcal}\to \brac{S,\Tcal_S}$ be continuous. Since $\induc{f}_A\defn f\circ j_A$, by theorem Sup-A-3 $\induc{f}_A:\brac{A, \induc{\Tcal}_A}\to\brac{S,\Tcal_S}$ is continuous.\\

\label{thm:local_cont} \noindent \textbf{Theorem} Sup-A-5 (Local formulation of continuity).
Let $\brac{\Omega,\Tcal}$ and $\brac{S,\Tcal_S}$ be topological spaces. Suppose $f:\Omega\to S$ is such, that there is $\brac{U_i}_{i\in I}\in \Tcal$ with $\Omega = \bigcup_{i\in I} U_i$ and $\induc{f}_{U_i}:\brac{U_i, \induc{\Tcal}_{U_i}}\to \brac{S, \Tcal_S}$ continuous for every $i\in I$. Then $f:\brac{\Omega, \Tcal}\to \brac{S, \Tcal_S}$ is continuous.

Let $f_i\defn \induc{f}_{U_i}$ and $V\in \Tcal_S$. On the one hand, $\Omega=\bigcup_{i\in I} U_i$, so if $x\in f^{-1}\brac{V}$, then $f\brac{x}\in V$ and $\exists j\in I$ with $x\in U_j$, whence $f_j\brac{x}\in V$ and $x\in f^{-1}_j\brac{V}$, because $f_j\brac{x}=f\brac{x}$. On the other -- if $x\in \bigcup_{i\in I} f^{-1}_i\brac{V}$ then there is $j\in I$ with $f_j\brac{x}\in V$, whence $x\in U_j$ and $f\brac{x}\in V$. Now, $f^{-1}_i\brac{V}\in \induc{\Tcal}_{U_i}$ for every $i\in I$, from where follows existence of $\brac{W_i}_{i\in I}\in\Tcal$ such that $f^{-1}_i\brac{V} = W_i \cap U_i$ for all $i\in I$. Since $\brac{U_i}_{i\in I}$ are open in $\Omega$, $f^{-1}_i\brac{V}$ are open in $\Omega$, whence $f^{-1}\brac{V}\in \Tcal$. Therefore $f:\brac{\Omega, \Tcal}\to \brac{S, \Tcal_S}$ is continuous.\\

\label{thm:cont_pasting} \noindent \textbf{Theorem} Sup-A-6 (Pasting lemma).
Let $\brac{U_i}_{i\in I}$ be non-empty and open in $\Omega$ with $\Omega =\bigcup_{i\in I}$ and $\brac{f_i}_{i\in I}:\brac{U_i, \induc{\Tcal}_{U_i}} \to \brac{S, \Tcal_S}$ be continuous and such that $x\in U_i\cap U_j$ implies $f_i\brac{x}=f_j\brac{x}$ for every $i,j\in I$.

Define the map $f:\Omega\to S$ as follows: for any $x\in \Omega$ let $f\brac{x}\defn f_j\brac{x}$ where $j\in I$ is such that $x\in U_j$. The map $f$ is well defined everywhere on $\Omega$, since $f_i$ coincide on the non-empty pairwise intersections of their domains and $\Omega = \bigcup_{i\in I} U_i$. 

Let $i\in I$ and pick any $x\in U_i$. Then $x\in \Omega$, whence there is $j\in I$ with $x\in U_j$. Thus $U_i\cap U_j$ is non-empty and $\induc{f}_{U_i} = f\brac{x} = f_j\brac{x}$. Since $f_i$ and $f_j$ coincide on non-empty $U_i\cap U_j$, $\induc{f}_{U_i} = f_i\brac{x}$. Therefore $\induc{f}_{U_i}=f_i$ for all $i\in I$.

By theorem Sup-A-5 continuity of $f_i$ with respect to $\brac{U_i, \induc{\Tcal}_{U_i}}$--$\brac{S, \Tcal_S}$ for each $i\in I$ implies that the map $f:\Omega\to S$, obtained by ``pasting'' $\brac{f_i}_{i\in I}$ together, is $\brac{\Omega, \Tcal}$--$\brac{S, \Tcal_S}$ continuous.\\

\label{thm:cont_concat} \noindent \textbf{Theorem} Sup-A-7.
Let $\brac{\Omega_i, \Tcal_i}_{i\in I}$ be a family of topological spaces and $\brac{\Omega, \Tcal}$ -- a topological space. Then $\brac{f_i}_{i\in I}:\brac{\Omega, \Tcal}\to\brac{\Omega_i, \Tcal_i}$ are continuous if and only if the map $f:\Omega\to \prod_{i\in I} \Omega_i$ defined by $f\brac{\omega}\defn \brac{f_i\brac{\omega}}_{i\in I}$ is $\Tcal$--$\bigodot_{i\in I} \Tcal_i$ continuous.

First, for any $A\in \coprod_{i\in I} \Tcal_i$, $A = \prod_{i\in I} A_i$ and $J_A\defn \obj{ \induc{i\in I} A_i\neq \Omega_i }$ is finite. Since $f^{-1}_i\brac{\Omega_i} = \Omega$ for every $i\in I$, $f^{-1}\brac{\prod_{i\in I} A_i} = \bigcap_{i\in I} f^{-1}_i\brac{A_i} = \bigcap_{i\in J_A} f^{-1}_i\brac{A_i}$. Thus, if $\brac{f_i}_{i\in I}:\brac{\Omega, \Tcal}\to\brac{\Omega_i, \Tcal_i}$ are continuous then $f^{-1}\brac{A}\in \Tcal$.

Conversely, if $f:\brac{\Omega, \Tcal}\to \brac{\prod_{i\in I} \Omega_i, \bigodot_{i\in I} \Tcal_i}$ is measurable then for all $j \in I$ and every $B_j\in \Tcal_j$, $B\defn B_j\times \prod_{i\neq j,\,i\in I} \Omega_i$ is such that $B \in \bigodot_{i\in I} \Tcal_i$, whence $f_j^{-1}\brac{B_j} = f^{-1}\brac{B} \in \Tcal$.\\

% section sup_a (end)

\section{Supplementary results - B} % (fold)
\label{sec:sup_b}

It is necessary to venture briefly into the topic of infinite summation. Let $\brac{a_i}_{i\in I}\in \Zinf$ be an arbitrary collection of non-negative numbers with $I\neq \emptyset$. The sum of this collection is defined as \[\sum_{i\in I} a_i \defn \sup\obj{ \induc{ \sum_{i\in F} a_i } F\subseteq I\,\text{-- finite} }\]

\label{thm:inf_sum_consistency} \noindent \textbf{Theorem} Sup-B-1.
This definition of infinite summation of non-negative values is consistent with finite summation.

Indeed, in the case of a finite collection of numbers $\sum_{k\in F} a_k \leq \sum_{k\in I}^\text{finite} a_k$ for any finite $F\subseteq I$ because $a_k$ are non-negative. On the other hand $\sum_{k\in I}^\text{finite} a_k \leq \sum_{k\in I} a_k$ because $I$ is a finite set. Therefore $\sum_{k\in I} a_k = \sum_{i=1}^n a_{k\brac{i}}$, where $k:\obj{1\ldots n}\to I$ is an arbitrary bijection and $n=\abs{I}<+\infty$.\\

\label{thm:inf_sum_permutation} \noindent \textbf{Theorem} Sup-B-2.
The result of infinite summation of non-negative values is invariant under permutations.

Indeed, if $\phi:I\to I$ an arbitrary bijection, then $\phi\brac{F}$ is finite if and only if $F\subseteq I$ is finite. Therefore by definition of the infinite sum of non-negative values $\sum_{i\in I} a_{\phi\brac{i}} = \sum_{i\in I} a_i$.\\

\label{thm:non_neg_series_partition} \noindent \textbf{Theorem} Sup-B-3.
The generalized partition-invariance property of infinite sums of non-negative numbers.

If $\brac{I_\lambda}_{\lambda\in \Lambda}$ is a partition of $I$, then $\sum_{i\in F} a_i\leq \sum_{\lambda\in \Lambda} \sum_{i\in I_\lambda} a_i$ for any finite $F\subseteq I$ because numbers are non-negative. Therefore \[\sum_{i\in I} a_i \leq \sum_{\lambda\in \Lambda} \sum_{i\in I_\lambda} a_i\]

If $\sum_{i\in I} a_i < +\infty$, then $I_\lambda \subseteq I$ implies that $\sum_{i\in I_\lambda} a_i < +\infty$ for all $\lambda\in \Lambda$. For any fixed finite $L\subseteq \Lambda$, for any $\epsilon>0$ there exist $F_\lambda\subseteq I_\lambda$, $\lambda\in L$, such that $\sum_{i\in I_\lambda} a_i - \sfrac{\epsilon}{\abs{L}} < \sum_{i\in F_\lambda} a_i$. Hence, $\sum_{\lambda\in L} \sum_{i\in I_\lambda} a_i - \epsilon < \sum_{\lambda\in L} \sum_{i\in F_\lambda} a_i \equiv \sum_{i\in F} a_i \leq \sum_{i\in I} a_i$, where $F=\uplus_{\lambda \in L} I_\lambda$. This implies that \[\sum_{\lambda\in \Lambda} \sum_{i\in I_\lambda} a_i \leq \sum_{i\in I} a_i\] If $\sum_{i\in I} a_i = +\infty$, then the inequality it trivially true.\\

This generalized partition-invariance property of infinite sums of non-negative numbers implies, for instance, that $\sum_{i\in I} \sum_{j\in J} a_{ij} = \sum_{\brac{i,\,j}\in I\times J} a_{ij} = \sum_{j\in J} \sum_{i\in I} a_{ij}$. Also nothing precludes $\Lambda$ and $I_\lambda$ for all $\lambda\in \Lambda$ from being finite, whence, for instance, $\sum_{i\in I} a_i+b_i = \sum_{i\in I} a_i+\sum_{i\in I} b_i$.

%%% OVERFULL HERE
In particular, for any countable $\brac{a_n}_{n\geq 1}\in \Zinf$, the sum $\sum_{n=1}^\infty a_n$ reduces to $\lim_{n \to \infty} \sum_{k=1}^n a_k$ which, in turn, is equal to $\sup_{n\geq 1} \sum_{k=1}^n a_k \equiv \sum_{n\geq 1} a_n$ as finite partial sums are non-decreasing, and $\sum_{k = 1}^\infty a_k \equiv \sum_{k = 1}^\infty a_{\phi(k)}$ for any bijection $\phi:\mathbb{N}\to \mathbb{N}$. Also $\sum_{i\geq 1}\sum_{j\geq 1} a_{ij} \equiv \sum_{j\geq 1}\sum_{i\geq 1} a_{ij}$ for a double indexed sequence $\brac{a_{ij}}_{i,j\geq 1}\in \Zinf$.

% section sup_b (end)

\section{Supplementary results - Urysohn's lemma} % (fold)
\label{sec:urysohn_s_lemma}

\noindent \textbf{Definition}
A topological space $\brac{\Omega, \Tcal}$ is a $T_1$ space if for any $x\neq y\in \Omega$ there exist $U,V\in \Tcal$ with $x\in U$ and $y\in V$ such that $y\notin U$ and $x\notin V$.

\label{thm:t1_closed_atom}\noindent \textbf{Theorem} Urh-1.
In a $T_1$ space every point is a closed set.

Indeed, consider $x\in\Omega$ and let $y\neq x$. Then there exists an open neighbourhood $V\in \Tcal$ with $y\in V$ such that $x\notin V$, whence $y\notin\clo{x}$. Thus $\obj{x}$ coincides with its closure in $\brac{\Omega, \Tcal}$ and, consequently, is closed. Furthermore every finite subset must be closed in $\brac{\Omega,\Tcal}$, since every finite union of closed sets remains closed.\\

\noindent \textbf{Definition}
A limit point of a set $M\subseteq \Omega$ is such point $x\in \Omega$ that every open neighbourhood of $x$ contains points of $M$ other than $x$: $U\cap \brac{M\setminus \obj{x}}\neq \emptyset$ for every $U\in \Tcal$ with $x\in U$.


\label{thm:t1_limit_pt} \noindent \textbf{Theorem} Urh-2.
Let $\brac{\Omega, \Tcal}$ be a $T_1$ topological space and $M$ be some subset of $\Omega$. Then $x\in \Omega$ is a limit point of $M$ if and only if every open neighbourhood of $x$ contains infinitely many points of $M$ other than $x$.

Suppose $\omega\in \Omega$ is such that there is $V\in \Tcal$ with $\omega\in V$ such that $V\cap F$ is finite, where $F\defn M\setminus \obj{\omega}$. Since the space is $T_1$, theorem Urh-1 implies that $V\cap F$ is closed in $\brac{\Omega,\Tcal}$. Since a complement of a closed set is open by definition, a finite intersection of open sets is open, and $\omega\notin F$, $W\defn V\setminus F$ is an open neighbourhood of $\omega$. By construction $W\cap F=\emptyset$, whence for this $\omega$ there exists $W\in \Tcal$ with $\omega\in W$ such that $W\cap M\setminus \obj{\omega} = \emptyset$, implying that $\omega$ is not a limit point of $M$. Therefore a limit point of $M$ is such $x\in \Omega$, for which its every open neighbourhood contains infinitely many points of $M$ other than $x$. The converse is true by the very definition of a limit point.\\

\noindent \textbf{Definition}
A topological space $\brac{\Omega, \Tcal}$ is called $T_2$ space, or Hausdorff, if for any $x\neq y\in \Omega$ there exist $U,V\in \Tcal$ with $x\in U$ and $y\in V$ such that $U\cap V = \emptyset$. Obviously every $T_2$ space is a $T_1$ space.

\noindent \textbf{Definition}
A topological space $\brac{\Omega, \Tcal}$ is called $T_3$ space if for any set $X$ closed in $\brac{\Omega, \Tcal}$ and $y\in \Omega \setminus X$ there exist $U,V\in \Tcal$ with $X\subseteq U$ ($U$ is a ``neighbourhood'' of $X$) and $y\in V$ such that $U\cap V = \emptyset$. A topological space that is both $T_3$ and $T_1$ is called \emph{regular}.

Suppose $x\in \Omega$ and $U\in \Tcal$ is such that $x\in U$. Then $F\defn U^c$ is closed in $\brac{\Omega,\Tcal}$ and $x\notin F$. If the enclosing space is $T_3$ then there exist $V,W\in \Tcal$ with $F\subseteq V$ and $x\in W$ such that $V\cap W = \emptyset$. Note that the closure of $W$ cannot contain points from $F$, since $x\in V$ for every $x\in F$ and $V\cap W = \emptyset$. Therefore $\clo{W}\subseteq F^c = U$ and $x\in W\subseteq \clo{W}\subseteq U$. Thus in a $T_3$ space inside any open set around some point there is always the closure of a smaller open neighbourhood of the same point.

The converse is also true. Indeed, if $F$ is a closed set and $x\notin F$, then $F^c$ is an open neighbourhood of $x$, and thus there is $U\in \Tcal$, such that $x\in U$ and $\clo{U}\subseteq F^c$. Let $W\defn \Omega\setminus \clo{U}$ and note that $F\subseteq W$ and $W\in \Tcal$. Now $U\cap W = \emptyset$, since $U\subseteq \clo{U}$.

\noindent \textbf{Definition}
A topological space $\brac{\Omega, \Tcal}$ is called $T_4$ space if for any sets $X, Y$ closed in $\brac{\Omega, \Tcal}$ with $X\cap Y=\emptyset$ there exist $U,V\in \Tcal$ with $X\subseteq U$ and $Y\subseteq V$ such that $U\cap V = \emptyset$. This property is a straightforward generalisation of $T_3$ axiom. Similarly, a topological space that is both $T_4$ and $T_1$ is called \emph{normal}.

Analogously to the case of $T_3$ space, $T_4$ is equivalent to the following property: for any set closed $X$ and its open neighbourhood $U$ in $\brac{\Omega, T}$ there is another smaller open neighbourhood $V$ of $X$ such that $X\subseteq V\subseteq \clo{V}\subseteq U$. The proof is identical to the $T_3$ case except that instead of a point a closed set is considered.

The following lemma is rather complicated, but is so awesome, that I just have to prove it here.

\label{thm:urysohn} \noindent \textbf{Therefore} Urh-3.
Let $\brac{\Omega,\Tcal}$ be a $T_4$ topological space. Then for any pair of closed sets $A,B\subseteq \Omega$ with $A\cap B = \emptyset$, there exists a continuous function $\phi:\brac{\Omega, \Tcal}\to\brac{\Real,\Tcal_\Real}$ such that $\phi\brac{A}=0$ and $\phi\brac{B}=1$.

The $T_4$ axiom, or normality of the space in other words, allows one to construct a collection of prototypical level sets on which the function is to be defined. But prior to the actual construction of the level sets consider \[\Lambda\defn\obj{ \induc{\frac{m}{2^n}}\, n\geq 0, m = 1\ldots 2^n}\] This set is countable, since $\abs{\Lambda} \leq \abs{\mathbb{N}\times\mathbb{N}} = \abs{\mathbb{N}}$. This is also a set of so-called binary rational numbers and is known to be dense in $\clo{0,1}$ with respect to the natural subspace topology.

Indeed, for any $x \in \brac{0,1}$ and any $\epsilon>0$ take any $n \geq \lfloor -\log_2 \brac{\epsilon\wedge x} \rfloor + 1$ (for negative $n$ just take $n=1$) and $m = \lfloor x \cdot 2^n \rfloor$. Then $\frac{m}{2^n} \leq x < \frac{m+1}{2^n}$ , which implies that $0\leq x - \frac{m}{2^n} < \frac{1}{2^n} < \epsilon$. Therefore $\brac{0,1}\subseteq \clo{\Lambda}$.

If $x = 0$, then for any $\epsilon > 0$, there is $n \geq \lfloor -\log_2 \epsilon \rfloor + 1$ such that, $\frac{1}{2^n} - x < \epsilon$. Thus $0\in \clo{\Lambda}$. Finally, for $x=1$ and any $\epsilon > 0$, there is $n \geq \lfloor -\log_2 \epsilon \rfloor + 1$, $x - \frac{2^n-1}{2^n} = \frac{1}{2^n} < \epsilon$. Thus $\clo{0,1} \subseteq \clo{\Lambda}$.

The construction of the prototypical level sets is performed inductively. 
Start with $n=0$ and $m=1\ldots 2^n$, which makes $m=1$. From $A\cap B = \emptyset$ follows that $A\subset \Omega \setminus B$, so put $V_0 \defn A$ and $V_{\frac{m}{2^n}} = V_1 \defn \Omega \setminus B$. Note that $V_0 = \clo{V_0}\subset V_1$ and $V_1$ is open in $\brac{\Omega, \Tcal}$, while $V_0$ is not.

Now suppose that the collection of sets has been constructed for $n-1 \geq 0$. Let $m = 1\ldots 2^n$. If $m=2k$ for some $k=1\ldots 2^{n-1}$, then set $V_{\frac{m}{2^n}}\defn V_{\frac{k}{2^{n-1}}}$, which, by induction, is open in $\brac{\Omega, \Tcal}$, since $k\neq 0$. If $m=2k+1$ for some $k=1\ldots 2^{n-1}-1$, then $V_{\frac{m}{2^n}}$ is obtained from the normality of the topological space $\brac{\Omega, \Tcal}$. Indeed, by construction, $V_{\frac{m-1}{2^n}} \defn V_{\frac{k}{2^{n-1}}}$ and $V_{\frac{m+1}{2^n}} \defn V_{\frac{k+1}{2^{n-1}}}$ are open in $\brac{\Omega, \Tcal}$ and $\clo{V_{\frac{k}{2^{n-1}}}} \subset V_{\frac{k+1}{2^{n-1}}}$ Consequently, the normality of $\brac{\Omega, \Tcal}$ implies that there is a set $V_{\frac{m}{2^n}}\subseteq \Omega$ open in $\brac{\Omega, \Tcal}$, such that \[\clo{V_{\frac{m-1}{2^n}}} \subset V_{\frac{m}{2^n}} \subset \clo{V_{\frac{m}{2^n}}} \subset V_{\frac{m+1}{2^n}}\]

For any $A,B\subseteq \Omega$ closed in $\brac{\Omega, \Tcal}$ with $A\cap B = \emptyset$, this inductive procedure assigns an open set in $\brac{\Omega, \Tcal}$ to every binary rational from $\Lambda$. The resulting countable collection $\brac{V_\lambda}_{\lambda\in \Lambda} \in \Tcal$ is such that for any $\lambda_1 < \lambda_2$ from $\Lambda$ \[A = V_0 \subset V_{\lambda_1} \subset \clo{V_{\lambda_1}} \subset V_{\lambda_2} \subseteq V_1 = \Omega \setminus B\]

With the collection of level sets constructed, define $\phi:\brac{\Omega,\Tcal}\to \brac{\clo{0,1},\Tcal_{\clo{0,1}}}$ as follows: for $x\notin V_1$ put $\phi(x) = 1$, otherwise define it as \[\phi\brac{x} \defn \inf \obj{\induc{\lambda \in \Lambda}\, x\in V_\lambda}\] By definition, $\phi(x) = 1$ for every $x\notin V_1$, which means that $\phi = 1$ everywhere on $B$, since $\Omega \setminus V_1 = B$.

The map $\phi$ has a couple of very useful properties due to the density of $\Lambda$ in $\clo{0,1}$ and the way the collection $\brac{V_\lambda}_{\lambda \in \Lambda}$ has been constructed. First, for any $x\in V_0$ it is true that $x\in V_\lambda$ for every $\lambda \in \Lambda$. Furthermore, for every $\epsilon > 0$ there is $\lambda \in \Lambda$ such that $0 < \lambda < \epsilon$. Thus $x\in V_\lambda$ and, by definition, $\phi\brac{x} \leq \lambda$. Hence $\phi\brac{x} < \epsilon$ for every $\epsilon > 0$, implying that $\phi\brac{x}\leq 0$. Therefore $\phi = 0$ on $V_0$, which coincides with $A$.

Now let $x\in \clo{V_\beta}$ for some $\beta \in \Lambda$. Assume that $\phi\brac{x} > \beta$. Then there is some $\lambda \in \Lambda$ with $\beta < \lambda < \phi\brac{x}$, whence $\clo{V_\beta} \subset V_\lambda$. Thus $x \in V_\lambda$ and $\phi\brac{x} \leq \lambda$, which contradicts $\lambda < \phi\brac{x}$. Therefore in this case it must be $\phi\brac{x}\leq \beta$. To summarize for any $\beta\in \Lambda \cup \obj{0}$ it is true that $\phi\brac{x} > \beta$ implies $x \notin \clo{V_\beta}$.

This time, the fact that $\Lambda$ is dense in $\clo{0,1}$ is not used, just the properties of the greatest lower bound in $\clo{0,1}$. Let $\beta \in \Lambda$ and $\phi\brac{x} < \beta$. Then, by definition of $\phi$, there is $\lambda\in \Lambda$ with $\phi\brac{x} \leq \lambda < \beta$ such that $x\in V_\lambda$. By construction $\clo{V_\lambda} \subset V_\beta$ which implies that $x\in V_\beta$. Consequently, if $x\notin V_\beta$ then $\phi\brac{x}\geq \beta$. For $\beta = 0$ the property holds since $\phi\geq 0$ by definition.

The properties, shown above allow one to show that $\phi$ is $\Tcal$-$\Tcal_{\clo{0,1}}$ continuous.

Let $x\in \Omega$ be such that $\phi\brac{x} \in \brac{a,b}$, with $a,b \in \brac{0,1}$. Since $\brac{a,b}$ is open in $\Real$, there is some $\epsilon > 0$ small enough to fit $\brac{\phi\brac{x}-\epsilon, \phi\brac{x}+\epsilon}$ inside $\brac{a,b}$. Since $\Lambda$ is dense in $\clo{0,1}$, there are $p,q\in \Lambda$ with $0<a<\phi\brac{x}-\epsilon<p<\phi\brac{x}$ and $\phi\brac{x}<q<\phi\brac{x}+\epsilon<b<1$. Therefore $x\in V_q$ and $x\notin \clo{V_p}$. Consequently $x\in V_q\setminus \clo{V_p}$, which, being a finite intersection of open sets, is open in $\brac{\Omega, \Tcal}$. Now, if $y\in V_q\setminus \clo{V_p}$, then $y\in V_q$ and $y\notin \clo{V_p}$. The former implies that $y\in \clo{V_q}$ yielding $\phi(y)\leq q$, while the latter gives $\phi(y) \geq p$, since $y\notin \clo{V_p}$ implies that $ y\notin V_p$. Therefore $\phi(y)\in \clo{p,q}$ and the interval is contained in $\brac{\phi\brac{x}-\epsilon, \phi\brac{x}+\epsilon} \subseteq \brac{a,b}$. Hence for any $x\in \phi^{-1}\brac{\brac{a,b}}$ there are $p,q\in \Lambda$ such that $x \in V_q\setminus \clo{V_p} \subseteq \phi^{-1}\brac{\brac{a,b}}$.

Let $x\in \phi^{-1}\brac{\ploc{a,1}}$, for some $a\in \brac{0,1}$. If $\phi\brac{x}<1$ then the previous case applies. If, however, $\phi\brac{x} = 1$, then, since $\ploc{a,1}$ is open in $\clo{0,1}$ and $\Lambda$ is dense, there exists $p\in \Lambda$ such that $a < p < 1$. But then $\phi\brac{x} > p$, whence $x\in \Omega\setminus \clo{V_p}$, which is open in $\brac{\Omega, \Tcal}$. Now, if $y\in \Omega\setminus \clo{V_p}$, then $p\leq \phi\brac{y}\leq 1$. Thus $\phi\brac{y} \in \clo{p,1} \subset \ploc{a,1}$, whence $\Omega\setminus \clo{V_p}\subseteq \phi^{-1}\brac{\ploc{a,1}}$. To sum up, for every $x\in \phi^{-1}\brac{\ploc{a,1}}$ there is $p\in \Lambda$ such that $x\in \Omega\setminus \clo{V_p} \subset \phi^{-1}\brac{\ploc{a,1}}$.

Finally, let $x\in \phi^{-1}\brac{\clop{0,b}}$, for some $b\in \brac{0,1}$. Once again, if $\phi\brac{x} > 0$ then the first case, studied above, applies. If however, $\phi(x)=0$ then the openness of $\clop{0,b}$ in $\clo{0,1}$ and density of $\Lambda$ guarantee the existence of $q\in \Lambda$, such that $0 < q < b$. Thus $\phi(x) < q$ implies that $x\in V_q$. If $y\in V_q\subset \clo{V_q}$, then $0\leq \phi(y) \leq q < b$. Therefore $\phi(y) \in \clop{0,b}$ whence $V_q \subseteq \phi^{-1}\brac{\clop{0,b} }$. Therefore for any $x\in \phi^{-1}\brac{\clop{0,b}}$ there is $q\in \Lambda$ such that $x\in V_q \subset \phi^{-1}\brac{\clop{0,b}}$.

In summary for any $U\subseteq \clo{0,1}$ of the form $\brac{a,b}$, $\clop{0,b}$ or $\ploc{a,1}$ for $a,b \in \brac{0,1}$ it is true that for any $x\in \phi^{-1}\brac{U}$ there is $W\subseteq \Omega$ open in $\brac{\Omega, \Tcal}$ such that $x\in W\subset \phi^{-1}\brac{U}$.

Therefore $\phi^{-1}\brac{U}$ can be represented as an arbitrary union $\bigcup_{W\in \Gamma} W$, where $\Gamma$ is defined as $\obj{ \induc{W\in \Tcal}\, V\subseteq \phi^{-1}\brac{U}}$. However such sets form a topological basis of the natural topology on $\clo{0,1}$, induced by the natural topology on $\Real$ generated by the basis, defined by the sets of the form $\brac{a,b}$ for $a,b \in \Real$. Therefore $\phi:\brac{\Omega,\Tcal} \to \brac{\clo{0,1},\Tcal_{\clo{0,1}}}$ is a continuous map, since any open set in $\clo{0,1}$ is a union of arbitrary collection of sets from the basis of the topology.\\


% section urysohn_s_lemma (end)


\end{document}
